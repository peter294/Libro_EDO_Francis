% Notaciones.tex
\chapter*{Notación}
\addcontentsline{toc}{chapter}{Notación}
\markboth{Notación}{Notación}

% Que la tabla use todo el ancho
\setlength{\LTleft}{0pt}
\setlength{\LTright}{0pt}

\begin{longtable}{p{0.28\textwidth} p{0.68\textwidth}}
\hline
\textbf{Notación} & \textbf{Significado} \\
\hline
\endfirsthead
\hline
\textbf{Notación} & \textbf{Significado} \\
\hline
\endhead
\hline
\endfoot

$\mathbb{R}, \mathbb{C}$ & Conjuntos de números reales y complejos. \\
$[a,b]$ & Intervalo cerrado de $a$ a $b$. \\
$\Gamma(a)$ & Función gamma; ver \eqref{representacion de Gamma}. \\
$B(a,b)$ & Función beta. \\
$(a)_n$ & Símbolo de Pochhammer (factorial ascendente). \\
$\partial_x f$ & Derivada parcial de $f$ respecto a $x$. \\
$\dot{x}(t)$ & Derivada temporal de $x(t)$. \\
$C^k(\Omega)$ & Funciones con derivadas continuas hasta orden $k$ en $\Omega$. \\
$\nabla f$ & Gradiente de $f$. \\
$\Delta f$ & Laplaciano de $f$. \\
$\mathrm{diag}(a_1,\dots,a_n)$ & Matriz diagonal con entradas $a_1,\dots,a_n$. \\
% --- Añade tus propias filas debajo:
% \alpha & Tu descripción aquí. \\
\end{longtable}
