\documentclass{artbook}
%%%%%%%%%%%%%%%%%%%  Fancy Capitulos %%%%%%%%
%\usepackage[Sonny]{fncychap}
%Options (styles): Sonny, Lenny, Glenn, Conny, Rejne, Bjarne, Bjornstrup
\usepackage[tikz]{bclogo}
\usepackage{fancyvrb}
\usepackage{tikz}
\usetikzlibrary{arrows.meta}
\usepackage[boxedchapnumR]{settings/fncystyle} 
\usepackage{subcaption}

%\usepackage[tikz]{bclogo}
% inlinechap,  roundchapnum, boxedchapnum, boxedchapnumN, 
% boxedchapnumL, boxedchapnumR, quotchpI, quotchapII
%\graphicspath{{graphics/}}

%%%%%%%%%%%%%%%%%%%%%%%%%%%%%%%%%%%%%%%%%%%%%%%%%%%

% Definir estilo "plain" 
\fancypagestyle{plain}{
  \fancyhf{} % limpia encabezado y pie de página
  \renewcommand{\headrulewidth}{0pt}
  \renewcommand{\footrulewidth}{0pt}
  \fancyfoot[C]{\thepage} % número de página centrado
}

% Estilo para la página de la parte (sin encabezado ni número)
\fancypagestyle{partpage}{
	\fancyhf{}
	\renewcommand{\headrulewidth}{0pt}
	\renewcommand{\footrulewidth}{0pt}
}

% Para poner en negrita la numeración de las secciones
\renewcommand{\cftsecfont}{\bfseries}   % Numeración de secciones en negrita
\renewcommand{\cftsecpagefont}{\bfseries} % Páginas en negrita

% Configuración de mini TOC
\newcommand{\chaptertoc}{%

	\noindent\textbf{\large Índice}\\[0.2cm] % Título encima de la línea
	\noindent\rule{\linewidth}{0.4pt} % línea superior
	\startcontents[chapters]
	\printcontents[chapters]{}{1}{\setcounter{tocdepth}{1}}\renewcommand{\cftchapnumwidth}{\dimexpr\cftchapnumwidth+1em}
	\noindent\rule{\linewidth}{0.4pt} % línea inferior
	\vspace{0.5cm}
}

% Para el inicio de cada capitulo 
% Comando personalizado de introducción por capítulo
\newcommand{\fullpage}[1]{%

	\begin{fullwidth}[%\begin{fullwidth}[%
		width=\dimexpr\textwidth+\marginparsep+\marginparwidth,
		outermargin=\dimexpr-\marginparsep-\marginparwidth,
		]%
		
		% Aquí puedes colocar el contenido que desees
		#1
		
	\end{fullwidth}%
}

% Comando para el encabezado de cada capítulo 

\newcommand{\mychapter}[2]{%
	\begin{fullwidth}[
		width=\dimexpr\textwidth+\marginparsep+\marginparwidth,
		outermargin=\dimexpr-\marginparsep-\marginparwidth
		]%
		
		% El título del capítulo que el usuario pasa como argumento
		\chapter{#1}  % Aquí se coloca el título del capítulo
		
		\noindent
		\begin{minipage}[t]{0.90\textwidth}
			#2
			% Aquí se coloca el contenido de la primera columna
			% Puedes agregar más contenido aquí si lo deseas
		\end{minipage}%
		\hfill
		
		
		
	\end{fullwidth}%

\newpage
}

\newcommand{\mygeometry}[1]{
	\newgeometry{
		top=2.5cm,
		bottom=2.5cm,
		left=2.5cm,
		right=2.5cm
	}
	
	
}

% Paquetes
\usepackage[titles]{tocloft} % Personaliza títulos en la TOC
\usepackage{titletoc}        % Permite mini TOC por capítulos
\usepackage{minitoc}         % Para \startcontents y \printcontents
\usetikzlibrary{calc}
\usepackage{xparse} % Para argumentos más flexibles
%\contentsmargin{1em}
\setcounter{tocdepth}{2}%imprime solo hasta secciones (nivel 1)


% Estilo para capítulos en el índice
\titlecontents{chapter}
[2em]
{\bfseries\large}
{\hrule\vspace{0.25ex}\textcolor{uasdblue}{\contentslabel{0.5em}}\quad\bfseries\MakeUppercase}
{\hrule\vspace{0.25ex}\bfseries\MakeUppercase}
{\hfill\contentspage}
[\hrule \vskip 2.0em] % aquí va la línea final vamos agregar un hbox relleno con color uasdblue y blanco donde esta la numeracion de capitulo


% Configurar formato de capítulos en la tabla de contenido

% InsertBoxL definido en insbox.tex
\input{insbox/insbox.tex}



% Macro para insertar imagen en la ToC
\newcommand{\figuretoc}[1]{%
	\InsertBoxL{1}{%
		\begin{tikzpicture}[remember picture, overlay]
			\node[anchor=west] at (-0.2,-1.2) {% ← Cambia (1,0) para mover
				\includegraphics[height=4.0cm]{#1}
			};
		\end{tikzpicture}%
	}%
}

%\newcommand{\figureintoc}[1]{
	%	\begin{figure}
		%		\includegraphics[height=3cm]{#1}%
		%\end{figure}}
		
		% Definición para insertar una imagen pequeña en la ToC
		\newcommand{\figureintoc}[1]{%
			\protect\makebox[1.5em][l]{\protect\includegraphics[height=4cm]{#1}}%
		}
		
		
		\titlecontents{section}[14pc]{}{\contentslabel[\textbf{\color{uasdblue}\thecontentslabel}]{2pc}}{}{ \textbf{\textit{ \contentspage}}}[]
		%\dottedcontents{section}[6cm]{}{2.3em}{15.5pt}
		%\titlecontents{section}%
		%[6cm] % sangría
		%{\normalsize} % formato antes del número
		%{\contentslabel{2em}} % número como "1.1"
		%{} % si no hay número
		%{\hspace{1em} \textbf{\textit{\thecontentspage}}} % número de página
		%[] % espacio entre secciones
		%------ Agregar imagen en el capitulo -- 
		\author{Pedro Guzman}
		\title{Prueba del libro }
		
		\titlecontents{subsection}%
		[6.7cm] % sangría mayor que la sección
		{\normalsize} % texto más pequeño
		{\contentslabel{2.7em}} % espacio para el número (como 1.1.1)
		{}
		{\hspace{1.5em}\textbf{\textit{\thecontentspage}}} % número de página en negrita cursiva
		[] % espacio después de cada entrada
		
% ---------- Secciones  % 

\usepackage{etoolbox} % Para condicionales útiles
\usepackage{paracol}
\def\stackalignment{l} % Alineación izquierda para stackengine
\usepackage{stackengine}
\usetikzlibrary{shapes, positioning} % <-- ¡Importante!
\newcommand{\CustomSectionFormat}[1]{%
	\noindent
	\begin{tikzpicture}[baseline=(current bounding box.north)]
		% Número de sección dentro del óvalo
		\node[ellipse, draw=black, fill=uasdblue, text=white,
		inner xsep=4pt, inner ysep=2pt, anchor=west] (num) {\textbf{\large\thesection}};
		% Título al lado derecho del número
		\node[anchor=west] (title) at ([xshift=0.5em]num.east) {\textbf{\large #1}};
		% Línea azul debajo del conjunto
		%\draw[uasdblue, line width=1pt] (num.south west) -- ([xshift=title.east-|num.south west] title.south east);
	\end{tikzpicture}%
}

\titleformat{\section}
[block]
{\normalfont}
{}
{0pt}
{\CustomSectionFormat{#1}}

% Esquema de códigos para la seccion de ejercicios 

\usepackage{ifthen}
\newboolean{firstanswerofthechapter}  

\usepackage{xcolor}
\colorlet{lightcyan}{cyan!40!white}

\usepackage{chngcntr}
\usepackage{stackengine}

\usepackage{tasks}
\newlength{\longestlabel}
\settowidth{\longestlabel}{\bfseries 8.}
\settasks{label=\arabic*., label-format={\bfseries}, label-width=\longestlabel,
	item-indent=1.0cm, label-offset=0.2cm, column-sep={0.8cm}}

\usepackage[lastexercise,answerdelayed]{exercise}
\counterwithin{Exercise}{section}
\counterwithin{Answer}{section}
\renewcounter{Exercise}[section]
\newcommand{\QuestionNB}{\itshape}
\renewcommand{\ExerciseName}{EXERCISES}
\renewcommand{\ExerciseHeader}{\noindent\def\stackalignment{l}% code from https://tex.stackexchange.com/a/195118/101651
	\stackunder[0pt]{\colorbox{uasdblue}{\textcolor{white}{\textbf{\large\ExerciseHeaderNB\;\large\ExerciseName}}}}{\textcolor{uasdblue}{\rule{\linewidth}{1pt}  }} \par\vspace{1.5em} }
\renewcommand{\AnswerName}{Exercises}

\renewcommand{\AnswerHeader}{\ifthenelse{\boolean{firstanswerofthechapter}}%
	{\bigskip\noindent\textcolor{black}{\textbf{Capitulo \thechapter}}\newline\newline%
		\noindent\bfseries\emph{\textcolor{black}{\AnswerName\ \ExerciseHeaderNB, pagina %
				\pageref{\AnswerRef}}}\smallskip}
	{\noindent\bfseries\emph{\textcolor{black}{\AnswerName\ \ExerciseHeaderNB, pagina \pageref{\AnswerRef}}}\smallskip}}
\setlength{\QuestionIndent}{16pt}

%Estilo de pagian para respuesta 
\usepackage{fancyhdr}

\fancypagestyle{respstyle}{%
	\fancyhf{}
	% Páginas impares: borde a la derecha
	\fancyhead[RO]{%
		\begin{tikzpicture}[remember picture,overlay]
			% Rectángulo azul en todo el borde derecho
			\fill[uasdblue] 
			([xshift=-1.2cm]current page.east|-current page.north) 
			rectangle 
			([xshift=0cm]current page.east|-current page.south);
			
			% Texto vertical "RESPUESTAS"
			\node[rotate=90, text=white, font=\bfseries\large] 
			at ([xshift=-0.6cm]current page.east) {RESPUESTAS A LOS PROBLEMAS SELECCIONADOS};
			
			% En el nodo del estilo de página impar
			\node[text=black, font=\bfseries\large] 
			at ([xshift=-2.2cm,yshift=0.8cm]current page.south east) {Resp-\arabic{resp}};
			
		\end{tikzpicture}%
	}
	
	% Páginas pares: borde a la izquierda
	\fancyhead[LE]{%
		\begin{tikzpicture}[remember picture,overlay]
			% Rectángulo azul en todo el borde izquierdo
			\fill[uasdblue] 
			([xshift=0cm]current page.west|-current page.north) 
			rectangle 
			([xshift=1.2cm]current page.west|-current page.south);
			
			% Texto vertical "RESPUESTAS"
			\node[rotate=270, text=white, font=\bfseries\large] 
			at ([xshift=0.6cm]current page.west) {RESPUESTAS A LOS PROBLEMAS SELECCIONADOS};
			
			% En el nodo del estilo de página par
			\node[text=black, font=\bfseries\large] 
			at ([xshift=2.2cm,yshift=0.8cm]current page.south west) {Resp-\arabic{resp}};
		\end{tikzpicture}%
	}
}


\begin{document}
\makefrontcover
\frontmatter
%'------ Inicio del libro ---- %%%%%%%%%%


% -- titlepage -- %

\begin{titlepage}
\chapter*{Primera Edición}

%\Huge{\textsc{Ecuaciones Diferenciales}}
{\bfseries\fontsize{24}{30}\selectfont Curso Práctico De Ecuaciones Diferenciales Ordinarias Y Funciones
	Especiales, Con Aplicaciones.}
\vspace{0.2cm}

%{\bfseries\fontsize{24}{30}\selectfont { Y Funciones
%	Especiales, Con Aplicaciones.}}
%{\fontsize{30}{36}\selectfont \textsc{Ecuaciones Diferenciales}} 
\vspace{0.0cm}

%{\bfseries \fontsize{24}{30} \selectfont con problemas con valores  en la frontera}
%\vspace{2.0cm}
{\fontsize{30}{36}\selectfont \textcolor{uasdblue}{\textsc{Francis Álvarez P.}}} \\
\vspace{0.2cm}
{\fontsize{16}{24}\selectfont Universidad Autónoma De Santo Domingo }\\
\vspace{0.2cm}
{\fontsize{30}{36}\selectfont \textcolor{uasdblue}{\textsc{Jos\'e Angel Gomez}}}\\ 
\vspace{0.2cm}
{\fontsize{16}{24}\selectfont Universidad Autónoma De Santo Domingo }
\vspace{0.2cm}
{\fontsize{30}{36}\selectfont \textcolor{uasdblue}{\textsc{Juan Toribio Milane}}} \\
\vspace{0.2cm}
{\fontsize{16}{24}\selectfont Universidad Autónoma De Santo Domingo }
\vspace{0.2cm}
%\vspace{0.5cm}

{\fontsize{30}{36}\selectfont \textcolor{uasdblue}{\textsc{Pedro Guzmán G.}}}
\vspace{0.2cm}

%{\fontsize{16}{24}\selectfont  Matemático de aguita}

\vfill

%\begin{center}
%	\begin{minipage}{0.2\textwidth}
%		\includegraphics[width=\linewidth]{Imagen/uasd_escudo.jpg}
%	\end{minipage}
%	\hspace{0.5cm}
%	\begin{minipage}{0.7\textwidth}
%		\textcolor{uasdblue}{	{\fontsize{36}{42}\selectfont \textbf{UASD}}\\[1ex]
%			{\oldenglish \fontsize{16}{18}\selectfont Universidad Autónoma\\ de Santo Domingo}\\[1ex]
%			{\fontsize{10}{12}\selectfont PRIMADA DE AMÉRICA  |  Fundada el 28 de octubre de 1538}}
%	\end{minipage}
%\end{center}


	
\end{titlepage}



%%%%% Dedicatoria %%%%%%


\chapter*{Dedicatoria}


\begin{marginfigure}
	\includegraphics[width=\marginparwidth]{imagen/back1.png}
\end{marginfigure}

\lipsum[1-2]


 

%%%%% Agradecimientos %%%%%%

\fullpage{\chapter*{Agradecimientos}

La culminación de este libro sobre ecuaciones diferenciales no habría sido posible sin el apoyo y la colaboración de muchas personas e instituciones. 

En primer lugar, agradezco profundamente a mi familia, cuyo amor y apoyo incondicional me han dado la fuerza para completar este proyecto. Su paciencia y comprensión durante las largas horas de trabajo han sido fundamentales.

A mis colegas y estudiantes, quienes con sus preguntas, comentarios y debates enriquecieron el contenido de este libro. Sus aportes me motivaron a profundizar en los temas y a buscar formas más claras y didácticas de presentar los conceptos.

Agradezco también a la Universidad Autónoma de Santo Domingo (UASD) por brindarme los recursos y el entorno académico necesario para desarrollar este trabajo. Su compromiso con la educación y la investigación ha sido una fuente constante de inspiración.

Finalmente, expreso mi gratitud a todos los lectores que, con su interés en las ecuaciones diferenciales, dan sentido a este esfuerzo. Espero que este libro sea una herramienta útil en su aprendizaje y desarrollo académico.

\begin{flushright}
\textit{Pedro Guzmán} \\
\textit{Mayo de 2025}
\end{flushright}}

%%%%% Prefacio %%%%%%
%\chapter*{Prefacio}
%\addcontentsline{toc}{chapter}{Prefacio}

\chapter*{Acerca de los Autores}


\textcolor{uasdblue}{Juan Toribio Milanes} 
\begin{marginfigure} 
	\includegraphics[width=\marginparwidth]{imagen/back.png} 
	\caption*{Juan Toribio Milane}
	%\caption*{Juan Toribio Milane}
\end{marginfigure} 
\lipsum[1]

\vspace{0.5cm}
\textcolor{uasdblue}{Francis Álvarez Paulino}  
\begin{marginfigure} 
	\includegraphics[width=\marginparwidth]{imagen/back.png} 
	 \caption*{Francis Alvarez Paulino}
\end{marginfigure} 
\lipsum[1]

\vspace{0.5cm}
\textcolor{uasdblue}{Pedro Guzmán Guzmán}  
\begin{marginfigure} 
	\includegraphics[width=\marginparwidth]{imagen/back.png} 
	 \caption*{Pedro Guzmán Guzmán }
\end{marginfigure} 
\lipsum[1-2]



%--------- Tabla de contenido -------%
\newgeometry{	
	paperwidth=213mm,
	left=2cm,
	right=1cm,
	paperheight=279mm,
	top=1.65cm,
	bottom=1.3cm,
}

%\noindent
%\renewcommand\contentsname{Contenido}
%\addtocontents{toc}{~\hfill\textbf{Pagina}\par}
\dominitoc

\tableofcontents
\adjustmtc
\restoregeometry



\listoffigures
\listoftables

\mainmatter
% -- Capítulos -- %
%\mygeometry{
%	\part{Ecuaciones Diferenciales Ordinarias}
%	\thispagestyle{empty}
%}

% Notaciones.tex
\chapter*{Notación}
\addcontentsline{toc}{chapter}{Notación}
\markboth{Notación}{Notación}

% Que la tabla use todo el ancho
\setlength{\LTleft}{0pt}
\setlength{\LTright}{0pt}

\begin{longtable}{p{0.28\textwidth} p{0.68\textwidth}}
\hline
\textbf{Notación} & \textbf{Significado} \\
\hline
\endfirsthead
\hline
\textbf{Notación} & \textbf{Significado} \\
\hline
\endhead
\hline
\endfoot

$\mathbb{R}, \mathbb{C}$ & Conjuntos de números reales y complejos. \\
$[a,b]$ & Intervalo cerrado de $a$ a $b$. \\
$\Gamma(a)$ & Función gamma; ver \eqref{representacion de Gamma}. \\
$B(a,b)$ & Función beta. \\
$(a)_n$ & Símbolo de Pochhammer (factorial ascendente). \\
$\partial_x f$ & Derivada parcial de $f$ respecto a $x$. \\
$\dot{x}(t)$ & Derivada temporal de $x(t)$. \\
$C^k(\Omega)$ & Funciones con derivadas continuas hasta orden $k$ en $\Omega$. \\
$\nabla f$ & Gradiente de $f$. \\
$\Delta f$ & Laplaciano de $f$. \\
$\mathrm{diag}(a_1,\dots,a_n)$ & Matriz diagonal con entradas $a_1,\dots,a_n$. \\
% --- Añade tus propias filas debajo:
% \alpha & Tu descripción aquí. \\
\end{longtable}

\include{Cuerpo/Capitulos/Capitulo 1/Capitulo 1}

\mychapter{Funciones Especiales a Partir EDO de Segundo Orden}{ 
	
	\begin{wrapfigure}{r}{0.35\textwidth} %this figure will be at the right
		\centering
		\includegraphics[width=0.35\textwidth]{imagen/img5.png}
	\end{wrapfigure} En este capítulo se exploran diversas ecuaciones diferenciales especiales y sus soluciones, muchas de las cuales aparecen de manera recurrente en física matemática, mecánica cuántica, teoría de potenciales y análisis numérico. A lo largo de las secciones se estudiarán las ecuaciones de Laguerre, Jacobi, Chebyshev, Hermite, Bessel y Legendre, junto con sus polinomios asociados y propiedades fundamentales. Estos polinomios ortogonales no solo ofrecen soluciones exactas a problemas concretos, sino que también proporcionan herramientas poderosas para aproximaciones y expansiones en series.
	
	\vspace{0.5cm}
	
	Asimismo, se abordarán ecuaciones menos convencionales, como las de Airy, Gegenbauer y las funciones elípticas de SEzgo, que surgen en contextos más avanzados de análisis matemático y física aplicada. Cada sección incluye tanto la formulación general de la ecuación como las propiedades de sus soluciones y relaciones entre los distintos polinomios y funciones especiales. La comprensión de estas ecuaciones y sus soluciones es esencial para abordar problemas complejos que involucran simetrías, condiciones de contorno y series ortogonales, consolidando así una base sólida para estudios posteriores en matemáticas aplicadas y física teórica.
	}
	

 \addtocontents{toc}{\protect\figuretoc{imagen/img5.png}}
\section{Soluci\'on de una EDO entorno a un punto ordinario}
\subsection*{Repaso sobre series de potencias}	
	\Definition{EDO homog\'enea}{Una ecuaci\'on diferencial lineal de segundo orden es de la forma
		\begin{eqnarray}\label{EDL2}
			% \nonumber % Remove numbering (before each equation)
			y^{\prime\prime}(x)+p_{1}(x)y^{\prime}(x)+p_{2}(x)y(x) &=& 0
		\end{eqnarray}
		donde $p_{1}(x)\quad y \quad p_{2}(x)$ son funciones continuas en alg\'un intervalo J.}\label{EDO homog\'enea}
		
		\begin{fullwidth}[%
			width=\dimexpr\textwidth+\marginparsep+\marginparwidth,
			outermargin=\dimexpr-\marginparwidth,
			]
			
			\Definition{Series de potencia}{	Una serie de potencia es una serie de funciones de la forma
				\begin{eqnarray}\label{serie de potencia}
					% \nonumber % Remove numbering (before each equation)
					\displaystyle\sum_{m=0}^{\infty} c_{m}\left(x-x_{0}\right)^{m} &=& c_{0}+c_{1}\left(x-x_{0}\right)+c_{2}\left(x-x_{0}\right)^{2}+\cdots+c_{m}\left(x-x_{0}\right)^{m}+\cdots
				\end{eqnarray}
				donde los $c_{m}$ son coeficientes para $m=0,1,\cdots$ y el punto $x_{0}$ es independiente de $x$. El punto  $x_{0}$ se llama punto de expansi\'on de la serie.}
			
			
		\end{fullwidth}	
		
		Ahora mencionamos algunas propiedades importantes de series que se necesitar\'an mas adelante
		\Property{ Propiedades de series}{	
			\begin{enumerate}
				\item  Se dice que una serie de potencias
				\[
				\displaystyle\sum_{m=0}^{\infty} c_m\left(x - x_0\right)^m
				\]
				converge en un punto $x$ si
				\[
				\displaystyle\lim_{n \rightarrow \infty}\displaystyle \sum_{m=0}^{n} c_m\left(x - x_0\right)^m
				\]
				existe. Es claro que la serie converge en $x = x_0$; puede converger para todo $x$, o puede converger para algunos valores de $x$ y no para otros.
				\item Se dice que una serie de potencias
				\[
				\displaystyle\sum_{m=0}^{\infty} c_m\left(x - x_0\right)^m
				\]
				converge absolutamente en un punto $x$ si la serie
				\[
				\displaystyle\sum_{m=0}^{\infty} \left|c_m\left(x - x_0\right)^m\right|
				\]
				converge. Si la serie converge absolutamente, entonces la serie también converge; sin embargo, la recíproca no es necesariamente verdadera.
				\item Si la serie
				\[
				\displaystyle\sum_{m=0}^{\infty} c_m\left(x - x_0\right)^m
				\]
				converge absolutamente para $\displaystyle\left|x - x_0\right| < \mu$ y diverge para $\displaystyle\left|x - x_0\right| > \mu$, entonces a $\mu$ se le llama el radio de convergencia. Para una serie que no converge en ningún punto excepto en $x_0$, se define $\mu = 0$; y para una serie que converge para todo $x$, se dice que $\mu$ es infinito.
				\item  \textbf{Criterio del cociente}. Si, para un valor fijo de $x$,
				\[
				\displaystyle\lim_{m \rightarrow \infty} \left| \displaystyle\frac{c_{m+1}(x - x_0)^{m+1}}{c_m(x - x_0)^m} \right| = L,
				\]
				entonces la serie de potencias
				\[
				\displaystyle\sum_{m=0}^{\infty} c_m(x - x_0)^m
				\]
				converge absolutamente para los valores de $x$ tales que $L < 1$, y diverge cuando $L > 1$. Si $L = 1$, el criterio no permite concluir nada.
				\item \textbf{Criterio de comparación}. Si tenemos dos series de potencias
				\[
				\displaystyle\sum_{m=0}^{\infty} c_m(x - x_0)^m \quad \text{y} \quad \displaystyle\sum_{m=0}^{\infty} C_m(x - x_0)^m,
				\]
				donde $|c_m| \leq C_m$ para $m = 0, 1, \dots$, y si la serie
				\[
				\displaystyle\sum_{m=0}^{\infty} C_m(x - x_0)^m
				\]
				converge para $|x - x_0| < \mu$, entonces la serie
				\[
				\displaystyle\sum_{m=0}^{\infty} c_m(x - x_0)^m
				\]
				también converge para $|x - x_0| < \mu$.
				\item Si una serie
				\[
				\displaystyle\sum_{m=0}^{\infty} c_m(x - x_0)^m
				\]
				es convergente para $|x - x_0| < \mu$, entonces para cualquier $x$ tal que $|x - x_0| = \mu_0 < \mu$, existe una constante $M$ tal que
				\[
				\displaystyle |c_m| \, \mu_0^m \leq M \quad \text{para } m = 0, 1, \dots
				\]
				\item La derivada de una serie de potencias se obtiene derivando término a término; es decir, si
				\[
				\displaystyle f(x) = \displaystyle\sum_{m=0}^{\infty} c_m(x - x_0)^m,
				\]
				entonces
				\[
				\begin{aligned}
					f^{\prime}(x) &= c_1 + 2c_2(x - x_0) + 3c_3(x - x_0)^2 + \cdots \\
					&=\displaystyle \sum_{m=1}^{\infty} m c_m(x - x_0)^{m-1} =\displaystyle \sum_{m=0}^{\infty} (m+1) c_{m+1}(x - x_0)^m
				\end{aligned}
				\]
				
				Además, los radios de convergencia de estas dos series son iguales.
		\end{enumerate}}
		
		
		
		\textcolor{red}{Buscar diferentes definiciones de analiticidad y realizar una peque\~na introducci\'on. AGREGAR UNA GRAFICA DEL INTERVALO DE CONVERGENCIA DE LAS SERIES VER DENNI ZILL}
		
		\Definition{Función analítica}{Una funci\'on $f(x)$ es anal\'itica en $x=x_{0}$ si puede expresarse en serie de potencia en potencias de $(x-x_{0})$ en alg\'un intervalo de la forma $|x-x_{0}|<\mu$, donde $\mu>0$. Si $f(x)$ es anal\'itica en $x=x_{0}$, entonces
			\begin{eqnarray}\label{funcion en serie de potencia}
				% \nonumber % Remove numbering (before each equation)
				f(x)&=&\displaystyle\sum_{m=0}^{\infty} c_{m}\left(x-x_{0}\right)^{m}, \quad \left|x-x_{0}\right|<\mu
			\end{eqnarray}
			donde $c_{m}=f^{(m)}\left(x_0\right)/m !, m=0,1, \cdots$ que es la misma expansi\'on de Taylor de $f(x)$ en $x_{0}$.}
		Como sabemos de cursos de c\'alculo las siguientes funciones son anal\'iticas en el intervalo dado centrada en $x_{0}=0$.
		\begin{center}
			\begin{longtable}{|c|c|c|}
				\hline
				\textbf{Función} & \textbf{Serie de potencias} & \textbf{Intervalo de convergencia} \\
				\hline
				$e^x$ & $\displaystyle \sum_{n=0}^{\infty} \displaystyle\frac{x^n}{n!}$ & $(-\infty, \infty)$ \\
				\hline
				$\sin(x)$ & $\displaystyle \sum_{n=0}^{\infty} \displaystyle\frac{(-1)^n}{(2n+1)!}x^{2n+1}$ & $(-\infty, \infty)$ \\
				\hline
				$\cos(x)$ & $\displaystyle \sum_{n=0}^{\infty} \displaystyle\frac{(-1)^n}{(2n)!}x^{2n}$ & $(-\infty, \infty)$ \\
				\hline
				$\ln(1+x)$ & $\displaystyle \sum_{n=1}^{\infty} \displaystyle\frac{(-1)^{n+1}}{n}x^n$ & $(-1, 1]$ \\
				\hline
				$\displaystyle\frac{1}{1 - x}$ & $\displaystyle \sum_{n=0}^{\infty} x^n$ & $(-1, 1)$ \\
				\hline
				$\arctan(x)$ & $\displaystyle \sum_{n=0}^{\infty} \displaystyle\frac{(-1)^n}{2n+1}x^{2n+1}$ & $[-1, 1]$ \\
				\hline
				$\arcsin(x)$ & $\displaystyle \sum_{n=0}^{\infty} \displaystyle\frac{(2n)!}{4^n(n!)^2(2n+1)}x^{2n+1}$ & $[-1, 1]$ \\
				\hline
				$\sinh(x)$ & $\displaystyle \sum_{n=0}^{\infty} \displaystyle\frac{x^{2n+1}}{(2n+1)!}$ & $(-\infty, \infty)$ \\
				\hline
				$\cosh(x)$ & $\displaystyle \sum_{n=0}^{\infty}\displaystyle \frac{x^{2n}}{(2n)!}$ & $(-\infty, \infty)$ \\
				\hline
			\end{longtable}
		\end{center}
		%\Example{Titulo pendiente}{ejemplos de funciones anal\'iticas}
		
		\Definition{Puntos ordinario y singular}{Si en un punto $x=x_{0}$ las funciones $p_{1}(x)\quad y \quad p_{2}(x)$ son anal\'iticas, entonces $x_{0}$ se llama punto ordinario de la ecuaci\'on \ref{EDL2}. En caso de que las funciones $p_{1}(x)\quad y \quad p_{2}(x)$ no sean anal\'iticas en $x_{0}$ se dice que es un punto singular de \ref{EDL2}.}
		\Example{Punto ordinario y singular de una ecuación diferencial}{
			Determina si \( x = x_0 \) es un punto ordinario o singular de las siguientes ecuaciones diferenciales:
			
			\begin{enumerate}
				\item 
				\[
				x y^{\prime \prime} + \sin(x) y = 0, \quad x_0 = 0
				\]
				
				\item 
				\[
				(x^2 - 2x) y^{\prime \prime} + 5(x - 1) y^{\prime} + 3y = 0, \quad x_0 = 1
				\]
				
				\item 
				\[
				y^{\prime \prime} + e^x y^{\prime} + (1 + x^2) y = 0
				\]
				\item \textbf{Ecuación de Riccati-Bessel}
				\begin{equation}\label{RiccatiBessel}
					x^2 y^{\prime \prime} - (x^2 - k) y = 0, \quad -\infty < k < \infty
				\end{equation}
		\end{enumerate}}
		\begin{sol}
			\begin{enumerate}
				\item  
				
				$$
				\begin{aligned}
					& y^{\prime \prime}+\displaystyle\frac{\sin (x)}{x} y=0, x \neq 0\qquad \text{forma normal}\\
					& p_2(x)=\displaystyle\frac{\sin (x)}{x}  \hspace{6cm} \sin (x)=\displaystyle\sum_{n=0}^{\infty} \frac{(-1)^n}{(2 n+1)!} x^{2 n+1} \\
					& p_2(x)=\displaystyle\frac{1}{x} \sum_{n=0}^{\infty} \frac{(-1)^n}{(2 n+1)!} x^{2 n+1} \\
					& p_2(x)=\displaystyle\sum_{n=0}^{\infty} \frac{(-1)^n}{(2 n+1)!} x^{2 n}
				\end{aligned}
				$$
				Por lo tanto $x_0=0$ es un punto ordinario de $x y^{\prime\prime}+\sin (x) y=0$.
				\item $$
				\begin{aligned}
					& y^{\prime \prime}+\displaystyle\frac{5(x-1)}{x^2-2 x} y^{\prime}+\displaystyle\frac{3}{x^2-2 x}y=0, x \neq 0,2 \qquad \text{forma normal}\\
					& p_1(x)=5\displaystyle \frac{x-1}{x^2-2 x} \quad p_2(x)=\displaystyle\frac{3}{x^2-2 x}
				\end{aligned}
				$$
				Como $p_1(x)$ y $p_2(x)$ son diferenciable en $x=1$, entonces funciones analiticas.
				\item $$
				\begin{aligned}
					& y^{\prime \prime}+e^x y^{\prime}+\left(1+x^2\right) y=0 \\
					& p_1(x)=e^x\hspace{6cm} e^{x}=\displaystyle\sum_{n=0}^{\infty} \frac{x^n}{n!} \\
					& p_2(x)=1+x^2 \qquad \text{es diferenciable por lo tanto es analítica}
				\end{aligned}
				$$
				\item  $$
				\begin{aligned}
					&y^{\prime \prime}-\frac{x^2-k}{x^2} y=0 \\
					&p_2(x)=\displaystyle\frac{x^2-k}{x^2}
				\end{aligned}$$
				$p_{2}(x)$ no es analítica en $x_0=0$, tiene una singularidad esencial.
			\end{enumerate}
		\end{sol}
		Habiendo clasificado los puntos de diversas ecuaciones diferenciales como ordinarios o singulares, estamos en posición de aplicar ese análisis para construir soluciones. En particular, cuando el punto es ordinario, podemos resolver la ecuación mediante una serie de potencias centrada en dicho punto. A continuación, abordaremos este proceso retomando algunas de las ecuaciones ya estudiadas.\\
		Iniciamos este apartado presentando un teorema que presenta las condiciones suficientes para obtener una soluci\'on en series de potencia entorno a un punto ordinario.
		
		\Theorem{Soluci\'on de una EDO entorno a un punto ordinario}{Sean las funciones $p_{1}(x) \quad \text{y} \quad p_{2}(x)$ anal\'iticas en $x=x_{0}$; por lo tanto pueden ser expresadas como series de potencia en $(x-x_{0})$ en alg\'un intervalo $|x-x_{0}|<\mu$. Entonces, la ecuaci\'on definida en \ref{EDL2} con las condiciones iniciales
			\begin{eqnarray}
				% \nonumber % Remove numbering (before each equation)
				y(x_{0}) = c_{0}\quad\text{y}\quad y^{\prime}(x_{0}) = c_{1}
			\end{eqnarray}
			tienen una \'unica soluci\'on $y(x)$ anal\'itica en $x_{0}$, que puede ser expresada como
			\begin{eqnarray}\label{solenserie}
				y(x)&=& \displaystyle\sum_{m=0}^{\infty}c_{m}(x-x_{0})^{m}
			\end{eqnarray}
			en un intervalo $\mid x-x_{0}\mid<\mu$. Los coeficientes $c_{m}, m\geq 2$ de \ref{solenserie} se obtienen sustituyendolo directamente en la ecuaci\'on definida en \ref{EDO homog\'enea}}\label{teosolordinario}
		
		
		\begin{demo}
			\textcolor{red}{Escribir esta demostracion con la notacion del libro}\\
			Sin perder generalidad, asumiremos $x_{0}=0$\, sean
			\begin{eqnarray*}
				% \nonumber % Remove numbering (before each equation)
				p(x)=\sum_{m=0}^{\infty}\bar{p}_{m}x^{m}\quad q(x)=\sum_{m=0}^{\infty}\widetilde{q}_{m}x^{m}\quad |x|<\mu
			\end{eqnarray*}
			y
			\begin{eqnarray}\label{38}
				% \nonumber % Remove numbering (before each equation)
				y(x)&=&\sum_{m=0}^{\infty}c_{m}x^{m}\quad \text{con}\quad y(x_{0})=c_{0}\quad y'(x_{0})=c_{1}
			\end{eqnarray}
			Entonces
			\begin{eqnarray*}
				y^{\prime}(x)&=&\sum_{m=0}^{\infty}(m+1)c_{m+1}x^{m}\\
				y^{\prime\prime}(x)&=&\sum_{m=0}^{\infty}(m+2)(m+1)c_{m+2}x^{m}
			\end{eqnarray*}
			As\'i
			\begin{eqnarray*}
				p_{1}(x)y^{\prime}(x)&=&\sum_{m=0}^{\infty}\left(\sum_{k=0}^{m}(k+1)c_{k+1}\bar{p}_{m-k}\right)x^{m}\\
				p_{2}(x)y(x)&=&\sum_{m=0}^{\infty}\left(\sum_{k=0}^{m}c_{k}\widetilde{p}_{m-k}\right)x^{m}
			\end{eqnarray*}
			Sustituyendo estas expresiones en  la ecuaci\'on diferencial \ref{EDL2}, tenemos:
			\begin{eqnarray*}
				% \nonumber % Remove numbering (before each equation)
				\sum_{m=0}^{\infty}\left[(m+2)(m+1)c_{m+2}+\sum_{k=0}^{m}(k+1)c_{k+1}\bar{p}_{m-k}+\sum_{k=0}^{m}c_{k}\widetilde{p}_{m-k}\right]x^{m}&=&0
			\end{eqnarray*}
			
			As\'i, $y(x)$\, es una soluci\'on de la ecuaci\'on diferencial \ref{EDL2} si, y solo si la constante $c_{m}$, satisface la relaci\'on de recurrencia:
			\begin{eqnarray*}
				% \nonumber % Remove numbering (before each equation)
				c_{m+2}&=&-\frac{1}{(m+2)(m+1)}\left[\sum_{k=0}^{m}(k+1)c_{k+1}\bar{p}_{m-k}+c_{k}\widetilde{p}_{m-k}\right]\quad m \geq{0}
			\end{eqnarray*}
			Luego, con $m\rightarrow m-2:$
			\begin{eqnarray}\label{31}
				c_{m}&=&-\frac{1}{(m)(m-1)}\left[\sum_{k=0}^{m-2}(k+1)c_{k+1}\bar{p}_{m-k-2}+c_{k}\widetilde{p}_{m-k-2}\right]\quad m \geq{2}
			\end{eqnarray}
			A trav\'es de esta relaci\'on $c_{2},\, c_{3}, \cdots $\, pueden ser obtenidos sucesivamente como combinaci\'on lineal de $c_{0}\, \text{y} \, c_{1}$.\\
			
			Ahora debemos probar que la serie con estos coeficientes converge para $|x|<\mu.$\, Ya que las series de $p_{1}(x) \, \text{y} \, p_{1}(x)$, \, convergen para $|x|<\mu$, \, para alg\'un $|x|=\mu_{0}<\mu$, \, existe una constante $M>0$\, tal que:
			\begin{eqnarray}\label{32}
				|\bar{p}_{j}|\mu_{0}^{j}\leq{M}\quad \text{y}\quad |\widetilde{p}_{j}|\mu_{0}^{j}\leq{M}\quad j=0,1,2,\cdots.
			\end{eqnarray}
			Utilizando \eqref{31} y \eqref{32} encontramos:
			\begin{eqnarray}\label{33} |c_{m}|\leq{\frac{M}{m(m-1)}}\left[\sum_{k=0}^{m-2}\left\{\frac{(k+1)|c_{k+1}|}{\mu_{0}^{m-k-2}}+\frac{|c_{k}|}{\mu_{0}^{m-k-2}}\right\}\right]+\frac{M|c_{m-1}|\mu_{0}}{m(m-1)} \quad m \geq{2}
			\end{eqnarray}
			donde el t\'ermino $\displaystyle\frac{M|c_{m-1}|\mu_{0}}{m(m-1)}$, \, ha sido agregado con un prop\'osito que aclararemos m\'as adelante.
			Ahora definiremos la constante positiva $C_{m}$\, por la ecuaci\'on $C_{0}=|c_{0}|,$\\
			$ C_{1}=|c_{1}|$,
			\begin{eqnarray}\label{34} C_{m}={{M}{m(m-1)}}\left[\sum_{k=0}^{m-2}\left\{\frac{(k+1)C_{k+1}}{\mu_{0}^{m-k-2}}+\frac{C_{k}}{\mu_{0}^{m-k-2}}\right\}\right]+\frac{MC_{m-1}\mu_{0}}{m(m-1)},  \, m \geq{2}
			\end{eqnarray}
			De \ref{33} y \ref{34} es evidente que $|c_{m}|\leq C_{m},\, m=0,1,2, \cdots.$
			Luego, haciendo tender $m\rightarrow m+1$\, en \ref{34} obtenemos:
			\begin{eqnarray*} C_{m+1}&=&\frac{M}{m(m+1)}\left[\sum_{k=0}^{m-1}\left\{\frac{(k+1)C_{k+1}}{\mu_{0}^{m-k-1}}+\frac{C_{k}}{\mu_{0}^{m-k-1}}\right\}\right]+\frac{MC_{m}\mu_{0}}{m(m+1)}
			\end{eqnarray*}
			y ya que
			
			\begin{fullwidth}[%
				width=\dimexpr\textwidth+\marginparsep+\marginparwidth,
				outermargin=\dimexpr-\marginparsep-\marginparwidth,
				]
				\begin{eqnarray}\label{35}
					\mu_{0}C_{m+1}&=&\frac{M\mu_{0}}{m(m+1)}\left[\sum_{k=0}^{m-2}\left\{\frac{(k+1)C_{k+1}}{\mu_{0}^{m-k-1}}+\frac{C_{k}}{\mu_{0}^{m-k-1}}\right\}\right]+\frac{M\mu_{0}}{m(m+1)}[mC_{m}+C_{m-1}]+\frac{MC_{m}\mu_{0}^2}{m(m+1)}
				\end{eqnarray}
			\end{fullwidth}
			
			
			
			Combinando \ref{34} y \ref{35} conseguimos:
			\begin{eqnarray*} \mu_{0} C_{m+1}&=&\frac{M}{m(m+1)}\left[\frac{m(m-1)}{M}C_{m}-\mu_{0}C_{m-1}\right]+\frac{M\mu_{0}}{m(m+1)}[mC_{m}+C_{m-1}]+\frac{MC_{m}\mu_{0}^2}{m(m+1)}
			\end{eqnarray*}
			Simplificando:
			\begin{eqnarray}\label{36}
				\mu_{0} C_{m+1}&=&\frac{m-1}{m+1}C_{m}+\frac{mM\mu_{0}C_{m}}{m(m+1)}+\frac{MC_{m}\mu_{0}^2}{m(m+1)}
			\end{eqnarray}
			As\'i, como consecuencia de haber agregado la expresi\'on $\frac{M|c_{m-1}|\mu_{0}}{m(m-1)}$\, en \ref{33} hemos llegado a una relaci\'on de recurrencia de dos t\'erminos \ref{36} desde la cual tenemos:
			\begin{eqnarray*}
				\arrowvert {\frac{C_{m+1}x^{m+1}}{C_{m}x^{m}}}\arrowvert &=&\frac{m(m-1)+mM\mu_{0}+M\mu_{0}^2}{\mu_{0}m(m+1)}|x|\\
				\Rightarrow \lim_{m\rightarrow\infty}\arrowvert {\frac{C_{m+1}x^{m+1}}{C_{m}x^{m}}}\arrowvert &=&\frac{|x|}{\mu_{0}}.
			\end{eqnarray*}
			Entonces, la prueba del radio establece que las series  $\sum_{m=0}^{\infty}C_{m}x^{m}$\, convergen para $|x|<\mu_{0},$\, y por la prueba de comparaci\'on se sigue que las series $\sum_{m=0}^{\infty}c_{m}x^{m}$\, convergen absolutamente en $|x|<\mu_{0}.$\, Ya que $\mu_{0}\in(0,\mu_{0})$\, es arbitrario, las series convergen absolutamente en el intervalo $|x|<\mu_{0}.$\\
			
			Por lo tanto, hemos demostrado que la funci\'on la cual es anal\'itica en $x=x_{0}$\, es una soluci\'on del problema \ref{EDL2}, con valor inicial \ref{38} si, y solo si los coeficientes de la expansi\'on en series de potencias satisfacen la relaci\'on \ref{33}.
			Tambi\'en se sigue que esta ser\'a la \'unica soluci\'on.\\
		\end{demo}
		\Example{}{ Resuelva la ED
			\begin{equation*}
				\left(x^2+2\right) y^{\prime \prime}+3 x y^{\prime}-y=0
			\end{equation*} alrededor del punto ordinario $x_{0}=0$}
		\begin{sol}
			Por el teorema (\ref{teosolordinario}) la soluci\'on alrededor de $x_{0}=0$ es de la forma (\ref{solenserie}), de manera que
			\begin{equation*}
				\begin{aligned}
					& \displaystyle\left(x^2 + 2\right) y^{\prime \prime}
					= \displaystyle\sum_{m=0}^{\infty} m(m-1)c_m x^m 
					+ 2 \displaystyle\sum_{m=0}^{\infty} (m+2)(m+1)c_{m+2} x^m \\[1ex]
					& \displaystyle 3x y^{\prime} 
					= 3 \displaystyle\sum_{m=0}^{\infty} m c_m x^m
				\end{aligned}
			\end{equation*}
			Reemplazando en la ED
			\begin{equation*}
				\begin{aligned}
					&\displaystyle\sum_{m=0}^{\infty}\left[\left(m^{2}+2m-1\right)c_{m}+2\left(m+2\right)\left(m+1\right)c_{m+2} \right]x^{m}=0\\[1ex]
					&c_{m+2}=\displaystyle\frac{-\left(m^{2}+2m-1\right)}{2\left(m+2\right)\left(m+1\right)}c_{m}\quad \forall m\geq 0 \quad \text{Relaci\'on de recurrencia}
				\end{aligned}
			\end{equation*}
			Desarrollando los t\'erminos de la relaci\'on anterior
			\[
			\begin{array}{l|l}
				\begin{aligned}
					m &= 0 \\ 
					c_2 &= \dfrac{-(-1)}{2(2)(1)} c_0
				\end{aligned}
				&
				\begin{aligned}
					m &= 1 \\
					c_3 &= \dfrac{(-2)}{2(3)(2)}c_1
				\end{aligned}
				\\[5pt]
				\begin{aligned} 
					m &= 2 \\
					c_4 &= \dfrac{-(7)}{2(4)(3)} c_2=\dfrac{(-1)^2(-1)(7)}{2^2(4)(3)(2)(1)} c_0
				\end{aligned}
				&
				\begin{aligned}
					m &= 3 \\
					c_5 &= \dfrac{(-14)}{2(5)(4)}c_3=\dfrac{(-1)^2 2 \cdot 14}{2^2(5)(4)(3)(2)}c_1
				\end{aligned}
				\\[5pt]
				\begin{aligned}
					m &= 4 \\
					c_6 &= \dfrac{-(23)}{2(6)(5)} c_4=\dfrac{(-1)^3(-1)(7)(23)}{2^3(6)(5)(4)(3)(2)(1)} c_0
				\end{aligned}
				&
				\begin{aligned}
					m &= 5 \\
					c_7 &= \dfrac{(-34)}{2(7)(6)} c_5 = \dfrac{(-1)^3 2 \cdot 14\cdot 34}{2^3(7)(6)(5)(4)(3)(2)}c_1
				\end{aligned}
				\\[5pt]
				\begin{aligned}
					m &= 6 \\
					c_8 &= \dfrac{-(47)}{2(8)(7)} c_6 = \dfrac{(-1)^4(-1)(7)(23)(47)}{2^4(8)(7)(6)(5)(4)(3)(2)(1)} c_0
				\end{aligned}
				&
				\begin{aligned}
					m &= 7 \\
					c_9 &= \dfrac{(-62)}{2(9)(8)}c_7=\frac{(-1)^4 2 \cdot 14 \cdot 34 \cdot 62}{2^4(9)(8)(7)(6)(5)(4)(3)(2)}c_{1}
				\end{aligned}
				\\[5pt]
				\vdots & \vdots
				\\[5pt]
				\begin{aligned}
					c_{2m} &= \dfrac{(-1)^m(-1)(7) \ldots\left(4 m^2-4 m-1\right)}{2^m m!}c_0 \quad \forall m\geq 1
				\end{aligned}
				&
				\begin{aligned}
					c_{2m+1} &= \dfrac{(-1)^m 2 \cdot 14 \cdot 34 \ldots\left(4 m^2-2\right)}{2^m m!}c_{1} \quad \forall m\geq 1
				\end{aligned}
			\end{array}
			\]
			De esta forma la soluci\'on es 
			\[
			\begin{aligned}
				y(x) &= c_0\left[1+ \displaystyle\sum_{m=1}^{\infty} \dfrac{(-1)^m(-1)(7) \ldots\left(4 m^2-4 m-1\right)}{2^m m!} x^{2m}\right]
				+ c_1 \left[1+\displaystyle\sum_{m=1}^{\infty} \dfrac{(-1)^m 2 \cdot 14 \cdot 34 \ldots\left(4 m^2-2\right)}{2^m m!}c_{1} x^{2m+1}\right]
			\end{aligned}\]
		\end{sol}
		\Example{}{ Resuelva la ED
			\begin{equation*}
				y^{\prime \prime}+e^x y^{\prime}+\left(1+x^2\right) y=0, \quad y(0)=1, \quad y^{\prime}(0)=0
		\end{equation*}}
		\begin{sol}
			
		\end{sol}
		\Example{ Solución en serie no centrada en el origen }{ Resuelva la ED 
			\begin{equation*}
				\left(x^2-2 x\right) y^{\prime \prime}+5(x-1) y^{\prime}+3 y=0 
			\end{equation*} entorno al punto ordinario $ x_{0}=1 $}
		\begin{sol}
			Como $x_0=1$ es un punto ordinario, tenemos Una solución en serie de la forma $y(x)=\displaystyle\sum_{m=0}^{\infty} c_m(x-1)^m$
			
			\begin{multline*}
				\left(x^2-2 x\right) y^{\prime \prime}=(x-1)^2\displaystyle \sum_{m=0}^{\infty}m(m-1) c_{m}(x-1)^{m-2} \\
				-\displaystyle\sum_{m=0}^{\infty}(m+2)(m+1) c_{m+2}(x-1)^m \\
			\end{multline*}
			
			\begin{multline}\label{mo}
				\left(x^2-2 x\right) y^{\prime \prime}=\displaystyle\sum_{m=0}^{\infty}m(m-1) c_{m}(x-1)^{m} \\
				-\displaystyle\sum_{m=0}^{\infty}(m+2)(m+1) c_{m+2}(x-1)^m
			\end{multline}
			\begin{eqnarray}\label{moo}
				5(x-1) y^{\prime}=5(x-1) \displaystyle\sum_{m=0}^{\infty} m c_m(x-1)^{m-1}=5 \displaystyle\sum_{m=0}^{\infty} m c_m(x-1)^m 
			\end{eqnarray}
			\begin{eqnarray}\label{mooo}
				3 y=3\displaystyle \sum_{m=0}^{\infty} c_m(x-1)^m
			\end{eqnarray}
			Sumando las expresiones (\ref{mo}),(\ref{moo}) y (\ref{mooo})
			\begin{multline*}
				\displaystyle \sum_{m=0}^{\infty} m(m-1) c_m(x-1)^m-\displaystyle\sum_{m=0}^{\infty}(m+2)(m+1) c_{m+2}(x-1)^m \\
				+5 \displaystyle\sum_{m=0}^{\infty} m c_m(x-1)^m+3\displaystyle \sum_{m=0}^{\infty} c_m(x-1)^m=0
			\end{multline*}
			\begin{eqnarray*}
				\displaystyle\sum_{m=0}^{\infty}\left[(m(m-1)+5 m+3) c_m-(m+2)(m+1) c_{m+2}\right](x-1)^m=0 \\
				\displaystyle\sum_{m=0}^{\infty}\left[(m+1)(m+3) c_m-(m+1)(m+2) c_{m+2}\right](x-1)^m=0
			\end{eqnarray*}
			De la expresi\'on anterior obtenemos la relaci\'on de recurrencia
			\begin{equation*}
				c_{m+2}=\displaystyle\frac{m+3}{m+2}c_m \quad \forall m \geq 0
			\end{equation*}
			Desarrollando los t\'erminos de la relaci\'on anterior
			\[
			\begin{array}{l|l}
				\begin{aligned}
					m &= 0 \\
					c_2 &= \dfrac{3}{2} c_0
				\end{aligned}
				&
				\begin{aligned}
					m &= 1 \\
					c_3 &= \dfrac{2^1 \cdot 2}{3} c_1
				\end{aligned}
				\\[5pt]
				\begin{aligned}
					m &= 2 \\
					c_4 &= \dfrac{5}{4} c_2 = \dfrac{5 \cdot 3}{2^2 \cdot 2} c_0
				\end{aligned}
				&
				\begin{aligned}
					m &= 3 \\
					c_5 &= \dfrac{6}{5} c_3 = \dfrac{2^2 \cdot 3 \cdot 2}{5 \cdot 3} c_1
				\end{aligned}
				\\[5pt]
				\begin{aligned}
					m &= 4 \\
					c_6 &= \dfrac{7}{6} c_4 = \dfrac{7 \cdot 5 \cdot 3}{2^3 \cdot 3 \cdot 2} c_0
				\end{aligned}
				&
				\begin{aligned}
					m &= 5 \\
					c_7 &= \dfrac{8}{7} c_5 = \dfrac{2^3 \cdot 4 \cdot 3 \cdot 2}{7 \cdot 5 \cdot 3} c_1
				\end{aligned}
				\\[5pt]
				\begin{aligned}
					m &= 6 \\
					c_8 &= \dfrac{9}{8} c_6 = \dfrac{9 \cdot 7 \cdot 5 \cdot 3}{2^4 \cdot 4 \cdot 3 \cdot 2} c_0
				\end{aligned}
				&
				\begin{aligned}
					m &= 7 \\
					c_9 &= \dfrac{10}{9} c_7 = \dfrac{2^4 \cdot 5 \cdot 4 \cdot 3 \cdot 2}{9 \cdot 7 \cdot 5 \cdot 3} c_1
				\end{aligned}
				\\[5pt]
				\vdots & \vdots
				\\[5pt]
				\begin{aligned}
					c_{2m} &= \dfrac{3 \cdot 5 \cdot 7 \cdots (2m+1)}{2^m m!} c_0 \quad \forall m\geq 0
				\end{aligned}
				&
				\begin{aligned}
					c_{2m+1} &= \dfrac{2^m (2m+1)!}{3 \cdot 5 \cdot 7 \cdots (2m+1)} c_1 \quad \forall m\geq 0
				\end{aligned}
			\end{array}
			\]
			La soluci\'on podemos expresarla como
			\[
			\begin{aligned}
				y(x) &= \left(c_0 + c_2(x-1)^2 + c_4(x-1)^4 + \cdots \right) \\
				&\quad + \left(c_1(x-1) + c_3(x-1)^3 + c_5(x-1)^5 + \cdots \right) \\
				y(x) &= c_0 \displaystyle\sum_{m=0}^{\infty} \frac{3 \cdot 5 \cdot 7 \cdots (2m+1)}{2^m m!} (x-1)^{2m}
				+ c_1 \displaystyle\sum_{m=0}^{\infty} \frac{2^m (2m+1)!}{3 \cdot 5 \cdot 7 \cdots (2m+1)} (x-1)^{2m+1}.
			\end{aligned}
			\]
		\end{sol}
		En el estudio de soluciones en serie de ecuaciones diferenciales ordinarias, resulta particularmente ventajoso desarrollar dichas soluciones en torno al punto \( x_0 = 0 \). Esta elección no obedece a una mera preferencia arbitraria, sino a las \textbf{ventajas operacionales} que ofrece el origen en los procesos algebraicos y analíticos implicados: la manipulación de potencias, derivadas sucesivas y el desarrollo de coeficientes se simplifican sustancialmente cuando la serie está centrada en el cero.
		
		Por tal motivo, cuando la ecuación diferencial posee un \textbf{punto ordinario} \( x = x_0 \neq 0 \), es común y matemáticamente justificado efectuar un \textbf{cambio de variable} del tipo \( t = x - x_0 \). Con esta transformación, el problema se expresa en términos de una variable desplazada, permitiendo así que la \textbf{serie de potencias resultante esté centrada en \( t = 0 \)}, lo cual preserva la estructura analítica del problema original, pero facilita considerablemente el trabajo técnico.
		
		A continuación, se presenta la justificación formal de este procedimiento y se demuestra que dicha transformación conserva la naturaleza del punto ordinario, así como la equivalencia entre las soluciones desarrolladas en ambas variables.
		Sea $x=x_0$ un punto ordinario de (\ref{EDL2}), entonces existe una solución en serie de potencia garantizada en el teorema (\ref{solenserie}) de la forma
		\begin{equation*}
			y(x)=\sum_{n=0}^{\infty} c_n\left(x-x_0\right)^n \quad\left|x-x_0\right|<\mu
		\end{equation*}
		Realizando el cambio de variable $x=t+x_{0}$
		$$
		\begin{aligned}
			\displaystyle\frac{d y}{d t} & =\displaystyle\frac{d y}{d x}\displaystyle \frac{d x}{d t} \Rightarrow
			\displaystyle\frac{d y}{d t} & =\displaystyle\frac{d y}{d x} \\
			\displaystyle\frac{d}{d t}\left(\frac{d y}{d t}\right) & =\displaystyle\frac{d}{d t}\left(\frac{d y}{d x}\right)=\displaystyle\frac{d^2 y}{d x^2}
		\end{aligned}
		$$
		Reemplazando en (\ref{EDL2})
		$$\begin{aligned}
			&\displaystyle \frac{d^2 y}{d t^2}+p_1\left(t+x_0\right) \displaystyle\frac{d y}{d t}+p_2\left(t+x_0\right) y(t)=0 \\
			& p_1(x)=p_1\left(t+x_0\right)=\bar{p}_1(t) \\
			& p_2(x)=p_2\left(t+x_0\right)=\bar{p}_2(t) \\
			&\displaystyle \frac{d^2 y}{d t^2}+\bar{p}_1(t)\displaystyle \frac{d y}{d t}+\bar{p}_2(t) y(t)=0
		\end{aligned}$$
		$t=0$ es un punto ordinario de esta ecuaci\'on diferencial y tiene una soluci\'on en serie centrada en cero dada por la expresi\'on (\ref{solenserie}).
		\Example{ Solución en serie no centrada en el origen }{ Resuelva la ED 
			\begin{equation*}
				y^{\prime \prime}-2(x+3) y^{\prime}-3 y=0
			\end{equation*} entorno al punto ordinario $ x_{0}=-3 $}
		\begin{sol}
			Utilizando la t\'ecnica explicada anteriormente la ecuaci\'on diferencial se transforma en 
			\begin{equation*}
				y^{\prime \prime}-2t y^{\prime}-3y=0
			\end{equation*} donde $t_{0}=0$ es un punto ordinario. as\'i tenemos
			\[
			\begin{aligned}
				3y &= 3 \displaystyle\sum_{m=0}^{\infty} c_m t^m, \\
				2t y^{\prime} &= 2 \displaystyle\sum_{m=0}^{\infty} m c_m t^m, \\
				y^{\prime\prime} &= \displaystyle\sum_{m=0}^{\infty} (m+2)(m+1) c_{m+2} t^m
			\end{aligned}
			\]
			Sumando estas serie
			\begin{equation*}
				\displaystyle\sum_{m=0}^{\infty}\left[(m+2)(m+1) c_{m+2}-(2 m+3) c_m\right] t^m=0
			\end{equation*}
			Igualando el coeficiente a cero obtenemos la relación de recurrencia
			\begin{equation*}
				c_{m+2}=\displaystyle\frac{2 m+3}{(m+2)(m+1)} c_m \quad \forall m \geq 0
			\end{equation*}
			\[
			\begin{array}{l|l}
				\begin{aligned}
					m &= 0 \\
					c_2 &= \dfrac{3}{(2)(1)} c_0
					\\[5pt]
					m &= 2 \\
					c_4 &= \dfrac{7}{(4)(3)} c_2 = \dfrac{7 \cdot 3}{(4)(3)(2)(1)} c_0
					\\[5pt]
					m &= 4 \\
					c_6 &= \dfrac{11}{(6)(5)} c_4 = \dfrac{11 \cdot 7 \cdot 3}{(6)(5)(4)(3)(2)(1)} c_0
					\\[5pt]
					\vdots & \\
					c_{2m} &= \dfrac{3 \cdot 7 \cdot 11 \cdots (4m-1)}{(2m)!} c_0 \quad \forall m\geq 1
				\end{aligned}
				&
				\begin{aligned}
					m &= 1 \\
					c_3 &= \dfrac{5}{(3)(2)} c_1
					\\[5pt]
					m &= 3 \\
					c_5 &= \dfrac{9}{(5)(4)} c_3 = \dfrac{9 \cdot 5}{(5)(4)(3)(2)} c_1
					\\[5pt]
					m &= 5 \\
					c_7 &= \dfrac{13}{(7)(6)} c_5 = \dfrac{13 \cdot 9 \cdot 5}{(7)(6)(5)(4)(3)(2)} c_1
					\\[5pt]
					\vdots & \\
					c_{2m+1} &= \dfrac{5 \cdot 9 \cdot 13 \cdots (4m+1)}{(2m+1)!} c_1 \quad \forall m\geq 1
				\end{aligned}
			\end{array}
			\]
			As\'i la soluci\'on es 
			\[
			\begin{aligned}
				y(t) &= c_0\left[1+ \displaystyle\sum_{m=1}^{\infty} \dfrac{3 \cdot 7 \cdot 11 \cdots (4m-1)}{(2m)!} t^{2m}\right]
				+ c_1 \displaystyle\sum_{m=0}^{\infty} \dfrac{5 \cdot 9 \cdot 13 \cdots (4m+1)}{(2m+1)!} t^{2m+1}\\
				y(x) &= c_0\left[1+ \displaystyle\sum_{m=1}^{\infty} \dfrac{3 \cdot 7 \cdot 11 \cdots (4m-1)}{(2m)!} (x+3)^{2m}\right]
				+ c_1 \displaystyle\sum_{m=0}^{\infty} \dfrac{5 \cdot 9 \cdot 13 \cdots (4m+1)}{(2m+1)!} (x+3)^{2m+1}  
			\end{aligned}
			\]
		\end{sol}
		
		\Definition{Punto singular regular}{Un punto singular $x_{0}$ en el cual las funciones $p(x)=(x-x_{0})p_{1}(x)$ y $q(x)=(x-x_{0})^{2}p_{2}(x)$ sean anal\'iticas se llama un punto singular regular de \ref{EDL2}. As\'i una ecuaci\'on diferencial de segundo orden con un punto singular regular $x_{0}$ tiene la forma
			\begin{eqnarray}\label{12}
				% \nonumber % Remove numbering (before each equation)
				y^{\prime \prime}+\displaystyle\frac{p(x)}{(x-x_{0})}y^{\prime}+\displaystyle\frac{q(x)}{(x-x_{0})^{2}}y &=&0
			\end{eqnarray}
			donde las funciones $p(x)$ y $q(x)$ son anal\'iticas en $x=x_{0}$.\\
			Si un punto singular $x_{0}$ no es un regular, entonces se llama punto singular irregular.}\label{defsingular}
		\Example{ Punto singular e irregular de una ED }{ Determina si el punto indicado es singular o irregular
			\begin{enumerate}
				\item \begin{eqnarray}\label{nnd}
					2 x y^{\prime \prime}-y^{\prime}+2 y&=&0 \qquad x_{0}=0
				\end{eqnarray}
				\item \begin{eqnarray*} 
					y^{\prime \prime}+\dfrac{1}{x(x-1)^2} y^{\prime}+\dfrac{8}{x(x-1)} y&=&0 \qquad x_{0}=0 ,\; x_{0}=1
				\end{eqnarray*}
		\end{enumerate}}
		\begin{sol}
			\begin{enumerate}
				\item 
				\begin{equation*}
					\begin{aligned}
						& y^{\prime \prime} - \dfrac{1}{2x} y^{\prime} + \dfrac{1}{x} y = 0 
						\hspace{3cm} \text{Forma normal} \\
						& p_1(x) = \dfrac{1}{2x}, \qquad p_2(x) = \dfrac{1}{x} \\
						& p(x) = x \cdot p_1(x) = x \left(\dfrac{1}{2x} \right) = \dfrac{1}{2}, \qquad
						q(x) = x^2 \cdot p_2(x) = x^2 \left(\dfrac{1}{x} \right) = x \\
						& \text{En } x_0 = 0, \; \text{tanto } p(x) \text{ como } q(x) \text{ son analíticas, es un punto singular regular } 
					\end{aligned}
				\end{equation*}
				
				\item 
				\begin{equation*}
					\begin{aligned}
						& p_1(x) = \dfrac{1}{x(x - 1)^2}, \qquad 
						p_2(x) = \dfrac{8}{x(x - 1)} 
						\hspace{2cm} 
						\text{Singularidades en } x_0 = 0,\; x_0 = 1 \\
						& \text{Para } x_0 = 0 \\
						& p(x) = x \cdot p_1(x) = x \left( \dfrac{1}{x(x - 1)^2} \right) = \dfrac{1}{(x - 1)^2}, \\
						& q(x) = x^2 \cdot p_2(x) = x^2 \left( \dfrac{8}{x(x - 1)} \right) = \dfrac{8x}{x - 1} \\
						& \text{En } x_0 = 0, \;  p(x) \;  \text{y}  \; q(x)  \text{ son analíticas, es un punto singular regular} \\
						& \text{Para } x_0 = 1 \\
						& p(x) = \left(x-1\right) \cdot p_1(x) = \left(x-1\right) \left( \dfrac{1}{x(x - 1)^2} \right) = \dfrac{1}{x(x - 1)}, \\
						& q(x) = \left(x-1\right)^2 \cdot p_2(x) = \left(x-1\right)^2 \left( \dfrac{8}{x(x - 1)} \right) = x\left(x-1\right) \\
						& \text{En } x_0 = 1, \;  p(x) \text{ no es analíticas, es un punto singular irregular} 
					\end{aligned}
				\end{equation*}
			\end{enumerate}
		\end{sol}
		Si tenemos la ecuación de Cauchy-Euler de segundo orden $x^2 y^{\prime \prime}(x)-3 x y^{\prime}(x)+4 y(x)=0$, tiene un punto singular regular en $x=0$. Las soluciones de esta ecuaci\'on est\'a dada por la expresi\'on {\ref{cau1}} en el intervalo $(0, \infty)$ donde $y_1=x^2$ y $y_2=x^2 \ln x$. Si se intenta encontrar una solución en serie de potencias respecto al punto singular regular $x=0$ (en particular, $y=\sum_{n=0}^{\infty} c_n x^n$ ), se tendría éxito en obtener sólo la solución polinomial $y_1=x^2$. El hecho de que no se obtuviera la segunda solución no es sorprendente porque $\ln x$ (y en consecuencia $\left.y_2=x^2 \ln x\right)$ no es analítica en $x=0$, es decir, $y_2$ no tiene un desarrollo en serie de Taylor centrado en $x=0$.\\
		
		Para resolver ecuaciones diferenciales entorno a un punto singular regular, se emplea el siguiente teorema, llamado de teorema de Frobenius.
		\subsection{Soluci\'on de una EDO entorno a un punto singular regular}
		\Theorem{Teorema de Frobenius}{	Si $x=x_{0}$ \, es un punto singular regular de \ref{EDL2}, entonces existe al menos una soluci\'on en serie de la forma
			\begin{eqnarray}\label{solucionfrobenius}
				% \nonumber % Remove numbering (before each equation)
				y(x)&=&\displaystyle\sum_{m=0}^{\infty}c_{m}(x-x_{0})^{m+r}
			\end{eqnarray}
			donde $r$\, es la constante por determinar. Esta serie converge al menos en un intervalo del tipo\\ $0<x-x_{0}<R.$}\label{Teorema de Frobenius}
		\begin{demo}
			Derivando la expresi\'on (\ref{solucionfrobenius}) obtenemos
			\begin{equation*}
				\begin{aligned}
					y & =\displaystyle\sum_{m=0}^{\infty} c_m\left(x-x_0\right)^{m+r} \quad c_0 \neq 0 \\
					y^{\prime} & =\displaystyle\sum_{m=0}^{\infty}(m+r) c_m\left(x-x_0\right)^{m+r-1} \quad c_0 \neq 0 \\
					y^{\prime \prime} & =\displaystyle\sum_{m=0}^{\infty}(m+r)(m+r-1) c_m\left(x-x_0\right)^{m+r-2} \quad c_0 \neq 0
				\end{aligned}
			\end{equation*}
			Ahora para la demostraci\'on tomaremos la forma general de la ecuaci\'on de Euler dada en (\ref{cauchyeuler}) con un punto singular regular en $x_{0}=0$. Reemplazando en la forma normal obtenida en (\ref{cauchyeulernormal}), tenemos
			\begin{multline*}
				x^{r-2}\left(\displaystyle\sum_{m=0}^{\infty}(m+r)(m+r-1) c_m x^m\right)+\displaystyle\frac{1}{x}\left(\displaystyle\sum_{m=0}^{\infty} p_m x^m\right)\left(x^{r-1}\displaystyle \sum_{m=0}^{\infty}(m+r) c_m x^m\right) \\
				\left(\displaystyle\sum_{m=0}^{\infty} q_m x^m\right)\left(x^r \displaystyle\sum_{m=0}^{\infty} c_m x^m\right)=0
			\end{multline*}
			Realizando las operaciones y un corrimiento nos queda
			\begin{equation*}
				x^{r-2}\displaystyle \sum_{m=0}^{\infty}\left\{(m+r)(m+r-1) c_m+\displaystyle\sum_{k=0}^m\left[(k+r) p_{m-k}+q_{m-k}\right] c_k\right\} x^m=0
			\end{equation*}
			Igualando a cero los coeficientes de la serie, tenemos la relaci\'on de recurrencia
			\begin{equation}\label{as}
				(m+r)(m+r-1) c_m+\displaystyle\sum_{k=0}^m\left[(k+r) p_{m-k}+q_{m-k}\right] c_k=0 \quad \forall m \geq 1
			\end{equation}
			Como queremos despejar a $\boldsymbol{c}_{\boldsymbol{m}}$, aislamos el término que lo contiene y notamos que cuando $\boldsymbol{k}=\boldsymbol{m}$, $p_0
			\; y\; q_0$ aparecen de esta forma $\left[(m+r) p_0+q_0\right] c_m$
			Entonces, la sumatoria se separa
			\begin{equation*}
				\displaystyle\sum_{k=0}^m\left[(k+r) p_{m-k}+q_{m-k}\right] c_k=\left[(m+r) p_0+q_0\right] c_m+\displaystyle\sum_{k=0}^{m-1}\left[(k+r) p_{m-k}+q_{m-k}\right] c_k
			\end{equation*}
			
			Ahora factorizando a $c_m$ y despejando
			\begin{equation*}
				\left[(m+r)(m+r-1)+(m+r) p_0+q_0\right] c_m=-\displaystyle\sum_{k=0}^{m-1}\left[(k+r) p_{m-k}+q_{m-k}\right] c_k
			\end{equation*}
			Entonces, esto se puede expresar como
			\begin{equation}
				\begin{aligned}\label{idn}
					F(m+r) c_m & =\left[(m+r)(m+r-1)+(m+r) p_0+q_0\right] c_m \\
					& =-\displaystyle\sum_{k=0}^{m-1}\left[(k+r) p_{m-k}+q_{m-k}\right] c_k \quad \forall m \geq 1
				\end{aligned}
			\end{equation}
			Adem\'as, un cálculo simple nos muestra que
			\begin{equation}\label{saa}
				F(m+r)=F(r)+m\left(2 r+p_0+m-1\right)=0
			\end{equation}
			Analizando que ocurre cuando una de estas expresiones se anula para algún $m$ :
			Ya sabemos que $F(r)=0$,
			\begin{equation*}
				F(m+r)=m\left(2 r+p_0+m-1\right)
			\end{equation*}
			Entonces, $F(m+r)=0$, si:
			\begin{equation*}
				\begin{aligned}
					& m\left(2 r+p_0+m-1\right)=0 \\
					&  m=0 \quad \vee \quad\left(2 r+p_0+m-1\right)=0
				\end{aligned}
			\end{equation*}
			Como estamos analizando $m \geq 1$, la única posibilidad es
			\begin{equation*}
				2 r+p_0+m-1=0 \quad \Longrightarrow \quad m=1-2 r-p_0
			\end{equation*}
			De donde si tenemos que
			\begin{equation*}
				r_1+r_2=1-p_0 \Rightarrow 2 r=r_1+r_2-2\left(r_2-r\right)
			\end{equation*}
			
			Si $r=r_1 \vee r_2$, entonces de la expresi\'on (\ref{saa})
			\begin{equation*}
				m= \pm\left(r_2-r_1\right)
			\end{equation*}
		\end{demo}
		\Theorem{Punto Singular}{	Sean las funciones $p(x)\quad\text{y}\quad q(x)$ anal\'itica en $x=0$, y por lo tanto pueden ser expresada como serie de potencia dada en \ref{solenserie} para $\mid x \mid <\mu$. Adem\'as, sean $r_{1} \text{y} r_{2}$ las soluciones de la ecuaci\'on indicial
			\begin{eqnarray}\label{ecuacionindicial}
				% \nonumber % Remove numbering (before each equation)
				F(r) &=& r(r-1)+p_{0}r+q_{0}
			\end{eqnarray}
			entonces
			\begin{enumerate}
				\item Si $Re(r_{1})\geq Re(r_{2})$ y $r_{1}-r_{2}$ no es un entero no negativo, entonces las dos soluciones linealmente independiente de \ref{12} son
				\begin{eqnarray}\label{frobenius1}
					% \nonumber % Remove numbering (before each equation)
					y_{1}(x)&=&|x|^{r_{1}} \sum_{m=0}^{\infty} c_{m} x^{m}
				\end{eqnarray}
				y
				\begin{eqnarray}\label{frobenius2}
					y_{2}(x)=|x|^{r_{2}} \sum_{m=0}^{\infty} \bar{c}_{m} x^{m}
				\end{eqnarray}
				\item Si las raices de la ecuaci\'on indicial son iguales, es decir, $r_{2}=r_{1}$, entonces las dos dos soluciones linealmente independiente de \ref{12} son \ref{frobenius1} y
				\begin{eqnarray}\label{frobenius3}
					y_{2}(x)&=&y_{1}(x) \ln |x|+|x|^{r_{1}} \sum_{m=1}^{\infty} d_{m} x^{m}
				\end{eqnarray}
				\item Si las ra\'ices de la ecuaci\'on indicial cumplen que $r_{1}-r_{2}=n$(un entero positivo) entonces las dos dos soluciones linealmente independiente de \ref{12} son \ref{frobenius1} y
				\begin{eqnarray}\label{frobenius4}
					y_{2}(x)&=&c y_{1}(x) \ln |x|+|x|^{r_{2}} \displaystyle\sum_{m=0}^{\infty} e_{m} x^{m}
				\end{eqnarray}
				
				donde los coeficientes $c_{m},\bar{c}_{m}, d_{m}, e_{m}$ y la constante c se determinan sustituyendo la serie de $y(x)$ en la ecuaci\'on \ref{12}.
		\end{enumerate}}\label{teosolpuntosingular}
		\begin{demo}
			\begin{enumerate}
				\item \textbf{Caso 1:}\\
				De la expresi\'on (\ref{saa}) tenemos
				\begin{eqnarray*}
					F(r)=\left(r-r_1\right)\left(r-r_2\right) \Rightarrow F\left(m+r_1\right) & =\left(m+r_1-r_1\right)\left(m+r_1-r_2\right) \\
					& =m\left(m+r_1-r_2\right)
				\end{eqnarray*}
				Por lo tanto tomando la norma
				\begin{equation}\label{mn}
					\left|F\left(r_1+m\right)\right|  \geq m\left(m-\left|r_1-r_2\right|\right)
				\end{equation}
				Ahora vamos a asumir que existen constantes $M>0$ y $\mu_0<\mu$ tales que
				\begin{equation*}
					\left|p_i\right| \mu_0^i \leq M \quad \wedge \quad\left|q_i\right| \mu_0^i \leq M \quad \forall i \geq 0
				\end{equation*}
				
				Entonces, a partir de (\ref{mn}) y la relación de recurrencia de (\ref{idn}), obtenemos
				\begin{equation*}
					m\left(m-\left|r_1-r_2\right|\right)\left|c_m\right| \leq M \displaystyle\sum_{k=0}^{\infty}\left(k+\left|r_1\right|+1\right) \mu_0^{-m+k}\left|c_k\right| \quad \forall m \geq 1
				\end{equation*}
				Si planteamos una secuencia auxiliar $C_i$ que acota a $\left|c_i\right|$ 
				\begin{equation*}
					C_i=\left|c_i\right| \quad i=0,1,2, \cdots, n-1
				\end{equation*}
				Entonces
				\begin{equation}\label{po}
					i\left(i-\left|r_1-r_2\right|\right) C_i \leq M \sum_{k=0}^{i-1}\left(k+\left|r_1\right|+1\right) \mu_0^{-i+k} C_k \quad i=n, n+1, \cdots
				\end{equation}
				Esta relación por inducción permite probar que
				\begin{equation*}
					\left|c_m\right|=C_m \quad \forall m \geq 0
				\end{equation*}
				A partir de la expresi\'on  (\ref{po}), se obtiene la relación
				\begin{equation*}
					\displaystyle\frac{C_m}{C_{m-1}}=\displaystyle\frac{(m-1)\left(m-1-\left|r_1-r_2\right|\right)+M\left(m+\left|r_1\right|\right)}{\mu_0 m\left(m-\left|r_1-r_2\right|\right)}
				\end{equation*}
				Esto implica que
				\begin{equation*}
					\displaystyle\lim _{m \rightarrow \infty}\left|\displaystyle\frac{C_m x^m}{C_{m-1} x^{m-1}}\right|=\displaystyle\frac{|x|}{\mu_0} \quad \Rightarrow \quad \text { Converge si }|x|<\mu_0
				\end{equation*}
				Por el criterio de la razón, la serie $\displaystyle\sum_{m=0}^{\infty} c_m x^m$ converge si $|x|<\mu_0$, por tanto también converge $\displaystyle\sum_{m=0}^{\infty} c_m x^m$ por comparación.
				
				Sin embargo, si $\mu_0$ es arbitrario, $\displaystyle\sum_{m=0}^{\infty} c_m x^m$ converge para $|x|<\mu$.
				Finalmente, la presencia del factor $|x|^{r_1}$ introduce un Punto Singular en el origen de la solución. Ya podemos decir que $|x|^{r_1}\displaystyle \sum_{m=0}^{\infty} c_m x^m$ es una solución de la EDO  y es analítica para $0<|x|<\mu$.
				
				Si reemplazamos $r_1$ por $r_2$, con las consideraciones ya mencionadas, entonces $|x|^{r_2} \displaystyle\sum_{m=0}^{\infty} \bar{c}_m x^m$ es la segunda solución de la EDO  y también es analítica para $0<|x|<\mu$.
				\item \textbf{Caso 2:}\\
				Tenemos la EDO, donde
				\begin{equation*}
					\mathcal{L}_2[y(x)]=y^{\prime \prime}+\displaystyle\frac{p(x)}{x} y^{\prime}+\displaystyle\frac{q(x)}{x^2} y
				\end{equation*}
				
				Al aplicar el método de Frobenius, se obtiene una ecuación indicial $F(r)=0$, la cual tiene raíces repetidas $r_1=r_2$. Ya sabemos que
				\begin{equation*}
					F\left(r_1\right)=\left(\frac{\partial F}{\partial r}\right)_{r=r_1}=0
				\end{equation*}
				Y también sabemos que
				\begin{equation*}
					y_1(x)=|x|^{r_1} \sum_{m=0}^{\infty} c_m x^m \quad \text { en el intervalo } 0<|x|<\mu
				\end{equation*}
				De (\ref{as}) no asumimos a $\boldsymbol{r}$ como una solución de la ecuación indicial, es decir, que la tomamos con una parámetro variable, no necesariamente igual a $r_1$.
				Entonces, aplicando el operador $\mathcal{L}_2$ y suponiendo que los coeficientes $c_m$ satisfacen la relación de recurrencia (\ref{as}), se obtiene
				\begin{equation*}
					\mathcal{L}_2[y(x)]=c_0 x^{r-2} F(r)
				\end{equation*}
				Donde $\quad y(x)=x^r\displaystyle \sum_{m=0}^{\infty} c_m x^m$
				Diferenciando a (25) respecto a $\boldsymbol{r}$
				\begin{equation*}
					\begin{aligned}
						\displaystyle\frac{\partial}{\partial r} \mathcal{L}_2[y(x)] & =\mathcal{L}_2\left[\frac{\partial y(x)}{\partial r}\right] \\
						& =c_0 x^{r-2}\left[\frac{d F(r)}{d r}+F(r) \ln x\right]
					\end{aligned}
				\end{equation*}
				Como sabemos que $F\left(r_1\right)=0 \quad \wedge \quad\left(\frac{\partial F}{\partial r}\right)_{r=r_1}=0$, se obtiene
				\begin{equation*}
					\mathcal{L}_2\left[\left(\displaystyle\frac{\partial F}{\partial r}\right)_{r=r_1}\right]=0
				\end{equation*}
				Lo que implica que $\left(\displaystyle\frac{\partial F}{\partial r}\right)_{r=r_1}$ es la segunda solución formal Dicho esto, partimos de que:
				\begin{equation*}
					y(x)=x^r \displaystyle\sum_{m=0}^{\infty} c_m x^m
				\end{equation*}
				Entonces,
				\begin{equation*}
					\displaystyle\frac{\partial y(x)}{\partial r}=\frac{\partial}{\partial r}\left(x^r \sum_{m=0}^{\infty} c_m x^m\right)
				\end{equation*}
				Derivando, se obtiene
				\begin{equation*}
					\displaystyle\frac{\partial y(x)}{\partial r}=x^r \ln x \sum_{m=0}^{\infty} c_m x^m+x^r \sum_{m=0}^{\infty} \frac{\partial c_m}{\partial r} x^m
				\end{equation*}
				Factorizamos
				\begin{equation*}
					\displaystyle\frac{\partial y(x)}{\partial r}=x^r\left(\ln x \sum_{m=0}^{\infty} c_m x^m+\sum_{m=0}^{\infty} \frac{\partial c_m}{\partial r} x^m\right)
				\end{equation*}
				Ahora evaluamos $r=r_1$ 
				\begin{equation*}
					\left.\frac{\partial y(x)}{\partial r}\right|_{r=r_1}=x^{r_1}\left(\ln x \displaystyle\sum_{m=0}^{\infty} c_m x^m+\left.\sum_{m=0}^{\infty} \frac{\partial c_m}{\partial r}\right|_{r=r_1} x^m\right)
				\end{equation*}
				Separando los términos
				\begin{equation*}
					\left.\frac{\partial y(x)}{\partial r}\right|_{r=r_1}=\underbrace{x^{r_1} \sum_{m=0}^{\infty} c_m x^m}_{y_1(x)} \cdot \ln x+\left.x^{r_1} \sum_{m=0}^{\infty} \frac{\partial c_m}{\partial r}\right|_{r=r_1} x^m
				\end{equation*}
				Entonces, nos queda
				\begin{equation*}
					y_2(x)=y_1(x) \cdot \ln x+x^{r_1}\displaystyle \sum_{m=0}^{\infty} d_m x^m
				\end{equation*}
				Donde, como notamos
				\begin{equation*}
					y_2(x)=\left.\frac{\partial y(x)}{\partial r}\right|_{r=r_1} ; \quad y_1(x)=x^{r_1} \sum_{m=0}^{\infty} c_m x^m ; \quad d_m=\left.\frac{\partial c_m}{\partial r}\right|_{r=r_1} \quad \forall m \geq 0
				\end{equation*}
				Ya aquí, como $c_0$ no depende de $\boldsymbol{r}$, tenemos que $d_0=0$ y entonces
				\begin{equation*}
					\sum_{m=0}^{\infty} d_m x^m=\sum_{m=1}^{\infty} d_m x^m
				\end{equation*}
				Entonces, nos queda que la segunda solución será de la forma
				\begin{equation*}
					y_2(x)=y_1(x) \ln |x|+|x|^{r_1} \sum_{m=1}^{\infty} d_m x^m
				\end{equation*}
				
				\item \textbf{Caso 3:}\\
				Tenemos la EDO , el Método de Frobenius nos lleva a resolver la ecuación indicial $F(r)=0$, cuyas raíces son $r_1$ y $r_2$, y se cumple que: $r_1-r_2=n \in \mathbb{Z}^{+}$
				
				Como es sabido, el método nos garantiza una primera solución correspondiente a $r_1$, de la forma (\ref{frobenius1}) y esta es analítica para $0<|x|<\mu$.
				\begin{equation*}
					y_1(x)=x^{r_1}\displaystyle \sum_{m=0}^{\infty} c_m x^m
				\end{equation*}
				En correspondencia con $r_2$, podemos obtener $c_m\left(r_2\right)$ para $m=1,2,3, \cdots, n-1$ como un múltiplo lineal de $c_0$ a partir de la relación de recurrencia (\ref{idn}). Sin embargo, al intentar calcular $c_m\left(r_2\right)$, se obtiene
				\begin{equation*}
					F\left(r_2+m\right)=F\left(r_1\right) \quad \text { Donde, sabemos que } F\left(r_1\right)=0
				\end{equation*}
				Esto hace que $c_m\left(r_2\right)$ sea indeterminado, lo que interrumpe la construcción directa de la segunda solución.
				Ahora como en el caso anterior de las raíces repetidas, introducimos la derivada con respecto a $\boldsymbol{r}$ para generar una segunda solución L.I. a la que ya tenemos.
				
				Donde
				\begin{equation*}
					y_2(x)=\left.\displaystyle\frac{\partial y(x)}{\partial r}\right|_{r=r_2}
				\end{equation*}
				
				\begin{equation*}
					y(x)=x^r \displaystyle\sum_{m=0}^{\infty} c_m(r) x^m
				\end{equation*}
				
				Entonces, esto nos queda
				\begin{equation*}
					\displaystyle\frac{\partial y(x)}{\partial r}=\displaystyle\frac{\partial}{\partial r}\left(x^r \sum_{m=0}^{\infty} c_m(r) x^m\right)
				\end{equation*}
				Aplicando la regla del producto:
				\begin{equation*}
					\displaystyle\frac{\partial y(x)}{\partial r}=\left(x^r \ln |x|\displaystyle \sum_{m=0}^{\infty} c_m x^m\right)+\left(x^r \displaystyle\sum_{m=0}^{\infty} \frac{\partial c_m}{\partial r} x^m\right)
				\end{equation*}
				
				
				Ahora evaluamos la derivada en $r=r_2$, para obtener la segunda solución $y_2(x)$ 
				
				$$
				\begin{aligned}
					y_2(x) & =\left.\displaystyle\frac{\partial y(x)}{\partial r}\right|_{r=r_2} \\
					& =\underbrace{\left(x^{r_2} \ln x \displaystyle\sum_{m=0}^{\infty} c_m\left(r_2\right) x^m\right)}_{1^{\text {er }} \text { término }}+\underbrace{\left(\left.x^{r_2} \sum_{m=0}^{\infty} \frac{\partial c_m}{\partial r}\right|_{r=r_2} x^m\right)}_{2^{\text {do }} \text { término }}
				\end{aligned}
				$$
				
				
				El primer término es un múltiplo logarítmico de la primera solución $y_1(x)$ si consideramos que $\displaystyle\sum_{m=0}^{\infty} c_m\left(r_2\right) x^m$ y $\sum_{m=0}^{\infty} c_m\left(r_1\right) x^m$ son proporcionales.
				
				El segundo término es una serie regular
				
				\begin{equation*}
					x^{r_2} \displaystyle\sum_{m=0}^{\infty} e_m x^m \quad \text { donde }: e_m=\left.\frac{\partial c_m}{\partial r}\right|_{r=r_2} \quad \forall m \geq 0
				\end{equation*}
				Ya aquí tenemos la misma forma de (\ref{frobenius4}), donde
				\begin{equation*}
					C=\displaystyle\lim _{r \rightarrow r_2}\left(r-r_2\right) c_m(r)
				\end{equation*}
				
				
				Por lo que, finalmente nos queda
				
				\begin{equation*}
					y_2(x)=C y_1(x) \ln |x|+|x|^{r_2} \sum_{m=0}^{\infty} e_m x^m
				\end{equation*}
				
				
				Donde, claramente esta solución $y_2(x)$ también es analítica para $0<|x|<\mu$.
			\end{enumerate}
		\end{demo}
		
		%\textcolor{red}{agregar ejemplos de cada caso, luego hacer las demostraciones...}
		\Example{ Ecuación Diferencial con diferencia de ra\'ices indiciales no entera  }{ Resuelva la ecuaci\'on diferencial dada por la expresi\'on
			\begin{eqnarray*}
				2 x y^{\prime \prime}-y^{\prime}+2 y&=&0
		\end{eqnarray*}}
		\begin{sol}
			Del ejemplo(enumerar ejemplo) (\ref{nnd}) $x_{0}=0$ es un punto singular regular y por el 
			teorema (\ref{Teorema de Frobenius}) sabemos que al menos hay una soluci\'on de la forma (\ref{solucionfrobenius}) 
			\[
			\begin{aligned}
				& 2 x y^{\prime \prime}=2 \displaystyle\sum_{m=0}^{\infty}(m+r)(m+r-1) c_m x^{m+r-1}\\
				& y^{\prime}= \displaystyle\sum_{m=0}^{\infty}(m+r) c_m x^{m+r-1}
			\end{aligned}
			\]
			Reemplazando en la ED
			\[
			\begin{aligned}
				&\displaystyle\sum_{m=0}^{\infty}[2(m+r)(m+r-1)-(m+r)] c_m x^{m+r-1}+2 \displaystyle\sum_{m=0}^{\infty} c_m x^{m+r}=0 \\
				& \underbrace{\left[2 r^2-3 r\right]}_{\text {Ecuación Indicial }}c_0 x^{r-1}+\displaystyle\sum_{m=1}^{\infty} \underbrace{\left[[2(m+r)(m+r-1)-(m+r)] c_m+2 c_{m-1}\right]}_{\text {Relación de Recurrencia }} x^{m+r-1}=0
			\end{aligned}
			\]
			Resolviendo la ecuaci\'on indicial
			\[
			\begin{aligned}
				&2 r^2-3 r  =0 \qquad r_{1}=0\; r_{2}=\displaystyle\frac{3}{2}\quad r_{1}-r_{2}\notin\mathbb{Z}^{+}
			\end{aligned}
			\]
			La ecuaci\'on de recurrencia es
			\begin{equation}\label{recu}
				c_m=\displaystyle\frac{-2}{2(m+r)(m+r-1)-(m+r)} c_{m-1} \quad \forall m \geq 1
			\end{equation}
			Si $r_{1}=0$ en (\ref{recu}) 
			\[
			\begin{aligned}
				& c_m=\displaystyle\frac{-2}{(2 m-3) m} c_{m-1} \quad \forall m \geq 1
			\end{aligned}
			\]
			Desarrollamos los t\'erminos
			\[
			\begin{aligned}
				& c_1=\displaystyle\dfrac{-2}{(2(1)-3) 1} c_0=\displaystyle\dfrac{-2}{-1} c_0=\displaystyle\dfrac{2}{1}c_{0} \\
				& c_2=\displaystyle\dfrac{-2}{(2(2)-3) 2} c_1=\displaystyle\dfrac{-2(2)}{(1) 2} c_0=-\displaystyle\dfrac{2^2}{(1)(2)} c_0 \\
				& c_3=\displaystyle\dfrac{-2}{(2(3)-3) 3} c_2=\displaystyle\dfrac{-2 \cdot(-2)^2}{(1)(2)(3) 3} c_0=\displaystyle\dfrac{2^3}{3!\cdot 3} c_0 \\
				& c_4=\displaystyle\dfrac{-2}{(2(4)-3) 4} c_3=\displaystyle\dfrac{-2 \cdot(2)^3}{(1)(2)(3)(3)(5) 4} c_0= -\displaystyle\dfrac{2^4}{4!\cdot 3 \cdot 5} c_0
			\end{aligned}
			\]
			La soluci\'on para este caso es
			\begin{equation}\label{soler}
				y_1(x)=\displaystyle\sum_{m=0}^{\infty}\displaystyle \dfrac{(-1)^{m+1}(2)^m}{m!\displaystyle\prod_{k=1}^{m-2}(2 k+1)} x^m
			\end{equation}
			Si $r_{2}=\frac{3}{2}$ en (\ref{recu})
			\[
			\begin{aligned}
				& c_m=\displaystyle\frac{-1}{(m+3 / 2) m} c_{m-1}
			\end{aligned}
			\]
			Desarrollando esta relaci\'on de recurrencia
			\[\begin{aligned}
				& c_1=\displaystyle\dfrac{-1}{(1+3 / 2) 1} c_0=\displaystyle\dfrac{-1}{1 \cdot \displaystyle\dfrac{5}{2}} c_0=-\displaystyle\dfrac{2}{1 \cdot 5} c_0 \\
				& c_2=\displaystyle\dfrac{-1}{(2+3 / 2) 2} c_1=\displaystyle\dfrac{-1(-2)}{\left(\displaystyle\dfrac{7}{2}\right) \cdot 1 \cdot 2 \cdot 5} c_0 =\displaystyle\dfrac{(2)^2}{1 \cdot 2 \cdot 5 \cdot 7} c_0\\
				& c_3=\displaystyle\dfrac{-1}{(3+3 / 2) 3} c_2=\displaystyle\dfrac{-1(2)^2}{\left(\displaystyle\dfrac{9}{2}\right) 3 \cdot 1 \cdot 2 \cdot 5 \cdot 7} c_0=-\displaystyle\dfrac{(2)^3}{3!\cdot 5 \cdot 7 \cdot 9} c_0 \\
				& c_4=\displaystyle\dfrac{-1}{(4+3 / 2)^4} c_3=\displaystyle\dfrac{-1 \cdot-(2)^3}{\left(\displaystyle\dfrac{11}{2}\right) 4 \cdot 3!\cdot 5 \cdot 7 \cdot 9} c_0=\displaystyle\dfrac{(2)^4}{4!\cdot 5 \cdot 7 \cdot 9 \cdot 11} c_0
			\end{aligned}\]
			La segunda soluci\'on es 
			\begin{equation}\label{ugg}
				y_2(x)=x^{\displaystyle\frac{3}{2}} \displaystyle\sum_{m=0}^{\infty}\displaystyle\frac{(-2)^{m}}{m!\displaystyle\prod_{k=1}^{m-1}\left(3+2k\right)} x^m
			\end{equation}
			Por lo tanto la soluci\'on general es
			\begin{equation}\label{solgnsljjd}
				y(x)=c_1\displaystyle\sum_{m=0}^{\infty} \displaystyle\frac{(-1)^{m+1}(2)^m}{m!\displaystyle\prod_{k=1}^{m-2}(2 k+1)} x^m+c_{2}x^{\displaystyle\frac{3}{2}} \displaystyle\sum_{m=0}^{\infty}\displaystyle\frac{(-2)^{m}}{m!\displaystyle\prod_{k=1}^{m-1}\left(3+2k\right)} x^m
			\end{equation}
		\end{sol}
		\Example{ Ecuación Diferencial con Raíces Indiciales Repetidas }{ Resuelva la ecuaci\'on diferencial dada por la expresi\'on
			\begin{eqnarray*}
				x^2 y^{\prime \prime}+\left(x^2-x\right) y^{\prime}+y&=&0
		\end{eqnarray*}}
		
		\begin{sol}
			De la definición (\ref{defsingular}), \( x_0 = 0 \) es un punto singular regular. Derivando la solución en serie y reemplazando
			
			\[
			\begin{aligned}
				& \displaystyle\sum_{m=1}^{\infty} c_{m-1}(m-1+r) x^{m+r} 
				+ \displaystyle\sum_{m=0}^{\infty} c_m \left[(m+r)((m+r-2)+1)\right] x^{m+r} = 0 \\
				& \displaystyle x^r \left[ c_0 \left( r(r-2)+1 \right) \right] 
				+ \displaystyle\sum_{m=1}^{\infty} \left[ c_m\left((m+r)(m+r-2)+1\right) + c_{m-1}(m-1+r) \right] x^{m} = 0
			\end{aligned}
			\]
			
			Igualando los coeficientes a cero
			
			\[
			\begin{aligned}
				& \left[ r(r-2)+1 \right] c_0 = 0, \quad c_0 \ne 0 \Rightarrow r = 1 
				\hspace{2cm} \text{(Ecuación indicial)} \\[1ex]
				& c_m = \displaystyle\frac{1 - m - r}{(m+r)(m+r-2) + 1} c_{m-1} 
				\hspace{2cm} \text{(Relación de recurrencia)} \\[1ex]
				& \text{Para } r = 1: \quad c_m = \displaystyle\frac{-m}{m^2} c_{m-1}
			\end{aligned}
			\]
			
			Obteniendo los términos de la relación anterior
			
			\[
			\begin{aligned}
				& c_1 = \displaystyle\frac{-1}{1} c_0 = -c_0 \\[1ex]
				& c_2 = \displaystyle\frac{-2}{4} c_1 = -\displaystyle\frac{1}{2} c_1 = \displaystyle\frac{(-1)^2}{2!} c_0 \\[1ex]
				& c_3 = \displaystyle\frac{-3}{9} c_2 = -\displaystyle\frac{1}{3} c_2 = \displaystyle\frac{(-1)^3}{3!} c_0 \\[1ex]
				& \vdots \\
				& c_m = \displaystyle\frac{(-1)^m}{m!} c_0
			\end{aligned}
			\]
			
			Así, una solución está dada por:
			
			\[
			y_0(x) = c_0 \displaystyle\sum_{m=0}^{\infty} \dfrac{(-1)^m}{m!} x^m
			\]
			
			Para obtener la segunda solución linealmente independiente, utilizamos la expresión (\ref{frobenius2}) dada en el teorema (\ref{teosolpuntosingular}).
			
			De la relación de recurrencia obtenemos
			
			\[
			\begin{aligned}
				& c_m(r) = \displaystyle\frac{-1}{m + r - 1} c_{m-1}, \quad \forall m \geq 1 
				\hspace{1cm} \text{(Desarrollando los términos)} \\[1ex]
				& c_1(r) = \displaystyle\frac{-1}{r} c_0 \\[1ex]
				& c_2(r) = \displaystyle\frac{-1}{r+1} \cdot \displaystyle\frac{-1}{r} c_0 
				= \displaystyle\frac{(-1)^2}{r(r+1)} c_0 \\[1ex]
				& c_3(r) = \displaystyle\frac{(-1)^3}{r(r+1)(r+2)} c_0 \\[1ex]
				& \vdots \\
				& c_m(r) = \displaystyle\frac{(-1)^m}{\displaystyle\prod_{k=0}^{m-1}(r+k)} c_0
			\end{aligned}
			\]
			
			Por propiedad de los logaritmos
			
			\[
			\begin{aligned}
				& \ln |c_m(r)| = \ln(c_0) - \ln \left( \displaystyle\prod_{k=0}^{m-1}(r+k) \right) 
				= \ln(c_0) - \displaystyle\sum_{k=0}^{m-1} \ln |r+k| \\[1ex]
				& \displaystyle\frac{c_m'(r)}{c_m(r)} = - \displaystyle\sum_{k=0}^{m-1} \dfrac{1}{r+k} \\[1ex]
				& c_m'(r) = - c_m(r) \displaystyle\sum_{k=0}^{m-1} \dfrac{1}{r+k} \\[1ex]
				& d_m = \displaystyle\frac{(-1)^{m+1}}{m!} \displaystyle\sum_{k=0}^{m-1} \dfrac{1}{k+1}
			\end{aligned}
			\]
			
			Sustituyendo esta expresión en (\ref{frobenius2}), obtenemos la segunda solución
			
			\[
			y_1(x) = y_0(x) \ln(x) + x \displaystyle\sum_{m=1}^{\infty} 
			\left( \dfrac{(-1)^{m+1}}{m!} \displaystyle\sum_{k=0}^{m-1} \dfrac{1}{k+1} \right) x^m
			\]
			
			Por lo tanto, la solución general está dada por
			
			\[
			y(x) = c_0 \displaystyle\sum_{m=0}^{\infty} \dfrac{(-1)^m}{m!} x^m 
			+ c_1 \left[ \displaystyle\sum_{m=0}^{\infty} \dfrac{(-1)^m}{m!} x^m \ln(x)
			+ x \displaystyle\sum_{m=1}^{\infty} \left( \dfrac{(-1)^{m+1}}{m!} 
			\displaystyle\sum_{k=0}^{m-1} \dfrac{1}{k+1} \right) x^m \right]
			\]
			
		\end{sol}
		\Example{Ecuación Diferencial con diferencia de ra\'ices indiciales entera positiva }{ Resuelva la ecuaci\'on diferencial dada por la expresi\'on
			\begin{eqnarray*}
				x^2 y^{\prime \prime}+2 x y^{\prime}+x y&=&0
		\end{eqnarray*}}
		\begin{sol}
			En primer lugar verificamos que si $x_{0}=0$ es un punto singular regular, de forma que la forma normal de la ED
			\begin{equation*}
				y^{\prime \prime}-\displaystyle\frac{2}{x} y^{\prime}+\displaystyle\frac{1}{x} y=0
			\end{equation*}
			es facil verificar que $p(x)$ y $q(x)$ son anal\'itica en $x_{0}=0$, por lo tanto es un punto singular regular. As\'i est\'a garantizada una soluci\'on de la forma dada en (\ref{solucionfrobenius}), derivando y sustituyendo en la ecuaci\'on diferencial
			\begin{equation*}
				\displaystyle\sum_{m=0}^{\infty}(m+r)(m+r-1) c_m x^{m+r}+2\displaystyle \sum_{m=0}^{\infty}(m+r) c_m x^{m+r}+\displaystyle\sum_{m=0}^{\infty} c_m x^{m+r+1}=0
			\end{equation*}
			
			Factorizando las dos primeras series
			
			\begin{equation*}
				\displaystyle\sum_{m=0}^{\infty}[(m+r)(m+r+1)] c_m x^{m+r}+\displaystyle\sum_{m=0}^{\infty} c_m x^{m+r+1}=0
			\end{equation*}
			Igualamos los índices y exponentes de las dos sumatorias:
			\begin{equation*}
				[(r)(r+1)] c_0 x^r+\displaystyle\sum_{m=1}^{\infty}[(m+r)(m+r+1)] c_m x^{m+r}+\displaystyle\sum_{m=1}^{\infty} c_{m-1} x^{m+r}=0
			\end{equation*}
			Factorizando las series
			\begin{equation*}
				\underbrace{[(r)(r+1)]}_{\text {Ecuación Indicial }} c_0 x^r+\displaystyle\sum_{m=1}^{\infty} \underbrace{\left[\left((m+r)(m+r+1) c_m\right)+c_{m-1}\right]}_{\text {Relación de Recurrencia }} x^{m+r}=0
			\end{equation*}
			La ecuaci\'on indicial est\'a dada por
			\begin{equation*}
				r(r+1)=0
			\end{equation*} Con soluciones  $r_{1}=0$ y $r_{2}=-1$ cuya diferencia es uno.
			Igualando a cero el coeficiente de la sumatoria, tenemos
			\begin{equation*}
				(m+r)(m+r+1) c_m+c_{m-1}=0
			\end{equation*} Despejando $c_{m}$
			\begin{equation}\label{lml}
				c_m=\displaystyle\frac{-1}{(m+r)(m+r+1)} c_{m-1} \quad \forall m \geq 1
			\end{equation}
			Ahora sustituyendo $r_{1}=0$ en (\ref{lml})
			\begin{equation*}
				c_m=\displaystyle\frac{-1}{(m)(m+1)} c_{m-1} \quad \forall m \geq 1
			\end{equation*}
			Desarrollando los coeficientes
			$$\begin{aligned}
				& c_1=\displaystyle\frac{-1}{(1)(1+1)} c_0=\displaystyle \frac{-1}{(1)(2)} c_0 \\
				& c_2=\displaystyle\frac{-1}{(2)(2+1)} c_1=\displaystyle\frac{(-1)^2}{(1)(2)(2)(3)} c_0 \\
				& c_3=\displaystyle\frac{-1}{(3)(3+1)} c_2=\displaystyle\frac{(-1)^3}{(1)(2)(2)(3)(3)(4)} c_0 \\
				& c_4=\displaystyle\frac{-1}{(4)(4+1)} c_3=\displaystyle\frac{(-1)^4}{(1)(2)(2)(3)(3)(4)(4)(5)} c_0 \\
				& \vdots \\
				& c_m=\displaystyle\frac{(-1)^m}{m!(m+1)!} c_0
			\end{aligned}$$
			De igual forma desarrollamos los coeficientes de (\ref{lml}) para $r_{2}=-1$
			\begin{equation*}
				c_m=\displaystyle\frac{-1}{(m-1)(m)} c_{m-1} \quad \forall m \geq 1
			\end{equation*}
			Desarrollando
			$$\begin{array}{ll}
				c_1=\displaystyle\frac{-1}{(1-1)(1)} c_0=\displaystyle\frac{-1}{(0)(1)} c_0 \\
				c_2=\displaystyle\frac{-1}{(1)(2)} c_1=\displaystyle\frac{(-1)^2}{(0)(1)(1)(2)} c_0 \\
				c_3=\displaystyle\frac{-1}{(2)(3)} c_2 =\displaystyle\frac{(-1)^3}{(0)(1)(1)(2)(2)(3)} c_0 \\
				c_4=\displaystyle\frac{-1}{(3)(4)} c_3 =\displaystyle\frac{(-1)^4}{(0)(1)(1)(2)(2)(3)(3)(4)} c_0 \\
			\end{array}$$
			\begin{equation}\label{r2}
				c_m=\displaystyle\frac{(-1)^m}{(m-1)!m!} c_0 
			\end{equation}
			Ahora reemplazando en (\ref{solucionfrobenius}) obtenemos la soluci\'on
			\begin{equation*}
				y_1(x)=\displaystyle\sum_{m=0}^{\infty} \displaystyle\frac{(-1)^m}{m!(m+1)!} x^m
			\end{equation*}
			Como la diferencia de las ra\'ices indiciales es un entero positivo la segunda soluci\'on linealmente independiente se obtiene utilizando la expresi\'on dada en (\ref{frobenius4}), donde
			\begin{equation*} e_m=\left.\displaystyle\frac{\partial c_m}{\partial r}\right|_{r=r_2} \quad \forall m \geq 0
			\end{equation*}
			Ahora vamos a desarrollar la relaci\'on de recurrencia (\ref{lml}) para obtener una expresi\'on que dependa unicamente de $r$
			$$\begin{array}{rlrl}
				c_1 & =\displaystyle\frac{-1}{(1+r)(r+2)} c_0 =\displaystyle\frac{-1}{(r+1)(r+2)} c_0 \\
				c_2 & =\displaystyle\frac{-1}{(2+r)(r+3)} c_1 =\displaystyle\frac{(-1)^2}{(r+1)(r+2)(2+r)(r+3)} c_0 \\
				c_3 & =\displaystyle\frac{-1}{(3+r)(r+4)} c_2 =\displaystyle\frac{(-1)^3}{(r+1)(r+2)(2+r)(r+3)(3+r)(r+4)} c_0 \\
				c_4 & =\displaystyle\frac{-1}{(4+r)(r+5)} c_3=\displaystyle\frac{(-1)^4}{(r+1)(r+2)(2+r)(r+3)(3+r)(r+4)(4+r)(r+5)} c_0\\
			\end{array}$$
			\begin{equation}\label{win}
				c_m =\displaystyle\frac{(-1)^m}{(r+1)(r+2)^2 \cdots(r+m)^2(m+r+1)} c_0
			\end{equation}
			Para simplificar el proceso, escogemos $c_{0}=1$. Ahora vamos a calcular la constante $C$
			$$\begin{aligned}
				C & =\displaystyle\lim _{r \rightarrow r_2}\left(r-r_2\right) c_m(r) \\
				C & =\displaystyle\lim _{r \rightarrow-1}(r-(-1)) c_m(r) \\
				C & =\displaystyle\lim _{r \rightarrow-1}(r+1) c_m(r)
			\end{aligned}$$
			Recordando que
			$$\begin{aligned}
				c_m(r) & =\displaystyle\frac{(-1)^m}{(r+1)(r+2)^2 \cdots(r+m)^2(m+r+1)} c_0 \\
				c_1 & =\displaystyle\frac{-(1)}{(r+1)(r+2)}
			\end{aligned}$$
			$$\begin{aligned}
				C & =\lim _{r \rightarrow-1}(r+1) c_1(r) \\
				& =\lim _{r \rightarrow-1} \frac{(-1)(r+1)}{(r+1)(r+2)} \\
				& =\lim _{r \rightarrow-1} \frac{(-1)}{(r+2)} \\
				& =\frac{(-1)}{(-1+2)} \\
				C & =-1
			\end{aligned}$$
			Luego de realizar esto, la expresi\'on (\ref{win}) nos queda
			\begin{equation}\label{iwn}
				c_m(r)=\displaystyle\frac{(-1)^m}{(r+2)^2 \cdots(r+m)^2(m+r+1)}
			\end{equation}
			
			Diferenciando logar\'itmicamente a (\ref{iwn}) para obtener a $e_m$, tenemos lo siguiente
			
			$$
			\begin{aligned}
				\ln \left(c_m(r)\right) & =\ln (F(m, r)) \\
				& =\ln \left(\frac{(-1)^m}{(r+2)^2 \cdots(r+m)^2(m+r+1)}\right) \\
				\ln \left(c_m(r)\right) & =\ln (-1)^m-\sum_{i=2}^m \ln (r+i)^2-\ln (m+r+1) \\
				\frac{\partial\left(\ln \left(c_m(r)\right)\right)}{\partial r} & =\frac{\partial}{\partial r}\left[\ln (-1)^m-\sum_{i=2}^m \ln (r+i)^2-\ln (m+r+1)\right] \\
				\frac{c_m^{\prime}(r)}{c_m(r)} & =\left[0-2 \sum_{i=2}^m \frac{1}{r+i}-\frac{1}{m+r+1}\right] \\
				c_m^{\prime}(r) & =c_m(r)\left[-2 \sum_{i=2}^m \frac{1}{r+i}-\frac{1}{m+r+1}\right]
			\end{aligned}
			$$
			
			Como $e_m=\left.\displaystyle\frac{\partial c_m}{\partial r}\right|_{r=r_2}$, entonces
			$e_m=c_m^{\prime}\left(r_2\right)$
			$$
			c_m^{\prime}(r)=c_m(r)\left[-2 \sum_{i=2}^m \frac{1}{r+i}-\frac{1}{m+r+1}\right] \quad \text { Sabemos que } e_m=c_m^{\prime}\left(r_2\right) \text { : }
			$$
			
			
			Convenientemente establecemos que $e_0=c_0^{\prime}(-1)=1$ :
			\begin{equation*}
				c_m^{\prime}\left(r_2\right)=c_m\left(r_2\right)\left[-2 \sum_{i=2}^m \frac{1}{r+i}-\frac{1}{m+r+1}\right]_{r=r_2}
			\end{equation*}
			
			
			Ya calculamos $c_m\left(r_2\right)$ en (\ref{r2}), recordando que $r_2=-1$ :
			
			\begin{equation*}
				e_m=\frac{(-1)^m}{(m-1)!m!}\left[-2 \sum_{i=2}^m \frac{1}{-1+i}-\frac{1}{m}\right]
			\end{equation*}
			
			
			Simplificando la sumatoria, proponemos la siguiente sustitución:
			
			$$
			\begin{aligned}
				-1+i=k \quad \text { Despejando } i:-1+1+i & =k+1 \\
				i & =k+1
			\end{aligned}
			$$
			
			
			Ajustando los límite de la sumatoria:
			
			$$
			\begin{aligned}
				& \text { Si } i=m \rightarrow k=m-1 \\
				& \text { Si } i=2 \rightarrow k=1
			\end{aligned}
			$$
			
			
			Por lo que nos queda:
			
			\begin{eqnarray*}
				e_m&=&\displaystyle\frac{(-1)^m}{(m-1)!m!}\left[-2 \sum_{k=1}^{m-1} \frac{1}{k}-\frac{1}{m}\right]\\
				e_m&=&\displaystyle\frac{(-1)^m}{(m-1)!m!}\left[-2 \sum_{k=1}^{m-1} \frac{1}{k}-\frac{1}{m}\right]
			\end{eqnarray*}
			
			Como ya hemos determinado a $e_m$, ahora podemos construir la segunda solución en términos de series de potencias, la cual recordando :
			
			$$
			\begin{aligned}
				& y_2(x)=C y_1(x) \ln |x|+|x|^{r_2} \sum_{m=0}^{\infty} e_m x^m \quad C=-1 \quad, \quad r_2 \\
				& y_2(x)=-y_1(x) \ln (x)+x^{-1}\left[1-\sum_{m=1}^{\infty} \frac{(-1)^m}{(m-1)!m!}\left(2 \sum_{k=1}^{m-1} \frac{1}{k}+\frac{1}{m}\right) x^m\right]
			\end{aligned}
			$$
			
			
			$$
			r_2=-1
			$$
			
			
			Soluciones de la EDO dada en términos de Series de Potencias 
			
			$$
			\begin{aligned}
				& y_1(x)=\sum_{m=0}^{\infty} \frac{(-1)^m}{m!(m+1)!} x^m \\
				& y_2(x)=-y_1(x) \ln (x)+x^{-1}\left[1-\sum_{m=1}^{\infty} \frac{(-1)^m}{(m-1)!m!}\left(2 \sum_{k=1}^{m-1} \frac{1}{k}+\frac{1}{m}\right) x^m\right]
			\end{aligned}
			$$
			%\end{aligned}
		\end{sol}
%%%%%%%%%%%%%%%
%%%%%%%%%%%%%%
	\section{Ecuaci\'on de Hermite. Polinomios de Hermite.}\label{Hermite}
		\textcolor[rgb]{1.00,0.00,0.00}{Tesis milane de fisica}\\
		La ecuaci\'on diferencial de Hermite est\'a dada por:
		\begin{equation}\label{H}
			y^{\prime \prime}(x)-2x y^{\prime}(x)+\lambda y(x)=0
		\end{equation}
		
		
		N\'otese que el punto $x_{0}=0$\, es un punto ordinario de la ecuaci\'on diferencial \eqref{H}. Esto implica, por el teorema \eqref{Teorema de Frobenius} que \eqref{H} admite una soluci\'on de la forma\\ $y(x)=\displaystyle\sum_{m=0}^\infty c_{m}(x-x_{0})^{m}.$
		Ahora, sea\, \begin{equation} \label{H1} y(x)=\sum_{m=0}^\infty c_{m}x^{m},\,\text{ con}\, c_{0}\neq{0},\end{equation}
		\,\text{ soluci\'on \,de} \eqref{H} alrededor del punto $x_{0}=0.$ \,Entonces, derivando \eqref{H1} se obtiene:
		\begin{equation} \label{H2} y'(x)=\sum_{m=1}^\infty (m)c_{m}x^{m-1}\end{equation} \, y \, \begin{equation} \label{H3}y''(x)=\sum_{m=2}^\infty (m)(m-1)c_{m}x^{m-2}\end{equation}
		Luego, sustituyendo\eqref{H1},\,\eqref{H2},\, y\eqref{H3}\, en \eqref{H}\, se obtiene:
		\begin{equation*}
			\sum_{m=2}^\infty m(m-1)c_{m}x^{m-2}-2\sum_{m=1}^\infty mc_{m}x^{m}+2a\sum_{m=0}^\infty c_{m}x^{m}=0
		\end{equation*}
		Ahora, con\, $m\rightarrow m+2$ \, se obtiene:
		$$ \sum_{m=0}^\infty (m+2)(m+1)c_{m+2}x^{m}-2\sum_{m=1}^\infty mc_{m}x^{m}+\lambda\sum_{m=0}^\infty c_{m}x^{m}=0$$
		$$ \Rightarrow  2c_{2}+\sum_{m=1}^\infty (m+2)(m+1)c_{m+2}x^{m}-2\sum_{m=1}^\infty mc_{m}x^{m}+2ac_{0}+\lambda\sum_{m=1}^\infty c_{m}x^{m}=0$$
		Factorizando se obtiene:
		\begin{equation*}
			2c_{2}+\lambda c_{0}+\left( \sum_{m=1}^\infty (m+2)(m+1)c_{m+2}+(-2m+\lambda))c_{m}\right)x^{m}=0
		\end{equation*}
		Esto es posible si: $2c_{2}+\lambda c_{0}=0$\,y\,$(m+2)(m+1)c_{m+2}+(-2m+\lambda))c_{m}=0$,\\
		de donde se obtiene la relaci\'on de recurrencia
		\begin{equation}\label{H4} c_{m+2}=-\frac{(\lambda-2m)}{(m+2)(m+1)}c_{m}  \, \forall{m}\in{N}/m\geq{0}
		\end{equation}
		Ahora d\'andole valores pares a $m$, \, de \eqref{H4} \, se obtiene:
		\begin{align*}
			Si\, m=0 \Rightarrow c_{2}&=-\frac{\lambda}{2}c_{0}.\\
			Si \,m=2 \Rightarrow c_{4}&=-\frac{(\lambda-4)}{4\cdot 3}c_{2}=\frac{\lambda(\lambda-4)}{4\cdot{3}\cdot{2}}c_{0}.\\
			Si \,m= 4\Rightarrow c_{6}&=-\frac{(\lambda-8)}{6\cdot{5}}c_{4}=-\frac{\lambda(\lambda-8)(\lambda-4)}{6\cdot{5}\cdot{4}\cdot{3}\cdot{2}}c_{0}.\\
			Si \,m= 6\Rightarrow c_{8}&=-\frac{(\lambda-12)}{8\cdot{7}}c_{6}=\frac{\lambda(\lambda-12)(\lambda-8)(\lambda-4)}{8\cdot{7}\cdot{6}\cdot{5}\cdot{4}\cdot{3}\cdot{2}}c_{0}.\\
			\vdots
		\end{align*}
		
		\begin{equation}\label{He1}
			\Rightarrow c_{2m}=(-1)^{m}\frac{\lambda(\lambda-4)(\lambda-8)(\lambda-12)\cdots(\lambda+4-4m)}{(2m)!}c_{0}\quad m\geq 1
		\end{equation}
		\textcolor[rgb]{1.00,0.00,0.00}{Analizar la expresion equiv para $\lambda=2a$}
		\begin{equation*}Luego,\, (a+2-2m)\cdots(a-4)(a-2)a=\prod_{k=0}^{m}(a+2-2k)=2^{m}\prod_{k=0}^{m}(\frac{a}{2}+1-k)=2^{m}\frac{\Gamma{(\frac{a}{2}+1)}}{\Gamma{(\frac{a}{2}-m+1)}}.\end{equation*}
		
		As\'i, \, \begin{equation}\label{H5} c_{2m}=\frac{(-1)^{m}2^{2m}}{(2m)!} \frac{\Gamma{(\frac{a}{2}+1)}}{\Gamma{(\frac{a}{2}-m+1)}}c_{0},\, m=0,1,2,\cdots\end{equation}
		
		Ahora d\'andole valores impares a $m$, \, de \eqref{H4} \, se obtiene:
		\begin{align*}
			Si\, m=1 \Rightarrow c_{3}&=-\frac{\lambda-2}{3\cdot{2}}c_{1}.\\
			Si \,m=3 \Rightarrow c_{5}&=-\frac{(\lambda-6)}{5\cdot(4)}c_{3}=\frac{(\lambda-2)(\lambda-6)}{5\cdot{4}\cdot{3}\cdot{2}}c_{1}.\\
			Si \,m= 5\Rightarrow c_{7}&=-\frac{(\lambda-10)}{7\cdot{6}}c_{5}=-\frac{(\lambda-10)(\lambda-6)(\lambda-2)}{7\cdot{6}\cdot{5}\cdot{4}\cdot{3}\cdot{2}}c_{1}.\\
			Si \,m= 7\Rightarrow c_{9}&=-\frac{(\lambda-14)}{9\cdot{8}}c_{7}=\frac{(\lambda-14)(\lambda-10)(\lambda-6)(\lambda-2)}{9\cdot{8}\cdot{7}\cdot{6}\cdot{5}\cdot{4}\cdot{3}\cdot{2}}c_{1}.\\
			\vdots
		\end{align*}
		\begin{equation}\label{Her2}
			\Rightarrow c_{2m+1}=(-1)^{m}\frac{(\lambda-2)(\lambda-6)(\lambda-10)(\lambda-14)\cdots(\lambda-(4m-2))}{(2m+1)!}c_{1}\quad m\geq 1
		\end{equation}
		\textcolor[rgb]{1.00,0.00,0.00}{Analizar la expresion equiv para $\lambda=2a$}
		\begin{equation*}Luego,\, (a-(2m-1))\cdots(a-7)(a-5)(a-3)(a-1)=\prod_{k=0}^{m}(a-2k+1)=2^{m}\prod_{k=0}^{m}(\frac{a}{2}-k+\frac{1}{2}).\end{equation*}
		\begin{equation*}\Rightarrow (a-(2m-1))\cdots(a-7)(a-5)(a-3)(a-1)=2^{m}\frac{\Gamma{(\frac{a}{2}+\frac{1}{2})}}{\Gamma{(\frac{a}{2}-m+\frac{1}{2})}},\end{equation*}
		As'i, \, \begin{equation}\label{H6} c_{2m+1}=\frac{(-1)^{m}}{2\cdot(2m+1)!}\frac{2^{2m+1}\Gamma{(\frac{a}{2}+\frac{1}{2})}}{\Gamma{(\frac{a}{2}-m+\frac{1}{2})}}c_{1},\, m=0,1,2,\cdots \end{equation}
		Luego, sustituyendo \eqref{H5} y \eqref{H6} en \eqref{H1} $y(x)=\sum_{m=0}^\infty c_{m}x^{m},\, con\, c_{0}\neq{0},$\, se obtienen respectivamente:
		\begin{equation}\label{H20} c_{0}y_{0}(x)=c_{0}x^{0}\left(1-\frac{2a}{2!}x^{2}+\frac{2^2(a-2)}{4!}x^{4}-\cdots \right)=c_{0}\sum_{m=0}^\infty \frac{(-1)^{m}}{(2m)!} \frac{\Gamma{(\frac{a}{2}+1)}}{\Gamma{(\frac{a}{2}-m+1)}}(2x)^{2m};\, y\end{equation}
		\begin{equation*} c_{1}y_{1}(x)=c_{1}\left(x-\frac{2(a-1)}{3!}x^{3}+\frac{2^2(a-1)(a-3)}{5!}x^{5}- \cdots\right) \end{equation*}
		\begin{equation}\Rightarrow \label{H21} c_{1}y_{1}(x)=c_{1}\sum_{m=0}^\infty \frac{(-1)^{m}}{2\cdot(2m+1)!}\frac{\Gamma{(\frac{a}{2}+\frac{1}{2})}}{\Gamma{(\frac{a}{2}-m+\frac{1}{2})}}(2x)^{2m+1},\end{equation}
		As'i, la soluci'on general de la ecuaci'on diferencial \eqref{H} queda determinada por \eqref{H20} y \eqref{H21}:
		$$y(x)=c_{0}y_{0}(x)+c_{1}y_{1}(x)=c_{0}\sum_{m=0}^\infty \frac{(-1)^{m}}{(2m)!} \frac{\Gamma{(\frac{a}{2}+1)}}{\Gamma{(\frac{a}{2}-m+1)}}(2x)^{2m}+$$
		\begin{equation} \label{H22}
			c_{1}\sum_{m=0}^\infty \frac{(-1)^{m}}{2\cdot(2m+1)!}\frac{\Gamma{(\frac{a}{2}+\frac{1}{2})}}{\Gamma{(\frac{a}{2}-m+\frac{1}{2})}}(2x)^{2m+1};
		\end{equation}
		donde $c_{0}y_{0}(x)$\, y $c_{1}y_{1}(x)$\, representan las soluciones par e impar, respectivamente, de la ecuaci'on diferencial \eqref{H}.
		
		Finalmente, una forma para obtener los polinomios de Hermite, es tomar la constante de normalizaci\'on $c_{n}=2^{n}$,\, e iniciar un proceso regresivo a partir de la f'ormula de recurrencia \eqref{H4}:
		\begin{equation*} c_{m+2}=-\frac{2(a-m)}{(m+2)(m+1)}c_{m}  \, \forall{m}\in{N}/m\geq{0}. \end{equation*}
		Ahora, despejando a $c_{m}$ \, de \eqref{H4}, \,haciendo $m\rightarrow m-2$\,y tomando $a=n$\, se obtiene:
		\begin{equation} \label{H7} c_{m-2}=-\frac{m(m-1)}{2(n-m+2)}c_{m}, \forall{m}\in{N}/m\leq{n+2}. \end{equation}
		As'i, haciendo $m=n$ \, y \, sustituyendo la constante de normalizaci'on en la expresi'on anterior  se tiene:
		\begin{equation*}  c_{n-2}=-\frac{n(n-1)}{2(n-n+2)}c_{m}=-\frac{n(n-1)}{2^{2}}{2^{n}}=-n(n-1){2^{n-2}}. \end{equation*}
		De igual forma, tomando en \eqref{H7} $m\rightarrow n-2$\, y sustituyendo  $c_{n-2}$\, respectivamente se obtiene:
		\begin{equation*} c_{n-4}=-\frac{(n-2)(n-3)}{2(n-n+4)}c_{n-2}=\frac{(n-2)(n-3)}{2\cdot 2^{2}}n(n-1){2^{n-2}}=\frac{n(n-1)(n-2)(n-3){2^{n-4}}}{2}.\end{equation*}
		As'i mismo, tomando en \eqref{H7} $m\rightarrow n-4$\, y sustituyendo  $c_{n-4}$\, respectivamente se obtiene:
		\begin{equation*} c_{n-6}=-\frac{(n-4)(n-5)}{2(n-n+6)}c_{n-4}=\frac{(n-4)(n-5)}{3\cdot 2^{2}}\frac{n(n-1)(n-2)(n-3){2^{n-4}}}{2}.\end{equation*}
		\begin{equation*}\Rightarrow c_{n-6}=\frac{n(n-1)(n-2)(n-3)(n-4)(n-5){2^{n-6}}}{3\cdot{2}}.\end{equation*}
		Siguiendo este procedimiento de manera iterativa se obtienen los llamados \textbf{coeficientes de los polinomios de Hermite:}\, $c_{n-2m}=\frac{(-1)^{m}\,n!\,2^{n-2m}}{m!\,(n-2m)!}$,\, que al sustituirlos en \eqref{H1}
		$ y(x)=\sum_{m=0}^\infty c_{m}x^{m}$\, se concluye:
		$$ y(x)=\sum_{m=0}^\infty c_{m}x^{m}=\sum_{m=0}^{M} c_{n-2m}x^{n-2m}=\sum_{m=0}^{M} \frac{(-1)^{m}\,n!\,2^{n-2m}}{m!\,(n-2m)!}\,x^{n-2m};$$
		donde $M=\frac{n}{2}$\, si n es par y $M=\frac{n-1}{2}$\, si n es impar $\Rightarrow M=\lceil\frac{n}{2} \rceil$ (funci'on mayor entero).
		\begin{equation}\label{H8}\Rightarrow H_{n}(x)=\sum_{m=0}^{\lceil\frac{n}{2} \rceil} \frac{(-1)^{m}\,n!\,2^{n-2m}}{m!\,(n-2m)!}\,x^{n-2m}. \textbf{Polinomios de Hermite.}\end{equation}
		
		
		\Example{Calcule los primeros cuatro polinomios de Hermite}{
			
			Calcule los primeros cuatro polinomios de Hermite utilizando la expresi\'on (\ref{H8})}
		
		
		\begin{sol}
			Para $n=0$
			$$
			\begin{aligned}
				& H_0(x)=\sum_{m=0}^0\displaystyle \frac{(-1)^m 0 ! 2^{-2 m}}{m !(-2 m) !} x^{-2 m} \\
				& H_0(x)=1
			\end{aligned}
			$$
			Para $n=1$
			$$
			H_1(x)=\sum_{m=0}^{0}\displaystyle \frac{(-1)^m 1 ! 2^{1-2 m}}{m !(1-2 m) !} x^{1-2 m}
			$$
			Desarrollando la Sumatoria
			$$
			\begin{aligned}
				& H_1(x)=\frac{(-1)^0(1) 2}{(1)(1)} x \\
				& H_1(x)=2 x
			\end{aligned}
			$$
			Para $n=2$
			\begin{eqnarray*}
				H_2(x)&=&\displaystyle\sum_{m=0}^1\displaystyle \frac{(-1)^m 2 ! 2^{2-2 m}}{m !(2-2 m) !} x^{2-2 m}
			\end{eqnarray*}
			Desarrollando la serie
			\begin{eqnarray*}
				H_2(x)&=&\displaystyle\frac{(-1)^0 2 ! 2^2}{0 ! 2 !} x^2+\displaystyle\frac{(-1)^1 2 ! 2^0}{2 ! 0 !} x^0
			\end{eqnarray*}
			Simplificando
			\begin{eqnarray*}
				H_2(x)&=&4 x^2-2
			\end{eqnarray*}
			Para $n=3$
			\begin{eqnarray*}
				H_3(x)&=&\displaystyle\sum_{m=0}^1\displaystyle \frac{(-1)^m 3 ! 2^{3-2 m}}{m !(3-2 m) !} x^{3-2 m}
			\end{eqnarray*}
			Desarrollando
			\begin{eqnarray*}
				H_3(x)&=&\displaystyle\frac{(-1)^0 3 ! 2^3}{0 ! 3 !} x^3+\displaystyle\frac{(-1)^1 3 !(2)}{(1)(1)} x
			\end{eqnarray*}
			Simplificando
			\begin{eqnarray*}
				H_3(x)&=&8 x^3-12 x
			\end{eqnarray*}
		\end{sol}
		
		
		Siguiendo el mismo proceso se obtienen los dem\'as polinomios. En la siguiente tabla se presenta hasta el polinomio de grado nueve.
		\begin{center}
			\begin{tabular}{||cc||}
				\hline
				$n$ & $H_{n}(x)$\\
				\hline
				$0$&$H_0(x)=1$ \\
				\hline
				$1$&$H_1(x)=2 x$ \\
				\hline
				$2$&$H_2(x)=4 x^2-2$ \\
				\hline
				$3$&$H_3(x)=8 x^3-12 x$ \\
				\hline
				$4$&$H_4(x)=16 x^4-48 x^2+12$ \\
				\hline
				$5$&$H_5(x)=32 x^5-160 x^3+120 x$ \\
				\hline
				$6$&$H_6(x)=64 x^6-480 x^4+720 x^2-120$ \\
				\hline
				$7$&$H_7(x)=128 x^7-1344 x^5+3360 x^3-1680 x$ \\
				\hline
				$8$&$H_8(x)=256 x^8-3584 x^6+13440 x^4-13440 x^2+1680$ \\
				\hline
				$9$&$H_9(x)=512 x^9-9216 x^7+48384 x^5-80640 x^3+30240 x$ \\
				\hline
			\end{tabular}
			\label{dwdd}
		\end{center}
		\textcolor{red}{Agregar graficos de los polinomios}
		$$
		\begin{aligned}
			&H \epsilon_1(x)=1 \\
			&H e_1(x)=x^2 \\
			&H e_2(x)=x^2-1 \\
			&H e_{\mathrm{n}}(x)=x^3-3 x \\
			&H e_4(x)=x^4-6 x^2+3 \\
			&H e_5(x)=x^5-10 x^3+15 x \\
			&H e_6(x)=x^6-15 x^4+45 x^2-15 \\
			&H e_7(x)=x^7-21 x^5+105 x^3-105 x \\
			&H e_8(x)=x^8-28 x^6+210 x^4-420 x^2+105 \\
			&H e_9(x)=x^9-36 x^7+378 x^5-1260 x^3+945 x_1, \\
			&H e_{10}(x)=x^{20}-45 x^8+630 x^6-3150 x^4+4725 x^2-945
		\end{aligned}
		$$
		Demostrar como obtienen los m\'onicos
		\subsection{Funci\'on Generatriz}
		\Proposition{La funci\'on generatriz de los polinomios de Hermite}{	La funci\'on generatriz de los polinomios de Hermite est\'a dada por
			\begin{eqnarray}\label{generatrizHermite}
				% \nonumber % Remove numbering (before each equation)
				G(x, t)&=&e^{2 x t-t^2}=\sum_{n=0}^{\infty} \frac{H_n(x) t^n}{n !}
		\end{eqnarray}}
		
		
		\begin{demo}
			Recordemos que
			$e^x=\displaystyle\sum_{n=0}^{\infty} \frac{t^n}{n !}$. As\'i, expandiendo la serie $e^{2 x t-t^2}$, tenemos:
			\begin{eqnarray*}
				G(x, t)&=&e^{2 x t-t^2}\\
				&=&\sum_{n=0}^{\infty} \frac{(2 x t)^n}{n !} \sum_{m=0}^{\infty} \frac{\left(-t^2\right)^m}{m !}\\
				&=&\sum_{n=0}^{\infty} \sum_{m=0}^{\infty} \frac{(-1)^m(2 x)^n}{n !} \frac{(t)^{n+2 m}}{m !}
			\end{eqnarray*}
			Tomando el cambio de \'indices $n \rightarrow n-2 m$
			\begin{eqnarray*}
				% \nonumber % Remove numbering (before each equation)
				G(x, t)&=&e^{2 x t-t^2} \\
				&=& \sum_{m=0}^{\infty} \sum_{n=2 m}^{\infty} \frac{(-1)^m(2 x)^{n-2 m} t^n}{m !(n-2 m) !} \frac{n !}{n !} \\
				&=& \sum_{n=0}^{\infty} \sum_{m=0}^{\infty} \frac{(-1)^m n !(2 x)^{n-2 m}}{m !(n-2 m) !} \frac{t^n}{n !}
			\end{eqnarray*}
			\begin{eqnarray*}
				G(x, z)=e^{2 x t-t^2}&=&\sum_{n=0}^{\infty} H_n(x) \frac{t^n}{n !}
			\end{eqnarray*}
			Como se deseaba probar.
		\end{demo}
		\subsection{F\'ormula de Rodrigues}
		\Theorem{La f\'ormula de Rodrigues para los polinomios de Hermite}{	La f\'ormula de Rodrigues para los polinomios de Hermite es
			\begin{eqnarray}\label{hermiterod}
				% \nonumber % Remove numbering (before each equation)
				H_n(x)&=&(-1)^n e^{x^2}\displaystyle \frac{d^n}{d x^n} e^{-x^2}
		\end{eqnarray}}\label{teohermiterod}
		
		
		
		\begin{demo}
			Para demostrar el teorema (\ref{teohermiterod}), supondremos una soluci\'on de la ecuaci\'on de Hermite dada en (\ref{Hermite}) de la forma
			\begin{eqnarray}\label{hermyte}
				% \nonumber % Remove numbering (before each equation)
				y&=&(-1)^n e^{x^2} D^n e^{-x^2},(D=d / d x)
			\end{eqnarray}
			Si expresamos la ecuaci\'on de Hermite como
			\begin{eqnarray}\label{hermite2}
				\left(e^{-x^2} y^{\prime}\right)^{\prime}+2 n e^{-x^2} y&=&0
			\end{eqnarray}
			Derivando la expresi\'on (\ref{hermyte})
			\begin{eqnarray*}
				y^{\prime}&=&(-1)^n e^{x^2}\left(2 x D^n e^{-x^2}+D^n\left(D e^{-x^2}\right)\right)
			\end{eqnarray*}
			Aplicando la derivada de un producto y tomando factor com\'un obtenemos la siguiente expresi\'on
			$$
			\begin{aligned}
				\left(e^{-x^2} y^{\prime}\right)^{\prime} & =(-1)^n\left[2 D^n e^{-x^2}+2 x D^n\left(D e^{-x^2}\right)+D^{n+1}\left(-2 x e^{-x^2}\right)\right] \\
				\text{Simplificando la expresi\'on en el corchete}\\
				\left(e^{-x^2} y^{\prime}\right)^{\prime} & =2 n(-1)^{n+1} D^n e^{-x^2}=-2 n e^{-x^2} y
			\end{aligned}
			$$
			Reemplazando la expresi\'on anterior en (\ref{hermite2}) demostramos que (\ref{hermyte}) satisface la ecuaci\'on de Hermite.\\
			Como hemos demostrado que la soluci\'on de la ecuaci\'on Hermite son los polinomios de Hermite dado por $H_{n}(x)$, lo que significa que
			\begin{eqnarray*}
				% \nonumber % Remove numbering (before each equation)
				H_n(x)&=&c(-1)^n e^{x^2} D^n e^{-x^2}
			\end{eqnarray*}
			Donde $c$ es una constante. Ahora vamos a proceder a determinar el valor de $c$.\\
			Para $n=0$
			$$
			\begin{aligned}
				H_0(x) & =c(-1)^0 e^{x^2} D^0 e^{-x^2} \\
				& =c e^{x^2} e^{-x^2} \\
				H_0(x) & =c
			\end{aligned}
			$$
			para $n=1$
			$$
			\begin{aligned}
				H_1(x) & =c(-1)^{1} e^{x^2}D^{\prime}e^{-x^2} \\
				& \left.=-c e^{x^2}(-2 xe^{-x^2}\right) \\
				H_1(x) & =2 c x
			\end{aligned}
			$$
			Para $n=2$
			$$
			\begin{aligned}
				H_2(x) & =c(-1)^2 e^{x^2} D^2 e^{-x^2} \\
				& =c e^{x^2}\left(4 x^2-2\right) e^{-x^2} \\
				H_2(x) & =c\left(4 x^2-2\right)
			\end{aligned}
			$$
			Para $n=3$
			$$
			\begin{aligned}
				H_3(x) & =c(-1)^3 e^{x^2} D^3 e^{-x^2} \\
				& =-c e^{x^2}\left(-8 x^3+12 x\right) e^{-x^2} \\
				H_3(x) & =c\left(8 x^3-12 x\right)
			\end{aligned}
			$$
			Para $n=4$
			$$
			\begin{aligned}
				& H_4(x)=c(-1)^4 e^{x^2} b^4 e^{-x^2} \\
				& =c e^{x^2}\left(16 x^4-48 x^2+12\right) e^{-x^2} \\
				& H_4(x)=c\left(16 x^4-48 x^2+12\right) \\
				&
			\end{aligned}
			$$
			Siguiendo el mismo proceso obtenemos los demas polinomios. Ahora comparando con los polinomios obtenidos  anteriormente con los de la tabla (\ref{dwdd})\\
			\begin{center}
				\begin{tabular}{||cc||}
					\hline
					$H_0(x)=1$ & $c(1)=H_0(x)$ \\
					\hline
					$H_1(x)=2 x$ & $c(2x)=H_1(x)$ \\
					\hline
					$ H_2(x)=4 x^2-2$ & $c\left(4 x^2-2\right)=H_2(x)$ \\
					\hline
					$H_3(x)=8 x^3-12 x$ & $c\left(8 x^3-12 x\right)=H_3(x)$ \\
					\hline
					$H_4(x)=16 x^4-48 x^2+12$ & $c\left(16 x^4-48 x^2+12\right)=H_4(x)$ \\
					\hline
				\end{tabular}
			\end{center}
			Comparando el coeficiente principal de los polinomios, vemos que el valor de la constante $c$ es uno, de esta manera queda demostrado el teorema.
		\end{demo}
		\subsection{Relaci\'on de recurrencia de los polinomios de Hermite}
		\Theorem{Los polinomios de Hermites}{Los polinomios de Hermites satisfacen la relaci\'on de tres t\'erminos dadas por
			\begin{eqnarray}\label{hermiterelacion}
				% \nonumber % Remove numbering (before each equation)
				H_{n+1}(x)&=&2 x H_n(x)-2 n H_{n-1}(x), \quad H_0(x)=1, \quad H_1(x)=2 x
		\end{eqnarray}}
		
		
		\begin{demo}
			Derivando respecto a la variable  $t$ la expresi\'on (\ref{generatrizHermite})
			\begin{eqnarray*}
				(2 x-2 t) e^{2 x t-t^2}&=&\displaystyle\sum_{n=1}^{\infty} H_n(x) \frac{n t^{n-1}}{n !}
			\end{eqnarray*}
			Aplicando propiedad distributiva y reescribiendo la sumatoria
			\begin{eqnarray*}
				2x\left(e^{2 x t-t^2}\right)-2 t\left(e^{2 x t-t^2}\right)&=&\displaystyle\sum_{n=0}^{\infty} H_{n+1}(x)\displaystyle\frac{(n+1) t^n}{(n+1)!}
			\end{eqnarray*}
			Por la expresi\'on (\ref{generatrizHermite}), tenemos
			$$
			\begin{aligned}
				& 2 x \displaystyle\sum_{n=0}^{\infty} H_n(x) \frac{t^n}{n !}-2 t\displaystyle \sum_{n=0}^{\infty} H_n(x) \frac{t^n}{n !}=\displaystyle\sum_{n=0}^{\infty} H_{n+1}(x) \frac{(n+1) t^n}{(n+1) !} \\
				& 2 x\displaystyle \sum_{n=0}^{\infty} H_n(x) \frac{t^n}{n !}-2\displaystyle \sum_{n=0}^{\infty} H_n(x) \frac{t^{n+1}}{n !}=\displaystyle\sum_{n=0}^{\infty} H_{n+1}(x) \frac{(n+1) t^n}{(n+1) !}
			\end{aligned}
			$$
			Agrupando los t\'erminos semejantes
			\begin{eqnarray*}
				\displaystyle\sum_{n=0}^{\infty}\left[2 x \displaystyle\frac{H_n(x)}{n !}-\frac{(n+1) H_{n+1}(x)}{(n+1) !}\right] t^n-2\displaystyle \sum_{n=0}^{\infty} H_n(x) \frac{t^{n+1}}{n !}=0
			\end{eqnarray*}
			Desarrollando el primer t\'ermino
			\begin{eqnarray*}
				\left(2 x H_0(x)-H_1(x)\right)+\displaystyle\sum_{n=1}^{\infty}\left[2 x \frac{H_n(x)}{n !}-\displaystyle\frac{(n+1) H_{n+1}(x)}{(n+1) !}\right] t^n-2\displaystyle \displaystyle\sum_{n=1}^{\infty} H_{n-1}(x) \frac{t^n}{(n-1) !}&=&0
			\end{eqnarray*}
			Sumando nueva vez t\'erminos Semejantes
			\begin{eqnarray*}
				\left(2 x H_0(x)-H_1(x)\right)+\displaystyle\sum_{n=1}^{\infty}\left[2 x \frac{H_n(x)}{n !}-\displaystyle\frac{(n+1) H_{n+1}(x)}{(n+1) !}-\displaystyle\frac{2 H_{n-1}(x)}{(n-1) !}\right] t^n&=&0
			\end{eqnarray*}
			Igualando a cero, tenemos
			$$
			\left\{\begin{array}{l}
				2 x H_0(x)-H_1(x)=0 \\
				2 x\displaystyle \frac{H_n(x)}{n !}-\displaystyle\frac{(n+1) H_{n+1}(x)}{(n+1)!}-\displaystyle\frac{2 H_{n-1}(x)}{(n-1) !}=0
			\end{array}\right.
			$$
			Ahora despejando el polinomio de mayor grado
			$$
			\begin{aligned}
				& H_1(x)=2 x H_0(x)  \\
				& H_{n+1}(x)=n !\left[2 x \displaystyle\frac{H_n(x)}{n !}-\displaystyle\frac{2 H_{n-1}(x)}{(n-1) !}\right]=2 x H_n(x)-2 n H_{n-1}(x)
			\end{aligned}
			$$
			con lo que queda demostrado el teorema.
		\end{demo}
		
		
		%% Verificamos desde aqui 
		
		\Example{	Utiliza la relaci\'on de recurrencia de los polinomios de Hermite}{
			
			Utiliza la relaci\'on de recurrencia de los polinomios de Hermite para calcular los polinomios $H_{2}(x)$ y $H_{3}(x)$}
		
		\begin{sol}
			Para $n=1$, tenemos
			\begin{eqnarray*}
				H_2(x)&=&2 x H_1(x)-2(1) H_0(x)
			\end{eqnarray*}
			Reemplazando
			$$
			\begin{aligned}
				& H_2(x)=2 x(2 x)-2(1) \\
				& H_2(x)=4 x^2-2
			\end{aligned}
			$$
			Para $n=2$
			\begin{eqnarray*}
				H_3(x)&=&2 x H_2(x)-2(2) H_1(x)
			\end{eqnarray*}
			Reemplazando
			$$
			\begin{aligned}
				& H_3(x)=2 x\left(4 x^2-2\right)-4(2 x) \\
				& H_3(x)=8 x^3-4 x-8 x
			\end{aligned}
			$$
			Simplificando
			\begin{eqnarray*}
				H_3(x)&=&8 x^3-12 x
			\end{eqnarray*}
		\end{sol}
	\subsection{Representaci\'on integral de los polinomios de Hermite}	
		\Theorem{Los polinomios de Hermite}{Los polinomios de Hermite
			\begin{equation*}
				H_n(x)=\sum_{m=0}^{\left[\frac{N}{2}\right]} \dfrac{(-1)^m n ! 2^{n-2 m}}{m !(n-2 m) !} x^{n-2 m}
			\end{equation*}
			satisfacen la expresi\'on
			\begin{equation}\label{Hermiteintegral}
				H_n(x)=\frac{n !}{2 \pi i} \oint_C \dfrac{e^{2 x t-t^2}}{t^{n+1}} d t
			\end{equation}
			donde C es un contorno que rodea el origen en direcci\'on contraria a las agujas del reloj.}
		
		
		\begin{demo}
			Recordemos la f\'ormula generatriz de los polinomios de Hermite \ref{generatrizHermite} la cual est\'a dada por:
			$$
			G(x, t)=e^{2 x t-t^2}=\sum_{n=0}^{\infty} \frac{H_n(x) t^n}{n !}
			$$
			As\'i, calculando la derivada k-\'esima con respecto a  t  tenemos:
			\begin{eqnarray*}
				\frac{d^k}{d t^k}\left[e^{2 x t-t^2}\right]&=&\frac{d^k}{d t^k}\left[\sum_{n=0}^{\infty} \frac{H_n(x) t^n}{n !}\right]\\
				\frac{d}{d t}\left[e^{2 x t-t^2}\right]&=&\sum_{n=1}^{\infty} \frac{n H_n(x) t^{n-1}}{n !} ; \frac{d^2}{d t^2}\left[e^{2 x t-t^2}\right]\\
				&=&\sum_{n=2}^{\infty} \frac{n(n-1) H_n(x) t^{n-2}}{n !} \\
				\frac{d^k}{d t^k}\left[e^{2 x t-t^2}\right]&=&\sum_{n=k}^{\infty} \frac{n !}{(n-k) !} \frac{H_n(x) t^{n-k}}{n !}\\
				\frac{d^k}{d t^k}\left[e^{2 x t-t^2}\right]&=&\sum_{n=0}^{\infty} \frac{H_{n+k}(x) t^n}{(n) !}\\
				\frac{d^k}{d t^k}\left[e^{2 x t-t^2}\right]_{t=0}&=&\left[H_k(x)+\sum_{n=1}^{\infty} \frac{H_{n+k}(x) t^n}{(n) !}\right]_{t=0}\\
				\frac{d^n}{d t^n}\left[e^{2 x t-t^2}\right]_{t=0}&=&H_n(x)\\
				\frac{1}{2 \pi i} \oint_C \frac{f(z)}{\left(z-z_0\right)^{n+1}} d z&=&\frac{f^{(n)}\left(z_0\right)}{n !} \\
				f^{(n)}\left(z_0\right)&=&\frac{n !}{2 \pi i} \oint_C \frac{f(z)}{\left(z-z_0\right)^{n+1}} d z \\
				\frac{d^n}{d t^n}\left[e^{2 x t-t^2}\right]_{t=0}&=&f^{(n)}(0)=H_n(x) \\
				H_n(x)&=&\frac{n !}{2 \pi i} \oint_C \frac{e^{2 x t-t^2}}{t^{n+1}} d t
			\end{eqnarray*}
		\end{demo}
		
		\Example{Utiliza la f\'ormula integral de los polinomios de Hermite}{
			
			Utiliza la f\'ormula integral de los polinomios de Hermite para calcular $H_{0}(x), H_{1}(x),\,H_{2}(x),\,\text{y}\, H_{3}(x)$}
		
		\begin{sol}
			\begin{itemize}
				\item Para $n=0$
				\begin{eqnarray*}
					H_{0}(x)&=&\frac{1}{2 \pi i} \oint_c \frac{e^{2 x t-t^2}}{t} d t
				\end{eqnarray*}
				Como to $=0$ es un polo simple, tenemos del teorema (\ref{cauchy}) que
				\begin{eqnarray*}
					H_0(x)&=&\displaystyle\frac{2 \pi i}{2 \pi i} \operatorname{Res}(f(t), 0)
				\end{eqnarray*}
				Por la expresi\'on (\ref{polosimple})
				\begin{eqnarray*}
					H_0(x)=\lim _{t \rightarrow 0} t \left[\frac{e^{2 x t-t^2}}{t}\right]
				\end{eqnarray*}
				
				Evaluando el l\'imite y simplificando
				\begin{eqnarray*}
					H_0(x)&=&1
				\end{eqnarray*}
				\item Para $n=1$
				\begin{eqnarray*}
					H_1(x)&=&\displaystyle\frac{1}{2 \pi i} \oint_c \frac{e^{2 x t-t^2}}{t^2} d t
				\end{eqnarray*}
				podemos ver que $t_{0}=0$ es un polo de orden 2, asi del teorema (\ref{cauchy}) tenemos
				\begin{eqnarray*}
					H_1(x)&=&\displaystyle\frac{2 \pi i}{2 \pi i} \operatorname{Res}(f(t), 0)
				\end{eqnarray*}
				Aplicando la expresi\'on (\ref{poloN})
				\begin{eqnarray*}
					H_1(x)&=&\lim _{t \rightarrow 0} \frac{d}{d t}\left[t^2 \frac{e^{2 x t}-t^2}{t^2}\right]=\lim _{t \rightarrow 0} \frac{1}{d t}\left[e^{2 x t-t^2}\right]
				\end{eqnarray*}
				Derivando
				\begin{eqnarray*}
					H_1(x)&=&\lim _{t \rightarrow 0} e^{2 x t-t^2}(2 x-2 t)
				\end{eqnarray*}
				Evaluando el l\'imite
				\begin{eqnarray*}
					H_1(x)&=&2 x
				\end{eqnarray*}
				\item Para $n=2$
				\begin{eqnarray*}
					H_2(x)&=&\displaystyle\frac{2}{2 \pi i} \oint_c \frac{e^{2 x t-t^2}}{t^3} d t
				\end{eqnarray*}
				Podemos notar que  $t_{0}=0$ es un polo de orden 3, utilizando la expresi\'on (\ref{cauchyteo} ) dada en el torema  (\ref{cauchy}), obtenemos que
				\begin{eqnarray*}
					H_2(x)&=&2 \operatorname{Res}(f(t), 0)
				\end{eqnarray*}
				Utilizando la expresi\'on del residuo (\ref{poloN} ) dada en el teorema (\ref{PoloN}), tenemos
				\begin{eqnarray*}
					H_2(x)&=&2\displaystyle \frac{1}{2} \lim _{t \rightarrow 0} \frac{d^2}{d t^2}\left[t^3 \frac{e^{2 x t-t^2}}{t^3}\right]
				\end{eqnarray*}
				Simplificando y calculando las derivadas
				\begin{eqnarray*}
					H_2(x)&=&\displaystyle\lim _{t \rightarrow 0} e^{2 x t-t^2}\left(4 t^2-8 x t+4 x^2-2\right)
				\end{eqnarray*}
				Evaluando el l\'imite
				\begin{eqnarray*}
					H_2(x)=4 x^2-2
				\end{eqnarray*}
				\item Para $n=3$
				\begin{eqnarray*}
					H_3(x)&=&\frac{6}{2 \pi i} \oint_c \frac{e^{2 x t-t^2}}{t^4} d t
				\end{eqnarray*}
				Es evidente que $t_0=0$ es un polo de orden 4 , de tal manera que aplicando la expresi\'on (\ref{cauchyteo}) llegamos a
				\begin{eqnarray*}
					H_3(x)&=&6 \operatorname{Res}(f(t), 0)
				\end{eqnarray*}
				Por la expresi\'on (\ref{poloN}) tenemos
				\begin{eqnarray*}
					H_3(x)&=&6\displaystyle \frac{1}{6} \lim _{t \rightarrow 0} \frac{d^3}{d t^3}\left[t^4 \frac{e^{2 x t-t^2}}{t^4}\right]
				\end{eqnarray*}
				Simplificando y derivando
				\begin{eqnarray*}
					H_3(x)&=&\displaystyle\lim _{t \rightarrow 0}\left[-4 e^{2 x t-t^2}\left(2 t^3-6 t^2 x+t\left(6 x^2-3\right)-2 x^3+3 x\right)\right]
				\end{eqnarray*}
				Evaluando el l\'imite
				\begin{eqnarray*}
					H_3(x)&=&-4\left(-2 x^3+3 x\right)
				\end{eqnarray*}
				Simplificando
				\begin{eqnarray*}
					H_3(x)&=&8 x^3-12 x
				\end{eqnarray*}
			\end{itemize}
		\end{sol}
\subsection{Ecuaci\'on asociada de los polinomios de Hermite}
Trabajar esta parte
		\section{Ecuaci\'on de Laguerre}
		\begin{eqnarray}\label{Laguerre equation}
			% \nonumber % Remove numbering (before each equation)
			x y^{\prime \prime}+(\alpha+1-x) y^{\prime}+b y&=&0
		\end{eqnarray}
		Expresando \ref{Laguerre equation} en su forma normal se obtiene
		\begin{eqnarray}\label{Laguerre equation normal form}
			y^{\prime \prime}+\frac{(\alpha+1-x)}{x} y^{\prime}+\frac{x b}{x^2}y&=&0
		\end{eqnarray}
		Comparando \ref{Laguerre equation normal form} con la expresi\'on general de una ecuaci\'on ordinaria de segundo orden dada en su forma normal, se tiene que
		$P(x)=(\alpha+1-x)\quad y\quad q(x)=b x$, de donde $x_{0}=0$ es un punto singular regular de \ref{Laguerre equation}, por lo tanto existe una soluci\'on en serie de potencias de la forma\\ $$y(x)=\left(x-x_0\right)^{r} \sum_{n=0}^{\infty} c_n\left(x-x_0\right)^n \quad c_0 \neq 0$$
		La ecuaci\'on indicial de \ref{Laguerre equation} viene dada por
		\begin{eqnarray*}
			% \nonumber % Remove numbering (before each equation)
			r(r-1)+(\alpha+1) r+0 &=& 0 \\
			r(x-1+\alpha+1)= &=& 0 \\
			r(r+\alpha) &=& 0
		\end{eqnarray*}
		con $r=0 \quad \text{y}\quad r=-\alpha$ son las soluciones de la ecuaci\'on indicial.
		Tomando soluciones en series de potencias y calculando sus derivadas tenemos
		\begin{eqnarray*}
			% \nonumber % Remove numbering (before each equation)
			y(x)&=&\sum_{n=0}^{\infty} c_n x^{n+r}\\
			y^{\prime}(x)&=&\sum_{n=0}^{\infty}(n+r) x^{n+r-1}\\
			y^{\prime \prime}(x)&=&\sum_{n=0}^{\infty}(n+r)(n+r-1) c_n x^{n+r-2}
		\end{eqnarray*}
		Sustituyendo en \ref{Laguerre equation}, tenemos
		\begin{eqnarray*}
			% \nonumber % Remove numbering (before each equation)
			x^r\left[\sum_{n=0}^n(n+r)(n+r-1)c_n x^{n-1}+(\alpha+1) \sum_{n=0}^{\infty}(n+r)c_n x^{n-1}-\sum_{n=0}^{\infty} c_n(n+r) x^n+b \sum_{n=0}^1 c_n x^n\right]&=&0\\
			\sum_{n=0}^{\infty}c_n(n+r)[n+r-1+\alpha+1] x^{n-1}+\sum_{n=0}^{\infty}c_n[b-n-r] x^n&=&0\\
			\sum_{n=0}^{\infty}c_n(n+r)(n+r+\alpha)x^{n-1}+\sum_{n=0}^{\infty}c_n[b-n-r] x^n&=&0\\
			x^{-1}c_{0}r(r+\alpha)+\sum_{n=0}^{\infty}c_{n+1}(n+1+r)(n+1+r+\alpha)x^{n}\sum_{n=0}^{\infty}c_n[b-n-r] x^n&=&0\\
			c_{0}r(r+\alpha)x^{-1}+\sum_{n=0}^{\infty}[c_{n+1}(n+1+r)(n+1+r+\alpha)+c_n(b-n-r)] x^n&=&0\\
		\end{eqnarray*}
		De lo anterior se llega a que la ecuaci\'on indicial es
		\begin{eqnarray*}
			r(r+\alpha)&=&0
		\end{eqnarray*}
		cuyas soluciones son $r_{1}=0 \quad \text{y} \quad r_{2}=-\alpha$. Ahora igualando el coeficiente de $x^{n}$ a cero, obtenemos
		\begin{eqnarray}\label{laguerre}
			c_{n+1}(n+1+r)(n+1+r+\alpha)+c_{n}(b-n-r)=0\quad n\geq 0
		\end{eqnarray}
		Como la expresi\'on anterior contiene a $r$, y conocemos los valores al resolver la ecuaci\'on indicial,  reemplazaremos estos valores para obtener una expresi\'on en cada caso.
		\begin{itemize}
			\item Para $r=0$
			\begin{eqnarray}\label{ro}
				% \nonumber % Remove numbering (before each equation)
				c_{n+1}(n+1)(n+1+\alpha)+c_n(b-n)&&=0 \quad \forall n \geq 0
			\end{eqnarray}
			\item Para $ r=-\alpha$
			\begin{eqnarray}\label{ra}
				% \nonumber % Remove numbering (before each equation)
				c_{n+1}(n+1-\alpha)(n+1)+c_n(b-n+\alpha)&=&0 \quad n \geq 0
			\end{eqnarray}
		\end{itemize}
		Haciendo un cambio de variable en las expresiones anteriores y despejando a $c_{n+1}$
		\begin{eqnarray}\label{Laguerre ro}
			c_{m}=\frac{m-b-1}{m(m+\alpha)}c_{m-1}\quad \forall m\geq 1
		\end{eqnarray}
		\begin{eqnarray}\label{Laguerre ra}
			c_{m}=\frac{m-\alpha-b-1}{m(m-\alpha)}c_{m-1} \quad \forall m\geq 1
		\end{eqnarray}
		Ahora desarrollaremos las expresiones (\ref{Laguerre ro}) y (\ref{Laguerre ra}) asignando valores a la variable $m$
		\begin{itemize}
			\item Desarrollando la expresi\'on \ref{Laguerre ro}\\
			Para $m=1$
			\begin{eqnarray*}
				c_1&=&\displaystyle\frac{1-b-1}{\alpha+1} c_0=\displaystyle\frac{(0-b)}{\alpha+1} c_0
			\end{eqnarray*}
			Para $m=2$
			\begin{eqnarray*}
				c_2&=&\displaystyle\frac{1-b}{2(\alpha+2)} c_1=\displaystyle\frac{(0-b)(1-b)}{2(\alpha+1)(\alpha+2)} c_0
			\end{eqnarray*}
			Para $m=3$
			\begin{eqnarray*}
				c_3&=&\displaystyle\frac{2-b}{3(\alpha+3)} c_2=\displaystyle\frac{(0-b)(1-b)(2-b)}{3 \cdot 2(\alpha+1)(\alpha+2)(\alpha+3)} c_0
			\end{eqnarray*}
			Para $m=4$
			\begin{eqnarray*}
				c_4&=&\displaystyle\frac{3-b}{4(\alpha+4)} c_3=\displaystyle\frac{(0-b)(1-b)(2-b)(3-b)}{4 \cdot 3 \cdot 2(\alpha+1)(\alpha+2)(\alpha+3)(\alpha+4)} c_0
			\end{eqnarray*}
			Siguiendo el mismo desarrollo, llegamos a que
			\begin{eqnarray}\label{rofinal}
				c_m&=&\displaystyle\frac{(0-b)(1-b)(2-b) \cdots(m-1-b)}{m !(\alpha+1)(\alpha+2) \cdots(\alpha+m)} c_0 \quad \forall m \geq 1
			\end{eqnarray}
			Expresando (\ref{rofinal}) como productoria
			\begin{eqnarray*}
				% \nonumber % Remove numbering (before each equation)
				c_{m} &=& \displaystyle\prod_{i=1}^{m}\displaystyle\frac{i-1-b}{i(\alpha+i)}c_{0}\quad m\geq1
			\end{eqnarray*}
			Utilizando la relaci\'on de la funci\'on gamma con la productoria dada en (\ref{Funci\'on gamma en t\'erminos de la productoria})
			\begin{eqnarray}
				% \nonumber % Remove numbering (before each equation)
				c_{m} &=& \displaystyle\frac{(-1)^{m}\Gamma(\alpha+1)\Gamma(b+1)}{m!\Gamma(m+\alpha+1)\Gamma(b+1-m)}c_{0}\quad \forall m\geq 0
			\end{eqnarray}
			De manera que primera soluci\'on es
			\begin{eqnarray}\label{Laguerre 1ra solution}
				y_{1}(x)=c_{0}\displaystyle\sum_{m=0}^{\infty}\displaystyle\frac{(-1)^{m}\Gamma(\alpha+1)\Gamma(b+1)}{m!\Gamma(m+\alpha+1)\Gamma(b+1-m)}x^{m}
			\end{eqnarray}
			\item Ahora desarrollando la expresi\'on \ref{Laguerre ra}\\
			Para $m=1$
			\begin{eqnarray*}
				c_{1}&=&(-1)^{1}\displaystyle \frac{\alpha+b}{1!(1-\alpha)} c_0
			\end{eqnarray*}
			Para $m=2$
			\begin{eqnarray*}
				c_{2}&=&(-1)^2\displaystyle\frac{(\alpha+b)(\alpha+b-1)}{2!(1-\alpha)(2-\alpha)} c_0
			\end{eqnarray*}
			Para $m=3$
			\begin{eqnarray*}
				c_{3}&=&(-1)^3\displaystyle\frac{(\alpha+b)(\alpha+b-1)(\alpha+b-2)}{3!(1-\alpha)(2-\alpha)(3-\alpha)}c_0
			\end{eqnarray*}
			Para $m=4$
			\begin{eqnarray*}
				c_{4}&=&(-1)^4\displaystyle\frac{(\alpha+b-0)(\alpha+b-1)(\alpha+b-2)(\alpha+b-3)}{4!(1-\alpha)(2-\alpha)(3-\alpha)(4-\alpha)} c_{0}
			\end{eqnarray*}
			Siguiendo el desarrollo se llega a la expresi\'on general
			\begin{eqnarray*}
				c_{m}&=&\frac{(-1)^{m}\displaystyle \prod_{i=0}^{m-1}(\alpha+b-i)}{m!\displaystyle\prod_{i=1}^{m}(i-\alpha)}c_{0}
			\end{eqnarray*}
			\begin{eqnarray}\label{Laguerre cmra}
				c_{m}&=&\displaystyle\frac{(-1)^{m} \Gamma{(\alpha+b+1)} \Gamma{(1-\alpha)}}{m!\Gamma{(\alpha+b+1-m)}\Gamma{(m+1-\alpha)}}c_{0}
			\end{eqnarray}
		\end{itemize}
		As\'i la segunda soluci\'on est\'a por la expresi\'on
		\begin{eqnarray}\label{Laguerre 2da solution}
			y_{2}(x)=c_{0}|x|^{-\alpha}\displaystyle\sum_{m=0}^{\infty}\frac{(-1)^{m}\Gamma{(\alpha+b+1)} \Gamma{(1-\alpha)}}{\Gamma{(\alpha+b+1-m)}\Gamma{(m+1-\alpha)}}x^{m}
		\end{eqnarray}
		Por lo tanto la soluci\'on general de la ecuaci\'on de Laguerre dada en (\ref{Laguerre equation}) es
		\begin{eqnarray}\label{laguesolgnal}
			% \nonumber % Remove numbering (before each equation)
			y(x)=\left. A\displaystyle\sum_{m=0}^{\infty}\displaystyle\frac{(-1)^{m}\Gamma(\alpha+1)\Gamma(b+1)}{m!\Gamma(m+\alpha+1)\Gamma(b+1-m)}x^{m} +B|x|^{-\alpha}\displaystyle\sum_{m=0}^{\infty}\frac{(-1)^{m}\Gamma{(\alpha+b+1)} \Gamma{(1-\alpha)}}{\Gamma{(\alpha+b+1-m)}\Gamma{(m+1-\alpha)}}x^{m}\right.
		\end{eqnarray}
		\textcolor{red}{determinar el intervalo de convergencia de las soluciones en series}\\
		Para obtener los polinomios de Laguerre tomamos $\displaystyle b=n$ en la expresi\'on  (\ref{Laguerre 1ra solution} ) y la mutiplicamos por $\displaystyle\frac{\Gamma(\alpha+n+1)}{n ! \Gamma(\alpha+1)}$
		\begin{eqnarray*}
			L_{n}^{(\alpha)}(x)&=&\displaystyle\frac{\Gamma(\alpha+n+1)}{n ! \Gamma(\alpha+1)} \sum_{m=0}^{n} \frac{(-1)^m \Gamma(\alpha+1) \Gamma^{(b+1)} x^m}{m ! \Gamma(m+\alpha+1) \Gamma(n+1-m)}
		\end{eqnarray*}
		Introduciendo el factor $\displaystyle\frac{1}{n ! \Gamma(\alpha+1)}$ en la sumatoria
		\begin{eqnarray*}
			L_{n}^{(\alpha)}(x)&=&\displaystyle\Gamma(\alpha+n+1) \sum_{m=0}^{n} \frac{(-1)^m x^m}{m ! \Gamma(m+\alpha+1) \Gamma(n+1-m)}
		\end{eqnarray*}
		Por la propiedad (\ref{Relaci\'on de recurrencia de la funci\'on gamma}) tenemos
		\begin{eqnarray}\label{lagueralpha}
			L_{n}^{(\alpha)}(x)&=&\Gamma(\alpha+n+1) \sum_{m=0}^{n} \frac{(-1)^m x^m}{m !(n-m) ! \Gamma(m+\alpha+1)}
		\end{eqnarray}
		Tomando $\alpha=0$, obtenemos los polinomios de Laguerre de orden $n$ que se representa por $L_{n}(x)$ y su expresi\'on es
		\begin{eqnarray}\label{laguer0}
			% \nonumber % Remove numbering (before each equation)
			L_n(x)&=&\displaystyle\sum_{m=0}^n \frac{1}{m !}\left(\begin{array}{c}
				n \\
				m
			\end{array}\right)(-x)^m
		\end{eqnarray}
		\textcolor{red}{programar wx maxima los polinomios asociados de laguerre y graficar para diferentes valores de $\alpha$ incluyendo $\alpha=0$}
		
		\Example{Polinomio de Laguerre}{Utiliza la expresi\'on (\ref{laguer0}) para obtener los primeros 7 polinomios de Laguerre.}
		
		\begin{sol}
			sdd
		\end{sol}
		\begin{center}
			\begin{tabular}{||cc||}
				\hline
				% after \\: \hline or \cline{col1-col2} \cline{col3-col4} ...
				% \hline
				$n$ & $L_{n}(x)$\\
				\hline
				$0$ & $1$ \\
				\hline
				$ 1$ & $-x+1$ \\
				\hline
				$2$ & $\left(x^2-4 x+2\right) / 2$ \\
				\hline
				$3$ &$ \left(-x^3+9 x^2-18 x+6\right) / 6$ \\
				\hline
				$4$ & $\left(x^4-16 x^3+72 x^2-96 x + 24\right) / 24 $\\
				\hline
				$5$ & $\left(-x^5+25 x^4-200 x^3 + 600 x^2-600 x+120\right) / 120$ \\
				\hline
				$ $6 & $\left(x^6-36 x^5+450 x^4-2400 x^3+5400 x^2-4320 x+720\right) / 720$ \\
				\hline
			\end{tabular}
		\end{center}
		\textcolor{red}{Aclarar que pueden haber mas de una funcion generadoora}
\subsection{Funciones Generadoras de los polinomios de Laguerre}
\subsubsection{Funci\'on Generatriz de los polinomios de Laguerre}
		\Proposition{La funci\'on generatriz de los polinomios de Laguerre}{La funci\'on generatriz de los polinomios de Laguerre est\'a dada por la expresi\'on:
			\begin{eqnarray}\label{laguerregeneratriz}
				% \nonumber % Remove numbering (before each equation)
				G(x, z)&=&\sum_{n=0}^{\infty} L_n(x) z^n
		\end{eqnarray}}
		
		
		\begin{demo}
			Sea la funci\'on $G(x, z)=\sum_{n=0}^{\infty} L_n(x) z^n$ desarrollando esta funci\'on obtenemos:
			$$
			G(x, z)=\sum_{n=0}^{\infty} L_n(x) z^n=\sum_{n=0}^{\infty} \sum_{m=0}^n \frac{1}{m !}\left(\begin{array}{c}
				n \\
				m
			\end{array}\right)(-x)^m z^n=\sum_{m=0}^{\infty} \sum_{n=m}^{\infty} \frac{1}{m !}(-x)^m\left(\begin{array}{c}
				n \\
				m
			\end{array}\right) z^n
			$$
			Tomando el cambio de \'indices $k=n-m$ :
			$$
			G(x, z)=\sum_{m=0}^{\infty} \sum_{k=0}^{\infty} \frac{1}{m !}(-x)^m\left(\begin{array}{c}
				m+k \\
				m
			\end{array}\right) z^{m+k}=\sum_{m=0}^{\infty} \frac{1}{m !}(-x)^m z^m \sum_{k=0}^{\infty}\left(\begin{array}{c}
				m+k \\
				m
			\end{array}\right) z^k
			$$
			Pero $\displaystyle\sum_{k=0}^{\infty}\left(\begin{array}{c}m+k \\ m\end{array}\right) z^k=\left(\frac{1}{1-z}\right)^{m+1}$, cuando $|x|<1$. Lo que implica
			\begin{eqnarray*}
				% \nonumber % Remove numbering (before each equation)
				G(x, z) &=& \sum_{m=0}^{\infty} \frac{1}{m !}(-x)^m z^m\left(\frac{1}{1-z}\right)^{m+1} \\
				&=& \sum_{m=0}^{\infty}\left(\frac{-x z}{1-z}\right)^m\left(\frac{1}{1-z}\right) \frac{1}{m !}\\
				G(x, z)&=&\left(\frac{1}{1-z}\right) \sum_{m=0}^{\infty}\left(\frac{-x z}{1-z}\right)^m \frac{1}{m !}
			\end{eqnarray*}
			Recordemos que
			\begin{eqnarray*}
				e^x&=&\sum_{m=0}^{\infty} \frac{x^m}{m !}
			\end{eqnarray*} lo que nos lleva a:
			\begin{eqnarray*}
				% \nonumber % Remove numbering (before each equation)
				G(x, z)&=&\left(\frac{1}{1-z}\right) e^{\displaystyle\frac{-x z}{1-z}} \\
				G(x, z)&=&\displaystyle\sum_{n=0}^{\infty} L_n(x) z^n
			\end{eqnarray*}
		\end{demo}
%%%%%%%%%%%%%%%%%%%%%%%%%%%%%%%%%
		\subsubsection{Funci\'on Generadora para los polinomios de Laguerre}
		\Theorem{Función generadora de los polinomios de Laguerre}{La funci\'on generadora de los polinomios de Laguerre es
			\begin{eqnarray}\label{laguergen}
				G(x,t)&=&\displaystyle\sum_{n=0}^{\infty} L_n^{(\alpha)}(x) t^n
			\end{eqnarray}
			donde
			\begin{eqnarray}\label{gxt}
				% \nonumber % Remove numbering (before each equation)
				G(x,t)&=&(1-t)^{-1-\alpha}\displaystyle \exp \left(-\displaystyle\frac{x t}{(1-t)}\right)
		\end{eqnarray}}
		
		
		\begin{demo}
			\textcolor{red}{dar el intervalo de convergia de la serie}\\
			Para realizar esta demostraci\'on, demostraremos que la expresi\'on de la sumatoria en (\ref{laguergen}) es la expresi\'on dada en (\ref{gxt}).\\
			Reemplazando la expresi\'on (\ref{lagueralpha}) en la sumatoria
			\begin{eqnarray*}
				\displaystyle\sum_{n=0}^{\infty} L_n^{(\alpha)}(x) t^n&=&\displaystyle\sum_{n=0}^{\infty}\left[\sum_{m=0}^n \frac{\Gamma(n+\alpha+1)(-x)^m}{m !(n-m) ! \Gamma(m+\alpha+1)}\right] t^n
			\end{eqnarray*}
			La sumatoria se puede expresar como
			\begin{eqnarray*}
				\displaystyle\sum_{m=0}^{\infty}\left[\displaystyle\sum_{n=m}^{\infty}\displaystyle \frac{\Gamma(n+\alpha+1) t^n}{m !(n-m) ! \Gamma(m+\alpha+1)}\right](-x)^m
			\end{eqnarray*}
			Tomando el \'indice $k=n-m$
			\begin{eqnarray*}
				\displaystyle\sum_{m=0}^{\infty}\left[\displaystyle\sum_{k=0}^{\infty} \displaystyle\frac{\Gamma(k+m+\alpha+1) t^{k+m}}{m ! k ! \Gamma(m+\alpha+1)}\right](-x)^m
			\end{eqnarray*}
			Separando las dos sumatorias
			\begin{eqnarray*}
				\displaystyle\sum_{m=0}^{\infty}\left[\displaystyle\frac{t^m(-x)^m}{m ! \Gamma(m+\alpha+1)}\right] \displaystyle\sum_{k=0}^{\infty}\left[\displaystyle\frac{\Gamma(k+m+\alpha+1)}{k !} t^k\right]
			\end{eqnarray*}
			Expresando la expresi\'on de gamma a factorial
			\begin{eqnarray*}
				\displaystyle\sum_{m=0}^{\infty}\left[\displaystyle\frac{t^m(-x)^m}{m !(m+\alpha) !}\right]\displaystyle \sum_{k=0}^{\infty}\left[\frac{(m+k+\alpha) !}{k !} t^k\right]
			\end{eqnarray*}
			Multiplicando por $\displaystyle\frac{(m+\alpha) !}{(m+\alpha) !}$ y por defunici\'on de coeficiente binomial, llegamos a
			\begin{eqnarray*}
				\displaystyle\sum_{m=0}^{\infty}\left[\displaystyle\frac{(-x t)^m}{m !}\right] \displaystyle\sum_{k=0}^{\infty}\left[\left(\begin{array}{c}
					k+m+\alpha \\
					k
				\end{array}\right) t^{k}\right]
			\end{eqnarray*}
			Sabemos que $\displaystyle\sum_{k=0}^{\infty}\left[\left(\begin{array}{c}k+m+\alpha \\ k\end{array}\right) t^k\right]=\displaystyle\left(\frac{1}{1-t}\right)^{m+\alpha+1}$ reemplazando tenemos
			\begin{eqnarray*}
				\displaystyle\left(\frac{1}{1-t}\right)^{\alpha+1}\displaystyle \sum_{m=0}^{\infty}\left[\displaystyle\frac{1}{m !}\left(\frac{-x t}{1-t}\right)^m\right]
			\end{eqnarray*}
			Aplicando la serie de la funci\'on exponencial
			\begin{eqnarray*}
				(1-t)^{-1-\alpha} e^{\displaystyle\frac{-x t}{1-t}}
			\end{eqnarray*}
			Por lo tanto
			\begin{eqnarray*}
				G(x,t)&=&(1-t)^{-1-\alpha} e^{\displaystyle\frac{-xt}{1-t}}
			\end{eqnarray*}
			Lo cual demuestra el teorema.
		\end{demo}
\subsection{ F\'ormula de Rodrigues para los polinomios de Laguerre}
		\Theorem{F\'ormula de Rodrigues}{La f\'ormula de Rodrigues para los polinomios de Laguerre est\'a dada por
			\begin{eqnarray}\label{laguerod}
				% \nonumber % Remove numbering (before each equation)
				L_n^{(\alpha)}(x)&=&\displaystyle\frac{e^x x^{-\alpha}}{n !} \displaystyle\frac{d^n}{d x^n}\left(e^{-x} x^{n+\alpha}\right)
		\end{eqnarray}}
		
		
		\begin{demo}
			Para demostrar la f\'ormula de Rodrigues, en primer lugar aplicamos la regla de Leibniz para derivada e-n\'esima, es decir
			$$
			\displaystyle\frac{d^n}{d x^n}\left(e^{-x} x^{n+\alpha}\right)=\displaystyle\sum_{k=0}^n\left(\begin{array}{l}
				n \\
				k
			\end{array}\right)(-1)^k e^{-x}\displaystyle \frac{\Gamma(\alpha+n+1)}{\Gamma(\alpha+k+1)} x^{\alpha+k}
			$$
			Sustituyendo la expresi\'on anterior en (\ref{laguerod}), tenemos que
			$$
			L_n^{(\alpha)}(x)=\displaystyle\frac{e^x x^{-\alpha}}{n !} \sum_{k=0}^n\left(\begin{array}{l}
				n \\
				k
			\end{array}\right)(-1)^k\displaystyle \frac{\Gamma(\alpha+n+1)}{\Gamma(\alpha+k+1)} e^{-x} x^{\alpha+k}
			$$
			Simplificando obtenemos la expresi\'on (\ref{lagueralpha})
			$$
			L_n^{(\alpha)}(x)=\displaystyle\sum_{k=0}^n \displaystyle\frac{\Gamma(\alpha+n+1)(-x)^k}{k !(n-k) ! \Gamma(\alpha+k+1)}
			$$
			Con lo cual queda demostrado el teorema.
			Como caso particular para $\alpha=0$, tenemos
			$$
			L_n(x)=\displaystyle\sum_{k=0}^n \frac{\Gamma(n+1)(-x)^k}{k !(n-k) ! \Gamma(k+1)}
			$$
			
		\end{demo}
		
		\Example{Utiliza la f\'ormula de Rodrigues}{
			
			Utiliza la f\'ormula de Rodrigues dada en (\ref{laguerod}) para obtener $L_0^{(\alpha)}(x), L_1^{(\alpha)}(x)$ y $L_2^{(\alpha)}(x)$}\label{14}
		
		
		\begin{sol}
			tomando el valor de $n=0$, tenemos
			$$
			L_0^{(\alpha)}(x)=\displaystyle\frac{e^x x^{-\alpha}}{(0) !}\displaystyle \frac{d^{(0)}}{d x^{(0)}}\left(e^{-x} x^\alpha\right)
			$$
			El caso $n=0$ significa no derivar la funci\'on, as\'i
			$$
			L_0^{(\alpha)}(x)=e^x x^{-\alpha}\left(e^{-x} x^\alpha\right)
			$$
			Simplificando
			$$
			L_0^{(\alpha)}(x)=1
			$$
			Para el caso de $n=1$, tenemos
			$$
			L_1^{(\alpha)}(x)=\displaystyle\frac{e^x x^{-\alpha}}{1 !}\displaystyle \frac{d}{d x}\left(e^{-x} x^{1+\alpha}\right)
			$$
			Realizando la derivada de un producto, tenemos que
			$$
			L_1^{(\alpha)}(x)=e^x x^{-\alpha}\left((\alpha+1) e^x x^\alpha-e^{-\alpha} x^{\alpha+1}\right)
			$$
			Simplificando\\
			$L_1^{(\alpha)}(x)=\alpha+1-x$\\
			Para $n=2$
			$$
			\begin{gathered}
				L_2^{(\alpha)}(x)=\displaystyle\frac{e^x x^{-\alpha}}{2 !}\displaystyle \frac{d^2}{d x^2}\left(e^{-x} x^{2+\alpha}\right) \\
			\end{gathered}
			$$
			Realizando la segunda derivada
			$$
			L_2^{(\alpha)}(x)=\displaystyle\frac{e^x x^{-\alpha}}{2}\left[e^{-x} x^\alpha\left(\alpha^2+\alpha(3-2 x)+x^2-4 x+2\right)\right]
			$$
			Simplificando
			$$
			L_{2}^{(\alpha)}(x)=\displaystyle\frac{1}{2}\left[x^2-(4+2 \alpha) x+\left(\alpha^2+3 \alpha+2\right)\right]
			$$
		\end{sol}
		\subsection{Relaci\'on de recurrencia de los polinomios de Laguerre}
		\Theorem{Polinomios de Laguerre}{	Los polinomios de Laguerre $L_{n}^{(\alpha)}(x)$ satisfacen la siguiente relaci\'on a tres t\'erminos
			\begin{eqnarray}\label{laguerecurrencia}
				L_{n+1}^{(\alpha)}(x)&= &\displaystyle \frac{2 n+\alpha+1-x}{n+1} L_n^{(\alpha)}(x)-\displaystyle\frac{n+\alpha}{n+1} L_{n-1}^{(\alpha)}(x)
			\end{eqnarray}
			donde
			\begin{eqnarray*}
				L_0^{(\alpha)}(x)=1, \quad L_1^{(\alpha)}(x)=1+\alpha-x
		\end{eqnarray*}}
		
		
		\begin{demo}
			Derivando ambos lados la expresi\'on (\ref{laguergen}) con respecto a t, tenemos
			\begin{eqnarray*}
				-x(1-t)^{-1-\alpha}(1-t)^{-2} e^{-\displaystyle\frac{x t}{1-t}}+(\alpha+1)(1-t)^{-2-\alpha} e^{-\displaystyle\frac{x t}{1-t}}&=&\displaystyle\sum_{n=0}^{\infty}(n+1) L_{n+1}^{(\alpha)}(x) t^n
			\end{eqnarray*}
			Mutiplicando por $(1-t)^2$
			\begin{eqnarray*}
				-x(1-t)^{-1-\alpha} e^{\displaystyle\frac{-x t}{1-t}}+(\alpha+1)(1-t)(1-t)^{-1-\alpha} e^{-\displaystyle\frac{x t}{1-t}}=(1-t)^2 \displaystyle\sum_{n=0}^{\infty}(n+1)L_{n+1}(x)t^{n}
			\end{eqnarray*}
			Por la funci\'on generadora de los polinomios de Laguerre dado (\ref{laguergen}), tenemos que
			\begin{eqnarray*}
				-x\displaystyle \sum_{n=0}^{\infty} L_n^{(\alpha)}(x) t^n+(\alpha+1)(1-t)\displaystyle \sum_{n=0}^{\infty} L_n^{(\alpha)}(x) t^n=(1-t)^2 \displaystyle\sum_{n=0}^{\infty}(n+1) L_{n+1}^{(\alpha)}(x) t^n
			\end{eqnarray*}
			Desarrollando la segunda y tercera serie
			$$
			\begin{aligned}
				-x\displaystyle \sum_{n=0}^{\infty} L_n^{(\alpha)}(x) t^n & +(\alpha+1)\displaystyle \sum_{n=0}^{\infty} L_n^{(\alpha)}(x) t^n-(\alpha+1)\displaystyle \sum_{n=0}^{\infty} L_n^{(\alpha)}(x) t^{n+1}=\displaystyle\sum_{n=0}^{\infty}(n+1) L_{n+1}^{(\alpha)}(x)t^{n} \\
				& -2\displaystyle \sum_{n=0}^{\infty}(n+1) L_{n+1}^{(\alpha)}(x) t^{n+1}+\displaystyle\sum_{n=0}^{\infty}(n+1) L_{n+1}^{(\alpha)}(x) t^{n+2}
			\end{aligned}
			$$
			Agrupando terminos semejantes
			$$
			\begin{aligned}
				\displaystyle\sum_{n=0}^{\infty}\left[-x L_n^{(\alpha)}(x)+(\alpha+1) L_n^{(\alpha)}(x)-(n+1) L_{n+1}^{(\alpha)}(x)\right] t^n=\displaystyle\sum_{n=0}^{\infty}\left[(\alpha+1) L_n^{(\alpha)}(x)-2(n+1) L_{n+1}^{(\alpha)}(x)\right] t^{n+1} \\
				+\displaystyle\sum_{n=0}^{\infty}(n+1) L_{n+1}^{(\alpha)}(x) t^{n+2}
			\end{aligned}
			$$
			Desarrollando los dos primeros t\'erminos
			$$\begin{aligned}
				& \left((\alpha-x+1) L_0^{(\alpha)}(x)-L_1^{(\alpha)}(x)\right)+\left((\alpha-x+1) L_1^{(\alpha)}(x)-2 L_2^{(\alpha)}(x)\right) t \\
				& +\sum_{n=2}^{\infty}\left[-x L_n^{(\alpha)}(x)+(\alpha+1) L_n^{(\alpha)}(x)-(n+1) L_{n+1}^{(\alpha)}(x)\right] t^n=\left((\alpha+1) L_0^{(\alpha)}(x)-2 L_1^{(\alpha)}(x)\right)t \\
				& +\sum_{n=1}^{\infty}\left[(\alpha+1) L_n^{(\alpha)}(x)-2(n+1) L_{n+1}^{(\alpha)}(\alpha)\right] t^{n+1}+\sum_{n=2}^{\infty}(n-1) L_{n-1}^{(\alpha)}(x) t^n
			\end{aligned}$$
			
			\begin{fullwidth}[%
				width=\dimexpr\textwidth+\marginparsep+\marginparwidth,
				outermargin=\dimexpr-\marginparsep-\marginparwidth,
				]
				Por agrupaci\'on de t\'erminos
				$$
				\begin{aligned}
					& \left((\alpha-x+1) L_0^{(\alpha)}(x)-L_1^{(\alpha)}(x)\right)+\left((\alpha-x+3) L_1^{(\alpha)}(x)-2 L_2^{(\alpha)}(x)-(\alpha+1) L_0^{(\alpha)}(\alpha)\right)t \\
					& +\displaystyle\sum_{n=2}^{\infty}\left[-x L_n^{(\alpha)}(x)+(\alpha+1) L_n^{(\alpha)}(x)-(n+1) L_{n+1}^{(\alpha)}(x)-(n-1) L_{n-1}^{(\alpha)}(x)\right] t^n=
					\displaystyle\sum_{n=2}^{\infty}\left[(\alpha+1) L_{n-1}^{(\alpha)}(x)-2 n L_n^\alpha(x)\right] t^n
				\end{aligned}
				$$
			\end{fullwidth}
			
			
			Agrupando t\'erminos nueva vez
			$$
			\begin{aligned}
				& {\left[(\alpha-x+1) L_0^{(\alpha)}(x)-L_1^{(\alpha)}(x)\right]+\left[(\alpha-x+3) L_{1}^{(\alpha)}(x)-2 L_2^{(\alpha)}(x)-(\alpha+1) L_0^{(\alpha)}(x)\right] t} \\
				&+\displaystyle \sum_{n=2}^{\infty}\left[(2n+\alpha+1+-x) L_n^{(\alpha)}(x)-(n+1) L_{n+1}^{(\alpha)}(x)-\left(n+\alpha\right) L_{n-1}^{(\alpha)}(x)\right] t^n=0
			\end{aligned}
			$$
			Iqualando a cero los coeficientes
			$$
			\left\{\begin{array}{l}
				(\alpha-x+1) L_0^{(\alpha)}(x)-L_1^{(\alpha)}(x)=0 \\
				(\alpha-x+3) L_1^{(\alpha)}(x)-2 L_2^{(\alpha)}(x)-(\alpha+1) L_0^{(\alpha)}(x)=0 \\
				(2n+\alpha+1+-x) L_n^{(\alpha)}(x)-(n+1) L_{n+1}^{(\alpha)}(x)-(n+\alpha) L_{n-1}^{(\alpha)}(x)=0
			\end{array}\right.
			$$
			Despejando el polinomio de mayor grado, tenemos
			$$\begin{aligned}
				& L_1^{(\alpha)}(x)=(\alpha-x+1) L_0^{(\alpha)}(x) \\
				& L_2^{(\alpha)}(x)=\displaystyle\frac{1}{2}\left[(\alpha+1) L_0^{(\alpha)}(x)-(\alpha-x+3) L_1^{(\alpha)}(x)\right] \\
				& L_{n+1}^{(\alpha)}(x)=\displaystyle\frac{1}{n+1}\left[(2n+\alpha-x+1) L_n^{(\alpha)}(x)-(n+\alpha) L_{n-1}^{(\alpha)}(x)\right] \quad \forall n>1
			\end{aligned}
			$$
			Se puede comprobar que la expresi\'on $L_{n+1}^{(\alpha)}(x)$, contiene la expresi\'on $L_{2}^{(\alpha)}(x)$, por lo tanto tenemos que
			$$\begin{aligned}
				& L_{n+1}^{(\alpha)}(x)=\displaystyle\frac{1}{n+1}\left[(2n+\alpha-x+1) L_n^{(\alpha)}(x)-(n+\alpha) L_{n-1}^{(\alpha)}(x)\right] \quad \forall n\geq1
			\end{aligned}
			$$
			Lo cual queda demostrado el teorema.
		\end{demo}
		
		\Example{Obtenga el polinomio}{
			
			Obtenga el polinomio $L_3^{(\alpha)}(x)$ de Laguerre utilizando la expresion (\ref{laguerecurrencia} ) y los resultados obtenidos en el ejemplo (\ref{14} )}
		
		
		\begin{sol}
			para $n=2$, tenemos
			$$
			L_3^{(\alpha)}(x)=\frac{5+\alpha-x}{3} L_2^{(\alpha)}(x)-\frac{2+\alpha}{\alpha+1} L_1^{(\alpha)}(x)
			$$
			Reemplazando $L_2^{(\alpha)}(x)$ y $L_1^{(\alpha)}(x)$, tenemos
			$$
			\begin{aligned}
				L_3^{(\alpha)}(x)=\frac{5+\alpha-x}{3} & {\left[\frac{1}{2}\left(x^2-(4+2 \alpha) x+\left(\alpha^2+3 \alpha+2\right)\right]\right.} \\
				- & \frac{2+\alpha}{\alpha+1}[\alpha+1-x]
			\end{aligned}
			$$
			Realizando el producto
			$$
			\begin{aligned}
				L_3^{(\alpha)}(x)= & \frac{5+\alpha}{6} x^2-2(2+\alpha)\frac{(5+\alpha)}{6} x+\frac{(5+\alpha)}{6}\left(\alpha^2+3 \alpha+2\right) \\
				& -\frac{x^3}{6}-\frac{2(\alpha+2)}{6} x^2+\frac{\left(\alpha^2+3 \alpha+2\right)}{6} x \\
				& -(2+\alpha)+\frac{(2+\alpha)}{\alpha+1} x
			\end{aligned}
			$$
			Sumando los t\'erminos semejantes
			\begin{multline*}
				L_3^{(\alpha)}(x)=-\displaystyle\frac{x^3}{6}+\displaystyle\frac{1}{6}(5+\alpha-2 \alpha-4) x^2+\displaystyle\frac{1}{6}\left(-(4+2 \alpha)(5+\alpha)
				+\alpha^2+3 \alpha+2+\frac{12+6 \alpha}{\alpha+1}\right)x \\
				+\displaystyle\frac{1}{6}\left[(5+\alpha)\left(\alpha^2+3 \alpha+2\right)-(\alpha+2)\right]
			\end{multline*}
			
			\begin{fullwidth}[%
				width=\dimexpr\textwidth+\marginparsep+\marginparwidth,
				outermargin=\dimexpr-\marginparsep-\marginparwidth,
				]
				Simplificando
				\begin{eqnarray*}
					% \nonumber % Remove numbering (before each equation)
					L_3^{(\alpha)}(x)&=&-\displaystyle\frac{x^3}{6}+\frac{1}{6}(1-\alpha) x^2+\displaystyle\frac{1}{6}\left(\frac{-6-23 \alpha-12 \alpha^2-\alpha^3}{\alpha+1}\right) x+\displaystyle\frac{1}{6}\left(\alpha^{3}+8\alpha^{2}+16\alpha+8\right)
				\end{eqnarray*}
				
			\end{fullwidth}
			
			
		\end{sol}
		

		\subsection{Representaci\'on integral de los polinomios de Laguerre}
		\Theorem{Los polinomios de Laguerre}{	Los polinomios de Laguerre
			$$
			L_n(x)=\displaystyle\sum_{m=0}^n \frac{1}{m !}\left(\begin{array}{c}
				n \\
				m
			\end{array}\right)(-x)^m,
			$$
			satisfacen la expresi\'on:
			\begin{equation}\label{laguerreintegral}
				L_n(x)=\displaystyle\frac{1}{2 \pi i} \oint_C \frac{e^{\displaystyle\frac{-x z}{1-z}}}{(1-z) z^{n+1}} d z
			\end{equation}
			donde $C$ es un contorno simple que rodea el origen en direcci\'on contraria a las agujas del reloj sin incluir la singularidad en $z=1$. La expresi\'on (\ref{laguerreintegral}) se conoce como \textbf{forma integral de los polinomios de Laguerre en variable compleja}.}
		
		
		\begin{demo}
			Recordemos que la f\'ormula generatriz de los polinomios de Laguerre \ref{laguerregeneratriz} est\'a dada por:
			$$
			\frac{e^{\displaystyle\frac{-x z}{1-z}}}{(1-z)}=\sum_{n=0}^{\infty} L_n(x) z^n
			$$
			Ahora, calculando la derivada en\'esima tenemos:
			$$
			\displaystyle\frac{d^n}{d z^n}\left[\displaystyle\frac{e^{\displaystyle\frac{-x z}{1-z}}}{(1-z)}\right]=\displaystyle\frac{d^n}{d z^n}\left[\sum_{n=0}^{\infty} L_n(x) z^n\right]
			$$
			\begin{eqnarray*}
				\displaystyle\frac{d}{d z}\left[\frac{e^{\displaystyle\frac{-x z}{1-z}}}{(1-z)}\right]&=&\displaystyle\sum_{n=1}^{\infty} n L_n(x) z^{n-1}\\
				\frac{d^2}{d z^2}\left[\frac{e^{\displaystyle\frac{-z z}{1-z}}}{(1-z)}\right]&=&\displaystyle\sum_{n=2}^{\infty} n(n-1) L_n(x) z^{n-2}\\
				\displaystyle\frac{d^k}{d z^k}\left[\frac{e^{\displaystyle\frac{-x z}{1-z}}}{(1-z)}\right]&=&\displaystyle\sum_{n=k}^{\infty} \frac{n !}{(n-k) !} L_n(x) z^{n-k} \\
				\displaystyle \frac{d^k}{d z^k}\left[\frac{e^{\displaystyle\frac{-x z}{1-z}}}{(1-z)}\right]&=&\displaystyle\sum_{n=0}^{\infty} \frac{(n+k) !}{(n) !} L_{n+k}(x) z^n\\
				\displaystyle\frac{d^k}{d z^k}\left[\frac{e^{\displaystyle\frac{-x z}{1-z}}}{(1-z)}\right]_{z=0}&=&\left[L_k(x)+\sum_{n=1}^{\infty}\displaystyle \frac{(n+k) !}{(n) !} L_{n+k}(x) z^n\right]_{z=0}\\
				\displaystyle\frac{d^n}{d z^n}\left[\frac{e^{\displaystyle\frac{-x z}{1-z}}}{(1-z)}\right]_{z=0}&=&L_n(x)
			\end{eqnarray*}
			As\'i, por el teorema de la integral de Cauchy [ver teorema A.1] tenemos:
			$$
			\begin{aligned}
				&\frac{1}{2 \pi i} \oint_C \frac{f(z)}{\left(z-z_0\right)^{n+1}} d z=\displaystyle\frac{f^{(n)}\left(z_0\right)}{n !} \Rightarrow f^{(n)}\left(z_0\right)=\frac{n !}{2 \pi i} \oint_C \frac{f(z)}{\left(z-z_0\right)^{n+1}} d z \\
				&\Rightarrow \displaystyle\frac{d^n}{d z^n}\left[\frac{e^{\displaystyle\frac{-x z}{1-z}}}{(1-z)}\right]_{z=0}=f^{(n)}(0)=L_n(x) \Rightarrow L_n(x)=\displaystyle\frac{n !}{2 \pi i} \oint_C \frac{e^{\displaystyle\frac{-x z}{1-z}}}{(1-z) z^{n+1}} dz
			\end{aligned}
			$$
		\end{demo}
		
		\Example{Polinomios de Laguerre}{
			
			Utiliza la f\'ormula de la integral de los polinomios de Laguerre para calcular los polinomios $L_{1}(x)\,\text{y},L_{2}(x)$}
		
		\begin{sol}
			\begin{itemize}
				\item Para $n=1$
				\begin{eqnarray*}
					L_1(x)&=&\displaystyle\frac{1}{2 \pi i} \oint_c \frac{e^{\displaystyle\frac{-x z}{1-z}}}{(1-z) z^2} d z
				\end{eqnarray*}
				Como podemos ver en la expresi\'on anterior $z_0=0$ es un polo de orden 2, aplicando la expresi\'on (\ref{cauchyteo}) dada en el teorema (\ref{cauchy}) tenemos
				\begin{eqnarray*}
					L_1(x)&=&\operatorname{Res}(f(t), 0)
				\end{eqnarray*}
				Por la expresi\'on (\ref{poloN}) tenemos
				\begin{eqnarray*}
					L_1(x)&=&\lim _{z \rightarrow 0} \frac{d}{d z}\left[z^2 \frac{e^{\displaystyle\frac{-x z}{1-z}}}{(1-z) z^2}\right]
				\end{eqnarray*}
				Simplificando y derivando obtenemos
				\begin{eqnarray*}
					L_1(x)&=&\lim _{z \rightarrow 0} \frac{e^{\displaystyle\frac{x z}{z-1}}(x+z-1)}{(z-1)^3}
				\end{eqnarray*}
				Evaluando el limite
				\begin{eqnarray*}
					L_1(x)&=&1-x
				\end{eqnarray*}
				\item Para $n=2$
				\begin{eqnarray*}
					L_2(x)&=&\displaystyle\frac{1}{2 \pi i} \oint_c \frac{e^{\displaystyle\frac{-x z}{1-z}}}{(1-z) z^3} d z
				\end{eqnarray*}
				Podemos notar en la expresi\'on anterior que $z_0=0$ es un polo de orden 3, aplicando la expresi\'on (\ref{cauchyteo}) dada en el teorema (\ref{cauchy} )
				\begin{eqnarray*}
					L_2(x)&=&\operatorname{Res}(f(z), 0)
				\end{eqnarray*}
				Por la expresi\'on (\ref{poloN}) tenemos
				\begin{eqnarray*}
					L_2(x)&=&\frac{1}{2} \lim _{z \rightarrow 0} \frac{d^2}{d z^2}\left[z^3 \frac{e^{\displaystyle\frac{-x z}{1-z}}}{(1-z) z^3}\right]
				\end{eqnarray*}
				
				\begin{fullwidth}[%
					width=\dimexpr\textwidth+\marginparsep+\marginparwidth,
					outermargin=\dimexpr-\marginparsep-\marginparwidth,
					]
					Simplificando y calculando la derivadao
					\begin{eqnarray*}
						L_2(x)&=&\frac{1}{2} e^{-\displaystyle\frac{x z}{1-z}}\left[-\frac{2 x z}{(1-z)^2} - \frac{2 x}{1-z}+\left(-\frac{x z}{(1-z)^2}-\frac{x}{1-z}\right)^2-\frac{2 x z}{(1-z)^3}-\frac{2 x}{(1-z)^2}+\frac{2}{(1-z)^3}\right]
					\end{eqnarray*}
				\end{fullwidth}
				
				
				
				Evaluando el l\'imite, tenemos
				\begin{eqnarray*}
					L_2(x)&=&\displaystyle\frac{1}{2}\left[-2 x+x^2-2 x^2+2\right]
				\end{eqnarray*}
				Simplificando
				\begin{eqnarray*}
					L_2(x)&=&\frac{1}{2} x^2-2 x+1
				\end{eqnarray*}
			\end{itemize}
		\end{sol}
		
		
		A continuaci\'on se presenta una definici\'on para obtener los polinomios
		m\'onicos de Laguerre utilizando los operadores de subida y bajada, para profundizar m\'as en el tema [ver \cite{Piheira}].
		
		\Definition{Operador Identidad}{Sea $I$ el operador identidad. Definimos los operadores de subida y bajada de los polinomios m\'onicos de Laguerre como
			\begin{eqnarray}\label{lagbajada}
				% \nonumber % Remove numbering (before each equation)
				\tilde{\mathcal{L}}_n^{\downarrow}&:=&\displaystyle\frac{x}{\gamma_{n}} \frac{d}{d x}-\frac{n}{\gamma_{n}} I
			\end{eqnarray}
			\begin{eqnarray}\label{lagsubida}
				% \nonumber % Remove numbering (before each equation)
				\tilde{\mathcal{L}}_n^{\uparrow}&:=&-x \displaystyle\frac{d}{d x}+(x-n-\alpha) I
			\end{eqnarray}
			
			donde $\gamma_{n}=n(n+\alpha)$,
			ademas se tiene para todo valor de $n \geq 1$
			\begin{eqnarray}\label{lagobajada}
				\tilde{\mathcal{L}}_n^{\downarrow}[L^{\alpha}_{n}(x)]&:=& L_{n-1}^\alpha(x)
			\end{eqnarray}
			\begin{eqnarray}\label{lagosubida}
				\tilde{\mathcal{L}}_n^{\uparrow}[L^{\alpha}_{n-1}(x)]&:=& L_{n}^\alpha(x)
		\end{eqnarray}}
		
		
		Ahora vamos a obtener algunos de los polinomios de Laguerre $L_n(x)$ utilizando los operadores de subida y bajada.
		
		\Example{Polinomio Mónico}{	Utiliza el operador escala de subida para obtener los polinomios m\'onicos $L_{2}(x), L_{3}(x)$ y $L_{4}(x)$ de Laquerre.}
		
		\begin{sol}
			Sabemos que $L_0(x)=1$ y $L_1(x)=x-1$
			entonces tenemos
			\begin{eqnarray*}
				L_2(x)&=&\tilde{\mathcal{L}}_n^{\uparrow}\left[L_1(x)\right]
			\end{eqnarray*}
			Aplicando la expresi\'on (\ref{lagosubida})
			$$
			\begin{aligned}
				& L_2(x)=-x\displaystyle \frac{d}{d x}(x-1)+(x-2)(x-1)  \\
				& L_2(x)=-x(1)+x^2-3 x+2 \\
				& L_2(x)=x^2-4 x+2
			\end{aligned}$$
			
			\begin{eqnarray*}
				L_3(x)&=&\tilde{\mathcal{L}}_n^{\uparrow}\left[L_2(x)\right]
			\end{eqnarray*}
			
			Aplicando el operador de subida
			$$
			\begin{aligned}
				& L_3(x)=-x\displaystyle \frac{d}{d x}\left(x^2-4 x+2\right)+(x-3)\left(x^2-4 x+2\right) \\
				& L_3(x)=-x(2 x-4)+x^3-7 x^2+14 x-6\\
				& L_3(x)=-2 x^2+4 x+x^3-7 x^2+14 x-6 \\
				& L_3(x)=x^3-9 x^2+18 x-6
			\end{aligned}
			$$
			\begin{eqnarray*}
				L_4(x)&=&\tilde{\mathcal{L}}_n^{\uparrow}\left[L_3(x)\right]
			\end{eqnarray*}
			$$
			\begin{aligned}
				& L_4(x)=-x \frac{d}{d x}\left[x^3-9 x^2+18 x-6\right]+(x-4)\left(x^3-9 x^2+18 x-6\right) \\
				& L_4(x)=-x\left(3 x^2-18 x+18\right)+x^4-13 x^3+54 x^2-78 x+24 \\
				& L_4(x)=-3 x^3+18 x^2-18 x+x^4-13 x^3+54 x^2-78 x+24\\
				&L_4(x)=x^4-16 x^3+72 x^2-96 x+24
			\end{aligned}
			$$
		\end{sol}
		
		\textcolor{blue}{orden de los polinomios de este capitulo: Chevyshhev y tipos. Laguerre y asociados. Jacobi y los ultraesfericos. laguerre y hermite. bernoulli. Airy. Hipergeometrica.}\\
		\textcolor{red}{dar ejemplos de aplicaciones de poliomios de Laguerre con referencia actualizada, mecanica cuantica, etc...}
		\textcolor{red}{Antes de los polinomios de Jacobi deben estar legendre, chevyshev y tipos y todo los casos que se obtienen a partir de los Jacobi.(trabajar los polinomios de intervalo acotado seguido, luego los no acotados.) En el caso legendre trabajar los legendre asociados y la generalizacion de fejer (ver sego)}\\
		\textcolor{purple}{seguir el esquema de Laguerre en los demas polinomios}
\subsection{Ecuaci\'on asociada de los polinomios de Laguerre}
Trabajar esta parte
		\section{Ecuaci\'on de Jacobi. Polinomios de Jacobi.}
		Ecuaci\'on de Jacobi:
		\begin{eqnarray}\label{Jacobiequation}
			% \nonumber % Remove numbering (before each equation)
			(1-x^2)y^{\prime \prime}(x)+[\beta-\alpha-(\alpha+\beta+2)x]y^{\prime}(x)+\gamma y(x)&=&0
		\end{eqnarray}
		con $\gamma=n(n+\alpha+\beta+1)$, de la forma normal se determina
		\begin{eqnarray*}
			p_{1}(x)&=&\displaystyle\frac{\beta-\alpha-(\alpha+\beta+2) x}{1-x^2} \\
			p_{2}(x)&=&\displaystyle\frac{\gamma}{1-x^2}
		\end{eqnarray*} por lo tanto $x_{0}=\pm 1$, entonces \ref{Jacobiequation} tiene una soluci\'on de la forma dada en \ref{solucionfrobenius}.\\
		Obteniendo las derivadas de \ref{solucionfrobenius}, tenemos
		$$\begin{gathered}
			y=\sum_{n=0}^{\infty} c_n(x-1)^{n+r} ; \quad y^{\prime}=\sum_{n=0}^{\infty} c_n(n+r)(x-1)^{n+r-1} \\
			y^{\prime \prime}=\sum_{n=0}^{\infty} c_n(n+r)(n+r-1)(x-1)^{n+r-2}
		\end{gathered}$$
		Reemplazando en cada t\'ermino se tiene
		$$\begin{aligned}
			&\left(1-x^2\right) y^{\prime \prime}=-(x+1)(x-1) \sum_{n=0}^{\infty} c_n(n+r)(n+r-1)(x-1)^{n+r-2} \\
			&\left(1-x^2\right) y^{\prime \prime}=-(x+1) \sum_{n=0}^{\infty} c_n(n+r)(n+r-1)(x-1)^{n+r-1}\\
			&{[\beta-\alpha-(\alpha+\beta+2) x] y^{\prime}=-[2(\alpha+1)+(\alpha+\beta+2)(x-1)] \sum_{n=0}^{\infty} c_n(n+r)(x-1)^{n+r-1}} \\
			&\quad=-2(\alpha+1) \sum_{n=0}^{\infty}c_n(n+r)(x-1)^{n+r-1}-(\alpha+\beta+2) \sum_{n=0}^{\infty} c_n(n+r)(x-1)^{n+r}
		\end{aligned}$$
		Sustituyendo los resultados anteriores en \ref{Jacobiequation}, tenemos
		$$\begin{aligned}
			-(x+1) & \sum_{n=0}^{\infty} c_n(n+r)(n+r-1)(x-1)^{n+r-1}-2(\alpha+1) \sum_{n=0}^{\infty} c_n(n+r)(x-1)^{n+r-1} \\
			&-(\alpha+\beta+2) \sum_{n=0}^{\infty} c_n(n+r)(x-1)^{n+r}+\gamma \sum_{n=0}^{\infty} c_n(x-1)^{n+r}=0
		\end{aligned}$$
		
		\begin{fullwidth}[%
			width=\dimexpr\textwidth+\marginparsep+\marginparwidth,
			outermargin=\dimexpr-\marginparsep-\marginparwidth,
			]
			Desarrollando el primer t\'ermino de la segunda y haciendo un corrimiendo
			$$\begin{aligned}
				-(x+1) \sum_{n=0}^{\infty} c_n(n+r)(n+r-1)(x-1)^{n+r-1}-2(\alpha+1) r c_0(x-1)^{r-1}-2(\alpha+1) \sum_{n=0}^{\infty} c_{n+1}(n+x+1)(x-1)^{n+r}\\
				-(\alpha+\beta+2) \sum_{n=0}^{\infty} c_n(n+r)(x-1)^{n+r}+y \sum_{n=0}^{\infty} c_n(x-1)^{n+r}=0
			\end{aligned}$$
			
			Llamando $ B=\displaystyle\sum_{n=0}^{\infty} c_{n}(n+r)(n+r-1)(x-1)^{n+r-1}$ y
			agrupando las series semejantes
			\begin{eqnarray*}
				-2(\alpha+1) r c_0(x-1)^{r-1}-(x+1) B+\sum_{n=0}^{\infty}\left[(\gamma-(\alpha+\beta+2)(n+r)) c_n-2(\alpha+1)(n+r+1) c_{n+1}\right](x-1)^{n+r}&=&0
			\end{eqnarray*}
			Acomodando
			\begin{eqnarray*}
				-(X+1) B&=&-X B+B-B-B\\
				-(X+1) B&=&-(x-1) B-2 B
			\end{eqnarray*}
			As\'i
			\begin{eqnarray*}
				-(X+1) B&=&-\sum_{n=0}^{\infty} c_n(n+r)(n+r-1)(x-1)^{n+r}-2 \sum_{n=0}^{\infty} c_n(n+r)(n+r-1)(x-1)^{n+r-1}
			\end{eqnarray*}
			Tomando el primer t\'ermino de la segunda serie
			\begin{eqnarray*}
				-(X+1) B&=&-2 c_0 r(r-1)(x-1)^{r-1}-\sum_{n=0}^{\infty} c_n(n+r)(n+r-1)(x-1)^{n+r}-2 \sum_{n=0}^{\infty} c_{n+1}(n+r+1)(n+r)(x-1)^{n+r}
			\end{eqnarray*}
		\end{fullwidth}

		Organizando todo los t\'erminos
		\begin{multline*}
			2(x-1)^{r-1}rc_{0}\left(-r-\alpha\right)+\sum_{n=0}^{\infty}[(\gamma-(\alpha+\beta+2)(n+r)-(n+r)(n+r-1))c_n \\
			-[2(\alpha+1)(n+r+1)+2(n+r)(n+r+1)]c_{n+1}](x-1)^{n+r}=0
		\end{multline*}
		Simplificando
		\begin{multline*}
			2(x-1)^{r-1}rc_{0}\left(-r-\alpha\right)+\sum_{n=0}^{\infty}[(\gamma-(n+r)(\alpha+\beta+n+r+1))c_n \\
			-2(n+r+1)(n+r+\alpha+1)c_{n+1}](x-1)^{n+r}=0
		\end{multline*}
		Iqualando los coeficients a cero.
		$$
		\left\{\begin{array}{l}
			2(x-1)^{r-1} rc_{0} (-r-\alpha)=0 \Rightarrow r=0 \wedge r=-\alpha \\
			{[\gamma-(n+r)(\alpha+\beta+n+r+1)] c_n-2(n+r+1)(n+r+\alpha+1) c_{n+1}=0 \quad \forall n \geq 0}
		\end{array}\right.
		$$
		Despejando a $c_{n+1}$
		\begin{eqnarray*}
			% \nonumber % Remove numbering (before each equation)
			c_{n+1}&=&\frac{\gamma-(n+r)(\alpha+\beta+n+r+1)}{2(n+r+1)(n+r+\alpha+1)} c_n \quad n\geq0
		\end{eqnarray*}
		Tomando el caso de $r=0$, tenemos
		$$\begin{aligned}
			& c_{n+1}=\frac{\gamma-n(\alpha+\beta+n+1)}{2(n+1)(n+\alpha+1)} c_n \\
			&c_{n+1}=\frac{\gamma_v-\gamma_n}{2(n+1)(n+\alpha+1)} c_n \quad \forall n\geq0
		\end{aligned}$$
		Desarrollando los t\'erminos
		$$
		\begin{aligned}
			n=0 \longrightarrow c_1 &=\frac{\gamma_v-\gamma_0}{2(1)(\alpha+1)} c_0 \\
			n=1 \longrightarrow c_2 &=\frac{\gamma_v-\gamma_1}{2(2)(\alpha+2)} \frac{\gamma_v-\gamma_0}{2(1)(\alpha+1)} c_0 \\
			& \vdots \\
			c_2 &=\frac{\left(\gamma_v-\gamma_0\right)\left(\gamma_v-\gamma_1\right)}{2^2 2 !(\alpha+1)(\alpha+2)} c_0 \\
			c_n &=\frac{\left(\gamma_v-\gamma_0\right)\left(\gamma_v-\gamma_1\right) \cdots\left(\gamma_v-\gamma_{n-1}\right)}{2^n n !(\alpha+1)(\alpha+2) \cdots(\alpha+n-1)(\alpha+n)} c_0
		\end{aligned}
		$$
		
		Simplificando las expresiones del numerador y denominador
		$$\begin{aligned}
			\gamma_v-\gamma_{n-1} &=v(\alpha+\beta+v+1)-(n-1)(\alpha+\beta+n) \\
			&=v(\alpha+\beta)+v(v+1)-(n-1)(\alpha+\beta)-n(n-1) \\
			&=(\alpha+\beta)(v-n+1)+v^2+v-n^2+n \\
			&=(\alpha+\beta)(v-n+1)+(v-n)(v+n)+(v+n) \\
			&=(\alpha+\beta)(v-n+1)+(v+n)(v-n+1) \\
			\gamma_v-\gamma_{n-1}&=(v-n+1)(\alpha+\beta+v+n) \\
			&(\alpha+1)(\alpha+2) \cdots(\alpha+n-1)(\alpha+n)=(\alpha+1)_{n}=\frac{\Gamma(\alpha+n+1)}{\Gamma(\alpha+1)} \\
		\end{aligned}$$
		\textcolor{red}{ver video en youtube}\\
		
		\vspace{1cm}
		
		\begin{fullwidth}[%
			width=16cm,
			outermargin=\dimexpr-\marginparwidth,
			]
			\subsection{Funci\'on Generadora de los polinomios de Jacobi}
			\Theorem{La funci\'on generadora de los polinomios de Jacobi}{La funci\'on generadora de los polinomios de jacobi $P_{n}^{(\alpha,\beta)}(x)$ est\'a dada por:
				\begin{eqnarray}\label{3}
					% \nonumber to remove numbering (before each equation)
					F(x,r) &=& 2^{\alpha + \beta}R^{-1}(1-r+R)^{-\alpha}(1+r+R)^{-\beta};R=(1-2xr+r^{2})^{1/2}
			\end{eqnarray}}
		\end{fullwidth}
		
	
			\begin{demo}
			Sea $\phi(y)=\frac{y^{2}-1}{2}$ en la expresi\'on \ref{1}, entonces
			$$r=(y-x)\frac{2}{y^{2}-1}=\frac{2y-2x}{y^{2}-1}$$
			$$ry^{2}-r=2y-2x$$
			$$ry^{2}-2y=r-2x$$ completando cuadrado
			$$(y-\frac{1}{r})^{2}=\frac{1-2xr+r^{2}}{r^{2}}$$
			$$y-\frac{1}{r}=-\frac{\left(1-2xr+r^{2}\right)^{1/2}}{r}$$
			$$y=\frac{1}{r}-\frac{\left(1-2xr+r^{2}\right)^{1/2}}{r}$$
			$$y=\frac{1}{r}-\frac{R}{r}$$
			Derivando la expresi\'on \ref{2} con respecto a la variable $x$, obtenemos $$f^{\prime}(y)\frac{dy}{dx}=f^{\prime}(x)+\sum_{n=1}^{\infty}\frac{r^{n}}{n!}\frac{d^{n}}{dx^{n}}\left\{f^{\prime}(x)\frac{(x^{2}-1)^{n}}{2^{n}}\right\}$$
			tomando $f^{\prime}(x)=(1-x)^{\alpha}(1+x)^{\beta}$ y sabemos que $\displaystyle\frac{dy}{dx}=\frac{1}{R}$ la expresi\'on anterior se convierte en \\ $$\frac{(1-y)^{\alpha}(1+y)^{\beta}}{R}=(1-x)^{\alpha}(1+x)^{\beta}+\sum_{n=1}^{\infty}\frac{r^{n}}{n!}\frac{d^{n}}{dx^{n}}\left\{(1-x)^{\alpha}(1+x)^{\beta}\frac{(x^{2}-1)^{n}}{2^{n}}\right\}$$
			$$\frac{(1-y)^{\alpha}(1+y)^{\beta}}{R}=(1-x)^{\alpha}(1+x)^{\beta}+\sum_{n=1}^{\infty}\frac{(-1)^{n}r^{n}}{2^{n}n!}\frac{d^{n}}{dx^{n}}\left\{(1-x)^{n+\alpha}(1+x)^{n+\beta}\right\}$$
			Aplicando la f\'ormula de rodrigues para los polinomios de jacobi, tenemos $$\frac{(1-y)^{\alpha}(1+y)^{\beta}}{R}=(1-x)^{\alpha}(1+x)^{\beta}+\sum_{n=1}^{\infty}r^{n}(1-x)^{\alpha}(1+x)^{\beta}P_{n}^{(\alpha,\beta)}(x)$$
			$$(\frac{1-y}{1-x})^{\alpha}(\frac{1+y}{1+x})^{\beta}=1+\sum_{n=1}^{\infty}r^{n}P_{n}^{(\alpha,\beta)}(x)$$
			Expresando $\displaystyle\frac{1-y}{1-x}$,$\displaystyle\frac{1+y}{1+x}$ en t\'ermino de $r$ y $s$, tenemos $$1-y=1-(\frac{1}{r}-\frac{R}{r})$$
			$$1-y=\frac{r-1+R}{r}$$
			Sabemos que $R^{2}=1-2xr+r^{2}$, despejando la variable $x$, tenemos $$x=\frac{1+r^{2}-R^{2}}{2r}$$
			De donde $$1-x=\frac{(r-1+R)(r-1-R)}{2r}$$
			y
			\begin{align*}
				\frac{1+y}{1+x} &= \frac{r - R + 1}{r} \cdot \frac{2r}{(r - R + 1)(r + R + 1)} = \frac{2}{r + R + 1} \\
				\intertext{Entonces tenemos}
				2^{\alpha + \beta} R^{-1} (1 - r + R)^{-\alpha} (1 + r + R)^{-\beta}
				&= 1 + \sum_{n=1}^{\infty} r^{n} P_n^{(\alpha, \beta)}(x) \\
				\intertext{Finalmente}
				2^{\alpha + \beta} R^{-1} (1 - r + R)^{-\alpha} (1 + r + R)^{-\beta}
				&= \sum_{n=0}^{\infty} r^{n} P_n^{(\alpha, \beta)}(x)
			\end{align*}
			
			Lo que demuestra el teorema
		\end{demo}
		
		\subsection{Operadores Diferenciales de Jacobi}
		\begin{fullwidth}[%
			width=\dimexpr\textwidth+\marginparsep+\marginparwidth,
			outermargin=\dimexpr-\marginparwidth,
			]
			\Definition{Operadores Diferenciales de Jacobi}{Sea $\mathfrak{I}$ el operador identidad. Definimos los dos operadores diferenciables de Jacobi como
				
				\begin{eqnarray}\label{jacobajada}
					\widehat{\mathfrak{L}}_n^{\downarrow}&:=&-\displaystyle\frac{\widehat{a}_n(x)}{\widehat{b}_n} \mathfrak{I}+\frac{1-x^2}{\widehat{b}_n} \frac{d}{d x} \quad \text { (operador diferencial de bajada de Jacobi), }
				\end{eqnarray}
				\begin{eqnarray}\label{jacosubida}
					\widehat{\mathfrak{L}}_n^{\uparrow}&:=&-\displaystyle\frac{\widehat{c}_n(x)}{\widehat{d}_n} \mathfrak{I}+\frac{1-x^2}{\widehat{d}_n} \frac{d}{d x} \quad \text { (operador diferencial de subida de Jacobi) }
				\end{eqnarray}
				donde
				\begin{eqnarray}\label{anx}
					\widehat{a}_n(x)&=&-\displaystyle\frac{n((2 n+\alpha+\beta) x+\beta-\alpha)}{2 n+\alpha+\beta}
				\end{eqnarray}
				\begin{eqnarray}\label{bnx}
					\widehat{b}_n&=&\frac{4 n(n+\alpha)(n+\beta)(n+\alpha+\beta)}{(2 n+\alpha+\beta)^2(2 n+\alpha+\beta-1)}
				\end{eqnarray}
				\begin{eqnarray}\label{cnx}
					\widehat{c}_n(x)&=&\displaystyle\frac{(n+\alpha+\beta)((2 n+\alpha+\beta) x+\alpha-\beta)}{2 n+\alpha+\beta}
				\end{eqnarray}
				y
				\begin{eqnarray}\label{dn}
					\widehat{d}_n&=&-(2 n+\alpha+\beta-1)
				\end{eqnarray}
				Si $n\geq 1$ se tiene que para toda secuencia $\left\{P_n^{\alpha, \beta}\right\}_{n \geqslant 0}$ se cumple
				\begin{eqnarray}\label{jacopbajada}
					\widehat{\mathfrak{L}}_n^{\downarrow}\left[P_n^{\alpha, \beta}(x)\right] & =-\displaystyle\frac{\widehat{a}_n(x)}{\widehat{b}_n} P_n^{\alpha, \beta}(x)+\displaystyle\frac{1-x^2}{\widehat{b}_n}\left(P_n^{\alpha, \beta}(x)\right)^{\prime}=P_{n-1}^{\alpha, \beta}(x)
				\end{eqnarray}
				\begin{eqnarray}\label{jacopsubida}
					\widehat{\mathfrak{L}}_n^{\uparrow}\left[P_{n-1}^{\alpha, \beta}(x)\right] & =-\displaystyle\frac{\widehat{c}_n(x)}{\widehat{d}_n} P_{n-1}^{\alpha, \beta}(x)+\displaystyle\frac{1-x^2}{\widehat{d}_n}\left(P_{n-1}^{\alpha, \beta}(x)\right)^{\prime}=P_n^{\alpha, \beta}(x)
			\end{eqnarray}}
		\end{fullwidth}
		
		\Example{Obtener los polinomios de Legendre}{Utiliza el operador diferencial de subida de Jacobi para obtener los polinomios de Legendre $P_1(x), P_2(x)$ y $P_3(x)$.}
		
		\begin{sol}
			Como los polinomios de Legendre se obtienen como caso particular de los de Jacobi para el caso $\alpha=\beta=0$, tenemos que:
			$$
			\begin{aligned}
				P_1(x)=P_{1}^{(0,0)}(x) & =\displaystyle\widehat{\mathfrak{L}}_n^{\uparrow}\left[P_0^{(0,0)}(x)\right] \\
				& =\displaystyle\left(-\frac{\hat{c}_1(x)}{\hat{d}_1} \mathfrak{I}+\frac{1-x^2}{\hat{d}_1} \frac{d}{d x}\right)\left(P_0^{(0,0)}(x)\right) \\
				& =-\displaystyle\frac{\hat{c}_1(x)}{\hat{d}_1} \mathfrak{I}\left(P_0^{(0,0)}(x)\right)+\frac{1-x^2}{\hat{d}_1} \frac{d}{d x}\left(P_0^{(0,0)}(x)\right)
			\end{aligned}
			$$
			Reemplazando $P_0^{(0,0)}(x)$
			$$
			\begin{aligned}
				& P_1(x)=P_1^{(0,0)}(x)=\displaystyle-\frac{\hat{c}_1(x)}{d_1}(1)+\frac{1-x^2}{d_n} \frac{d}{d x}(1)
			\end{aligned}
			$$
			Aplicando el operador derivada
			$$
			\begin{aligned}
				& =\displaystyle\frac{\hat{c}_1(x)}{\hat{d}_1} \\
			\end{aligned}
			$$
			obteniendo el valor de $\hat{c}_1(x)$ y $\hat{d}_1$, utilizando las expresiones (\ref{cnx}), (\ref{dn}) respectivamente
			$$
			\begin{aligned}
				P_1(x)=P_1^{(0,0)}(x) & =-\displaystyle\frac{ x}{-1} \\
				P_1(x)=P_1^{(0,0)}(x) & =x
			\end{aligned}
			$$
			Para $n=2$
			$$
			\begin{aligned}
				P_2(x)=P_2^{(0,0)}(x) & =\displaystyle\widehat{\mathfrak{L}}_n^{\uparrow}\left[P_1^{(0,0)}(x)\right] \\
				&=\left(-\displaystyle\frac{\hat{c}_n(x)}{\hat{d}_n} \mathfrak{I}+\frac{1-x^2}{\hat{d}_n} \frac{d}{d x}\right)\left(P_1^{(0,0)}(x)\right) \\
				& =-\displaystyle\frac{\hat{c}_n(x)}{\hat{d}_n} \mathfrak{I}\left(P_1^{(0,0)}(x)\right)+\frac{1-x^2}{\hat{d}_n} \frac{d}{d x}\left(P_1^{(0,0)}(x)\right)
			\end{aligned}
			$$
			Reemplazando $P_1^{(0,0)}(x)$ y aplicando la definici\'on de los operadores
			$$
			P_2(x)=P_2^{(0,0)}(x)=-\displaystyle\frac{\hat{c}_n(x)}{d_n}(x)+\frac{1-x^2}{\hat{d}_n}(1)
			$$
			Utilizando las expresiones (\ref{cnx})y (\ref{dn}) para obtener
			$\hat{c}_2(x)$ y $\hat{d}_2$, tenemos
			$$
			P_2(x)=P_2^{(0,0)}(x)=-\frac{2 x}{-3}(x)+\frac{1-x^2}{-3}
			$$
			Simplificando
			$$
			P_2(x)=P_2^{(0,0)}(x)=x^2-\frac{1}{3}
			$$
			Para $n=3$
			$$
			\begin{aligned}
				P_3(x)=P_3^{(0,0)}(x) & =\displaystyle\widehat{\mathfrak{L}}_n^{\uparrow}\left[P_2^{(0,0)}(x)\right] \\
				& =\left(-\frac{\hat{c}_3(x)}{\hat{d}_3} \mathfrak{I}+\frac{1-x^2}{\hat{d}_3} \frac{d}{d x}\right)\left(P_2^{(0,0)}(x)\right) \\
				& =-\frac{\hat{c}_2(x)}{\hat{d}_3} \mathfrak{I}\left(P_2^{(0,0)}(x)\right)+\frac{1-x^2}{\hat{d}_3} \frac{d}{d x}\left(P_2^{(0,0)}(x)\right)
			\end{aligned}
			$$
			Reemplazando $P_2^{(0,0)}(x)$ y aplicando los operadores
			$$
			P_3(x)=P_3^{(0,0)}(x)=-\frac{\hat{c}_3(x)}{\hat{d}_3}\left(x^2-\frac{1}{3}\right)+\frac{1-x^2}{\hat{d}_3}(2 x)
			$$
			Por las expresiones (\ref{cnx})y (\ref{dn}) tenemos
			$$
			P_2(x)=P_2^{(0,0)}(x)=-\dfrac{3 x}{-5}\left(x^2-\dfrac{1}{3}\right)+\dfrac{1-x^2}{-5}(2 x)
			$$
			Simplificando
			$$
			P_2(x)=P_2^{(0,0)}(x)=x^3-\dfrac{3}{5} x
			$$
		\end{sol}
		
		\subsection{Polinomios de Hermite como caso particular de los polinomios de Jacobi}
		En este apartado se obtendran los polinomios m\'onicos $H_{0}(x), H_{1}(x), . .$ de Hermite como particular de los polinomios de Jacobi.\\
		
		\Example{Polinomios m\'onico de Hermite}{
			
			Utiliza la definici\'on dada en (\ref{defijacobi}) para calcular los $3$ primeros polinomios m\'onico de Hermite}
		
		\begin{sol}
			Sabemos de \cite{Szego1939} los polinomios m\'onicos de Hermite se obtienen a partir de:
			\begin{align}\label{hermitemonico}
				H_{n}(x)=\displaystyle\lim _{\alpha \rightarrow \infty} \frac{n !}{\alpha^{\frac{n}{2}}} \operatorname{P_{n}}^{(\alpha, \alpha)}\left(\alpha^{-\frac{1}{2}} x\right)
			\end{align}
			As\'i:
			\begin{align*}
				H_{0}(x)=\lim _{\alpha \rightarrow \infty} \frac{0!}{\alpha^{\frac{0}{2}}} P_{0}^{(\alpha, \alpha)}\left(\alpha^{-\frac{1}{2}} x\right)
			\end{align*}
			Simplificando y por la expresi\'on \ref{p0(x)} obtenemos
			\begin{align*}
				% \nonumber % Remove numbering (before each equation)
				H_{0}(x) &= 1
			\end{align*}
			De igual manera
			\begin{align*}
				H_{1}(x)&=\lim _{\alpha \rightarrow \infty} \frac{1 !}{\alpha^{\frac{1}{2}}} P_{1}^{\left(\alpha_{1} \alpha\right)}\left(\alpha^{-\frac{1}{2}} x\right)\\
				\intertext{De la expresi\'on \ref{p1(x)} tenemos}\\
				H_{1}(x)&=\lim _{\alpha \rightarrow \infty} \frac{1}{\alpha^{\frac{1}{2}}}\times \frac{1}{2}\left[(2 \alpha+2) \frac{x}{\alpha^{1/2}}\right]\\
				\text{Simplificando y tomando l\'imite se obtiene}\\
				H_{1}(x)&=x
			\end{align*}
			De manera similar
			\begin{align*}
				H_{2}(x)&=\lim _{\alpha \rightarrow \infty} \frac{2 !}{\alpha^{\frac{2}{2}}} P_{2}^{(\alpha, \alpha)}\left(\alpha^{-\frac{1}{2}} x\right)
				\intertext{De la expresi\'on \ref{p2(x)}, tenemos}
				H_{2}(x)&=\lim _{\alpha \rightarrow \infty} \frac{2}{\alpha}\left[\frac{1}{8}\left\{-4-2 \alpha+(2 \alpha+4)(2 \alpha+3) \frac{x^{2}}{\alpha}\right\}\right]
				\intertext{Desarrollando y simplificando}
				H_{2}(x)&=\lim _{\alpha \rightarrow \infty}\left\{-\frac{1}{\alpha}-\frac{1}{2}+\frac{4 \alpha^{2}+14 x+12}{4 \alpha^{2}} x^{2}\right\}
				\intertext{Tomando l\'imite}
				H_{2}(x)&=x^{2}-\frac{1}{2}
			\end{align*}
		\end{sol}
		
		\subsection{Polinomios de Laguerre como caso particular de los polinomios de Jacobi}
		De \cite[4.2.18]{Beals2010} presenta una expresi\'on para obtener los polinomios de laguerre no m\'onico a partir de los de Jacobi, en este apartado obtendremos los mismos pero m\'onico.\\
		La expresi\'on para obtener los polinomios m\'onicos de Laguerre est\'a dada por:
		\begin{eqnarray}\label{polimonicolaguerre}
			% \nonumber % Remove numbering (before each equation)
			L_{n}^{(\alpha)}(x) &=& \lim _{\beta \rightarrow \infty}(-n)_{n} P_{n}^{\left(\alpha, \beta\right)}\left(1-2 \beta^{-1} x\right)
		\end{eqnarray}
		As\'i, para el caso de $n=0$, tenemos
		\begin{align*}
			L_{0}^{(\alpha)}(x)&=\lim _{\beta \rightarrow \infty}(0)_{0} P_{0}^{\left(\alpha, \beta\right)}\left(1-2 \beta^{-1} x\right)
		\end{align*}
		De \ref{p0(x)}
		\begin{align*}
			L_{0}^{(\alpha)}(x)&=1
		\end{align*}
		De igual manera se obtiene el polinomio  lineal de Laguerre
		\begin{align*}
			L_{1}^{(\alpha)}(x) &=\displaystyle\lim _{\beta \rightarrow \infty}(-1)_{1} P_{1}^{\left(\alpha_, \beta\right)}\left(1-2 \beta^{-1} x\right) \\ &=\lim _{\beta \rightarrow \infty}-\frac{1}{2}\left[\alpha-\beta+(\alpha+\beta+2)\left(1-2 \beta^{-1} x\right)\right] \\
			& \intertext{Simplicando}\\
			L_{1}^{(\alpha)}(x) &=\displaystyle\lim _{\beta \rightarrow \infty}-\frac{1}{2}\left[2\alpha+2+\left(-2 \alpha \beta^{-1}-2-4 \beta^{-1}\right) x\right]\\
			\intertext{Tomando l\'imite y simplificando la expresi\'on }\\
			L_{1}^{(\alpha)}(x)&=x-\alpha-1
		\end{align*}
		Si queremos el polinomio cuadr\'atico , tenemos que
		\begin{align*}
			L_{2}^{(\alpha)}(x)&=\lim _{\beta \rightarrow \infty}(-2)_{2}\, P_{2}^{\left(\alpha, \beta\right)}\left(1-2 {\beta}^{-1} x\right)\\
			L_{2}^{(\alpha)}(x) &=\lim _{\beta \rightarrow \infty} \frac{(-2)_{2}}{8}\left\{-4-(\alpha+\beta)+(\alpha-\beta)^{2}+2(\alpha-\beta)(\alpha+\beta+3)\left(1-2 {\beta}^{-1} x\right)\right\}\\
			&\quad +\frac{(-2)_{2}}{8}(\alpha+\beta+4)(\alpha+\beta+3)\left(1-2 \beta^{-1} x\right)^{2}\\
			L_{2}^{(\alpha)}(x)&=\lim _{\beta \rightarrow \infty} \frac{(-2)_{2}}{8}\left\{4(\alpha+1)(\alpha+2)-8 \beta^{-1}(\alpha+\beta+3)(\alpha+2) x+4 {\beta}^{-2}(\alpha+\beta+4)(\alpha+\beta+3) x^{2}\right\}\\
			\intertext{Tomando l\'imites y simplicando}
			L_{2}^{(\alpha)}(x)&=x^{2}-2(\alpha+2) x+(\alpha+1)(\alpha+2)
		\end{align*}
		\textcolor[rgb]{1.00,0.00,0.00}{Aplicacion en cuantica en thesis de geronimo y en la thesis de rodolgo la transformacion de jacobi en la ecuacion de calor}
		\subsection{Ecuaci\'on asociada de los polinomios de Jacobi}
Trabajar esta parte
		\section{Ecuaci\'on de Chebyshev. Polinomios de Chebyshev}
		\begin{eqnarray}\label{chebyshev equation}
			% \nonumber % Remove numbering (before each equation)
			\left(1-x^2\right) y^{\prime \prime}-x y^{\prime}+a^2 y&=&0
		\end{eqnarray}
		Expresando la ecuaci\'on \ref{chebyshev equation} en su forma normal
		\begin{eqnarray*}
			% \nonumber % Remove numbering (before each equation)
			y^{\prime \prime}+\frac{-x}{\left(1-x^2\right)} y^{\prime}+\frac{a^2}{1-x^2} y&=&0
		\end{eqnarray*}
		verifamos que $x_{0}=0$ es un punto ordinario, por el teorema tal.... existe una soluci\'on en serie de potencias.\\
		Derivando la soluci\'on en serie y reemplazandola en \ref{chebyshev equation} tenemos la expresi\'on
		\begin{eqnarray*}
			% \nonumber % Remove numbering (before each equation)
			\sum_{m=2}^{\infty}m(m-1)c_{m} x^{m-2}+\sum_{m=2}^{\infty}-m(m-1)c_{m} x^{m}+\sum_{m=1}^{\infty}-mc_{m} x^{m}+\sum_{m=0}^{\infty} a^{2}c_{m} x^{m}&=&0
		\end{eqnarray*}
		Desarrollando los dos primeros t\'erminos de las sumatorias con sub\'indices $m=0 \quad \text{y}\quad m=1$ y sumando las series semejantes se obtiene
		\begin{eqnarray*}
			% \nonumber % Remove numbering (before each equation)
			a^{2}c_0+\left(a^{2}-1\right)c_{1} x+\sum_{m=2}^{\infty}c_{m}\left[{a}^2-{m-m^2+m}\right] x^{m}+\sum_{m=2}^{\infty} m(m-1)c_{m}x^{m-2}&=&0
		\end{eqnarray*}
		Obteniendo los dos primeros t\'erminos de la \'ultima serie de la expresi\'on anterior y haciendo un cambio llegamos a
		
		\begin{fullwidth}[%
			width=\dimexpr\textwidth+\marginparsep+\marginparwidth,
			outermargin=\dimexpr-\marginparwidth,
			]
			\begin{eqnarray*}
				% \nonumber % Remove numbering (before each equation)
				(a^{2}c_0+2c_{2})+[\left(a^{2}-1\right)c_{1}+(2)(3)c_{3}] x+\sum_{m=2}^{\infty}\left[(a^{2}-m^{2})c_{m}+(m+1)(m+2)c_{m+2}\right]x^{m}&=&0
			\end{eqnarray*}
		\end{fullwidth}
		
		
		Aplicando igualdad de polinomios obtenemos
		$$\left \{\begin{array}{c}
			a^{2}c_{0}+(1)(2)c_{2}=0\\
			\left(a^{2}-1\right)c_{1}+(2)(3)c_{3}=0 \\
			\left(a^{2}-m^{2}\right)c_{m}+(m+1)(m+2) c_{m+2}=0 \quad \forall m \geq 2
		\end{array} \right.$$
		Desarrollando $c_{m+2}$
		\begin{eqnarray*}
			c_{m+2}&=&\frac{\left(m^{2}-a^{2}\right)}{(m+1)(m+2)}c_{m}
		\end{eqnarray*}
		Ahora desarrollaremos los t\'erminos pares
		\begin{eqnarray*}
			% \nonumber % Remove numbering (before each equation)
			\text{si}\: m=0 \rightarrow \quad c_{2}&=&\frac{0^{2}-a^{2}}{(1)(2)}c_{0}  \\
			\text{si}\:  m=2 \rightarrow \quad c_{4}&=&\frac{\left(2^{2}-a^{2}\right)}{(3)(4)} \frac{\left(-a^{2}\right)}{(1)(2)}c_{0}  \\
			c_{4}&=&\frac{-a^{2}\left(2^{2}-a^{2}\right)}{4!}c_{0}\\
			\text{si}\: m=4 \rightarrow \quad c_{6}&=&\frac{-a^{2}\left(2^{2}-a^{2}\right)\left(4^{2}-a^{2}\right)}{4!(5)(6)}c_{0}  \\
			c_{6}&=&\frac{-a^{2}\left(2^{2}-a^{2}\right)\left(4^{2}-a^{2}\right)}{6!}c_{0}
		\end{eqnarray*}
		Desarrollando los dem\'as t\'erminos llegamos a expresi\'on
		\begin{eqnarray}\label{coef pares chevished}
			% \nonumber % Remove numbering (before each equation)
			c_{2m}&=&\frac{-a^{2}\left(2^{2}-a^{2}\right)\left(4^{2}-a^{2}\right) \cdots \left[(2m-2)^{2}-a^{2}\right]}{(2m)!}c_{0} \quad m \geq 1
		\end{eqnarray}
		Desarrollando los t\'erminos impares
		\begin{eqnarray*}
			% \nonumber % Remove numbering (before each equation)
			\text{si}\: m=1 \rightarrow \quad c_{3}&=&\frac{1^{2}-a^{2}}{(2)(3)}c_{1}  \\
			\text{si}\:  m=3 \rightarrow \quad c_{5}&=&\frac{\left(3^{2}-a^{2}\right)}{(4)(5)} \frac{1^{2}-a^{2}}{(2)(3)}c_{1}  \\
			c_{5}&=&\frac{\left(1^{2}-a^{2}\right)\left(3^{2}-a^{2}\right)}{5!}c_{1}\\
			\text{si}\: m=5 \rightarrow \quad c_{7}&=& \frac{\left(5^{2}-a^{2}\right)}{(6)(7)}\frac{\left(1^{2}-a^{2}\right)\left(3^{2}-a^{2}\right)}{5!}c_{1} \\
			c_{7}&=&\frac{\left(1^{2}-a^{2}\right)\left(3^{2}-a^{2}\right)\left(5^{2}-a^{2}\right)}{7!}c_{1}
		\end{eqnarray*}
		Siguiendo ese mismo desarrollo se obtiene
		\begin{eqnarray}\label{coef impares chevished}
			% \nonumber % Remove numbering (before each equation)
			c_{2m+1}=\frac{\left(1^{2}-a^{2}\right)\left(3^{2}-a^{2}\right) \cdots\left[(2m-1)^{2}-a^{2}\right]}{(2m+1)!}c_{1} \quad m \geq 1
		\end{eqnarray}
		Reemplazando \ref{coef impares chevished} y \ref{coef pares chevished} en la expresi\'on de series citar, se obtiene la soluci\'on para la ecuaci\'on de Chevished dada por
		\begin{eqnarray}\label{solucion chevished}
			y=c_0\left[1+\sum_{m=1}^{\infty} \frac{1}{(2m)!} \prod_{i=1}^{m}\left[(2i-2)^{2}-a^{2}\right]x^{2m}\right]+c_{1}\left[x+\sum_{m=1}^{\infty}\frac{1}{(2m+1)!}\prod_{i=1}^{m}\left[(2i-1)^{2}-a^{2}\right]\right]x^{2m+1}
		\end{eqnarray}
\subsection{Funci\'on Generadoradora de los polinomios de Chevyshev}
		\Theorem{Funci\'on generadora de los polinomios de Chevyshev}{La funci\'on generadora de los polinomios de Chevyshev est\'a dada por
			\begin{eqnarray}
				% \nonumber % Remove numbering (before each equation)
				\displaystyle\frac{1-t^2}{1-2 t x+t^2}&=&T_0(x)+2\displaystyle \sum_{n=1}^{\infty} T_n(x) t^n
		\end{eqnarray}}
		
		\begin{demo}
			Por la serie geom\'etrica, tenemos que
			$$
			\left(1-t^2\right)(1-t(2 x-t))^{-1}=\left(1-t^2\right) \sum_{n=0}^{\infty} t^n(2 x-t)^n
			$$
			clearly the constant term is 1 , and the coefficient of $t^n$ is
			$$
			\begin{aligned}
				(2 x)^n & +\sum_{m=1}^{\left[\frac{n}{2}\right]}(-1)^m\left\{\left(\begin{array}{c}
					n-m \\
					m
				\end{array}\right)+\left(\begin{array}{c}
					n-m-1 \\
					m-1
				\end{array}\right)\right\}(2 x)^{n-2 m} \\
				& =\sum_{m=0}^{\left[\frac{n}{2}\right]} \frac{n(-1)^m(n-m-1) !}{m !(n-2 m) !}(2 x)^{n-2 m}=2 T_n(x)
			\end{aligned}
			$$
		\end{demo}
		
		
		\subsection{F\'ormula de Rodrigues para los polinomios de Chevyshev}
		\begin{fullwidth}[%
			width=16cm,
			outermargin=\dimexpr-\marginparsep-\marginparwidth,
			]
			\Theorem{La f\'ormula de Rodrigues para los polinomios de Chevyshev}{La f\'ormula de Rodrigues para los polinomios de Chevyshev es
				\begin{eqnarray}\label{rodrigueschevysev}
					% \nonumber % Remove numbering (before each equation)
					T_n(x)&=&\displaystyle\frac{(-2)^n n !}{(2 n) !}\left(1-x^2\right)^{1 / 2} \frac{d^n}{d x^n}\left(1-x^2\right)^{n-1 / 2}
			\end{eqnarray}}
			
			
		\end{fullwidth}
		
		
		\begin{demo}
			Sea $z=\left(1-x^2\right)^{n-1 / 2}$ para obtener
			$$
			\left(1-x^2\right) z^{\prime}+(2 n-1) z x=0
			$$
			Derivando esta relaci\'on $(n+1)$ veces, obtenemos
			$$
			\left(1-x^2\right) D^{n+2} z-3 x D^{n+1} z+\left(n^2-1\right) D^n z=0
			$$
			Ahora si tomamos $w=\left(1-x^2\right)^{1 / 2} D^n z$, llegamos a
			$$
			\begin{aligned}
				& \left(1-x^2\right) w^{\prime \prime}-x w^{\prime}+n^2 w=0 \\
				& \left(1-x^2\right)^{1 / 2}\left[\left(1-x^2\right) D^{n+2} z-3 x D^{n+1} z+\left(n^2-1\right) D^n z\right]=0
			\end{aligned}
			$$
			De esta expresi\'on podemos notar que $w(x)$ y $T_n(x)$ son las soluciones polinomiales de la ecuaci\'on de chevyshev, y por lo tanto $T_n(x)=k w(x)$, donde $k$ es una constante a determinar.\\
			
			Por la f\'ormula de Leibniz podemos expresar
			\begin{eqnarray}\label{mc}
				\left(1-x^2\right)^{1 / 2} D^n\left(1-x^2\right)^{n-1 / 2}&=&\left(1-x^2\right)^{1 / 2}\displaystyle \sum_{j=0}^n\left(\begin{array}{l}
					n \\
					j
				\end{array}\right) D^{n-j}(1+x)^{n-1 / 2} D^j(1-x)^{n-1 / 2}
			\end{eqnarray}
			Calculando las derivadas de la sumatoria
			$$
			\begin{aligned}
				& D^{(n-1)}(1+x)^{n-1 / 2}=\frac{(n-1 / 2) !}{\left(J-\frac{1}{2}\right) !}(x+1)^{j-1 / 2} \\
				& D^J(1-x)^{n-1 / 2}=(-1)^J \frac{(n-1 / 2) !}{(n-1 / 2-1) !}(1-x)^{n-1 / 2-J}
			\end{aligned}
			$$
			Reemplazando las derivadas en (\ref{mc}) y simplificando
			$$
			\left(1-x^2\right)^{1 / 2} D^n\left(1-x^2\right)^{n-1 / 2}=n !(-1)^n \sum_{j=0}^n\left(\begin{array}{c}
				n-1 / 2 \\
				j
			\end{array}\right)\left(\begin{array}{c}
				n-1 / 2 \\
				n-1
			\end{array}\right)(x-1)^{n-1}(x+1)^j
			$$
			Por la identidad de vandermonde tenemos
			$$
			\left(1-x^2\right)^{1 / 2} D^n\left(1-x^2\right)^{n-1 / 2}=(-1)^n n !\left(\begin{array}{c}
				2 n-1 \\
				n
			\end{array}\right)
			$$
			Comparando el coeficiente de $x^n$ en ambos lados de $T_n(x)=cw(x)$, determinamos que
			$$
			c=\frac{(-1)^n 2^n n !}{(2 n) !}
			$$
		\end{demo}
\subsection{Relaci\'on de recurrencia de los polinomios de Chevyshev}
		\Theorem{Los polinomios de Chevyshev}{Los polinomios de Chevyshev $(T_{n}(x))$ satisfacen la relaci\'on de recurrencia dada por
			\begin{eqnarray}\label{chevyshevrecurrencia}
				% \nonumber % Remove numbering (before each equation)
				T_{n+1}(x)&=&2 x T_n(x)-T_{n-1}(x), \quad n \geq 1
		\end{eqnarray}}
		
		\begin{demo}
			De la identidad trigonom\'etrica de producto a suma, tenemos
			$$
			\cos \left((n+1) \cos ^{-1}(x)\right)+\cos \left((n-1) \cos ^{-1}(x)\right)=2 \cos \left(n \cos ^{-1}(x)\right) \cos \left(\cos ^{-1}(x)\right)
			$$
			De la expresi\'on (\ref{Tn chevyshev}) Ilegamos a
			$$
			T_{n+1}(x)+T_{n-1}(x)=2 x T_n(x)
			$$
			Despejando $T_{n+1}(x)$ se demuestra la expresi\'on
			$$
			T_{n+1}(x)=2 x T_n(x)-T_{n-1}(x)
			$$
		\end{demo}
		
		\Example{ Polinomios de Chevyshev}{Utiliza la relaci\'on de recurrencia de los polinomios de chevyshev para obtener $T_2(x), T_3(x)$ y $T_4(x)$.}
		
		\begin{sol}
			Para $n=1$, tenemos
			$$
			T_2(x)=2 x T_1(x)-T_0(x)
			$$
			Reemplazando $T_1(x)$ y $T_0(x)$
			$$
			T_2(x)=2 x(x)-(1)
			$$
			Simplificando
			$$
			T_2(x)=2 x^2-1
			$$
			Para el caso de $n=2$
			$$
			T_3(x)=2 x T_2(x)-T_1(x)
			$$
			Reemplazando $T_2(x)$ y $T_1(x)$
			$$
			T_3(x)=2 x\left(2 x^2-1\right)-x
			$$
			Simplificando
			$$
			\begin{aligned}
				& T_3(x)=4 x^3-2 x-x \\
				& T_3(x)=4 x^3-3 x
			\end{aligned}
			$$
			Para $n=3$, tenemos
			$$
			T_4(x)=2 x T_3(x)-T_2(x)
			$$
			Sustituyendo $T_3(x)$ y $T_2(x)$
			$$
			t_4(x)=2 x\left(4 x^3-3 x\right)-\left(2 x^2-1\right)
			$$
			Aplicando propiedad distributiva
			$$
			T_4(x)=8 x^4-6 x^2-2 x^2+1
			$$
			simplificando
			$$
			T_4(x)=8 x^4-8 x^2+1
			$$
		\end{sol}

		\subsection{Polinomios de Chebyshev}
		Si en la expresi\'on \ref{chebyshev equation} se toma $a=n$
		\begin{eqnarray}\label{ecuacion polinomio chebyshev}
			% \nonumber % Remove numbering (before each equation)
			\left(1-x^2\right) y^{\prime \prime}-x y^{\prime}+n^2 y&=&0
		\end{eqnarray}
		Ahora tomaremos
		\begin{eqnarray}\label{sust chebyshev}
			x&=&\cos(t)\quad x\leq 1
		\end{eqnarray}
		derivando tenemos
		\begin{eqnarray*}
			\frac{d y}{d t}&=&\frac{d y}{d x} \frac{d x}{d t}\\
			\frac{d}{d t}\left[\frac{d y}{d t}\right] &=&\frac{d}{d t}\left[\frac{d y}{d x} \frac{d x}{d t}\right] \\
			\frac{d^{2}y}{d t^{2}} &=&\frac{d^{2} y}{d x^{2}}\left(\frac{d x}{d t}\right)^{2}+\frac{d y}{d x} \frac{d^{2} x}{d t^{2}}
		\end{eqnarray*}
		Obteniendo las derivadas indicadas
		$$
		\begin{gathered}
			\frac{d x}{d t}=-\operatorname{sen}(t) \quad \frac{d^{2} x}{d t^{2}}=-\cos (t) \\
			\frac{d y}{d x}=-\csc (t) \frac{d y}{d t}
		\end{gathered}
		$$
		reemplazando tenemos
		$$
		\begin{aligned}
			&\frac{d^{2} y}{d t^{2}}=\operatorname{sen}^{2}(t) \frac{d^{2} y}{d x^{2}}+(-\csc (t))(-\cos (t)) \frac{d y}{d t} \\
			&\frac{d^{2} y}{d t^{2}}=\operatorname{sen}^{2}(t) \frac{d^{2} y}{d x^{2}}+\cot (t) \frac{d y}{d t} \\
			&\frac{d^{2} y}{d x^{2}}=\csc ^{2}(t) \frac{d^{2} y}{d t^{2}}-\csc ^{2}(t) \cot (t) \frac{d y}{d t}
		\end{aligned}
		$$
		
		Sustituyendo las derivadas en \ref{ecuacion polinomio chebyshev}
		\begin{eqnarray*}
			\left(1-\cos ^{2}(t)\right)\left[\csc^{2}(t) \frac{d^{2} y}{dt^{2}}-\csc^{2}(t)\cot(t) \frac{dy}{dt}\right]+\cos(t)\csc (t) \frac{d y}{d t}+n^{2}y&=&0
		\end{eqnarray*}
		Por identidades trigonom\'etricas llegamos a la expresi\'on
		\begin{eqnarray}\label{shebyshev coef const}
			\frac{d^{2}y}{dt^{2}}+n^{2}y&=&0
		\end{eqnarray}
		lo cual es una ecuaci\'on de segundo orden con coeficientes constantes. Sabemos que las soluciones \ref{shebyshev coef const} es de la forma
		\begin{eqnarray*}
			y(t)&=&e^{rt}
		\end{eqnarray*}
		Obteniendo la segunda derivada y reemplazando en \ref{shebyshev coef const} se llega a la expresi\'on
		\begin{eqnarray*}
			e^{r^t} r^{2}+n^{2} e^{rt}&=&0 \\
			e^{rt}\left[r^{2}+n^{2}\right]&=&0
		\end{eqnarray*}
		como $e^{rt}\neq 0$, entonces
		\begin{eqnarray*}
			r^{2}+n^{2}&=&0\\
			r&=&\pm ni
		\end{eqnarray*}
		As\'i la soluci\'on de \ref{shebyshev coef const} es
		\begin{eqnarray*}
			y(t)=c_{0} e^{-nti}+c_{1}e^{nti}
		\end{eqnarray*}
		Expresando la soluci\'on en t\'erminos de seno y coseno
		\begin{eqnarray*}
			y(t)&=&A \cos (n t)+B \sin (n t) \quad \text{donde}\quad A=c_{1}+c_{0}\quad \text{y}\quad B=i(c_{1}-c_{0})
		\end{eqnarray*}
		Reemplazando \ref{sust chebyshev} en la expresi\'on anterior
		\begin{eqnarray*}
			y(t)&=&A \cos (n\cos^{-1}(x))+B \sin (n\cos^{-1}(x)) \quad x\leq1
		\end{eqnarray*}
		Escogiendo
		\begin{eqnarray}\label{Tn chevyshev}
			T_{n}(x)&=&\cos (n\cos^{-1}(x))\quad n\geq 0
		\end{eqnarray}
		obtenemos los polinomios de Chebyshev.
		\begin{eqnarray*}
			% \nonumber % Remove numbering (before each equation)
			\text{si}\: n=0 \rightarrow \quad T_{0}(x)&=&\cos (0) \\
			T_{0}(x)&=&1\\
			\text{si}\:  n=1 \rightarrow \quad T_{1}(x)&=&\cos (\cos^{-1}(x)) \\
			T_{1}(x)&=&x\\
			\text{si}\:  n=2 \rightarrow \quad T_{2}(x)&=&\cos (2\cos^{-1}(x))\\
			T_{2}(x)&=&2\cos^{2}(\cos^{-1}(x))-1\\
			T_{2}(x)&=&2x^{2}-1\\
			\text{si}\:  n=3 \rightarrow \quad T_{3}(x)&=&\cos (3\cos^{-1}(x))\\
			T_{3}(x)&=&4\cos^{3}(\cos^{-1}(x))-3\cos(\cos^{-1}(x))\\
			T_{3}(x)&=&4x^{3}-3x\\
		\end{eqnarray*}
		
		$$\begin{aligned}
			&T_0(x)=1 \\
			&T_1(x)=x \\
			&T_2(x)=2 x^2-1 \\
			&T_3(x)=4 x^3-3 x \\
			&T_4(x)=8 x^4-8 x^2+1 \\
			&T_5(x)=16 x^5-20 x^3+5 x \\
			&\mathrm{~T}_6(\mathrm{x})=32 \mathrm{x}^6-48 \mathrm{x}^4+18 \mathrm{x}^2-1 \\
			&T_7(x)=64 x^7-112 x^5+56 x^3-7 x \\
			&T_8(x)=128 x^8-256 x^6+160 x^4-32 x^2+1 \\
			&T_9(x)=256 x^9-576 x^7+432 x^5-120 x^3+9 x \\
			&T_{10}(x)=512 x^{10}-1280 x^8+1120 x^6-400 x^4+50 x^2-1 \\
			&T_{11}(x)=1024 x^{11}-2816 x^9+2816 x^7-1232 x^5+220 x^3-11 x
		\end{aligned}
		$$
		\textcolor{red}{agregar graficos en wxmaxima}
		
		
		
		
		\subsection{Propiedades de los polinomios de Chevyshev}
		\subsection{Chevyshev Primer tipo}
		Tomando $\alpha=\beta=-\frac{1}{2}$ en la ecuaci\'on de Jacobi \ref{Jacobiequation} se obtiene la ecuaci\'on de Chevyshed de primer tipo
		\begin{equation}\label{chevyshed1tipo}
			\left(1-x^2\right)y^{\prime\prime}(x)-x y^{\prime}(x)+\gamma y(x)=0
		\end{equation}
		Donde $x_{0}=0$ es un punto ordinario, de manera que por el toerema \ref{teosolordinario} existe una soluci\'on de la forma dada en \ref{solenserie}. Derivando \ref{solenserie} y reemplazando en \ref{chevyshed1tipo} se tiene
		$$\begin{gathered}
			\displaystyle\sum_{m=2}^{\infty}m(m-1) c_m x^{m-2}-\displaystyle\sum_{m=2}^{\infty}m(m-1) c_m x^m-\displaystyle\sum_{m=1}^{\infty}m c_m x^m+\gamma^2 \displaystyle\sum_{m=0}^{\infty} c_m x^m=0 \\
			\displaystyle\sum_{m=-2}^{\infty}(m+2)(m+1) c_{m+2} x^m-\displaystyle\sum_{m=2}^{\infty}m(m-1)c_m x^m-\displaystyle\sum_{m=1}^{\infty}m c_m x^m+\gamma^2 \displaystyle\sum_{m=0}^{\infty} c_m x^m=0 \\
			\displaystyle \sum_{m=0}^{\infty}(m+2)(m+1)c_{m+2} x^m-\displaystyle\sum_{m=0}^{\infty}m(m-1)c_m x^m-\displaystyle\sum_{m=0}^{\infty}m c_m x^m+\gamma^2 \displaystyle\sum_{m=0}^{\infty} c_m x^m=0 \\
			\displaystyle\sum_{m=0}^{\infty}\left[c_{m+2}(m+2)(m+1)-c_m\left(m(m-1)+m-\gamma^2\right)\right] x^m=0
		\end{gathered}$$
		Igualando los coeficientes a cero
		\begin{equation}\label{chevyshe1tiporelacion}
			c_{m+2}=\frac{m^2-\gamma^2}{(m+2)(m+1)} c_m \quad \forall m \geqslant 0
		\end{equation}
		
		Para el caso en que $m$ es par tenemos que se cumple que:
		$$
		\begin{aligned}
			m=0 \rightarrow c_2 & =\displaystyle\frac{\left(0^2-\gamma^2\right)}{2 !} c_0 \\
			m=2 \rightarrow c_4 & =\displaystyle\frac{\left(2^2-\gamma^2\right)\left(0^2-\gamma^2\right)}{4 !} c_0 \\
			\vdots \\
			c_{2 m} & =\displaystyle\frac{\displaystyle\prod_{i=0}^{m-1}\left((2 i)^2-\gamma^2\right)}{(2 m) !} c_0
		\end{aligned}
		$$
		Para los casos en que $m$ es impar tenemos que se cumple que:
		$$
		\begin{aligned}
			& m=1 \rightarrow c_3= \displaystyle\frac{\left(1^2-\gamma^2\right)}{3 !} c_1 \\
			& m=3 \rightarrow c_5=\displaystyle\frac{\left(3^2-\gamma^2\right)\left(1^2-\gamma^2\right)}{5 !} c_1 \\
			\vdots\\
			& c_{2 m+1}=\displaystyle\frac{\displaystyle\prod_{i=0}^{m-1}\left((2 i+1)^2-\gamma^2\right)}{(2 m+1) !} c_1
		\end{aligned}
		$$
		De manera que la soluci\'on general de \ref{chevyshed1tipo} es la expresi\'on
		\begin{equation}\label{solchevyshev1tipo}
			y(x)=c_0\left[1+\displaystyle\sum_{m=1}^{\infty} \displaystyle\frac{\displaystyle\prod_{i=0}^{m-1}\left((2 i)^2-\gamma^2\right)}{(2 m) !} x^{2 m}\right]+c_1\left[x+\displaystyle\sum_{m=1}^{\infty}\displaystyle \frac{\displaystyle\prod_{i=0}^{m-1}\left((2 i+1)^2-\gamma^2\right)}{(2 m+1) !} x^{2 m+1}\right]
		\end{equation}
		Los polinomios de Chevyshev $T_{m}$ de primer tipo se obtienen multiplicando $P_{m}$ por las constantes $c_{0}$ y $c_{1}$, donde  \cite{spencer} presenta que tales constantes son  $c_{0}=\displaystyle (-1)^{m/2}$ y $c_{1}=\displaystyle m(-1)^{m/2}$. A continuaci\'on presentamos los
		primeros diez polinomios de Chevyshev de primer tipo
		$$\begin{aligned}
			& T_0(x)=1 \\
			& T_1(x)=x \\
			& T_2(x)=2 x^2-1  \\
			& T_3(x)=4 x^3-3 x \\
			& T_4(x)=8 x^4-8 x^2+1 \\
			& T_5(x)=16 x^5-20 x^3+5 x \\
			& T_6(x)=32 x^6-48 x^4+18 x^2-1  \\
			& T_7(x)=64 x^7-112 x^5+56 x^3-7 x  \\
			& T_8(x)=128 x^8-256 x^6+160 x^4-32 x^2+1  \\
			& T_9(x)=256 x^9-576 x^7+432 x^5-120 x^3+9 x  \\
			& T_{10}(x)=512 x^{10}-1280 x^8+1120 x^6-400 x^4+50 x^2-1
		\end{aligned}$$
		\subsection*{Obtenci\'on de los polinomios de Chevishev de segundo tipo como proceso de Gramm-Schmidt}
		Los polinomios de Chebyshev del primer tipo $T_n(x)$ se pueden obtener usando el Proceso de ortogonalizaci\'on de Gram-Schmidt para polinomios en el dominio $(-1,1)$ con el factor de peso $1 / \sqrt{1-x^2}$ [13], pág. 61 . A continuaci\'on se presenta el proceso correspondiente:
		
		Tomando $R_0(x)=1, R_1(x)=x+a_{1,0}, R_2(x)=x^2+a_{2,1} x+a_{2,0}, R_3(x)=x^3+a_{3,2} x^2+a_{3,1} x+a_{3,0}$ hasta $R_n(x)$ donde las constantes o coeficientes $a_{n, m}$ son determinadas por la condici\'on de ortogonalidad.
		Antes de obtener los polinomios $R_{n}(x)$ demostraremos que estos polinomios son ortogonales.\\
		La ecuaci\'on diferencial \ref{chevyshed1tipo} para $y=R_m(x)$ se puede escribir en forma de un problema de Sturm Liouville [13] página 63:
		$$
		\left(1-x^2\right) R_m^{\prime \prime}(x)-x R_m^{\prime}(x)+\lambda R_m(x)=0
		$$
		Escribimos la ED en la forma estandar:
		$$
		R_m^{\prime \prime}(x)-\frac{x}{\left(1-x^2\right)} R_m^{\prime}(x)+\frac{\lambda}{\left(1-x^2\right)} R_m(x)=0
		$$
		Tomando a $$\mu=e^{\displaystyle\int P(x) d x}=e^{-\displaystyle\int \displaystyle\frac{x}{\left(1-x^2\right)} d x}=e^{\ln \left|1-x^2\right|^{1 / 2}} \rightarrow \mu=\left(1-x^2\right)^{1 / 2}$$
		Multiplicando nuestra \'ultima expresi\'on por $\mu$ :
		$$
		\begin{gathered}
			\left(1-x^2\right)^{1 / 2} R_m^{\prime \prime}(x)-\left(1-x^2\right)^{1 / 2} \frac{x}{\left(1-x^2\right)} R_m^{\prime}(x)+\left(1-x^2\right)^{1 / 2} \frac{\lambda}{\left(1-x^2\right)} R_m(x)=0 \\
			\left(1-x^2\right)^{1 / 2} R_m^{\prime \prime}(x)-\frac{x}{\left(1-x^2\right)^{1 / 2}} R_m^{\prime}(x)+\frac{\lambda}{\left(1-x^2\right)^{1 / 2}} R_m(x)=0
		\end{gathered}
		$$
		Es de conocimiento que:
		$$
		\frac{d}{d x}\left(\left(1-x^2\right)^{1 / 2}\right)=-\frac{x}{\left(1-x^2\right)^{1 / 2}}
		$$
		Por lo que:
		$$
		\frac{d}{d x}\left(\left(1-x^2\right)^{1 / 2} \frac{d}{d x}\left(R_m(x)\right)\right)=-\frac{\lambda}{\left(1-x^2\right)^{1 / 2}} R_m(x)
		$$
		$\operatorname{Para} \lambda=m^2$
		\begin{equation}\label{nkn}
			\frac{d}{d x}\left(\left(1-x^2\right)^{1 / 2} \frac{d}{d x}\left(R_m(x)\right)\right)=-\frac{m^2}{\left(1-x^2\right)^{1 / 2}} R_m(x)
		\end{equation}
		Multiplicando e integrando por $R_n(x)$ con $n \neq m$ :
		\begin{equation}\label{dwd}
			\int_{-1}^1 R_n(x) \frac{d}{d x}\left(\left(1-x^2\right)^{1 / 2} \frac{d}{d x}\left(R_m(x)\right)\right) d x=-m^2 \int_{-1}^1 \frac{1}{\sqrt{1-x^2}} R_m(x) R_n(x)
		\end{equation}
		$\mathrm{Al}$ integrar por parte la expresi\'on
		\begin{equation}\label{nf}
			\int_{-1}^1 R_n(x) \frac{d}{d x}\left(\left(1-x^2\right)^{1 / 2} \frac{d}{d x}\left(R_m(x)\right)\right) d x
		\end{equation}
		seleccionando a $u=R_n(x)$ y a $v=\displaystyle\int \frac{d}{d x}\left(\left(1-x^2\right)^{1 / 2} \frac{d}{d x}\left(R_m(x)\right)\right) d x$, se tiene que la misma es igual a:
		$$
		\left.\left(1-x^2\right)^{1 / 2} R_n(x) R_m^{\prime}(x)\right|_{-1} ^1-\int_{-1}^1\left(1-x^2\right)^{1 / 2} R_n^{\prime}(x) R_m^{\prime}(x) d x
		$$
		Puesto que al evaluar $\left(1-x^2\right)^{1 / 2}$ en $-1$ y $1$ el resultado es cero, la integral \ref{nf} queda reducida a:
		$$
		\int_{-1}^1 R_n(x) \frac{d}{d x}\left(\left(1-x^2\right)^{1 / 2} \frac{d}{d x}\left(R_m(x)\right)\right) d x=-\int_{-1}^1\left(1-x^2\right)^{1 / 2} R_n^{\prime}(x) R_m^{\prime}(x) d x
		$$
		Por lo que:
		$$
		\int_{-1}^1 \sqrt{1-x^2} R_n^{\prime}(x) R_m^{\prime}(x) d x=m^2 \int_{-1}^1 \frac{1}{\sqrt{1-x^2}} R_m(x) R_n(x)
		$$
		Si seguimos el mismo procedimiento que el anterior pero con $m$ y $n$ intercambiados, se obtendr\'a una ecuaci\'on identica, con la diferencia que el coeficiente $m^2$ ser\'a sustituido por un $n^2$. Si restamos una de estas ecuaciones de la otra:
		$$
		\left[m^2-n^2\right] \int_{-1}^1 \frac{R_n(x) R_m(x)}{\sqrt{1-x^2}} d x=0
		$$
		Para $n \neq m$
		\begin{equation}\label{lo}
			\int_{-1}^1 \frac{R_n(x) R_m(x)}{\sqrt{1-x^2}} d x=0
		\end{equation}
		Los polinomios $R_n(x)$ satisfacen as\'i las mismas relaciones de ortogonalidad que los polinomios de Chebyshev del primer tipo y por lo tanto deben ser m\'ultiplos de ellos.\\
		Ya demostrado que los polinomios $R_{n}(x)$ son ortogonales respecto a la funci\'on de peso,vamos a presentar la obtenci\'on de los primeros cuatro polinomios:
		\begin{itemize}
			\item $R_0(x)=1$
			\item Para la obtenci\'on de $R_1(x)$
			$$
			\begin{aligned}
				\int_{-1}^1 w(x) R_1(x) R_0(x) d x & =\int_{-1}^1 \frac{1}{\sqrt{1-x^2}}(1)\left(x+a_{1,0}\right) d x=0 \\
				& \rightarrow \int_{-1}^1 \frac{x}{\sqrt{1-x^2}} d x+a_{1,0} \int_{-1}^1 \frac{1}{\sqrt{1-x^2}} d x=0 \rightarrow a_{1,0}=0 \rightarrow R_1(x)=x
			\end{aligned}
			$$
			\item Para la obtenci\'on de $R_2(x)$
			$$
			\begin{aligned}
				& \int_{-1}^1 w(x) R_2(x) R_1(x) d x=\int_{-1}^1 \frac{1}{\sqrt{1-x^2}}(x)\left(x^2+a_{2,1} x+a_{2,0}\right) d x=0 \\
				& \int_{-1}^1 \frac{x^3}{\sqrt{1-x^2}} d x+a_{2,1} \int_{-1}^1 \frac{x^2}{\sqrt{1-x^2}} d x+a_{2,0} \int_{-1}^1 \frac{x}{\sqrt{1-x^2}} d x=0 \rightarrow a_{2,1}=0 \\
				& \int_{-1}^1 w(x) R_2(x) R_0(x) d x=\int_{-1}^1 \frac{1}{\sqrt{1-x^2}}\left(x^2+a_{2,1} x+a_{2,0}\right) d x=0 \\
				& \int_{-1}^1 \frac{x^2}{\sqrt{1-x^2}} d x+a_{2,1} \int_{-1}^1 \frac{x}{\sqrt{1-x^2}} d x+a_{2,0} \int_{-1}^1 \frac{d x}{\sqrt{1-x^2}}=0 \rightarrow a_{2,0}=-\frac{1}{2} \rightarrow R_2(x)=x^2-\frac{1}{2}
			\end{aligned}
			$$
			\item Para la obtenci\'on de $R_3(x)$
			$$
			\begin{aligned}
				\int_{-1}^1 w(x) R_3(x) R_0(x) d x=\int_{-1}^1 \frac{1}{\sqrt{1-x^2}}\left(x^3+a_{3,2} x^2+a_{3,1} x+a_{3,0}\right) d x=0 \\
				\quad \int_{-1}^1 \frac{x^3}{\sqrt{1-x^2}}+a_{3,2} \int_{-1}^1 \frac{x^2}{\sqrt{1-x^2}}+a_{3,1} \int_{-1}^1 \frac{x}{\sqrt{1-x^2}}+a_{3,0} \int_{-1}^1 \frac{1}{\sqrt{1-x^2}}=0 \rightarrow a_{3,0}=0
			\end{aligned}
			$$
			$$
			\begin{aligned}
				& \int_{-1}^1 w(x) R_3(x) R_1(x) d x=\int_{-1}^1 \frac{1}{\sqrt{1-x^2}}\left(x^3+a_{3,2} x^2+a_{3,1} x+a_{3,0}\right)(x) d x=0 \\
				& \int_{-1}^1 \frac{x^4}{\sqrt{1-x^2}} d x+a_{3,2} \int_{-1}^1 \frac{x^3}{\sqrt{1-x^2}} d x+a_{3,1} \int_{-1}^1 \frac{x^2}{\sqrt{1-x^2}} d x+a_{3,0} \int_{-1}^1 \frac{x}{\sqrt{1-x^2}} d x \rightarrow a_{3,1}=-\frac{3}{4}
			\end{aligned}
			$$
			$$
			\begin{aligned}
				& \int_{-1}^1 w(x) R_3(x) R_2(x) d x=\int_{-1}^1 \frac{1}{\sqrt{1-x^2}}\left(x^3+a_{3,2} x^2+a_{3,1} x+a_{3,0}\right)\left(x^2-\frac{1}{2}\right) d x=0 \\
				& \int_{-1}^1 \frac{x^5}{\sqrt{1-x^2}} d x-\frac{1}{2} \int_{-1}^1 \frac{x^3}{\sqrt{1-x^2}} d x+a_{3,2} \int_{-1}^1 \frac{x^4}{\sqrt{1-x^2}} d x-\frac{1}{2} a_{3,2} \int_{-1}^1 \frac{x^2}{\sqrt{1-x^2}} d x+ \\
				& a_{3,1} \int_{-1}^1 \frac{x^3}{\sqrt{1-x^2}} d x-\frac{1}{2} a_{3,1} \int_{-1}^1 \frac{x}{\sqrt{1-x^2}} d x+a_{3,0} \int_{-1}^1 \frac{x^2}{\sqrt{1-x^2}} d x-\frac{1}{2} a_{3,0} \int_{-1}^1 \frac{1}{\sqrt{1-x^2}} d x \\
				& \rightarrow a_{3,2}=0 \rightarrow R_3(x)=x^3-\frac{3}{4} x
			\end{aligned}
			$$
		\end{itemize}
		Siguiendo el mismo proceso obtenemos los dem\'as polinomios $R_{n}(x)$
		$$\begin{aligned}
			& R_0(x)=1 \\
			& R_1(x)=x \\
			& R_2(x)=x^2-\frac{1}{2} \\
			& R_3(x)=x^3-\frac{3}{4} x \\
			& R_4(x)=x^4-x^2+\frac{1}{8} \\
			& R_5(x)=x^5-\frac{20}{16} x^3+\frac{5}{16} x \\
			& R_6(x)=x^6-\frac{48}{32} x^4+\frac{18}{32} x^2-\frac{1}{32} \\
			& R_7(x)=x^7-\frac{112}{64} x^5+\frac{56}{64} x^3-\frac{7}{64} x \\
			& R_8(x)=x^8-\frac{256}{128} x^6+\frac{160}{128} x^4-\frac{32}{128} x^2+\frac{1}{128} \\
			& R_9(x)=x^9-\frac{576}{256} x^7+\frac{432}{256} x^5-\frac{120}{256} x^3+\frac{9}{250} x \\
			& R_{10}(x)=x^{10}-\frac{1280}{512} x^8+\frac{1120}{512} x^6-\frac{400}{512} x^4+\frac{50 x^2}{512}-\frac{1}{512}
		\end{aligned}$$
		
		
		\subsection*{Obtenci\'on de los polinomios de Chevishev de primer tipo a partir de la relaci\'on de recurrencia a tres t\'erminos}
		\begin{equation}\label{recutipo1}
			T_{n+1}(x)=2 x T_n(x)-T_{n-1}(x)
		\end{equation}
		ahora obtendremos los polinomios desde $T_2(x)$ a $T_7(x)$
		
		\begin{fullwidth}[%
			width=\marginparwidth+\marginparsep+\dimexpr\textwidth,
			outermargin=\dimexpr-\marginparwidth,
			]
			$$
			\begin{aligned}
				& T_2(x)=2 x T_1(x)-T_0(x)=2 x(x)-1=2 x^2-1 \\
				& T_3(x)=2 x T_2(x)-T_1(x)=2 x\left(2 x^2-1\right)-x=4 x^3-3 x \\
				& T_4(x)=2 x T_3(x)-T_2(x)=2 x\left(4 x^3-3 x\right)-\left(2 x^2-1\right)=8 x^4-8 x^2+1 \\
				& T_5(x)=2 x T_4(x)-T_3(x)=2 x\left(8 x^4-8 x^2+1\right)-\left(4 x^3-3 x\right)=16 x^5-20 x^3+5 x \\
				& T_6(x)=2 x T_5(x)-T_4(x)=2 x\left(16 x^5-20 x^3+5 x\right)-\left(8 x^4-8 x^2+1\right)=32 x^6-48 x^4+18 x^2-1 \\
				& T_7(x)=2 x T_6(x)-T_5(x)=64 x^7-112 x^5+56 x^3-7 x
			\end{aligned}
			$$
		\end{fullwidth}
		
		
		
		
		\subsection*{Obtenci\'on de los polinomios de Chevishev de primer tipo a partir de la f\'ormula generadora}
		Iniciamos a asignar valores a $n$ :
		$$
		\sum_{n=0}^{\infty} t^n T_n(x)=\frac{1-x t}{1-2 x t+t^2}=w(x, t)
		$$
		Para este proceso nos vamos a auxiliarnos de la definici\'on de polinomio de Maclaurin, gracias a la cual podemos plantear lo siguiente:
		$$
		T_n(x)=\frac{1}{n !} \frac{\partial^n}{\partial t^n} w(x, t) \equiv \frac{1}{n !} \frac{\partial^n}{\partial t^n}\left(\frac{1-x t}{1-2 x t+t^2}\right)_{t=0}
		$$
		$$
		\begin{aligned}
			& n=0 \rightarrow T_0(x)=\frac{1}{0 !} \frac{\partial^0}{\partial t^0}\left(\frac{1-x t}{1-2 x t+t^2}\right)_{t=0}=\frac{1}{0 !}\left[\frac{1-x t}{1-2 x t+t^2}\right]_{t=0}=1 \\
			& n=1 \rightarrow T_1(x)=\frac{1}{1 !} \frac{\partial^1}{\partial t^1}\left(\frac{1-x t}{1-2 x t+t^2}\right)_{t=0}=\frac{1}{1 !}\left[\frac{x t^2+x-2 t}{\left(1-2 x t+t^2\right)^2}\right]_{t=0}=x \\
			& n=2 \rightarrow T_2(x)=\frac{1}{2 !} \frac{\partial^2}{\partial t^2}\left(\frac{1-x t}{1-2 x t+t^2}\right)_{t=0}=\frac{1}{2 !}\left[\frac{2\left(2 x^2-x t-3 x t+3 t^2-1\right)}{\left(1-2 x t-t^2\right)^3}\right]_{t=0}=2 x^2-1 \\
			& n=3 \rightarrow T_3(x)=\frac{1}{3 !} \frac{\partial^3}{\partial t^3}\left(\frac{1-x t}{1-2 x t+t^2}\right)_{t=0}=\frac{1}{3 !}\left(6\left(4 x^3-3 x\right)\right)=4 x^3-3 x \\
			& n=4 \rightarrow T_4(x)=\frac{1}{4 !} \frac{\partial^4}{\partial t^4}\left(\frac{1-x t}{1-2 x t+t^2}\right)_{t=0}=\frac{1}{4 !}\left[24\left(8 x^4-8 x^2-1\right)\right]=8 x^4-8 x^2-1 \\
			& n=5 \rightarrow T_5(x)=\frac{1}{5 !} \frac{\partial^5}{\partial t^5}\left(\frac{1-x t}{1-2 x t+t^2}\right)_{t=0}=16 x^5-20 x^3+5 x
		\end{aligned}
		$$
		\subsection{Chevyshev Segundo tipo}
		La ecuaci\'on diferencial que satisface los polinomios de Chevyshev de segundo tipo, se obtiene a partir de \ref{Jacobiequation} con los par\'ametros $\alpha=\beta=\frac{1}{2}$
		\begin{equation}\label{chevysev2tipo}
			\left(1-x^2\right) y^{\prime\prime}-3 xy^{\prime}+\gamma y=0
		\end{equation}
		Derivando y reemplazando en \ref{chevysev2tipo} tenemos
		$$\begin{aligned}
			\displaystyle\sum_{m=2}^{\infty} c_m m(m-1) x^{m-2}-\displaystyle\sum_{m=2}^{\infty} c_m m(m-1) x^m-3 \displaystyle\sum_{m=1}^{\infty} c_m m x^m+\displaystyle\gamma \sum_{m=0}^{\infty} c_m x^m=0 \\
			\displaystyle \sum_{m=0}^{\infty} c_{m+2}(m+2)(m+1) x^m-\displaystyle\sum_{m=2}^{\infty} c_m m(m-1) x^m-3\displaystyle \sum_{m=1}^{\infty} c_m m x^m+\displaystyle\gamma \sum_{m=0}^{\infty} c_m x^m=0 \\
			2(1) c_2+(3)(2) c_3 x+\displaystyle\sum_{m=2}^{\infty} c_{m+2}(m+2)(m+1) x^m-\displaystyle\sum_{m=2}^{\infty} c_m m(m-1) x^m-3 c_1 x \\
			\displaystyle-3 \sum_{m=2}^{\infty} c_m m x^m+\gamma c_0+\gamma c_1 x+\gamma\displaystyle \sum_{m=2}^{\infty} c_m x^m=0 \\
		\end{aligned}$$
		De d\'onde se obtienen las siguientes expresiones:
		$$
		\begin{gathered}
			2 c_2+\gamma c_0=0 \\
			\left(6 c_3-3 c_1+\gamma c_1\right) x=0 \\
			\displaystyle\sum_{m=2}^{\infty}\left[c_{m+2}(m+2)(m+1)-c_m((m)(m-1)+3 m-\gamma)\right] x^m=0
		\end{gathered}
		$$
		Igualando cada una de las mismas a cero, se tiene la expresi\'on de recurrencia:
		$$
		c_{m+2}=\displaystyle\frac{m(m-1)+3 m-\gamma}{(m+2)(m+1)} c_m \equiv \displaystyle\frac{m(m+2)-\gamma}{(m+2)(m+1)} c_m
		$$
		Para el caso en que $m$ es par tenemos que se cumple que:
		$$
		\begin{aligned}
			& c_{2 m}=\displaystyle\frac{\displaystyle\prod_{i=0}^{m-1}[(2 i)(2 i+2)-\gamma]}{(2 m) !} c_0 \\
			&
		\end{aligned}
		$$
		Para los casos en que $m$ es impar tenemos que se cumple que:
		$$
		\begin{aligned}
			& m=1 \rightarrow c_3=\displaystyle\frac{(3)-\gamma}{(3)(2)} c_1 \\
			& m=3 \rightarrow c_5=\displaystyle\frac{(3)(5)-\gamma}{(5)(4)}\displaystyle \frac{(3)-\gamma}{(3)(2)} c_1 \\
			& m=5 \rightarrow c_7=\displaystyle\frac{(5)(7)-\gamma}{(7)(6)} \displaystyle\frac{(3)(5)-\gamma}{(5)(4)}\displaystyle \frac{(3)-\gamma}{(3)(2)} c_1 \\
			& c_{2 m+1}=\displaystyle\frac{\displaystyle\prod_{i=0}^{m-1}[(2 i+1)((2 i+1)+2)-\gamma]}{(2 m+1) !} c_1 \\
			&
		\end{aligned}
		$$
		La soluci\'on general de \ref{chevysev2tipo} es
		\begin{fullwidth}[%
			width=\marginparwidth+\marginparsep+\dimexpr\textwidth,
			outermargin=\dimexpr-\marginparwidth,
			]
			
			
			\begin{equation}\label{solgnal}
				y=c_0\left[1+\displaystyle\sum_{m=1}^{\infty} \frac{\displaystyle\prod_{i=0}^{m-1}[(2 i)(2 i+2)-\gamma]}{(2 m) !} x^{2 m}\right]+c_1\left[x+\displaystyle\sum_{m=1}^{\infty} \frac{\displaystyle\prod_{i=0}^{m-1}[(2 i+1)((2 i+1)+2)-\gamma]}{(2 m+1) !}x^{2m+1}\right]
			\end{equation}
			
		\end{fullwidth}
		
		
		
		Los polinomios de Chevyshev $U_{m}$ de segundo tipo con $\gamma=m(m+2)$ se obtienen multiplicando $P_{m}$ por las constantes $c_{0}$ y $c_{1}$, donde  \cite{spencer} presenta que tales constantes son  $c_{0}=\displaystyle (-1)^{m/2}$ y $c_{1}=\displaystyle 2^{m+1}(-1)^{m}$. A continuaci\'on presentamos
		los diez primeros polinomios de Chevyshev de segundo orden
		$$\begin{aligned}
			& U_0(x)=1 \\
			& U_1(x)=2 x \\
			& U_2(x)=4 x^2-1  \\
			& U_3(x)=8 x^3-4 x \\
			& U_4(x)=16 x^4-12 x^2+1 \\
			& U_5(x)=32 x^5-32 x^3+6 x \\
			& U_6(x)=64 x^6-80 x^4+24 x^2-1 \\
			& U_7(x)=128 x^7-192 x^5+80 x^3-8 x  \\
			& U_8(x)=256 x^8-448 x^6+240 x^4-40 x^2+1  \\
			& U_9(x)=512 x^9-1024 x^7+672 x^5-160 x^3+10 x  \\
			& U_{10}(x)=1024 x^{10}-2304 x^8+1792 x^6-560 x^4+60 x^2+1
		\end{aligned}$$
		
		\Theorem{Polinomios de Chevyshev de segundo tipo}{La funci\'on generadora de los polinomios de Chevyshev de segundo tipo es
			\begin{eqnarray}\label{generadora2tipochevyshev}
				% \nonumber % Remove numbering (before each equation)
				\displaystyle\frac{1}{1-2 t x+t^2}&=&\displaystyle\sum_{n=0}^{\infty} U_n(x) t^n
		\end{eqnarray}}
		
		
		\begin{demo}
			jjg
		\end{demo}
		
		\Theorem{Los polinomios de chevyshev de segundo}{Los polinomios de chevyshev de segundo tipo satisfacen la relaci\'on de recurrencia dada por
			\begin{eqnarray}\label{chevyshev2tiporelacion}
				% \nonumber % Remove numbering (before each equation)
				U_{n+1}(x)&=&2 x U_n(x)-U_{n-1}(x), \quad U_0(x)=1, \quad U_1(x)=2 x
		\end{eqnarray}}
		
		
		\begin{demo}
			Por la identidad trigonom\'etrica sabemos
			\begin{eqnarray}\label{it1}
				\displaystyle\sin ((n+2) \theta)+\sin (n \theta)&=&\displaystyle 2 \sin ((n+1) \theta) \cos (\theta)
			\end{eqnarray}
			Tomando $\theta=\cos ^{-1}(x)$ en la expresi\'on (\ref{it1})
			\begin{eqnarray*}
				\sin \left((n+2) \cos ^{-1}(x)\right)+\sin \left(n \cos ^{-1}(x)\right)&=&2 \sin \left((n+1) \cos ^{-1}(x)\right)
				\cos \left(\cos ^{-1}(x)\right)
			\end{eqnarray*}
			Por la definici\'on de los polinomios de Chevysev, tenemos que
			\begin{eqnarray*}
				\sin \left(\cos ^{-1}(x)\right) u_{n+1}(x)+\sin \left(\cos ^{-1}(x)\right) u_{n-1}(x)&=&2 x \sin \left(\cos ^{-1}(x)\right) u_n(x)
			\end{eqnarray*}
			Tomando factor com\'un
			\begin{eqnarray*}
				\sin \left(\cos ^{-1}(x)\right)\left[u_{n+1}(x)+u_{n-1}(x)\right]&=&\sin \left(\cos ^{-1}(x)\right)\left[2 x u_n(x)\right]
			\end{eqnarray*}
			Por igualdad de expresiones
			\begin{eqnarray*}
				u_{n+1}(x)+u_{n-1}(x)&=&2 x u_n(x)
			\end{eqnarray*}
			Despejando $u_{n+1}(x)$ obtenemos el resultado esperado
			\begin{eqnarray*}
				u_{n+1}(x)&=&2 x u_n(x)-u_{n-1}(x) \quad \forall n \geq 1
			\end{eqnarray*}
		\end{demo}
		
		\Theorem{Los polinomios de chevyshev de segundo tipo}{	Los polinomios de chevyshev de segundo tipo se obtienen partir de la siguiente f\'ormula de Rodrigues
			\begin{eqnarray}\label{chevyshev2tiporodrigues}
				% \nonumber % Remove numbering (before each equation)
				U_n(x)&=&\displaystyle\frac{(-2)^n(n+1) !}{(2 n+1) !}\left(1-x^2\right)^{-1 / 2}\displaystyle \frac{d^n}{d x^n}\left(1-x^2\right)^{n+1 / 2}
		\end{eqnarray}}
		
		\begin{demo}
			sjjs
		\end{demo}
		
		
		\subsection*{Obtenci\'on de los polinomios de Chevishev de segundo tipo como proceso de Gramm-Schmidt}
		Los polinomios de Chebyshev de segundo tipo $U_n(x)$ se pueden obtener usando el proceso de ortogonalizaci\'on de Gram-Schmidt para polinomios en el dominio $(-1,1)$ con la funci\'on de peso $\sqrt{1-x^2}$ [13] pág. 73. A continuaci\'on se presenta el proceso correspondiente:
		
		Tomando $R_0(x)=1, R_1(x)=x+a_{1,0}, R_2(x)=x^2+a_{2,1} x+a_{2,0}, R_3(x)=x^3+a_{3,2} x^2+a_{3,1} x+a_{3,0}$ hasta $R_n(x)$ donde las constantes  $a_{n, m}$ son determinadas por la condici\'on de ortogonalidad. Para este proceso, solo vamos a presentar la obtenci\'on de los primeros cuatro polinomios:
		\begin{itemize}
			\item $R_0(x)=1$
			\item Para la obtenci\'on de $R_1(x)$
			$$
			\begin{aligned}
				\int_{-1}^1 w(x) R_1(x) R_0(x) d x & =\int_{-1}^1 \sqrt{1-x^2}(1)\left(x+a_{1,0}\right) d x=0 \\
				& \rightarrow \int_{-1}^1(x) \sqrt{1-x^2} d x+a_{1,0} \int_{-1}^1 \sqrt{1-x^2} d x=0 \rightarrow a_{1,0}=0 \rightarrow R_1(x)=x
			\end{aligned}
			$$
			
			
			\begin{fullwidth}[%
				width=\marginparwidth+\marginparsep+\dimexpr\textwidth,
				outermargin=\dimexpr-\marginparwidth,
				]
				
				\item Para la obtenci\'on de $R_2(x)$
				$$
				\begin{aligned}
					& \int_{-1}^1 w(x) R_2(x) R_1(x) d x=\int_{-1}^1 \sqrt{1-x^2}(x)\left(x^2+a_{2,1} x+a_{2,0}\right) d x=0 \\
					& \int_{-1}^1 x^3 \sqrt{1-x^2} d x+a_{2,1} \int_{-1}^1\left(x^2\right) \sqrt{1-x^2} d x+a_{2,0} \int_{-1}^1(x) \sqrt{1-x^2} d x=0 \rightarrow a_{2,1}=0 \\
					& \int_{-1}^1 w(x) R_2(x) R_0(x) d x=\int_{-1}^1 \sqrt{1-x^2}\left(x^2+a_{2,1} x+a_{2,0}\right) d x=0 \\
					& \int_{-1}^1\left(x^2\right) \sqrt{1-x^2} d x+a_{2,1} \int_{-1}^1(x) \sqrt{1-x^2} d x+a_{2,0} \int_{-1}^1 \sqrt{1-x^2} d x=0 \rightarrow a_{2,0}=-\frac{1}{4} \rightarrow R_2(x)=x^2-\frac{1}{4}
				\end{aligned}
				$$
				\item Para la obtenci\'on de $R_3(x)$
				$$
				\begin{aligned}
					& \int_{-1}^1 w(x) R_3(x) R_0(x) d x=\int_{-1}^1 \sqrt{1-x^2}\left(x^3+a_{3,2} x^2+a_{3,1} x+a_{3,0}\right) d x=0 \\
					& \int_{-1}^1\left(x^3\right) \sqrt{1-x^2}+a_{3,2} \int_{-1}^1\left(x^2\right) \sqrt{1-x^2}+a_{3,1} \int_{-1}^1(x) \sqrt{1-x^2}+a_{3,0} \int_{-1}^1 \sqrt{1-x^2}=0 \\
					& \rightarrow a_{3,2}(\pi / 8)+(\pi / 2) a_{3,0}=0 \\
					& \int_{-1}^1 w(x) R_3(x) R_1(x) d x=\int_{-1}^1 \sqrt{1-x^2}\left(x^3+a_{3,2} x^2+a_{3,1} x+a_{3,0}\right)(x) d x=0 \\
					& \int_{-1}^1\left(x^4\right) \sqrt{1-x^2} d x+a_{3,2} \int_{-1}^1\left(x^3\right) \sqrt{1-x^2} d x+a_{3,1} \int_{-1}^1\left(x^2\right) \sqrt{1-x^2} d x+a_{3,0} \int_{-1}^1(x) \sqrt{1-x^2} d x \\
					& \rightarrow a_{3,1}=-\frac{1}{2} \\
					&
				\end{aligned}
				$$
				$$\begin{gathered}
					\int_{-1}^1 w(x) R_3(x) R_2(x) d x=\int_{-1}^1 \sqrt{1-x^2}\left(x^3+a_{3,2} x^2+a_{3,1} x+a_{3,0}\right)\left(x^2-\frac{1}{4}\right) d x=0 \\
					\int_{-1}^1\left(x^5\right) \sqrt{1-x^2} d x-\frac{1}{4} \int_{-1}^1\left(x^3\right) \sqrt{1-x^2} d x+a_{3,2} \int_{-1}^1\left(x^4\right) \sqrt{1-x^2} d x-\frac{1}{4} a_{3,2} \int_{-1}^1\left(x^2\right) \sqrt{1-x^2} d x+ \\
					a_{3,1} \int_{-1}^1\left(x^3\right) \sqrt{1-x^2} d x-\frac{1}{4} a_{3,1} \int_{-1}^1(x) \sqrt{1-x^2} d x+a_{3,0} \int_{-1}^1\left(x^2\right) \sqrt{1-x^2} d x-\frac{1}{4} a_{3,0} \int_{-1}^1(1) \sqrt{1-x^2} d x \\
					\rightarrow a_{3,2}=0 \rightarrow R_3(x)=x^3-\frac{1}{2} x
				\end{gathered}$$
				Siguiendo este mismo proceso obtenemos los dem\'as polinomios $R_{n}(x)$, a continuaci\'on se presentan los primeros diez.
				
			\end{fullwidth}
			
			
			$$\begin{aligned}
				& R_0(x)=1 \\
				& R_1(x)=x \\
				& R_2(x)=x^2-\frac{1}{4} \\
				& R_3(x)=x^3-\frac{4}{8} x \\
				& R_4(x)=x^4-\frac{12}{16} x^2+\frac{1}{16} \\
				& R_5(x)=x^5-x^3+\frac{6}{32} x \\
				& R_6(x)=x^6-\frac{80}{64} x^4+\frac{24}{64} x^2+\frac{1}{64} \\
				& R_7(x)=x^7-\frac{192}{128} x^5+\frac{80}{128} x^3-\frac{8}{128} x \\
				& R_8(x)=x^8-\frac{448}{256} x^6+\frac{240}{256} x^4-\frac{40}{256} x^2+\frac{1}{256} \\
				& R_9(x)=x^9-\frac{1024}{512} x^7+\frac{672}{512} x^5-\frac{160}{512} x^3+\frac{10}{512} x \\
				& R_{10}(x)=x^{10}-\frac{2304}{1024} x^8+\frac{1792}{1024} x^6-\frac{560}{1024} x^4+\frac{60}{1024} x^2+\frac{1}{1024}
			\end{aligned}$$
		\end{itemize}
		\subsection*{Obtenci\'on de los polinomios de Chevishev de tercer tipo a partir de la relaci\'on de recurrencia a tres t\'erminos}
		Los polinomios de Chebyshev al ser ortogonales tienen una relaci\'on de recurrrencia a tres t\'erminos dado por la siguiente expresi\'on
		\begin{equation}\label{recurrencia2tipo}
			U_{n+1}(x)+U_{n-1}(x)=2 x U_n(x)
		\end{equation}
		Obteniendo los primeros polinomios tenemos\\
		Para $n=1$
		$$
		\begin{aligned}
			& U_2(x)+U_0(x)=2 x U_1(x) \\
			& U_2(x)=2 x U_1(x)-U_0(x)
		\end{aligned}$$
		Reemplazando $U_1(x)\;\text{y}\; U_0(x)$
		$$
		\begin{aligned}
			& U_2(x)=2 x(2 x)-1 \\
			& U_2(x)=4 x^2-1
		\end{aligned}
		$$
		Para $n=2$
		$$
		\begin{aligned}
			& U_3(x)+U_1(x)=2 x U_2(x) \\
			& U_3(x)=2 x U_2(x)-U_1(x)
		\end{aligned}$$
		Reemplazando $U_2(x)\;\text{y}\; U_1(x)$
		$$
		\begin{aligned}
			& U_3(x)=2 x\left(4 x^2-1\right)-2 x \\
			& U_3(x)=8 x^3-4 x
		\end{aligned}
		$$
		Para $n=3$
		$$
		\begin{aligned}
			& U_4(x)+U_2(x)=2xU_3(x) \\
			& U_4(x)=2 x U_3(x)-U_2(x)
		\end{aligned}$$
		Reemplazando $U_3(x)\;\text{y}\; U_2(x)$
		$$
		\begin{aligned}
			& U_4(x)=2 x\left(8 x^3-4 x\right)-\left(4 x^2-1\right) \\
			& U_4(x)=16 x^4-12 x^2+1
		\end{aligned}
		$$
		\subsection*{Obtenci\'on de los polinomios de Chevishev de tercer tipo a partir de la f\'ormula generadora}
		La funci\'on generadora m\'as simple se puede derivar tomando la parte imaginaria de la identidad
		\begin{equation}\label{generadora2tipo}
			\sum_{n=0}^{\infty} t^n e^{i(n+1) \theta}=\sum_{n=0}^{\infty}\left(t e^{i \theta}\right)^n(e)^{i \theta}=\frac{(e)^{i \theta}}{1-t(e)^{i \theta}} ;|t|<1
		\end{equation}
		De donde se tiene que recordar la identidad $e^{i \theta}=\cos \theta+i \sin \theta$, por cual
		$$
		\sum_{n=0}^{\infty} t^n e^{i(n+1) \theta}=\sum_{n=0}^{\infty}\left[t^n \cos (n+1) \theta+i t^n \sin (n+1) \theta\right]
		$$
		$\mathrm{y}$
		$$
		\frac{(e)^{i \theta}}{1-t(e)^{i \theta}}=\frac{\cos \theta+i \sin \theta}{1-t[\cos \theta+i \sin \theta]}
		$$
		Esta \'ultima expresi\'on la multiplicamos por su conjugado para simplificar:
		$$
		\begin{aligned}
			\frac{\cos \theta+i \sin \theta}{1-t[\cos \theta+i \sin \theta]} & =\left(\frac{\cos \theta+i \sin \theta}{(1-t \cos \theta)+i(t \sin \theta)}\right)\left(\frac{(1-t \cos \theta)-i(t \sin \theta)}{(1-t \cos \theta)-i(t \sin \theta)}\right) \\
			& =\frac{\cos \theta+i \sin \theta+t \cos ^2 \theta+t \sin ^2 \theta}{(1-t \cos \theta)^2-(t \sin \theta)^2} \\
			& =\frac{\cos \theta+i \sin \theta+t}{1-2 t \cos \theta+t^2 \cos ^2 \theta+t^2-t^2 \cos ^2 \theta} \\
			& =\frac{(\cos \theta+t)+(\sin \theta) i}{1-2 t \cos \theta+t^2 \cos ^2 \theta+t^2-t^2 \cos ^2 \theta}
		\end{aligned}
		$$
		Sabiendo que $x=\cos \theta$ entonces,
		$$
		\frac{\cos \theta+i \sin \theta}{1-t[\cos \theta+i \sin \theta]}=\frac{(x+t)+(\sin \theta) i}{1-2 t x+t^2}
		$$
		Igualando las partes imaginarias, se tiene:
		$$
		\sum_{n=0}^{\infty} t^n \sin (n+1) \theta=\frac{\sin \theta}{1-2 t x+t^2}
		$$
		Como $U_n(x)=U_n(\cos \theta)=\displaystyle\frac{\sin (n+1) \theta}{\sin \theta}$
		\begin{equation}\label{mko}
			\displaystyle\sum_{n=0}^{\infty} t^n U_n(\cos \theta)=\frac{1}{1-2 t x+t^2}
		\end{equation}
		En lo adelante, presentamos la obtenci\'on de los polinomios por esta v\'ia:
		$$
		\sum_{n=0}^{\infty} t^n U_n(x)=\frac{1}{1-2 t x+t^2}=w(x, t)
		$$
		Para este proceso nos vamos a auxiliar de la definici\'on de polinomio de Maclaurin, gracias a la cual podemos plantear lo siguiente:
		$$
		U_n(x)=\frac{1}{n !} \frac{\partial^n}{\partial t^n} w(x, t) \equiv \frac{1}{n !} \frac{\partial^n}{\partial t^n}\left(\frac{1}{1-2 t x+t^2}\right)_{t=0}
		$$
		Iniciamos a asignar valores a $n$ :
		$$
		\begin{aligned}
			& n=0 \rightarrow U_0(x)=\frac{1}{0 !} \frac{\partial^0}{\partial t^0}\left(\frac{1}{1-2 t x+t^2}\right)_{t=0}=\frac{1}{0 !}\left[\frac{1}{1-2 x(0)+(0)^2}\right]=1 \\
			& n=1 \rightarrow U_1(x)=\frac{1}{1 !} \frac{\partial^1}{\partial t^1}\left(\frac{1}{1-2 t x+t^2}\right)_{t=0}=\frac{1}{1 !}\left[\frac{(-1)(2 t-2 x)}{\left(1-2 x(t)+(t)^2\right)^2}\right]_{t=0}=2 x \\
			& n=2 \rightarrow U_2(x)=\frac{1}{2 !} \frac{\partial^2}{\partial t^2}\left(\frac{1}{1-2 t x+t^2}\right)_{t=0}=\frac{1}{2 !}\left[\frac{(-2)\left(-3 t^2+6 t x+1-4 x^2\right)}{\left(1-2 x t+t^2\right)^2}\right]_{t=0}=4 x^2-1 \\
			& n=3 \rightarrow U_3(x)=\frac{1}{3 !} \frac{\partial^3}{\partial t^3}\left(\frac{1}{1-2 t x+t^2}\right)_{t=0}=\frac{1}{3 !}\left[(-24 x)\left(1-2 x^2\right)\right]=8 x^3-4 x \\
			& n=4 \rightarrow U_4(x)=\frac{1}{4 !} \frac{\partial^4}{\partial t^4}\left(\frac{1}{1-2 t x+t^2}\right)_{t=0}=\frac{1}{4 !}\left[(-24)\left(-16 x^4+12 x^2-1\right)\right]=16 x^4-12 x^2+1 \\
			& n=5 \rightarrow U_5(x)=\frac{1}{5 !} \frac{\partial^5}{\partial t^5}\left(\frac{1}{1-2 t x+t^2}\right)_{t=0}=32 x^5-32 x^3+6 x
		\end{aligned}
		$$
		\subsection{Chevyshev Tercer tipo}
		Si en la ecuaci\'on de Jacobi \ref{Jacobiequation} tomamos $\alpha=-\displaystyle\frac{1}{2} \wedge \beta=\displaystyle\frac{1}{2}$ tenemos
		$$\begin{aligned}
			& \left(1-x^2\right) y^{\prime \prime}+\left[\displaystyle\frac{1}{2}-\left(-\displaystyle\frac{1}{2}\right)-\left(-\displaystyle\frac{1}{2}+\frac{1}{2}+2\right) x\right] y^{\prime}+\gamma y=0 \\
		\end{aligned}$$
		\begin{equation}\label{chevyseg3tipo}
			\left(1-x^2\right) y^{\prime \prime}+(1-2 x) y^{\prime}+\gamma y=0
		\end{equation}
		Esta ecuaci\'on se conoce como ecuaci\'on de \textsf{Chevyshed de tercer tipo}, y a las soluciones de la misma \textsf{polinomios de Chevyshed de tercer tipo}, donde el par\'ametro gamma es
		$$\begin{aligned}
			\gamma & =n\left(n-\displaystyle\frac{1}{2}+\displaystyle\frac{1}{2}+1\right) ; n=0,1,2,3 \ldots  \\
			\gamma & =n(n+1) ; n=0,1,2,3 \ldots
		\end{aligned}$$
		Para analizar los puntos ordinarios y singulares, expresamos \ref{chevyseg3tipo} a la forma normal
		$$y^{\prime \prime}+\displaystyle\frac{1-2 x}{1-x^2} y^{\prime}+\displaystyle\frac{\gamma}{1-x^2} y=0$$
		donde $$P(x)=\displaystyle\frac{1-2 x}{1-x^2} \quad \text{y}\quad Q(x)=\frac{\gamma}{1-x^2}$$
		podemos ver que los puntos singulares de $ P(x)$ y $Q(x)$  son  $x_0= \pm 1$, de manera que ahora queremos determinar si estos puntos son singulares regulares, aplicando dicha definici\'on para $x_{0}=1$
		
		$$\begin{aligned}
			& p(x)=\left(x-x_0\right) P(x) \\
			& p(x)=(x-1) \frac{1-2 x}{1-x^2} \\
			& p(x)=(x-1) \frac{1-2 x}{-(x-1)(x+1)} \\
			& p(x)=-\frac{1-2 x}{1+x}
		\end{aligned}$$
		$$\begin{aligned}
			& q(x)=\left(x-x_0\right)^2 Q(x) \\
			& q(x)=(x-1)^2 \frac{\gamma}{1-x^2}\\
			& q(x)=(x-1)^2 \frac{\gamma}{-(x-1)(x+1)} \\
			& q(x)=\frac{\gamma(x-1)}{-(x+1)} \\
			& q(x)=\frac{\gamma(1-x)}{x+1}
		\end{aligned}$$
		$x_0=1$ punto ordinario de $p(x)$ y $q(x)$, entonces es punto singular regular de $P(x)$ y $Q(x)$.
		De igual manera analizamos $x_{0}=-1$
		$$\begin{aligned}
			& p(x)=\left(x-x_0\right) P(x) \\
			& p(x)=(x+1) \displaystyle\frac{1-2 x}{1-x^2} \\
			& p(x)=(x+1) \displaystyle\frac{1-2 x}{-(x-1)(x+1)} \\
			& p(x)=-\displaystyle\frac{1-2 x}{x-1}
		\end{aligned}$$
		$$\begin{aligned}
			& q(x)=\left(x-x_0\right)^2 Q(x) \\
			& q(x)=(x+1)^2 \displaystyle\frac{\gamma}{1-x^2} \\
			& q(x)=(x+1)^2\displaystyle \frac{\gamma}{-(x-1)(x+1)} \\
			& q(x)=\displaystyle\frac{\gamma(x+1)}{1-x}
		\end{aligned}$$
		Por lo que tambi\'en $x_0=-1$ punto ordinario de $p(x)$ y $q(x)$, entonces es punto singular regular de $P(x)$ y $Q(x)$.
		Como $x_0=1$ y $x_0=-1$ son puntos singulares, por el teorema de \ref{Teorema de Frobenius} existe una soluci\'on de la forma \ref{solucionfrobenius}.\\
		
		
		
		Tomando la soluci\'on de la forma $y=\displaystyle\sum_{v=0}^{\infty} a_v\left(x-x_0\right)^{v+r}$ con $x_{0}=1$, derivando y reemplazando en \ref{chevyseg3tipo}
		$$\begin{aligned}
			& \left(1-x^2\right)\displaystyle \sum_{v=0}^{\infty} a_v(v+r-1)(v+r)(x-1)^{v+r-2}+(1-2 x)\displaystyle \sum_{v=0}^{\infty} a_v(v+r)(x-1)^{v+r-1}+\gamma\displaystyle \sum_{v=0}^{\infty} a_v(x-1)^{v+r}=0
		\end{aligned}$$
		\begin{multline*}
			-(x+1)(x-1) \displaystyle\sum_{v=0}^{\infty} a_v(v+r-1)(v+r)(x-1)^{v+r-2}+[-(2(x-1)+1)]\displaystyle \sum_{v=0}^{\infty} a_v(v+r)(x-1)^{v+r-1}+ \\
			\gamma\displaystyle \sum_{v=0}^{\infty} a_v(x-1)^{v+r}=0 \\
			-(x+1) \displaystyle\sum_{v=0}^{\infty} a_v(v+r-1)(v+r)(x-1)^{v+r-1}-2(x-1)\displaystyle \sum_{v=0}^{\infty} a_v(v+r)(x-1)^{v+r-1}-\displaystyle\sum_{v=0}^{\infty} a_v(v+r)(x-1)^{v+r-1}+ \\
			\gamma\displaystyle \sum_{v=0}^{\infty} a_v(x-1)^{v+r}=0\\
			-(x+1)\displaystyle \sum_{v=0}^{\infty} a_v(v+r-1)(v+r)(x-1)^{v+r-1}+\displaystyle\sum_{v=0}^{\infty}-2 a_v(v+r)(x-1)^{v+r}-\displaystyle\sum_{v=0}^{\infty} a_v(v+r)(x-1)^{v+r-1}+ \\
			\displaystyle\sum_{v=0}^{\infty} \gamma a_v(x-1)^{v+r}=0
		\end{multline*}
		Para simplificar la expresi\'on llamaremos $B=\displaystyle\sum^{\infty} a_v(v+r-1)(v+r)(x-1)^{v+r-1}$ de manera que
		$$\begin{aligned}
			& -r a_0(x-1)^{r-1}-(x+1) B+\displaystyle\sum_{v=0}^{\infty}\left[(\gamma-2(v+r)) a_v\right](x-1)^{v+r}-\displaystyle\sum_{v=1}^{\infty} a_v(v+r)(x-1)^{v+r-1}=0 \\
			& -r a_0(x-1)^{r-1}-(x+1) B+\displaystyle\sum_{v=0}^{\infty}\left[(\gamma-2(v+r)) a_v\right](x-1)^{v+r}-\displaystyle\sum_{v=0}^{\infty} a_{v+1}(v+1+r)(x-1)^{v+1+r-1}=0
		\end{aligned}$$
		
		$$\begin{aligned}
			& -r a_0(x-1)^{r-1}-(x+1) B+\displaystyle\sum_{v=0}^{\infty}\left[(\gamma-2(v+r)) a_v\right](x-1)^{v+r}-\displaystyle\sum_{v=0}^{\infty} a_{v+1}(v+r+1)(x-1)^{v+r}=0
		\end{aligned}$$
		\begin{equation}\label{b}
			-r a_0(x-1)^{r-1}-(x+1) B+\displaystyle\sum_{v=0}^{\infty}\left[(\gamma-2(v+r)) a_v-(v+r+1) a_{v+1}\right](x-1)^{v+r}=0
		\end{equation}
		Simplificando la expresi\'on que contiene $B$
		$$\begin{aligned}
			-(x+1) B & =(-x B+B)-B-B \\
			& =-(x-1) B-2 B \\
			& =-(x-1)\displaystyle \sum_{v=0}^{\infty} a_v(v+r-1)(v+r)(x-1)^{v+r-1}-2\displaystyle \sum_{v=0}^{\infty} a_v(v+r-1)(v+r)(x-1)^{v+r-1} \\
			& =-2 a_0 r(r-1)(x-1)^{r-1}-\displaystyle\sum_{v=0}^{\infty} a_v(v+r-1)(v+r)(x-1)^{v+r}-2\displaystyle \sum_{v=1}^{\infty} a_v(v+r-1)(v+r)(x-1)^{v+r-1}
		\end{aligned}$$
		\begin{multline*}
			=-2 a_0 r(r-1)(x-1)^{r-1}-\displaystyle\sum_{v=0}^{\infty} a_v(v+r-1)(v+r)(x-1)^{v+r} \\
			-2\displaystyle \sum_{v=1-1}^{\infty} a_{v+1}(v+1+r-1)(v+1+r)(x-1)^{v+1+r-1}
		\end{multline*}
		$$\begin{aligned}
			& =-2 a_0 r(r-1)(x-1)^{r-1}-\displaystyle\sum_{v=0}^{\infty} a_v(v+r-1)(v+r)(x-1)^{v+r}-2\displaystyle \sum_{v=0}^{\infty} a_{v+1}(v+r)(v+r+1)(x-1)^{v+r} \\
			-(x+1) B& =-2 a_0 r(r-1)(x-1)^{r-1}-\displaystyle\sum_{v=0}^{\infty}\left[(v+r)(v+r-1) a_v+2(v+r)(v+r+1) a_{v+1}\right](x-1)^{v+r}
		\end{aligned}$$
		Reemplando esta expresi\'on en \ref{b}
		$$\begin{aligned}
			& -r a_0(x-1)^{r-1}-2 a_0 r(r-1)(x-1)^{r-1}-\displaystyle\sum_{v=0}^{\infty}\left[(v+r)(v+r-1) a_v+2(v+r)(v+r+1) a_{v+1}\right](x-1)^{v+r} \\
			& +\displaystyle\sum_{v=0}^{\infty}\left[(\gamma-2(v+r)) a_v-(v+r+1) a_{v+1}\right](x-1)^{v+r}=0 \\
			& (-1-2(r-1)) r a_0(x-1)^{r-1}\\
			& +\displaystyle\sum_{v=0}^{\infty}\left[(\gamma-2(v+r)-(v+r)(v+r-1)) a_v+\left(-(v+r+1)+(-2(v+r)(v+r+1)) a_{v+1}\right](x-1)^{v+r}=0\right. \\
			& (-1-2 r+2) r a_0(x-1)^{r-1}+\displaystyle\sum_{v=0}^{\infty}\left[(\gamma-(2+(v+r-1))(v+r)) a_v-(1+2(v+r))(v+r+1) a_{v+1}\right](x-1)^{v+r}=0 \\
			& (1-2 r) r a_0(x-1)^{r-1}+\displaystyle\sum_{v=0}^{\infty}\left[(\gamma-(v+r+1)(v+r)) a_v-2\left(\frac{1}{2}+v+r\right)(v+r+1) a_{v+1}\right](x-1)^{v+r}=0
		\end{aligned}$$
		
		Igualando los coeficientes a cero, tenemos la ecuaci\'on indicial
		$$\begin{aligned}
			& (1-2 r) r=0 \quad a_0 \neq 0 \\
			& r=0 \wedge 1-2 r=0 \\
			& r=0 \wedge r=\frac{1}{2}
		\end{aligned}$$
		y la relaci\'on de recurrencia dada por
		$$\begin{array}{lc}
			(\gamma-(v+r+1)(v+r)) a_v-2\left(\displaystyle\frac{1}{2}+v+r\right)(v+r+1) a_{v+1} \\
			(\gamma-(v+r+1)(v+r)) a_v=2\left(\displaystyle\frac{1}{2}+v+r\right)(v+r+1) a_{v+1} \\
			a_{v+1}=\displaystyle\frac{\gamma-(v+r+1)(v+r)}{2\left(\displaystyle\frac{1}{2}+v+r\right)(v+r+1)} a_v
		\end{array}$$
		para $r=0$
		$$\begin{gathered}
			a_{v+1}=\displaystyle\frac{\gamma-(v+0+1)(v+0)}{2\left(\frac{1}{2}+v+0\right)(v+0+1) a_{v+1}} a_v \\
			a_{v+1}=\displaystyle\frac{\gamma-v(v+1)}{2\left(\frac{1}{2}+v\right)(v+1)} a_v \quad
			\forall v \geq 0
		\end{gathered}$$
		Denotando
		\begin{equation}\label{yv}
			\displaystyle\gamma_{v}=v(v+1)
		\end{equation} la expresi\'on anterior queda como
		\begin{equation}\label{r0}
			a_{v+1}=\displaystyle\frac{\gamma_n-\gamma_v}{2\left(v+\frac{1}{2}\right)(v+1)} a_v
		\end{equation}
		Ahora daremos valores a $v$
		$$\begin{aligned}
			& \text{si}\, v=0\\
			& a_{0+1}=\displaystyle\frac{\gamma_n-\gamma_0}{2\left(0+\frac{1}{2}\right)(0+1)} a_0 \\
			& a_1=\frac{\gamma_n-\gamma_0}{2\left(0+\frac{1}{2}\right)(1)} a_0\\
			& \text { Si } v=1 \\
			& a_{1+1}=\displaystyle\displaystyle\frac{\gamma_n-\gamma_1}{2\left(1+\frac{1}{2}\right)(1+1)} a_1 \\
			& a_2=\displaystyle\frac{\left(\gamma_n-\gamma_0\right)\left(\gamma_n-\gamma_1\right)}{2(2)\left(1+\frac{1}{2}\right)(2)(1)\left(0+\frac{1}{2}\right)} a_0 \\
			& a_2=\displaystyle\frac{\left(\gamma_n-\gamma_0\right)\left(\gamma_n-\gamma_1\right)}{2^2(2) !\left(0+\frac{1}{2}\right)\left(1+\frac{1}{2}\right)} a_0 \\
			& \text { Si } v=2 \\
			& a_{2+1}=\displaystyle\frac{\gamma_n-\gamma_2}{2\left(2+\frac{1}{2}\right)(2+1)} a_2 \\
			& a_3=\displaystyle\frac{\left(\gamma_n-\gamma_0\right)\left(\gamma_n-\gamma_1\right)\left(\gamma_n-\gamma_2\right)}{2^2(2) ! 2(3)\left(0+\frac{1}{2}\right)\left(1+\frac{1}{2}\right)\left(2+\frac{1}{2}\right)} a_0 \\
			& a_3=\displaystyle\frac{\left(\gamma_n-\gamma_0\right)\left(\gamma_n-\gamma_1\right)\left(\gamma_n-\gamma_2\right)}{2^3(3) !\left(0+\frac{1}{2}\right)\left(1+\frac{1}{2}\right)\left(2+\frac{1}{2}\right)} a_0 \\
			\vdots\\
			& \text { Si } v=v-1
		\end{aligned}$$
		\begin{equation}\label{av}
			a_v=\displaystyle\frac{\left(\gamma_n-\gamma_0\right)\left(\gamma_n-\gamma_1\right) \ldots \ldots\left(\gamma_n-\gamma_{v-1}\right)}{2^v(v) !\left(0+\frac{1}{2}\right)\left(1+\frac{1}{2}\right) \ldots \ldots\left(v-1+\frac{1}{2}\right)} a_0
		\end{equation}
		Ahora simplificaremos la expresi\'on anterior, para el numerador tenemos
		$$\begin{aligned}
			\gamma_n-\gamma_{v-1} & =n(n+1)-v(v-1) \\
			& =n^2+n-v^2+v \\
			& =\left(n^2-v^2\right)+(n+v) \\
			& =(n+v)(n-v)+(n+v)
		\end{aligned}$$
		\begin{eqnarray}\label{yv-}
			\gamma_n-\gamma_{v-1}& =&(n-v+1)(n+v)
		\end{eqnarray}
		Para el denominador, usamos la propiedad  de la funci\'on Gamma $\frac{\Gamma(x+v)}{\Gamma(x)}=x(x+1)(x+2) \ldots(x+v-1)$, tomando $x=1/2$
		$$\displaystyle\frac{\Gamma\left(\frac{1}{2}+v\right)}{\Gamma\left(\frac{1}{2}\right)}=\frac{1}{2}\left(\frac{1}{2}+1\right)\left(\frac{1}{2}+2\right) \ldots\left(\frac{1}{2}+v-1\right)$$
		Sustituyendo en la expresio\'on de recurrencia \ref{av}
		$$
		a_v=\displaystyle\frac{(n-v+1)(n+v)}{2^v(v) ! \displaystyle\frac{\Gamma\left(\frac{1}{2}+v\right)}{\Gamma\left(\frac{1}{2}\right)}} a_0$$
		Simplificando y expresando el numerador en t\'ermino de productoria
		$$a_v=\displaystyle\frac{\Gamma\left(\frac{1}{2}\right)}{2^v(v) ! \Gamma\left(\frac{1}{2}+v\right)} a_0\displaystyle \prod_{k=1}^v(n-k+1)(n+k)$$
		Ahora expresando la productoria en t\'ermino de la funci\'on gamma
		$$a_v=\displaystyle\frac{\Gamma\left(\frac{1}{2}\right)}{2^v(v) ! \Gamma\left(\frac{1}{2}+v\right)} a_0 \frac{n !}{(n-v) !} \frac{\Gamma(n+1+v)}{\Gamma(n+1)}$$
		Por definici\'on de n\'umero combinatorio
		$$a_v=\left(\begin{array}{l}
			n \\
			v
		\end{array}\right) \displaystyle\frac{\Gamma\left(\frac{1}{2}\right) \Gamma(n+1+v)}{2^v \Gamma\left(\frac{1}{2}+v\right) \Gamma(n+1)} a_0$$
		Sustituyendo este valor en la suposici\'on de la soluci\'on tenemos
		$$y(x)=a_0+\displaystyle\sum_{v=1}^{\infty} a_v(x-1)^v$$
		Ahora si definimos $$V(x)=a_0 y(x) \wedge a_0=1$$
		$$\begin{gathered}
			a_0 y(x)=a_0+a_0 \displaystyle\sum_{v=1}^{\infty}\left(\begin{array}{l}
				n \\
				v
			\end{array}\right) \frac{\Gamma\left(\frac{1}{2}\right) \Gamma(n+1+v)}{2^v \Gamma\left(\frac{1}{2}+v\right) \Gamma(n+1)}(x-1)^v \\
			a_0 y(x)=a_0\left[1+\displaystyle\sum_{v=1}^n\left(\begin{array}{l}
				n \\
				v
			\end{array}\right) \frac{\Gamma\left(\frac{1}{2}\right) \Gamma(n+1+v)}{\Gamma\left(\frac{1}{2}+v\right) \Gamma(n+1)}\left(\frac{x-1}{2}\right)^v\right]
		\end{gathered}$$
		Simplificando
		$$\begin{aligned}
			& a_0 y(x)=\left[1+\displaystyle\sum_{v=1}^n\left(\begin{array}{l}
				n \\
				v
			\end{array}\right) \frac{\Gamma\left(\frac{1}{2}\right) \Gamma(n+1+v)}{\Gamma\left(\frac{1}{2}+v\right) \Gamma(n+1)}\left(\frac{x-1}{2}\right)^v\right] \\
			& a_0 y(x)=\displaystyle\sum_{v=0}^n\left(\begin{array}{l}
				n \\
				v
			\end{array}\right) \frac{\Gamma\left(\frac{1}{2}\right) \Gamma(n+1+v)}{\Gamma\left(\frac{1}{2}+v\right) \Gamma(n+1)}\left(\frac{x-1}{2}\right)^v
		\end{aligned}$$
		
		$$\begin{aligned}
			& a_0 y(x)=\Gamma\left(\frac{1}{2}\right) \displaystyle\sum_{v=0}^n\left(\begin{array}{l}
				n \\
				v
			\end{array}\right) \frac{\Gamma(n+1+v)}{\Gamma\left(\frac{1}{2}+v\right) \Gamma(n+1)}\left(\frac{x-1}{2}\right)^v \\
			& V_n(x)=\Gamma\left(\frac{1}{2}\right)\displaystyle \sum_{v=0}^n \frac{n ! \Gamma(n+1+v)}{v !(n-v) ! \Gamma\left(\frac{1}{2}+v\right) \Gamma(n+1)}\left(\frac{x-1}{2}\right)^v
		\end{aligned}$$
		Simplificando por la propiedad gamma
		$$V_n(x)=\Gamma\left(\frac{1}{2}\right)\displaystyle \sum_{v=0}^n \frac{(n+v) !}{v !(n-v) ! \Gamma\left(\frac{1}{2}+v\right)}\left(\frac{x-1}{2}\right)^v$$
		Ahora vamos a obtener los primeros polinomios a partir de la expresi\'on anterior.
		$$
		\begin{aligned}
			& \text { Si } n=0 \\
			& V_{(x)}=\displaystyle\Gamma(\left(\frac{1}{2}\right)) \displaystyle\sum_{v=0}^0 \frac{(0+v)!}{v!(0-v) ! \Gamma\left(\frac{1}{2}+v\right)}\left(\frac{x-1}{2}\right)^v \\
			& V_{0}(x)=\Gamma\left(\frac{1}{2}\right) \frac{1}{\Gamma\left( \frac{1}{2}\right)}=1
		\end{aligned}
		$$
		si $n=1$
		$$
		\begin{aligned}
			& V_1(x)=\displaystyle\Gamma\left(\frac{1}{2}\right)\displaystyle \sum_{v=0}^1 \frac{(1+v) !}{v!(1-v)! \Gamma\left(\frac{1}{2}+v\right)}\left(\frac{x-1}{2}\right)^v \\
			& =\displaystyle\Gamma\left(\frac{1}{2}\right)\bigg[\frac{1!}{0!(1!)\Gamma(\frac{1}{2})}(1)+\frac{2!}{(1!)(0!)\Gamma(\frac{1}{2}+1)}(\frac{x-1}{2})\bigg] \\
			& =1+\frac{2!\Gamma(\frac{1}{2})}{\left(\frac{1}{2}\right)\Gamma(\frac{1}{2})}\left(\frac{x-1}{2}\right) \\
			& V_1(x)=1+2(x-1)=2 x-1 \\
			&
		\end{aligned}
		$$
		Si $n=2$
		$$
		\begin{aligned}
			& V_{2}(x)=\Gamma(\frac{1}{2})\displaystyle\sum_{v=0}^{2}\frac{(2+v)!}{v!(2-v)!\Gamma(\frac{1}{2}+v)}(\frac{x-1}{2})^{v}\\
			&=\Gamma(\frac{1}{2})\bigg[\frac{2!}{0!(2!)\Gamma(\frac{1}{2})}(1)+\frac{3!}{1!(1!)\Gamma(\frac{1}{2}+1)}(\frac{x-1}{2})+\frac{4!}{2!(0!)\Gamma(\frac{1}{2}+2)}(\frac{x-1}{2})^{2} \bigg]\\
			&=1+\frac{3!\Gamma(\frac{1}{2})}{\frac{1}{2}\Gamma(\frac{1}{2})}(\frac{x-1}{2})+\frac{4!\Gamma(\frac{1}{2})}{2!(\frac{1}{2})(\frac{3}{2})\Gamma(\frac{1}{2})}(\frac{x-1}{2})^{2}\\
			&=1+6x-6+4x^{2}-8x+4\\
			&V_{2}(x)=4x^{2}-2x-1
		\end{aligned}
		$$
		Si $n=3$
		$$
		\begin{aligned}
			& V_{3}(x)=\Gamma(\frac{1}{2})\displaystyle\sum_{v=0}^{3}\frac{(3+v)!}{v!(3-v)!\Gamma(\frac{1}{2}+v)}(\frac{x-1}{2})^{v}\\
			&=\Gamma(\frac{1}{2})\bigg[\frac{3!}{0!(3!)\Gamma(\frac{1}{2})}(1)+\frac{4!}{1!(2!)\Gamma(\frac{1}{2}+1)}(\frac{x-1}{2})+\frac{5!}{2!(1!)\Gamma(\frac{1}{2}+2)}(\frac{x-1}{2})^{2} +\frac{6!}{3!(0!)\Gamma(\frac{1}{2}+3)}(\frac{x-1}{2})^{3} \bigg]\\
			&=1+\frac{4!\Gamma(\frac{1}{2})}{2!\frac{1}{2}\Gamma(\frac{1}{2})}(\frac{x-1}{2})+\frac{5!\Gamma(\frac{1}{2})}{2!(\frac{1}{2})(\frac{3}{2})\Gamma(\frac{1}{2})}(\frac{x-1}{2})^{2}
			+\frac{6!\Gamma(\frac{1}{2})}{3!(5/2)(3/2)(1/2)\Gamma(\frac{1}{2})}(\frac{x-1}{2})^{3} \\
			&V_{3}(x)=8x^{3}-4x^{2}-4x+1
		\end{aligned}
		$$
		Desarrollando ese mismo proceso obtenemos los dem\'as polinomios, acontinuaci\'on se presentan algunos m\'as
		$$\begin{aligned}
			& V_0(x)=1 \\
			& V_1(x)=2 x-1 \\
			& V_2(x)=4 x^2-2 x-1 \\
			& V_3(x)=8 x^3-4 x^2-4 x+1 \\
			& V_4(x)=16 x^4-8 x^3-12 x^2+4 x+1 \\
			& V_5(x)=32 x^5-16 x^4-32 x^3+12 x^2+6 x-1 \\
			& V_6(x)=64 x^6-32 x^5-80 x^4+32 x^3+24 x^2-6 x-1 \\
			& V_7(x)=128 x^7-64 x^6-192 x^5+80 x^4+80 x^3-24 x^2-8 x+1 \\
			& V_8(x)=256 x^8-128 x^7-448 x^6+192 x^5+240 x^4-80 x^3-40 x^2+8 x+1
		\end{aligned}$$
		\subsection*{Obtenci\'on de los polinomios de Chevishev de tercer tipo como proceso de Gramm-Schmidt}
		Los polinomios de Chebyshev de 3er tipo, se obtienen utilizando el proceso de ortogonalizaci\'on de Gram-Schmidt para polinomios en el dominio $(-1,1)$ con la funci\'on de peso $w(x)=$ $\sqrt{(1+x) /(1-x)}$.\\
		Los polinomios obtenidos con este proceso son de la forma
		$R_0(x)=1, R_1(x)=x+a_{1,0}, \quad R_2(x)=x^2+a_{2,1} x+a_{2,0}$, $R_3(x)=x^3+a_{3,2} x^2+a_{3,1} x+a_{3,0}$, y continuando de esta manera hasta $R_n(x)$, donde las constantes o coeficientes $a_{n, m}$ son determinadas por la condici\'on de ortogonalidad, de un polinomio dado con los restantes polinomios de grado menor al mismo. A continuaci\'on desarrollamos el proceso
		\begin{itemize}
			\item Vamos a obtener a $R_1(x)=x+a_{1,0}$, donde debemos encontrar el valor de la constante $\mathbf{a_{1,0}}$\\
			Como $R_0(x)$ y  $R_1(x)$ son ortogonales
			$$
			\left\langle R_0(x), R_1(x)\right\rangle_w=\int_{-1}^1 w(x) R_0(x) R_1(x) d x=\int_{-1}^1 w(x)\left(x+a_{1,0}\right) d x=\int_{-1}^1 x w(x) d x+a_{1,0} \int_{-1}^1 w(x) d x=0
			$$
			El valor de las integrales anteriores est\'an dados por
			\begin{eqnarray}\label{gramschmint}
				% \nonumber % Remove numbering (before each equation)
				\int_{-1}^1 x w(x) d x= &=& \displaystyle\frac{\pi}{2} \\
				\int_{-1}^1 w(x) d x &=& \pi
			\end{eqnarray}
			Reemplazando estos valores obtenidos
			\begin{eqnarray*}
				% \nonumber % Remove numbering (before each equation)
				\displaystyle\frac{\pi}{2}+a_{1,0} \pi&=&0
			\end{eqnarray*}
			\text{Despejando tenemos}
			\begin{eqnarray*}
				a_{1,0}&=&-\displaystyle\frac{1}{2}
			\end{eqnarray*}
			Por lo que el polinomio es
			\begin{equation*}
				R_1(x)=x-\displaystyle\frac{1}{2}
			\end{equation*}
			\item Ahora tomamos $R_2(x)=x^2+a_{2,1} x+a_{2,0}$, donde las constantes $a_{2,1}, a_{2,0}$ se obtienen del hecho de que $R_2(x)$ es ortogonal a $R_1(x)$ y $R_0(x)$, es decir
			\begin{eqnarray}\label{int1}
				% \nonumber % Remove numbering (before each equation)
				\left\langle R_2(x), R_1(x)\right\rangle_w&=&\int_{-1}^1 w(x) R_2(x) R_1(x) d x=\int_{-1}^1 w(x)\left(x^2+a_{2,1} x+a_{2,0}\right)(x-1 / 2) d x=0
			\end{eqnarray}
			\begin{eqnarray}\label{int2}
				\left\langle R_2(x), R_0(x)\right\rangle_w&=&\int_{-1}^1 w(x) R_2(x) R_0(x) d x=\int_{-1}^1 w(x)\left(x^2+a_{2,1} x+a_{2,0}\right) d x=0
			\end{eqnarray}
			Al resolver la primera integral dada en \ref{int1}
			\begin{eqnarray*}
				% \nonumber % Remove numbering (before each equation)
				\left\langle R_2(x), R_1(x)\right\rangle_w&=&\displaystyle\int_{-1}^1 w(x)\left(x^2+a_{2,1} x+a_{2,0}\right)(x-1 / 2) d x=0 \\
				&=& \displaystyle\int_{-1}^1 x^2(x-1 / 2) w(x) d x+a_{2,1} \displaystyle\int_{-1}^1 x(x-1 / 2) w(x) d x+a_{2,0}\displaystyle \int_{-1}^1(x-1 / 2) w(x) d x=0
			\end{eqnarray*}
			Resolviendo cada una de las integrales anteriores
			\begin{eqnarray*}
				% \nonumber % Remove numbering (before each equation)
				\displaystyle\int_{-1}^1 x^2(x-1 / 2) w(x) d x&=&\displaystyle\int_{-1}^1 x^2(x-1 / 2) \sqrt{(1+x) /(1-x)} d x=\displaystyle\frac{\pi}{8} \\
				\displaystyle\int_{-1}^1 x(x-1 / 2) w(x) d x&=&\displaystyle \int_{-1}^1 x(x-\displaystyle 1 / 2) \sqrt{(1+x) /(1-x)} d x=\displaystyle \frac{\pi}{4} \\
				\displaystyle \int_{-1}^1(x-1 / 2) w(x) d x&=&\displaystyle\int_{-1}^1(x-1 / 2) \sqrt{(1+x) /(1-x)} d x=0
			\end{eqnarray*}
			Sustituyendo los valores obtenidos
			\begin{eqnarray*}
				% \nonumber % Remove numbering (before each equation)
				\frac{\pi}{8}+a_{2,1} \frac{\pi}{4}&=&0
			\end{eqnarray*}
			Resolviendo la ecuaci\'on
			\begin{eqnarray*}
				a_{2,1}&=&-\frac{1}{2}
			\end{eqnarray*}
			Ahora vamos a resolver la integral dada en la expresi\'on \ref{int2}, donde usaremos los resultados obtenidos en \ref{gramschmint} y el valor de la integral
			\begin{eqnarray}\label{gramschmint2}
				% \nonumber % Remove numbering (before each equation)
				\displaystyle\int_{-1}^1 x^2 w(x) d x&=&\int_{-1}^1 x^2 \sqrt{(1+x) /(1-x)} d x=\frac{\pi}{2}
			\end{eqnarray}
			Con estos valores obtenidos tenemos la siguiente ecuaci\'on lineal
			\begin{equation*}
				\pi / 2+a_{2,1} \pi / 2+a_{2,0} \pi=0
			\end{equation*}
			Reemplazando el valor de $a_{2,1}$, llegamos a
			\begin{equation*}
				a_{2,0}=-\frac{1}{4}
			\end{equation*}
			As\'i obteniendo
			\begin{equation*}
				R_2(x)=x^2-x / 2-1 / 4
			\end{equation*}
		\end{itemize}
		Si continuamos de esta forma, tenemos:
		$$
		\begin{aligned}
			& R_3(x)=x^3-\frac{1}{2} x^2-\frac{1}{2} x+\frac{1}{8} \\
			& R_4(x)=x^4-\frac{1}{2} x^3-\frac{3}{4} x^2+\frac{1}{4} x+\frac{1}{16}  \\
			& R_5(x)=x^5-\frac{1}{2} x^4-x^3+\frac{3}{8} x^2+\frac{3}{16} x-\frac{1}{32} . \\
			& R_6(x)=x^6-\frac{1}{2} x^5-\frac{5}{4} x^4+\frac{1}{2} x^3+\frac{3}{8} x^2-\frac{3}{32} x-\frac{1}{64} . \\
			& R_7(x)=x^7-\frac{1}{2} x^6-\frac{3}{2} x^5+\frac{5}{8} x^4+\frac{5}{8} x^3-\frac{3}{16} x^2-\frac{1}{16} x+\frac{1}{128} . \\
			& R_8(x)=x^8-\frac{1}{2} x^7-\frac{7}{4} x^6+\frac{3}{4} x^5+\frac{15}{16} x^4-\frac{5}{16} x^3-\frac{5}{32} x^2+\frac{1}{22} x+\frac{1}{256} .
		\end{aligned}
		$$
		\subsection*{Obtenci\'on de los polinomios de Chevishev de tercer tipo a partir de la relaci\'on de recurrencia a tres t\'erminos}
		Sabemos que la relaci\'on de recurrencia est\'a dada por (relacionarla con la parte de ortogonalidad.citar)
		\begin{equation}\label{3chevishevrelacion}
			V_{n+1}(x)=2 x V_n(x)-V_{n-1}(x)
		\end{equation}
		para utilizar esta expresi\'on debemos conocer los primeros dos polinomios de Chevishev, y est\'an dado por las expresiones (citar en la parte obtenida de la ecuacion diferencial)\\
		$\operatorname{Para} n=1$
		$$
		V_2(x)=2 x V_1(x)-V_0(x)=2 x(2 x-1)-1=4 x^2-2 x-1
		$$
		$\operatorname{Para} n=2$
		$$
		V_3(x)=2 x V_2(x)-V_1(x)=2 x\left(4 x^2-2 x-1\right)-2 x+1=8 x^3-4 x^2-4 x+1
		$$
		$\operatorname{Para} n=3$
		$$
		V_4(x)=2 x V_3(x)-V_2(x)=2 x\left(8 x^3-4 x^2-4 x+1\right)-4 x^2+2 x+1=16 x^4-8 x^3-12 x^2+4 x+1
		$$
		Asi continua para los demas polinomios restantes...
		
		\subsection*{Obtenci\'on de los polinomios de Chevishev de tercer tipo a partir de la f\'ormula generadora}
		\textcolor[rgb]{1.00,0.00,0.00}{Ver aplicacion, video(Carlos felix y milane)}
		\subsection{Chevyshev Cuarto tipo}
		De la ecuaci\'on diferencial de Jacobi definida en \ref{Jacobiequation}, tomando $\alpha=\frac{1}{2} \wedge \beta=-\frac{1}{2}$ se transforma en la ecuaci\'on de Chevyshev de cuarto tipo
		\begin{equation}\label{chevysed4}
			\left(1-x^2\right) y^{\prime \prime}-(1+2 x) y^{\prime}+\gamma y=0
		\end{equation}
		con $\gamma=n(n+1) ; n=0,1,2,3 \ldots$. $x_{0}=\pm 1$ son puntos singulares regulares, el teorema \ref{Teorema de Frobenius} garantiza una soluci\'on de la forma \ref{solucionfrobenius}, dada por
		$$y=\displaystyle\sum_{v=0}^{\infty} a_v\left(x-x_0\right)^{v+r} \quad x_0=1$$
		Derivando y reemplazando en \ref{chevysed4}
		$$\begin{aligned}
			& \left(1-x^2\right) \displaystyle\sum_{v=0}^{\infty} a_v(v+r-1)(v+r)(x-1)^{v+r-2}-(1+2 x)\displaystyle \sum_{v=0}^{\infty} a_v(v+r)(x-1)^{v+r-1}+\gamma \displaystyle\sum_{v=0}^{\infty} a_v(x-1)^{v+r}=0 \\
			&\text{Podemos escribir $ -(1+2 x)=-2(x-1)-3$ y $\left(1-x^2\right)=-(x+1)(x-1)$}\\
			& -(x+1)(x-1) \displaystyle\sum_{v=0}^{\infty} a_v(v+r-1)(v+r)(x-1)^{v+r-2}+[-2(x-1)-3]\displaystyle \sum_{v=0}^{\infty} a_v(v+r)(x-1)^{v+r-1}+\gamma \displaystyle\sum_{v=0}^{\infty} a_v(x-1)^{v+r}=0
		\end{aligned}$$
		\begin{multline*}
			-(x+1)\displaystyle \sum_{v=0}^{\infty} a_v(v+r-1)(v+r)(x-1)^{v+r-1}-2(x-1)\displaystyle \sum_{v=0}^{\infty} a_v(v+r)(x-1)^{v+r-1}-3 \displaystyle\sum_{v=0}^{\infty} a_v(v+r)(x-1)^{v+r-1}+ \\
			\gamma \displaystyle\sum_{v=0}^{\infty} a_v(x-1)^{v+r}=0 \\
			-(x+1)\displaystyle \sum_{v=0}^{\infty} a_v(v+r-1)(v+r)(x-1)^{v+r-1}+\displaystyle\sum_{v=0}^{\infty}-2 a_v(v+r)(x-1)^{v+r}-3\displaystyle \sum_{v=0}^{\infty} a_v(v+r)(x-1)^{v+r-1}+ \\
			\displaystyle\sum_{v=0}^n \gamma a_v(x-1)^{v+r}=0
		\end{multline*}
		Ahora sea $$B=\displaystyle \sum_{v=0}^{\infty} a_v(v+r-1)(v+r)(x-1)^{v+r-1}$$
		$$\begin{aligned}
			& -3 r a_0(x-1)^{r-1}-(x+1) B+\displaystyle\sum_{v=0}^{\infty}\left[(\gamma-2(v+r)) a_v\right](x-1)^{v+r}-3\displaystyle \sum_{v=1}^{\infty} a_v(v+r)(x-1)^{v+r-1}=0 \\
			& -3 r a_0(x-1)^{r-1}-(x+1) B+\displaystyle\sum_{n=0}^{\infty}\left[(\gamma-2(v+r)) a_v\right](x-1)^{v+r}-3\displaystyle \sum_{\nu=0}^{\infty} a_{v+1}(v+1+r)(x-1)^{v+1+r-1}=0
		\end{aligned}$$
		Agrupando las series semejantes
		$$-3 r a_0(x-1)^{r-1}-(x+1) B+\displaystyle\sum_{v=0}^{\infty}\left[(\gamma-2(v+r)) a_v-3(v+r+1) a_{v+1}\right](x-1)^{v+r}=0$$
		Ahora vamos a realizar de manera separada la operaci\'on de la serie denotada por $B$
		$$\begin{aligned}
			-(x+1) B & =(-x B+B)-B-B \\
			& =-(x-1) B-2 B \\
			& =-(x-1)\displaystyle \sum_{v=0}^{\infty} a_v(v+r-1)(v+r)(x-1)^{v+r-1}-2\displaystyle \sum_{v=0}^{\infty} a_v(v+r-1)(v+r)(x-1)^{v+r-1} \\
			& =-2 a_0 r(r-1)(x-1)^{r-1}-\displaystyle\sum_{v=0}^{\infty} a_v(v+r-1)(v+r)(x-1)^{v+r}-2 \displaystyle\sum_{v=1}^{\infty} a_v(v+r-1)(v+r)(x-1)^{v+r-1}
		\end{aligned}$$
		\begin{multline*}
			-(x+1) B =-2 a_0 r(r-1)(x-1)^{r-1}-\displaystyle\sum_{v=0}^{\infty} a_v(v+r-1)(v+r)(x-1)^{v+r} \\
			-2\displaystyle \sum_{v=1-1}^{\infty} a_{v+1}(v+1+r-1)(v+1+r)(x-1)^{v+1+r-1}
		\end{multline*}
		$$\begin{aligned}
			&-(x+1) B =-2 a_0 r(r-1)(x-1)^{r-1}-\displaystyle\sum_{v=0}^{\infty} a_v(v+r-1)(v+r)(x-1)^{v+r}-2\displaystyle \sum_{v=0}^{\infty} a_{v+1}(v+r)(v+r+1)(x-1)^{v+r}
		\end{aligned}$$
		\begin{equation}\label{xb}
			-(x+1) B=-2 a_0 r(r-1)(x-1)^{r-1}-\displaystyle\sum_{v=0}^{\infty}\left[(v+r)(v+r-1) a_v+2(v+r)(v+r+1) a_{v+1}\right](x-1)^{v+r}
		\end{equation}
		Sustituyendo \ref{xb}
		$$\begin{aligned}
			& -3 r a_0(x-1)^{r-1}-2 a_0 r(r-1)(x-1)^{r-1}-\displaystyle\sum_{v=0}^{\infty}\left[(v+r)(v+r-1) a_v+2(v+r)(v+r+1) a_{v+1}\right](x-1)^{v+r} \\
			& +\displaystyle\sum_{v=0}^{\infty}\left[(\gamma-2(v+r)) a_v-3(v+r+1) a_{v+1}\right](x-1)^{v+r}=0
		\end{aligned}$$
		\begin{multline*}
			(-3-2(r-1)) r a_0(x-1)^{r-1}+\displaystyle\sum_{v=0}^{\infty}\left[(\gamma-2(v+r)-(v+r)(v+r-1)) a_v\right.+ \\
			\left(-3(v+r+1)+(-2(v+r)(v+r+1)) a_{v+1}\right](x-1)^{v+r}=0
		\end{multline*}
		$$\begin{aligned}
			& (-3-2 r+2) r a_0(x-1)^{r-1}+\displaystyle\sum_{v=0}^{\infty}\left[(\gamma-(2+(v+r-1))(v+r)) a_v-(3+2(v+r))(v+r+1) a_{v+1}\right](x-1)^{v+r}=0 \\
			& -(1+2 r) r a_0(x-1)^{r-1}+\displaystyle\sum_{v=0}^{\infty}\left[(\gamma-(v+r+1)(v+r)) a_v-(3+2(v+r))(v+r+1) a_{v+1}\right](x-1)^{v+r}=0
		\end{aligned}$$
		La ecuaci\'on indicial est\'a dada por
		$$\begin{aligned}
			& -(1+2 r) r=0 \quad a_0 \neq 0 \\
			& -r=0 \wedge 1+2 r=0 \\
			& r=0 \quad \wedge r=-\frac{1}{2}
		\end{aligned}$$
		y la relaci\'on de recurrencia
		$$\begin{gathered}
			(\gamma-(v+r+1)(v+r)) a_v-(3+2(v+r))(v+r+1) a_{v+1}=0 \\
			(3+2(v+r))(v+r+1) a_{v+1}=(\gamma-(v+r+1)(v+r)) a_v
		\end{gathered}$$
		\begin{eqnarray}\label{av4}
			% \nonumber % Remove numbering (before each equation)
			a_{v+1}&=&\displaystyle\frac{\gamma-(v+r+1)(v+r)}{(3+2(v+r))(v+r+1)} a_v
		\end{eqnarray}
		para el caso de $r=0$
		$$\begin{aligned}
			& a_{v+1}=\displaystyle\frac{\gamma-(v+0+1)(v+0)}{(3+2(v+0))(v+0+1)} a_v \\
			& a_{v+1}=\displaystyle\frac{\gamma-v(v+1)}{(3+2 v)(v+1)} a_v \\
			& a_{v+1}=\displaystyle\frac{\gamma-v(v+1)}{2\left(v+\frac{3}{2}\right)(v+1)} a_v ; \forall v \geq 0
		\end{aligned}$$
		Utilizando la expresi\'on \ref{yv}
		\begin{eqnarray}\label{av0recurrencia}
			% \nonumber % Remove numbering (before each equation)
			a_{v+1}&=&\displaystyle\frac{\gamma_n-\gamma_v}{2\left(v+\frac{3}{2}\right)(v+1)} a_v
		\end{eqnarray}
		Desarrollando los t\'erminos de \ref{av0recurrencia}\\
		$$
		\begin{aligned}
			& \text{Si} v=0 \\
			& a_{0+1}=\frac{\gamma_n-\gamma_0}{2\left(0+\frac{3}{2}\right)(0+1)} a_0 \\
			& a_1=\displaystyle\frac{\gamma_n-\gamma_0}{2\left(0+\frac{3}{2}\right)(1)} a_0\\
			& \text{Si}\; v=1 \\
			& a_{1+1}=\displaystyle\frac{\gamma_n-\gamma_1}{2\left(1+\frac{3}{2}\right)(1+1)} a_1 \\
			&a_2=\displaystyle\frac{\left(\gamma_n-\gamma_0\right)\left(\gamma_n-\gamma_1\right)}{2(2)\left(1+\frac{3}{2}\right)(2)(1)\left(0+\frac{3}{2}\right)} a_0 \\
			& a_2=\displaystyle\frac{\left(\gamma_n-\gamma_0\right)\left(\gamma_n-\gamma_1\right)}{2^2(2) !\left(0+\frac{3}{2}\right)\left(1+\frac{3}{2}\right)} a_0 \\
			& \text { Si } v=2 \\
			& a_{2+1}=\displaystyle\frac{\gamma_n-\gamma_2}{2\left(2+\frac{3}{2}\right)(2+1)} a_2 \\
			& a_3=\displaystyle\frac{\left(\gamma_n-\gamma_0\right)\left(\gamma_n-\gamma_1\right)\left(\gamma_n-\gamma_2\right)}{2^2(2)! 2(3)\left(0+\displaystyle\frac{3}{2}\right)\left(1+\frac{3}{2}\right)\left(2+\frac{3}{2}\right)} a_0 \\
			& a_3=\displaystyle\frac{\left(\gamma_n-\gamma_0\right)\left(\gamma_n-\gamma_1\right)\left(\gamma_n-\gamma_2\right)}{2^3(3) !\left(0+\frac{3}{2}\right)\left(1+\frac{3}{2}\right)\left(2+\frac{3}{2}\right)} a_0\\
			\vdots\\
			& \text{Si} v=v+1 \\
			&a_v=\displaystyle\frac{\left(\gamma_n-\gamma_0\right)\left(\gamma_n-\gamma_1\right) \ldots \ldots\left(\gamma_n-\gamma_{v-1}\right)}{2^v(v) !\left(0+\displaystyle\frac{3}{2}\right)\left(1+\frac{3}{2}\right) \ldots \ldots\left(v-1+\frac{3}{2}\right)} a_0
		\end{aligned}$$
		Tomando $a=3/2$   en \ref{Relaci\'on de recurrencia generalizada de la funci\'on gamma} y de la expresi\'on \ref{yv-}
		$$a_v=\displaystyle\frac{(n-v+1)(n+v)}{2^v(v) ! \displaystyle\frac{\Gamma\left(\frac{3}{2}+v\right)}{\Gamma\left(\frac{3}{2}\right)}} a_0$$
		Expresando el numerador en productoria
		$$a_v=\displaystyle\frac{\Gamma\left(\frac{3}{2}\right)}{2^v(v) ! \Gamma\left(\frac{3}{2}+v\right)} a_0\displaystyle \prod_{k=1}^v(n-k+1)(n+k)$$
		Ahora expresamos la productoria en t\'erminos de la funci\'on gamma
		$$a_v=\displaystyle\frac{\Gamma\left(\frac{3}{2}\right)}{2^v(v) ! \Gamma\left(\displaystyle\frac{3}{2}+v\right)} a_0 \frac{n !}{(n-v) !}
		\frac{\Gamma(n+1+v)}{\Gamma(n+1)}$$
		$$
		a_v =\displaystyle\frac{\Gamma\left(\frac{3}{2}\right)}{2^v \Gamma\left(\displaystyle\frac{3}{2}+v\right)} \frac{n !}{(v) !(n-v) !} \frac{\Gamma(n+1+v)}{\Gamma(n+1)} a_0 $$
		Por n\'umero combinatorio tenemos
		$$
		a_v  =\left(\begin{array}{l}
			n \\
			v
		\end{array}\right)\displaystyle \frac{\Gamma\left(\frac{3}{2}\right) \Gamma(n+1+v)}{2^v \Gamma\left(\frac{3}{2}+v\right) \Gamma(n+1)} a_0$$
		Para obtener los polinomios de Chevyshev de cuarto tipo, tomaremos la condici\'on
		\begin{equation}\label{chevyshevcond4}
			W(x)=a_0 y(x)\; \text{y}\; a_0=(2 n+1)
		\end{equation}
		$$\begin{aligned}
			& a_0 y(x)=a_0+a_0 \displaystyle\sum_{v=1}^{\infty}\left(\begin{array}{l}
				n \\
				v
			\end{array}\right) \displaystyle\frac{\Gamma\left(\frac{3}{2}\right) \Gamma(n+1+v)}{2^v \Gamma\left(\frac{3}{2}+v\right) \Gamma(n+1)}(x-1)^v \\
			& a_0 y(x)=a_0\left[1+\displaystyle\sum_{v=1}^n\left(\begin{array}{l}
				n \\
				v
			\end{array}\right)\displaystyle \frac{\Gamma\left(\frac{3}{2}\right) \Gamma(n+1+v)}{\Gamma\left(\frac{3}{2}+v\right) \Gamma(n+1)}\left(\frac{x-1}{2}\right)^v\right]
		\end{aligned}$$
		Reemplazando $a_{0}$
		$$\begin{aligned}
			& a_0 y(x)=(2 n+1)\left[1+\displaystyle\sum_{v=1}^n\left(\begin{array}{l}
				n \\
				v
			\end{array}\right) \frac{\Gamma\left(\frac{3}{2}\right) \Gamma(n+1+v)}{\Gamma\left(\frac{3}{2}+v\right) \Gamma(n+1)}\left(\frac{x-1}{2}\right)^v\right] \\
			& a_0 y(x)=(2 n+1)\displaystyle \sum_{v=0}^n\left(\begin{array}{l}
				n \\
				v
			\end{array}\right) \frac{\Gamma\left(\frac{3}{2}\right) \Gamma(n+1+v)}{\Gamma\left(\frac{3}{2}+v\right) \Gamma(n+1)}\left(\frac{x-1}{2}\right)^v
		\end{aligned}$$
		Sacando la constante de la sumatoria
		
		$$\begin{aligned}
			&\begin{aligned}
				& a_0 y(x)=(2 n+1) \Gamma\left(\frac{3}{2}\right) \sum_{v=0}^n\left(\begin{array}{l}
					n \\
					v
				\end{array}\right) \frac{\Gamma(n+1+v)}{\Gamma\left(\frac{3}{2}+v\right) \Gamma(n+1)}\left(\frac{x-1}{2}\right)^v
			\end{aligned}
		\end{aligned}$$
		Por propiedad de la funci\'on gamma en t\'ermino de n\'umeros combinatorios
		$$\begin{aligned}
			&\begin{aligned}
				& W_n(x)=(2 n+1) \Gamma\left(\frac{3}{2}\right)\displaystyle \sum_{v=0}^n \frac{n ! \Gamma(n+1+v)}{v !(n-v) ! \Gamma\left(\frac{3}{2}+v\right) \Gamma(n+1)}\left(\frac{x-1}{2}\right)^v
			\end{aligned}\\
			&W_n(x)=(2 n+1) \Gamma\left(\frac{3}{2}\right) \sum_{v=0}^n \frac{\Gamma(n+1+v)}{v !(n-v) ! \Gamma\left(\frac{3}{2}+v\right)}\left(\frac{x-1}{2}\right)^v
		\end{aligned}$$
		Esta expresi\'on representa los polinomios de cuarto tipo de Chevyshev.\\
		Ahora utilizaremos la expresi\'on obtenida para calcular los primeros 4 polinomios de Chevyshev de cuarto tipo.\\
		Si $n=0$\\
		\begin{eqnarray*}
			% \nonumber % Remove numbering (before each equation)
			W_{0}(x) &=& \Gamma(3/2)\displaystyle\sum_{v=0}^{0}\frac{\Gamma(1+v)}{v!(-v)!\Gamma(3/2+v)}(\frac{x-1}{3})^{v}
		\end{eqnarray*}
		Desarrollando la sumatoria y simplificando, tenemos que
		\begin{eqnarray*}
			W_{0}(x) &=& 1
		\end{eqnarray*}
		Si $n=1$\\
		\begin{eqnarray*}
			% \nonumber % Remove numbering (before each equation)
			W_{3}(x) &=& 3\Gamma(3/2)\displaystyle\sum_{v=0}^{1}\frac{\Gamma(2+v)}{v!(1-v)!\Gamma(3/2+v)}(\frac{x-1}{3})^{v}\\
		\end{eqnarray*}
		Desarrollando la sumatoria
		\begin{eqnarray*}
			W_{3}(x) &=&3 \Gamma(3/2)\displaystyle\{\frac{\Gamma(2)}{0!1!\Gamma(3/2)}+\frac{\Gamma(3)}{1! 0! \Gamma(3/2+1)}(\frac{x-1}{2})\}
		\end{eqnarray*}
		Simplificando, tenemos que
		\begin{eqnarray*}
			W_{3}(x) &=&3 \Gamma(3/2)\displaystyle\{\frac{1}{\Gamma(3/2)}+\frac{2}{3/2 \Gamma(3/2)}(\frac{x-1}{2})\}
		\end{eqnarray*}
		Aplicando distributiva llegamos a
		\begin{eqnarray*}
			W_{1}(x) &=& 2x+ 1
		\end{eqnarray*}
		Si $n=2$\\
		\begin{eqnarray*}
			% \nonumber % Remove numbering (before each equation)
			W_{2}(x) &=& 5\Gamma(3/2)\displaystyle\sum_{v=0}^{2}\frac{\Gamma(3+v)}{v!(2-v)!\Gamma(3/2+v)}(\frac{x-1}{2})^{v}
		\end{eqnarray*}
		Desarrollando la sumatoria
		\begin{eqnarray*}
			% \nonumber % Remove numbering (before each equation)
			W_{2}(x) &=& 5\Gamma(3/2)\displaystyle\{\frac{\Gamma(3)}{0!2!\Gamma(3/2)}+\frac{\Gamma(4)}{1!1!\Gamma(3/2+1)}(\frac{x-1}{2})+\frac{\Gamma(5)}{2! 0!\Gamma(3/2+2)}(\frac{x-1}{2})^{2}\}
		\end{eqnarray*}
		Distribuyendo y simplificando los t\'erminos, tenemos
		\begin{eqnarray*}
			% \nonumber % Remove numbering (before each equation)
			W_{2}(x) &=& 4x^{2}+2x-1
		\end{eqnarray*}
		Si $n=3$\\
		\begin{eqnarray*}
			% \nonumber % Remove numbering (before each equation)
			W_{3}(x) &=& 7\Gamma(3/2)\displaystyle\sum_{v=0}^{3}\frac{\Gamma(4+v)}{v!(3-v)!\Gamma(3/2+v)}(\frac{x-1}{2})^{v}
		\end{eqnarray*}
		Desarrollando la sumatoria
		\begin{eqnarray*}
			% \nonumber % Remove numbering (before each equation)
			W_{3}{x} &=& 7\Gamma(3/2)\{\frac{\Gamma(4)}{3!\Gamma(3/2)}+\frac{\Gamma(5)}{2!\Gamma(3/2+1)}(\frac{x-1}{2})+\frac{\Gamma(6)}{2!\Gamma(3/2+2)}(\frac{x-1}{2})^{2}+\frac{\Gamma(7)}{3!\Gamma(3/2+3)}(\frac{x-1}{2})^{3}\}
		\end{eqnarray*}
		Simplificando tenemos que
		\begin{eqnarray*}
			% \nonumber % Remove numbering (before each equation)
			W_{3}(x) &=& 8x^{3}+4x^{2}-4x-1
		\end{eqnarray*}
		Siguiendo ese mismo proceso obtenemos los dem\'as polinomios, a continuaci\'on presentamos los primeros 8 polinomios
		$$
		\begin{aligned}
			W_0(x)&=1 \\
			W_1(x) & =2 x+1 \\
			W_2(x) & =4 x^2+2 x-1 . \\
			W_3(x) & =8 x^3+4 x^2-4 x-1 . \\
			W_4(x) & =16 x^4+8 x^3-12 x^2-4 x+1 . \\
			W_5(x) & =32 x^5+16 x^4-32 x^3-12 x^2+6 x+1 . \\
			W_6(x) & =64 x^6+32 x^5-80 x^4-32 x^3+24 x^2+6 x-1 . \\
			W_7(x) & =128 x^7+64 x^6-192 x^5-80 x^4+80 x^3+24 x^2-8 x-1 . \\
			W_8(x) & =256 x^8+128 x^7-448 x^6-192 x^5+240 x^4+80 x^3-40 x^2-8 x+1 .
		\end{aligned}
		$$
		
		
		\subsection*{Obtenci\'on de los polinomios de Chevishev de cuarto tipo como proceso de Gramm-Schmidt}
		Siendo $w(x)=\sqrt{(1-x) /(1+x)}$, Tomamos $R_0(x)=1, R_1(x)=x+a_{1,0}, R_2(x)=x^2+$ $a_{2,1} x+a_{2,0}, R_3(x)=x^3+a_{3,2} x^2+a_{3,1} x+a_{3,0}$, y continuando de esta manera hasta $R_n(x)$, donde las constantes o coeficientes $a_{n, m}$ son determinadas por la condici\'on de ortogonalidad, de un polinomio dado con los restantes polinomios de grado menor al mismo, es decir:\\
		$$
		\left\langle R_0(x), R_1(x)\right\rangle_w=\int_{-1}^1 w(x) R_0(x) R_1(x) d x=\int_{-1}^1 w(x)\left(x+a_{1,0}\right) d x=\int_{-1}^1 x w(x) d x+a_{1,0} \int_{-1}^1 w(x) d x=0
		$$
		$\mathrm{Al}$ resolver estas integrales tenemos que
		$$
		\begin{aligned}
			& \int_{-1}^1 x w(x) d x=\int_{-1}^1 x \sqrt{\frac{1-x}{1+x}} d x=\left|-\frac{\operatorname{sen}^{-1}(x)+(2-x) \sqrt{1-x^2}}{2}\right|_{-1}^1=-\frac{\pi}{2} \\
			& a_{1,0} \int_{-1}^1 w(x) d x=a_{1,0} \int_{-1}^1 \sqrt{\frac{1-x}{1+x}} d x=a_{1,0}\left(\left|\operatorname{sen}^{-1}(x)+\sqrt{1-x^2}\right|_{-1}^1=a_{1,0} \pi\right.
		\end{aligned}
		$$
		Por lo que: $-\pi / 2+a_{1,0} \pi=0$, de donde tenemos que: $a_{1,0}=(\pi / 2) / \pi=1 / 2$, por lo que:
		$$
		R_1(x)=x+1/2
		$$
		Ahora tomamos $R_2(x)=x^2+a_{2,1} x+a_{2,0}$, donde las constantes $a_{2,1}, a_{2,0}$ se obtienen del hecho de que $R_2(x)$ es ortogonal a $R_1(x)$ y $R_0(x)$, es decir:
		\begin{equation}\label{vd}
			\left\langle R_2(x), R_1(x)\right\rangle_w=\int_{-1}^1 w(x) R_2(x) R_1(x) d x=\int_{-1}^1 w(x)\left(x^2+a_{2,1} x+a_{2,0}\right)\left(x+\frac{1}{2}\right) d x=0
		\end{equation}
		y
		\begin{equation}\label{hvd}
			\left\langle R_2(x), R_0(x)\right\rangle_w=\int_{-1}^1 w(x) R_2(x) R_0(x) d x=\int_{-1}^1 w(x)\left(x^2+a_{2,1} x+a_{2,0}\right) d x=0
		\end{equation}
		Al resolver la primera integral, tenemos:
		$$
		\int_{-1}^1 w(x)\left(x^2+a_{2,1} x+a_{2,0}\right)\left(x+\frac{1}{2}\right) d x
		=\int_{-1}^1 x^2(x+1 / 2) w(x) d x+a_{2,1} \int_{-1}^1 x(x+1 / 2) w(x) d x+a_{2,0} \int_{-1}^1\left(x+\frac{1}{2}\right) w(x) d x=0
		$$
		De donde:
		$$
		\begin{gathered}
			\int_{-1}^1 x^2\left(x+\frac{1}{2}\right) w(x) d x=\int_{-1}^1 x^2\left(x+\frac{1}{2}\right) \sqrt{(1-x) /(1+x)} d x=-\pi / 8 \\
			a_{2,1} \int_{-1}^1 x\left(x+\frac{1}{2}\right) w(x) d x=a_{2,1} \int_{-1}^1 x\left(x+\frac{1}{2}\right) \sqrt{(1-x) /(1+x)} d x=a_{2,1} \pi / 4 \\
			a_{2,0} \int_{-1}^1\left(x+\frac{1}{2}\right) w(x) d x=a_{2,0} \int_{-1}^1(x+1 / 2) \sqrt{(1-x) /(1+x)} d x=a_{2,0}(0)=0
		\end{gathered}
		$$
		Por lo que, al sumarlas e igualarlas a cero, tenemos la ecuaci\'on:
		$$
		-\pi / 8+a_{2,1} \pi / 4=0
		$$
		Sacando $\pi$ como factor com\'un y multiplicando por $8$ tenemos:
		$$
		2 a_{2,1}-1=0 ; a_{2,1}=1 / 2
		$$
		$\mathrm{Al}$ resolver la segunda integral, tenemos:
		$$
		\begin{gathered}
			\int_{-1}^1 x^2 w(x) d x=\int_{-1}^1 x^2 \sqrt{(1-x) /(1+x)} d x=\pi / 2 \\
			a_{2,1} \int_{-1}^1 x w(x) d x=a_{2,1} \int_{-1}^1 x \sqrt{(1-x) /(1+x)} d x=-a_{2,1} \pi / 2 \\
			a_{2,0} \int_{-1}^1 w(x) d x=a_{2,0} \int_{-1}^1 \sqrt{(1-x) /(1+x)} d x=a_{2,0} \pi
		\end{gathered}
		$$
		Por lo que, al sumarlas e igualarlas a cero, tenemos la ecuaci\'on:
		$$
		\pi / 2-a_{2,1} \pi / 2+a_{2,0} \pi=0
		$$
		Sacando $\pi$ como factor com\'un y multiplicando por $2$ tenemos:
		$$
		2 a_{2,0}-a_{2,1}+1=0
		$$
		$\mathrm{Al}$ resolver estas dos integrales tenemos un sistema de ecuaciones
		$$
		\left(\begin{array}{c}
			2 a_{2,0}-a_{2,1}+1=0 \\
			2 a_{2,1}-1=0
		\end{array}\right)
		$$
		Lo cual implica que: $a_{2,0}=-1 / 4$ y $a_{2,1}=1 / 2$. Por lo que:
		$$
		R_2(x)=x^2+\frac{x}{2}-\frac{1}{4} .
		$$
	
		\section{Ecuaci\'on Diferencial de Legendre}\label{Legendre}
		\textcolor[rgb]{1.00,0.00,0.00}{Aplicacion al potencial electroestatico.Libro(metodo matematico de Aiken)}
		\begin{eqnarray}\label{ecuacionlegendre}
			\left(1-x^2\right) y^{\prime \prime}-2 x y^{\prime}+\lambda y&=&0
		\end{eqnarray}
		\begin{sol}
			Expresandola en la forma normal
			\begin{eqnarray*}
				y^{\prime \prime}+\frac{-2 x}{1-x^2} y^{\prime}+\frac{\lambda}{1-x^2} y&=&0
			\end{eqnarray*}
			De donde es evidente que $x_{0}=0$ es un punto ordinario. As\'i la soluci\'on en series de potencias est\'a dada por  (\ref{solenserie}), derivando la expresi\'on se tiene
			
			\begin{eqnarray*}
				\left(1-x^2\right) \displaystyle\sum_{n=2}^{\infty} n(n-1) a_n x^{n-2}-2 x\displaystyle \sum_{n=1}^{\infty} n a_n x^{n-1}+\lambda \displaystyle\sum_{n=0}^{\infty} a_n x^n&=&0
			\end{eqnarray*}
			Distribuyendo
			\begin{eqnarray*}
				\displaystyle\sum_{n=2}^{\infty} n(n-1) a_n x^{n-2}-\displaystyle\sum_{n=2}^{\infty} n(n-1) a_n x^n-2 \displaystyle\sum_{n=1}^{\infty} n a_n x^n+\lambda \displaystyle\sum_{n=0}^{\infty} a_n x^n&=&0
			\end{eqnarray*}
			Realizando un corrimiendo
			\begin{eqnarray*}
				\displaystyle \sum_{n=0}^{\infty}(n+2)(n+1) a_{n+2} x^n-\displaystyle\sum_{n=0}^{\infty} n(n-1) a_n x^n-2\displaystyle \sum_{n=0}^{\infty} n a_n x^n +\lambda \displaystyle\sum_{n=0}^{\infty} a_n x^n&=&0
			\end{eqnarray*}
			tomando factor com\'un
			\begin{eqnarray*}
				\displaystyle\sum_{n=0}^{\infty}\left[(n+2)(n+1) a_{n+2}-(n(n-1)+2 n-\lambda) a_n\right] x^n&=&0
			\end{eqnarray*}
			Simplificando
			\begin{eqnarray*}
				\displaystyle\sum_{n=0}^{\infty}\left[(n+2)(n+1) a_{n+2}-\left(n^2+n-\lambda\right) a_n\right] x^n&=&0
			\end{eqnarray*}
			Esta expresi\'on es cero si y s\'olo si
			\begin{eqnarray*}
				(n+2)(n+1) a_{n+2}-\left(n^2+n-\lambda\right) a_n&=&0, \forall n \geq 0
			\end{eqnarray*}
			despejando $a_{n+2}$, tenemos
			\begin{eqnarray}\label{lambda}
				a_{n+2}&=&\displaystyle\frac{n(n+1)-\lambda}{(n+2)(n+1)} a_n, \forall n \geq 0
			\end{eqnarray}
			De manera que la soluci\'on de (\ref{ecuacionlegendre}) incluyendo t\'erminos pares e impares es
			\begin{eqnarray}\label{solegendrepimpar}
				% \nonumber % Remove numbering (before each equation)
				y(x) &=& a_{0}+a_{1}x+\displaystyle\sum_{n=2}^{\infty}\displaystyle\frac{n(n+1)-\lambda}{(n+2)(n+1)} a_nx^{n}
			\end{eqnarray}
			Ahora desarrollaremos los t\'erminos pares e impares de forma separada, as\'i de la expresi\'on (\ref{lambda}), tenemos\\
			\textbf{Si es n es par}\\
			$$\begin{aligned}
				& n=0 \\
				& a_2=\frac{(0-\lambda)}{(2)(1)} a_0 \\
				& n=2 \\
				& a_4=\frac{(6-\lambda)}{(4)(3)} a_2=\frac{(6-\lambda)(0-\lambda)}{(4)(3)(2)(1)} a_0 \\
				& n=4 \\
				& a_6=\frac{(20-\lambda)}{(6)(5)} a_4=\frac{(20-\lambda)(6-\lambda)(0-\lambda)}{(6)(5)(4)(3)(2)(1)} a_0 \\
				& n=6 \\
				& \quad a_8=\frac{(42-\lambda)}{(8)(7)} a_6=\frac{(42-\lambda)(20-\lambda)(6-\lambda)(0-\lambda)}{(8)(7)(6)(5)(4)(3)(2)(1)} a_0 \\
				& \vdots \\
				& n=2 k \\
				& \quad a_{2 k+2}=\frac{(0-\lambda)(6-\lambda) \ldots(2 k(2 k+1)-\lambda)}{(2 k) !} a_0 \quad k \geq 0
			\end{aligned}$$
			En t\'ermino de productoria podemos expresar c\'omo
			\begin{eqnarray}\label{lambda1}
				a_{2 k+2}&=&\displaystyle\frac{\displaystyle\prod_{i=0}^{k}(2 i(2 i+1)-\lambda)}{(2 k) !} a_0\quad \forall k \geq 0
			\end{eqnarray}
			De igual manera tenemos, \textbf{Si es n es impar}\\
			$$
			\begin{aligned}
				& n=1 \\
				& a_3=\frac{(2-\lambda)}{(3)(2)} a_1 \\
				& n=3 \\
				& a_5=\frac{(12-\lambda)}{(5)(4)} a_3=\frac{(12-\lambda)(2-\lambda)}{(5)(4)(3)(2)} a_1 \\
				& n=5 \\
				& a_7=\frac{(30 -\lambda)}{(7)(6)} a_5=\frac{(30-\lambda)(12-\lambda)(2-\lambda)}{(7)(6)(5)(4)(3)(2)} a_1 \\
				& n=7 \\
				& \quad a_9=\frac{(56-\lambda)}{(9)(8)} a_7=\frac{(56-\lambda)(30-\lambda)(12-\lambda)(2-\lambda)}{(9)(8)(7)(6)(5)(4)(3)(2)} a_1 \\
				& \vdots \\
				& n=2 k+1 \\
				& a_{2 k+3}=\frac{(2-\lambda)(12-\lambda) \ldots ((2k+2)(2k+1)-\lambda)}{(2 k+3) !} a_1\quad \forall k\geq 0
			\end{aligned}
			$$
			En t\'ermino de productoria
			\begin{eqnarray}\label{lambda2}
				a_{2 k+3}&=&\displaystyle\frac{\displaystyle\prod_{i=0}^k\left(\left(2i+2\right)(2i+1)-\lambda\right)}{(2 k+3) !} a_1 \quad \forall k \geq 0
			\end{eqnarray}
			Por lo que la soluci\'on general de la ecuaci\'on de legendre dada (\ref{ecuacionlegendre}) es
			\begin{multline}\label{solegendre}
				y(x)=a_0  {\left[1+\displaystyle\sum_{n=1}^{\infty} \displaystyle\frac{(0-\lambda)(6-\lambda) \cdots(2 n(2 n+1)-\lambda)}{(2 n) !} x^{2 n}\right] } \\
				+a_1\left[x+\displaystyle\sum_{n=1}^{\infty}\displaystyle \frac{(2-\lambda)(12-\lambda) \cdots((2n+2)(2 n+1)-\lambda)}{(2 n+3) !} x^{2 n+1}\right]
			\end{multline}
		\end{sol}
		
		\Example{Demuestre que la segunda soluci\'on de la ecuaci\'on}{
			
			Demuestre que la segunda soluci\'on de la ecuaci\'on (\ref{ecuacionlegendre}) se puede expresar como
			\begin{eqnarray}\label{legedre2}
				Q_n(x)&=&P_n(x)\displaystyle \int^x \displaystyle\frac{d t}{(1-t^2)(P_{n}(t))^2}
		\end{eqnarray}}
		
		\begin{sol}
			Como la segunda soluci\'on se puede expresar como (cilarlibro de regan) $$y_2(x)=y_1(x) \int^{x}\displaystyle\frac{1}{(y_1(t))^{2}} e^{-\displaystyle\int^{t} p_1(s) d s} d t$$
			con
			\begin{eqnarray*}
				P_1(s)&=&\frac{-2 s}{1-s^2}
			\end{eqnarray*}
			tenemos que
			\begin{eqnarray*}
				y_2(x)&=&P_{n}(x)\displaystyle \int \frac{1}{(P_{n}(x))^2} e^{-\displaystyle\int \frac{-2 s}{1-s^2} d s} d t
			\end{eqnarray*}
			Integrando
			\begin{eqnarray*}
				y_2(x)&=&P_n(x)\displaystyle \int^x \frac{1}{(P_{n}(t)^2} e^{\ln \left|\displaystyle\frac{1}{1-t^2}\right|} d t
			\end{eqnarray*}
			Por propiedad de la funci\'on exponencial
			$$
			\begin{aligned}
				& y_2(x)=P_{n}(x)\displaystyle \int^x \frac{1}{(P_{n}(t))^2} \frac{1}{1-t^2} d t \\
				& y_2(x)=P_{n}(x)\displaystyle \int^x \frac{d t}{\left(1-t^2\right)(P_{n}(t))^2}
			\end{aligned}
			$$
		\end{sol}
		
		Ahora obtendremos los primeros $5$ polinomios de Legendre utilizando la expresi\'on (\ref{legedre2}).\\
		Para $n=0$, tenemos
		$$
		Q_0(x)=P_0(x)\displaystyle \int^x \frac{d t}{\left(1-t^2\right)\left(P_0(t)\right)^2}
		$$
		Reemplazando el polinomio $P_0$ de Legendre
		$$
		Q_0(x)=\displaystyle\int^x \frac{d t}{1-t^2}
		$$
		Ahora Aplicando fracciones parciales
		$$
		Q_0(x)=\displaystyle\frac{1}{2} \int^x\left(\frac{1}{1+t}+\frac{1}{1-t}\right) d t
		$$
		Intengrando
		$$
		Q_0(x)=\displaystyle\frac{1}{2}[\ln (x+1)-\ln (x-x)]
		$$
		Simplificando por propiedad de los logar\'itmo
		$$
		Q_0(x)=\displaystyle\frac{1}{2} \ln \left(\frac{x+1}{1-x}\right)
		$$
		Tomando $n=1$
		$$
		Q_1(x)=P_1(x)\displaystyle \int^x \frac{d t}{\left(1-t^2\right)\left(P_1(t)\right)^2}
		$$
		Sustituyendo el polinomio $P_1$ de Legendre
		$$
		Q_1(x)=x \displaystyle\int \frac{d t}{\left(1-t^2\right) t^2}
		$$
		Por fracciones parciales, tenemos
		$$
		Q_1(x)=x\displaystyle \int^x\left[t^{-2}+\frac{1}{2(1+t)}+\frac{1}{2(1-t)}\right] d t
		$$
		Integrando
		$$
		\begin{aligned}
			& Q_1(x)=x\left[-x^{-1}+\displaystyle\frac{1}{2} \ln \left(\frac{1+x}{1-x}\right)\right]  \\
			& Q_1(x)=\displaystyle\frac{x}{2} \ln \left(\frac{x+1}{x-x}\right)-1
		\end{aligned}
		$$
		Para $n=2$
		$$
		Q_2(x)=P_2(x)\displaystyle \int^x \frac{d t}{\left(1-t^2\right)\left(P_2(t)\right)^2}
		$$
		Reemplazando en la expresi\'on anterior $P_2$ de Legendre
		$$
		\begin{aligned}
			& Q_2(x)=\displaystyle\frac{1}{2}\left(3 x^2-1\right) \int^x \frac{d t}{\left(1-t^2\right)\left(\frac{1}{2}\left(3 t^2-1\right)\right)^2} \\
			& Q_2(x)=4\left(3 x^2-1\right)\displaystyle \int^x \frac{d t}{\left(1-t^2\right)\left(3 t^2-1\right)^2}
		\end{aligned}
		$$
		Por fracciones parciales tenemos
		$$
		Q_2(x)=2\left(3 x^2-1\right)\displaystyle \int^x\left[\frac{3}{4\left(3 t^2-1\right)}+\frac{3}{2\left(3 t^2-1\right)^2}+\frac{1}{8(1+t)}+\frac{1}{8(1-t)}\right] d t
		$$
		integrando tenemos
		$$
		\begin{aligned}
			& Q_2(x)=\displaystyle\frac{1}{4}\left(3 x^2-1\right)\left[\ln \left(\frac{1+x}{1-x}\right)+\frac{6 x}{1-3 x^2}\right] \\
			& Q_2(x)=\displaystyle\frac{1}{4}\left(3 x^2-1\right) \ln \left(\frac{1+x}{1-x}\right)-\frac{3}{2} x
		\end{aligned}
		$$
		Para $n=3$
		$$
		Q_3(x)=P_3(x) \displaystyle\int^x \frac{d t}{\left(1-t^2\right)\left(P_3(t)\right)^2}
		$$
		Reemplazando el polinomio $P_3$ de Legendre
		$$
		\begin{aligned}
			& Q_3(x)=\left(\frac{1}{2}\left(5 x^3-3 x\right)\right)\displaystyle \int^x \frac{d t}{\left(1-t^2\right)\left[\frac{1}{2}\left(5 t^3-3 t\right)\right]^2} \\
			& Q_3(x)=2\left(5 x^3-3 x\right)\displaystyle \int^x \frac{d t}{\left(1-t^2\right)\left(5 t^3-3 t\right)^2}
		\end{aligned}
		$$
		Por fracciones parciales tenemos
		$$
		Q_3(x)=2\left(5 x^3-3 x\right)\displaystyle \int^x\left[\frac{25}{36\left(5 t^2-3\right)}+\frac{1}{9 t^2}+\frac{25}{6\left(5 t^2-3\right)^2}+\frac{1}{8(1+t)}+\frac{1}{8(1-t)}\right] dt
		$$
		Integrando
		$$
		Q_3(x)=2\left(5 x^3-3 x\right)\left[\frac{1}{72}\left(\frac{50 x}{3-5 x^2}-\frac{8}{x}+9 \ln \left(\frac{1+x}{1-x}\right)\right)\right]
		$$
		Simplificando
		$$
		Q_3(x)=\displaystyle\frac{1}{18}\left[8\left(3-5 x^2\right)-50 x^2-9 x\left(3-5 x^2\right) \ln \left(\frac{1+x}{1-x}\right)\right]
		$$
		Para $n=4$, tenemos
		$$
		Q_4(x)=P_4(x) \displaystyle \int^x \frac{d t}{\left(1-t^2\right)\left(P_4(t)\right)^2}
		$$
		Sustituyendo el polinomio $P_4$ de Legendre
		$$
		\begin{aligned}
			& Q_4(x)=\frac{1}{8}\left(35 x^4-30 x^2+3\right) \int^x \frac{d t}{\left(1-t^2\right)\left(\frac{1}{8}\left(35 t^4-30 t^2+3\right)\right)^2} \\
			& Q_4(x)=8\left(35 x^4-30 x^2+3\right) \int^x \frac{d t}{\left(1-t^2\right)\left(35 t^4-30 t^2+3\right)^2}
		\end{aligned}
		$$
		Aplicando fracciones parciales, tenemos
		$$
		Q_4(x)=8\left(35 x^4-30 x^2+3\right)\displaystyle \int^{x}\left[\frac{5\left(7 t^2+1\right)}{64\left(35 t^4-30t^{2}+3\right)}+\frac{5\left(7 t^2+1\right)}{8\left(35 t^4-30 t^2+3\right)^2}+\frac{1}{128(1+t)}+\frac{1}{128(1-t)}\right] d t
		$$
		Integrando
		$$
		Q_4(x)=8\left(35 x^4-30 x^2+3\right)\left[\frac{1}{384}\left(\frac{110 x-210 x^3}{35 x^4-30 x^2+3}+3 \ln \left(\frac{1+x}{1-x}\right)\right)\right]
		$$
		Simplificando
		$$Q_4(x)=\displaystyle\frac{1}{48}\left[110 x-210 x^3+3\left(35 x^4-30 x^2+3\right) \ln \left(\frac{1+x}{1-x}\right)\right] $$
		\begin{center}
			\begin{tabular}{||cc||}
				\hline
				% after \\: \hline or \cline{col1-col2} \cline{col3-col4} ...
				$ Q_0(x)$ & $\displaystyle\frac{1}{2} \ln \left(\frac{x+1}{1-x}\right)$\\
				\hline
				$ Q_1(x)$ & $\displaystyle\frac{x}{2} \ln \left(\frac{x+1}{x-x}\right)-1$ \\
				\hline
				$Q_2(x)$ & $\displaystyle\frac{1}{4}\left(3 x^2-1\right) \ln \left(\frac{1+x}{1-x}\right)-\frac{3}{2} x$ \\
				\hline
				$Q_3(x)$ & $\displaystyle\frac{1}{18}\left[8\left(3-5 x^2\right)-50 x^2-9 x\left(3-5 x^2\right) \ln \left(\frac{1+x}{1-x}\right)\right] $\\
				\hline
				$Q_4(x)$ & $\displaystyle\frac{1}{48}\left[110 x-210 x^3+3\left(35 x^4-30 x^2+3\right) \ln \left(\frac{1+x}{1-x}\right)\right]$ \\
				\hline
			\end{tabular}
		\end{center}
		
			\subsection{La f\'ormula de Rodrigues para los polinomios de Legendre}
		Los polinomios de Legendre $P_{n}(x)$ se pueden expresar en diferentes formas. En el siguiente teorema se demuestra la f\'ormula de Rodrigues.
		
		\Theorem{La f\'ormula de Rodrigues para los polinomios de Legendre}{	La f\'ormula de Rodrigues para los polinomios de Legendre $P_{n}(x)$ es
			\begin{eqnarray}\label{legendrerod}
				% \nonumber % Remove numbering (before each equation)
				P_n(x)&=&\displaystyle\frac{1}{2^n n !} \displaystyle\frac{d^n}{d x^n}\left(x^2-1\right)^n
		\end{eqnarray}}
		\begin{demo}
			Sea $v=\left(x^2-1\right)^n$, entonces
			\begin{eqnarray}\label{1}
				\left(x^2-1\right)\displaystyle\frac{d v}{d x}=2 n x v
			\end{eqnarray}
			Derivando la expresi\'on anterior (\ref{1}),$(n+1)$ veces por la f\'ormula de Leibniz, tenemos
			\begin{eqnarray*}
				\left(x^2-1\right)\displaystyle \frac{d^{n+2} v}{d x^{n+2}}+2(n+1) x \displaystyle\frac{d^{n+1} v}{d x^{n+1}}+n(n+1)\displaystyle \frac{d^n v}{d x^n}=2 n\left\{x \displaystyle\frac{d^{n+1} v}{d x^{n+1}}+(n+1) \displaystyle\frac{d^n v}{d x^n}\right\}
			\end{eqnarray*}
			La cual se puede escribir como
			\begin{eqnarray*}
				\left(1-x^2\right)\displaystyle \frac{d^{n+2} v}{d x^{n+2}}-2 x \displaystyle\frac{d^{n+1} v}{d x^{n+1}}+n(n+1)\displaystyle \frac{d^n v}{d x^n}&=&0
			\end{eqnarray*}
			Si sustituimos  $z=\displaystyle\frac{d^n v}{d x^n}$, entonces la ecuaci\'on anterior se tranforma en
			$$
			\left(1-x^2\right) \frac{d^2 z}{d x^2}-2 x \frac{d z}{d x}+n(n+1) z=0
			$$
			Esta ecuaci\'on obtenida es la Lengendre dada en (\ref{Legendre}) con  $\lambda=n(n+1)$. Por lo que es necesario que
			$$
			z=\frac{d^n v}{d x^n}=c P_n(x)
			$$
			donde $c$ es una constante. Como sabemos que  $P_n(1)=1$, tenemos
			$$
			\begin{aligned}
				c & =\left(\displaystyle\frac{d^n v}{d x^n}\right)_{x=1}=\left.\displaystyle\frac{d^n}{d x^n}\left(x^2-1\right)^n\right|_{x=1}=\left.\displaystyle\frac{d^n}{d x^n}(x-1)^n(x+1)^n\right|_{x=1} \\
				& =\left.\displaystyle\sum_{k=0}^n\left(\begin{array}{l}
					n \\
					k
				\end{array}\right)\displaystyle \frac{n !}{(n-k) !}(x-1)^{n-k}\displaystyle \frac{n !}{k !}(x+1)^k\right|_{x=1}=2^n n !
			\end{aligned}
			$$
			Por lo tanto, se sigue que
			\begin{eqnarray*}
				P_n(x)&=&\displaystyle\frac{1}{c} \frac{d^n v}{d x^n}=\displaystyle\frac{1}{2^n n !}\displaystyle \frac{d^n}{d x^n}\left(x^2-1\right)^n
			\end{eqnarray*}
		\end{demo}
		
		Ahora obtendremos los $6$ primeros polinomios de Legendre utilizando la f\'ormula de Rodrigues, demostrada en el teorema anterior
		
		\Example{Utiliza la f\'ormula de Rodrigues de los polinomios de Legendre}{
			
			Utiliza la f\'ormula de Rodrigues de los polinomios de Legendre para calcular los primeros $6$ polinomios.}
		
		\begin{sol}
			Para el caso de $n=0$
			$$
			\begin{aligned}
				P_0(x) & =\frac{1}{2^0 0 !} \frac{d^{(0)}}{d x^{(0)}}\left(x^2-1\right)^0 \\
				& =1(1) \\
				P_0(x) & =1
			\end{aligned}
			$$
			para el caso de $n=1$
			\begin{eqnarray*}
				P_1(x)&=&\displaystyle\frac{1}{2^{1} 1 !}\displaystyle \frac{d}{d x}\left(x^2-1\right)
			\end{eqnarray*}
			Realizando la derivada
			\begin{eqnarray*}
				P_1(x)&=&\frac{1}{2}(2 x)
			\end{eqnarray*}
			Simplificando
			\begin{eqnarray*}
				P_1(x)&=&x
			\end{eqnarray*}
			para el caso de $n=2$
			\begin{eqnarray*}
				P_2(x)&=&\displaystyle\frac{1}{2^2 2 !}\displaystyle \frac{d^2}{d x^2}\left(x^2-1\right)^2
			\end{eqnarray*}
			Realizando la derivada
			\begin{eqnarray*}
				P_2(x)&=&\displaystyle\frac{1}{8}\left(4\left(3 x^2-1\right)\right)
			\end{eqnarray*}
			Simplificando
			\begin{eqnarray*}
				P_2(x)&=&\displaystyle\frac{1}{2}\left(3 x^2-1\right)
			\end{eqnarray*}
			Para el caso de $n=3$
			\begin{eqnarray*}
				P_3(x)&=&\frac{1}{2^3 3 !} \frac{d^3}{d x^3}\left(x^2-1\right)^3
			\end{eqnarray*}
			Derivando
			\begin{eqnarray*}
				P_3(x)&=&\frac{1}{48}\left(24\left(10 x^3-3x\right)\right)
			\end{eqnarray*}
			Simplificando
			\begin{eqnarray*}
				P_3(x)&=&\frac{1}{2}\left(5 x^3-3 x\right)
			\end{eqnarray*}
			Para el caso de $n=4$
			\begin{eqnarray*}
				P_4(x)&=&\displaystyle\frac{1}{2^4 4 !} \frac{d^4}{d x^4}\left(x^2-1\right)^4
			\end{eqnarray*}
			Derivando obtenemos
			\begin{eqnarray*}
				P_4(x)&=&\displaystyle\frac{1}{384}\left(48\left(35 x^4-30 x^3+3\right)\right)
			\end{eqnarray*}
			Simplificando
			\begin{eqnarray*}
				P_4(x)&=&\displaystyle\frac{1}{8}\left(35 x^4-30 x^3+3\right)
			\end{eqnarray*}
			Para el caso de $n=5$
			\begin{eqnarray*}
				P_5(x)&=&\displaystyle\frac{1}{2^5 5 !} \frac{d^5}{d x^5}\left(x^2-1\right)^5
			\end{eqnarray*}
			Derivando
			\begin{eqnarray*}
				% \nonumber % Remove numbering (before each equation)
				P_{5}(x) &=& \displaystyle\frac{480}{3840} \left(63 x^5-70 x^3+15x\right)
			\end{eqnarray*}
			Simplificando
			\begin{eqnarray*}
				% \nonumber % Remove numbering (before each equation)
				P_{5}(x) &=& \displaystyle\frac{1}{8} \left(63 x^5-70 x^3+15x\right)
			\end{eqnarray*}
		\end{sol}


		\subsection{Funci\'on generadora de los polinomios de Legendre}
		\Proposition{La funci\'on generadora de los polinomios de Legendre}{La funci\'on generadora de los polinomios de Legendre est\'a dada por la expresi\'on
			\begin{eqnarray}\label{legendregen}
				% \nonumber % Remove numbering (before each equation)
				H(x,r) &=& \left(1-2 xr+r^2\right)^{-1 / 2}=\displaystyle\sum_{n=0}^{\infty} P_n(x) r^n
		\end{eqnarray}}
		
		
		\begin{demo}
			Si $|x| \leq k$ donde $k$ es arbitrario, y $|r|<\left(1+k^2\right)^{\displaystyle\frac{1}{2}}-k$, entonces se tiene por la desigualdad triangular
			\begin{eqnarray*}
				\left|2 x r-r^2\right| \leq 2|x||r|+\left|r^2\right|
			\end{eqnarray*}
			Por las suposiciones
			\begin{eqnarray*}
				\left|2 x \cdot r-r^2\right|<2 k\left[\left(1+k^2\right)^{\displaystyle\frac{1}{2}}-k\right]+\left[\left(1+k^2\right)^{\displaystyle\frac{1}{2}}-k\right]^2
			\end{eqnarray*}
			Simplificando el lado derecho de la desigualdad
			\begin{eqnarray*}
				\left|2 x r-r^2\right|<2 k\left(1+k^2\right)^{\displaystyle\frac{1}{2}}-2 k^2+\left(1+k^2\right)-2 k\left(1+k^2\right)^{1 / 2}+k^2
			\end{eqnarray*}
			Reduciendo t\'erminos semejantes
			\begin{eqnarray}\label{zh}
				\left|2 x r-r^2\right|<1 \quad(r)
			\end{eqnarray}
			Con el resultado obtenido en (\ref{zh}) podemos expresar $\left(1-2 x r+r^2\right)^{-1 / 2}$ en serie de potencias obteniendo
			$$\begin{aligned}
				{[1-r(2 x-r)]^{-\frac{1}{2}} } & =1+\displaystyle\frac{1}{2} r(2 x-r)+\displaystyle\frac{1}{2} \frac{3}{4} r^2(2 x-r)^2+\cdots \\
				& +\displaystyle\frac{(1)(3) \cdots(2 n-1)}{(2)(4) \cdots(2 n)} r^n(2 x-r)^n+\cdots
			\end{aligned}$$
			El coeficiente de $r^n$ en esta expansi\'on es
			$$
			\begin{aligned}
				& \displaystyle\frac{(1)(3) \cdots(2 n-1)}{(2)(4) \cdots(2 n)}(2 x)^n-\displaystyle\frac{(1)(3) \cdots(2 n-3)(n-1)}{(2)(4) \cdots(2 n-2) 1 !}(2 x)^{n-2} \\
				& \quad+\displaystyle\frac{(1)(3) \cdots(2 n-5)(n-2)(n-3)}{(2)(4) \cdots(2 n-4) 2 !}(2 x)^{n-4} \cdots \cdots
			\end{aligned}
			$$
			Simplificando se tiene
			\begin{eqnarray*}
				\displaystyle\frac{(1)(3) \cdots(2 n-1)}{n !}\left[x^n-\frac{n(n-1) x^{n-2}}{(2 n-1) (1)(2)}+\displaystyle\frac{n(n-1)(n-2)(n-3) x^{n-4}}{(2 n-1)(2 n-3)(2)(4)}-\cdots\right.
			\end{eqnarray*}
			Por lo tanto
			\begin{eqnarray*}
				\displaystyle[1-r(2 x-r)]^{-\frac{1}{2}}&=&\displaystyle\sum_{n=0}^{\infty} p_n(x) r^n
			\end{eqnarray*}
		\end{demo}
		\subsection{Relaci\'on de recurrencia de los polinomios de Legendre}
		\Proposition{Relaci\'on de recurrencia de los polinomios de Legendre}{$\forall  n\in \mathbb{Z}^+$,
			\begin{equation}\label{Legenrecurre}
			(n+1) P_{n+1}(x)-(2 n+1) x P_{n}(x)+n P_{n-1}(x)=0
		\end{equation}}
		
		\begin{demo}
			Iniciamos derivando la funci\'on generatriz de los polinomios de Legendre dada en (\ref{legendregen}) con respecto a $r$ :
			\begin{eqnarray*}
				% \nonumber % Remove numbering (before each equation)
				\displaystyle\frac{\partial H(x, r)}{\partial r}&=&-\displaystyle\frac{1}{2}\left(1-2 x r+r^{2}\right)^{-3 / 2}(-2 x+2 r)=\displaystyle\frac{x-r}{\left(1-2 x r+r^{2}\right)^{\displaystyle\frac{3}{2}}}
			\end{eqnarray*}
			Se puede verificar que
			\begin{eqnarray*}
				\left(1-2 x r+r^{2}\right)\textcolor{blue}{\displaystyle\displaystyle\frac{\partial H(x, r)}{\partial r}}-(x-r) \textcolor{blue}{H(x, r)}&=&0
			\end{eqnarray*}
			Ahora sustituyendo  (\ref{legendregen}) en la ecuaci\'on anterior
			\begin{eqnarray*}
				\left(1-2 x r+r^{2}\right) \textcolor{blue}{\displaystyle\sum_{n=1}^{\infty} n P_{n}(x) r^{n-1}}-(x-r) \textcolor{blue}{\displaystyle\sum_{n=0}^{\infty} P_{n}(x) r^{n}}&=&0
			\end{eqnarray*}
			Realizando la multiplicaci\'on anterior, obtenemos
			\begin{eqnarray*}
				\textcolor{blue}{\displaystyle\sum_{n=1}^{\infty} n P_{n}(x) r^{n-1}}-\displaystyle\sum_{n=1}^{\infty} 2 n x P_{n}(x) r^{n}+\textcolor{blue}{\displaystyle\sum_{n=1}^{\infty} n P_{n}(x) r^{n+1}}-\displaystyle\sum_{n=0}^{\infty} x P_{n}(x) r^{n}+\textcolor{blue}{\displaystyle\sum_{n=0}^{\infty} P_{n}(x) r^{n+1}}&=&0
			\end{eqnarray*}
			Reorganizando las series para tener igual potencia de $r$ en cada suma :
			$$
			\begin{aligned}
				&\textcolor{blue}{\displaystyle\sum_{n=0}^{\infty}(n+1) P_{n+1}(x) r^{n}}-\displaystyle\sum_{n=1}^{\infty} 2 n x P_{n}(x) r^{n}+\textcolor{blue}{\displaystyle\sum_{n=2}^{\infty}(n-1) P_{n-1}(x) r^{n}} \\
				&-\displaystyle\sum_{n=0}^{\infty} x P_{n}(x) r^{n}+\textcolor{blue}{\displaystyle\sum_{n=1}^{\infty} P_{n-1}(x) r^{n}}=0 .
			\end{aligned}
			$$
			Escribiendo los primeros dos t\'erminos, para que las sumatorias inicien en dos, tenemos lo siguiente
			$$
			\begin{aligned}
				&P_{1}(x)+2 P_{2}(x) r-2 x P_{1}(x) r-x P_{0}(x)-x P_{1}(x) r+P_{0}(x) r \\
				&+\displaystyle\sum_{n=2}^{\infty}\textcolor{blue}{\left[(n+1) P_{n+1}(x)-2 n x P_{n}(x)+(n-1) P_{n-1}(x)-x P_{n}(x)+P_{n-1}(x)\right]} r^{n}=0 .
			\end{aligned}
			$$
			Para que la serie de potencia en $r$ sea cero para todo valor de $r$ en alg\'un intervalo alrededor de $0$, los coeficientes de $r^{n}$ deben ser cero para $ n = 0,1,2, \ldots $ entonces
			$$
			\begin{array}{r}
				P_{1}(x)-x P_{0}(x)=0, \\
				2 P_{2}(x)-2 x P_{1}(x)-x P_{1}(x)+P_{0}(x)=0
			\end{array}
			$$
			y para
			$n=2,3, \ldots$,
			$$
			(n+1) P_{n+1}(x)-2 n x P_{n}(x)+(n-1) P_{n-1}(x)-x P_{n}(x)+P_{n-1}(x)=0
			$$
			Esto es
			$$
			\begin{aligned}
				&P_{1}(x)=x P_{0}(x) \\
				&P_{2}(x)=\frac{1}{2}\left(3 x P_{1}(x)-P_{0}(x)\right)
			\end{aligned}
			$$
			y, para $n=2,3, \ldots$,
			\textcolor{blue}{
				$$
				(n+1) P_{n+1}(x)-(2 n+1) x P_{n}(x)+n P_{n-1}(x)=0  \quad (3)
				$$
			}
			Como esta ecuaci\'on tambi\'en es v\'alida para $ n = 1 $, establece  la relaci\'on recursiva para todo entero positivo.
		\end{demo}
		




		\subsection{Polinomios asociados de Legendre }
		
		\Definition{La ecuaci\'on diferencial de Legendre}{	La ecuaci\'on diferencial de Legendre est\'a dada por la expresi\'on
			\begin{eqnarray}\label{legendreasoc}
				% \nonumber % Remove numbering (before each equation)
				\left(1-x^2\right) y^{\prime \prime}-2 x y^{\prime}+\left[n(n+1)-\displaystyle\frac{m^2}{1-x^2}\right] y&=&0
			\end{eqnarray}
			cuando $m$ y $n$ son enteros no negativos, la soluci\'on general es
			\begin{eqnarray}\label{legendreasocsol}
				y(x)&=&A P_n^m(x)+B Q_n^m(x)
			\end{eqnarray}
			donde $P_n^m(x)\;\text{y}\; Q_n^m(x)$ se llaman funciones asociada de Legendre de primer y segundo tipo respectivamente.}
		
		A continuaci\'on se resolver\'a la ecuaci\'on asociada de Legendre dada en (\ref{legendreasoc})
		\begin{sol}
			tomando la sustituci\'on
			\begin{eqnarray}\label{sus1}
				y&=&\left(1-x^2\right)^{\displaystyle\frac{m}{2}} u
			\end{eqnarray}
			Calculando la primera derivada y simplificando
			
			\begin{eqnarray*}
				y^{\prime}&=&-mx\left(1-x^2\right)^{\displaystyle\frac{m-2}{2}} u+\left(1-x^2\right)^{\displaystyle\frac{m}{2}}
			\end{eqnarray*}
			Tomando factor com\'un
			\begin{eqnarray}\label{der1}
				y^{\prime}&=&\left(1-x^2\right)^{\displaystyle\frac{m-2}{2}}\big[\left(1-x^2\right) u^{\prime}-m x u\big]
			\end{eqnarray}
			De igual manera para la segunda derivada
			$$
			\begin{aligned}
				y^{\prime \prime}=- & {\left[\left(1-x^2\right)^{\displaystyle\frac{m-2}{2}} u+\left(\frac{m-2}{2}\right)\left(-2 x^2\right)\left(1-x^2\right)^{\displaystyle\frac{m-4}{2}} u+x\left(1-x^2\right)^{\displaystyle\frac{m-2}{2}} u^{\prime}\right] } \\
				& +u^{\prime \prime}\left(1-x^2\right)^{\displaystyle\frac{m}{2}}-2 x\left(\frac{m}{2}\right) u^{\prime}\left(1-x^2\right)^{\displaystyle\frac{m-2}{2}}
			\end{aligned}
			$$
			Simplificando la expresi\'on anterior
			\begin{eqnarray}\label{der2}
				y^{\prime \prime}&=&\left(1-x^2\right)^{\displaystyle\frac{m-4}{2}}\left[\left(1-x^2\right)^2 u^{\prime \prime}-2 m x\left(1-x^2\right) u^{\prime}+\left(m(m-2) x^2-m\left(1-x^2\right)\right) u\right]
			\end{eqnarray}
			Reemplazando (\ref{sus1}),(\ref{der1}),(\ref{der2}) en (\ref{legendreasoc})  y tomando factor com\'un
			
			\begin{multline*}
				% \nonumber % Remove numbering (before each equation)
				\left(1-x^{2}\right)^{\displaystyle\frac{m-2}{2}}\bigg[\left(1-x^{2}\right)^{2}u^{\prime\prime} -2mx(1-x^{2})u^{\prime}+\left(m(m-2)x^{2}-m(1-x^{2})\right)u-2x\left(1-x^{2}\right)u^{\prime}\\
				+2mx^{2}u-m^{2}u+n(n+1)(1-x^{2})u\bigg]=0
			\end{multline*}
			Simplificando nueva vez
			\begin{eqnarray*}
				\left(1-x^2\right)^2 u^{\prime \prime}-2 x(m+1)\left(1-x^2\right) u^{\prime}+(n(n+1)-m(m+1))\left(1-x^2\right) u&=&0
			\end{eqnarray*}
			La expresi\'on anterior es equivalente a
			\begin{eqnarray}\label{legendreasimod}
				\left(1-x^2\right) u^{\prime\prime}-2 x(m+1) u^{\prime}+(n(n+1)-m(m+1)) u=0
			\end{eqnarray}
			Podemos ver que si $m=0$ en la expresion (\ref{legendreasimod}), obtenemos la ecuaci\'on de Legendre.
			Ahora resolveremos la ecuacion (\ref{legendreasimod}) por serie  entorno al punto $x_0=0$, por lo que la soluci\'on es de la forma (\ref{solenserie}). Derivando y reemplazando las derivadas en (\ref{legendreasoc}), tenemos
			$$
			\begin{gathered}
				\left(1-x^2\right) \displaystyle\sum_{k=0}^{\infty} k(k-1) a_k u^{k-2}-2 x(m+1)\displaystyle \sum_{k=0}^{\infty} k a_k u^{k-1} \\
				+(n(n+1)-m(m+1))\displaystyle \sum_{k=0}^{\infty} a_k u^k=0
			\end{gathered}
			$$
			Por distributiva
			$$
			\begin{gathered}
				\displaystyle\sum_{k=0}^{\infty} k(k-1) a_k u^{k-2}-\displaystyle\sum_{k=0}^{\infty} k(k-1) a_k u^k-2(m+1)\displaystyle \sum_{k=0}^{\infty} k a_k u^k \\
				+(n(n+1)-m(m+1))\displaystyle \sum_{k=0}^{\infty} a_k u^k=0
			\end{gathered}
			$$
			Redefiniendo el \'indice de la primera sumatoria
			$$
			\begin{aligned}
				& \displaystyle\sum_{k=0}^{\infty}(k+2)(k+1) a_{k+2} u^k-\displaystyle\sum_{k=0}^{\infty} k(k-1) a_k u^k \\
				& -2(m+1) \displaystyle\sum_{k=0}^{\infty} k a_k u^k+(n(n+1)-m(m+1))\displaystyle \sum_{k=0}^{\infty} a_k u^k=0
			\end{aligned}
			$$
			Se puede notar que la sumatoria anterior son semejantes, realizando las operaciones indicadas
			$$
			\displaystyle\sum_{k=0}^{\infty}\left[(k+2)(k+1) a_{k+2}+(n(n+1)-m(m+1)-2(m+1) k-k(k-1)) a_k\right] u^k=0
			$$
			Igualando a cero los coeficientes.
			$$
			(k+2)(k+1) a_{k+2}+(n(n+1)-m(m+1)-2(m+1) k-k(k-1)) a_k=0
			$$
			tomando factor com\'un para simplificar
			\begin{eqnarray*}
				(k+2)(k+1) a_{k+2}+(n(n+1)-m(m+1)-k(2 m+k+1)) a_k&=&0
			\end{eqnarray*}
			\begin{eqnarray}\label{qo}
				% \nonumber % Remove numbering (before each equation)
				a_{k+2}&=&\displaystyle\frac{k(2 m+k+1)+m(m+1)-n(n+1)}{(k+2)(k+1)} a_k \quad \forall k \geq 0
			\end{eqnarray}
			Ahora la expresi\'on (\ref{qo}) se desarrollar\'a para t\'erminos pares e impares. Para el caso de los pares tenemos
			$$
			\begin{aligned}
				& k=0 \\
				& a_2=\frac{m(m+1)-n(n+1)}{(2)(1)} a_0 \\
				& k=2 \\
				& a_4=\frac{2(2 m+3)+m(m+1)-n(n+1)}{(4)(3)} a_2 \\
				& a_4=\frac{(2(2 m+3)+m(m+1)-n(n+1))(m(m+1)-n(n+1))}{(4)(3)(2)(1)} a_0 \\
				& k=4 \\
				& a_6=\frac{4(2 m+5)+m(m+1)-n(n+1)}{(6)(5)} a_4 \\
				& a_6=\frac{(4(2 m+5)+m(m+1)-n(n+1))(2(2 m+3)+m(m+1)-n(n+1))}{(6)(5)(4)(3)(2)(1)}a_0
			\end{aligned}
			$$
			Siguiendo este proceso obtenemos
			\begin{eqnarray*}
				a_{2 k}&=&\displaystyle\frac{(m(m+1)-n(n+1)) \cdots(2(k-1)(2 m+2 k-1)+m(m+1)-n(n+1))}{(2 k)!}a_0
			\end{eqnarray*}
			Ahora realizaremos el mismo proceso para el caso impar.
			$$
			\begin{aligned}
				& k=1 \\
				& a_3=\frac{(1)(2 m+2)+m(m+1)-n(n+1)}{(3)(2)} a_1 \\
				& k=3 \\
				& a_5=\frac{3(2 m+4)+m(m+1)-n(n+1)}{(5)(4)} a_3 \\
				& a_5=\frac{((2 m+2)+m(m+1)-n(n+1))(3(2 m+4)+m(m+1)-n(n+1))}{(5)(4)(3)(2)} a_1 \\
				& k=5 \\
				& a_7=\frac{5(2 m+6)+m(m+1)-n(n+1)}{(7)(6)} a_5 \\
				& a_7=\frac{((2 m+2)+m(m+1)-n(n+1))(3(2 m+4)+m(m+1)-n(n+1))5(2 m+6)+m(m+1)-n(n+1)}{(7)(6)(5)(4)(3)(2)} a_1
			\end{aligned}
			$$
			De manera general tenemos para los coeficientes impares
			$$
			a_{2 k+1}=\frac{((2 m+2)+m(m+1)-n(n+1)) \cdots((2 k-1)(2 m+2 k)+m(m+1)-n(n+1)}{(2 k+1)!}a_1
			$$
		\end{sol}
		Ejemplos de polinomios asociados de Legendre\\
		\textcolor{red}{citar} sabemos que los polinomios de Legendre y Legendre asociados est\'an relacionados mediante la f\'ormula
		$$
		P_n^m(t)=\left(1-t^2\right)^{\frac{m}{2}} D^m P_n(t)
		$$
		$$\begin{aligned}
			&\text { As\'i, los primeros polinomios asociados de legendre, } \operatorname{con} t=\cos \theta \text {, son: }\\
			&P_n^0(t)=\left(1-t^2\right)^{\frac{0}{2}} D^0 P_n(t)=P_n(t) .\\
			&P_0^0(t)=P_1(t)=1 .\\
			&P_1^0(t)=P_1(t)=t=\cos \theta .\\
			&P_1^1(t)=\left(1-t^2\right)^{\frac{1}{2}} \frac{d}{d x} P_1(t)=\left(1-t^2\right)^{\frac{1}{2}} \frac{d}{d t} t=\left(1-t^2\right)^{\frac{1}{2}}=\operatorname{sen} \theta .\\
			&P_2^0(t)=P_2(t)=\frac{1}{2}\left(3 t^2-1\right)=\frac{1}{2}\left(3 \cos ^2 \theta-1\right) .\\
			&P_2^1(t)=\left(1-t^2\right)^{\frac{1}{2}} \frac{d}{d x} P_2(t)=\left(1-t^2\right)^{\frac{1}{2}} \frac{d}{d t}\left(\frac{1}{2}\left(3 t^2-1\right)\right)=3 t\left(1-t^2\right)^{\frac{1}{2}}=3 \cos \theta \operatorname{sen} \theta .\\
			&P_2^2(t)=\left(1-t^2\right)^{\frac{2}{2}} \frac{d^2}{d x^2} P_2(t)=\left(1-t^2\right) \frac{d^2}{d t^2}\left(\frac{1}{2}\left(3 t^2-1\right)\right)=3\left(1-t^2\right)=3 \operatorname{sen}^2 \theta .\\
			&P_3^0(t)=P_3(t)=\frac{1}{2}\left(5 t^3-3 t\right)=\frac{1}{2}\left(5 \cos ^3 \theta-3 \cos \theta\right)\\
			&P_3^1(t)=\left(1-t^2\right)^{\frac{1}{2}} \frac{d}{d x} P_3(t)=\left(1-t^2\right)^{\frac{1}{2}} \frac{d}{d x}\left(\frac{1}{2}\left(5 t^3-3 t\right)\right)=\frac{3}{2}\left(1-t^2\right)^{\frac{1}{2}}\left(5 t^2-1\right)\\
			&=\frac{3}{2} \operatorname{sen} \theta\left(5 \cos ^2 \theta-1\right) \text {. }\\
			&P_3^2(t)=\left(1-t^2\right)^{\frac{2}{2}} \frac{d^2}{d x^2} P_3(t)=\left(1-t^2\right) \frac{d^2}{d x^2}\left(\frac{1}{2}\left(5 t^3-3 t\right)\right)=15 t\left(1-t^2\right)=15 \cos \theta \operatorname{sen}^2 \theta .\\
			&P_3^3(t)=\left(1-t^2\right)^{\frac{3}{2}} \frac{d^3}{d x^3} P_3(t)=\left(1-t^2\right)^{\frac{3}{2}} \frac{d^3}{d x^3}\left(\frac{1}{2}\left(5 t^3-3 t\right)\right)=15\left(1-t^2\right)^{\frac{3}{2}}=15 \operatorname{sen}^3 \theta .
		\end{aligned}$$
		\subsection{Propiedades de los polinomios de Legendre}

		
		\Proposition{Sin nombre}{$\left(1-x^{2}\right) P_{n}^{\prime}(x)=(n+1) x P_{n}(x)-(n+1) P_{n+1}(x)$}
		
		\begin{demo}
			We can also obtain a recurrence formula involving $P_{n}^{\prime}$ by differentiating $H(x, r)$ with respect to $x$. This gives
			$$
			\frac{\partial H(x, r)}{\partial x}=\frac{r}{\left(1-2 x r+r^{2}\right)^{3 / 2}}=\displaystyle\sum_{n=0}^{\infty} P_{n}^{\prime}(x) r^{n},
			$$
			By explicit differentiation of $H(x, r)$ it is seen that
			$\left(1-2 x r+r^{2}\right) \textcolor{blue}{\frac{\partial H}{\partial x}}=r \textcolor{blue}{H(x, r)}$,  or
			$$
			\left(1-2 x r+r^{2}\right) \textcolor{blue}{\displaystyle\sum_{n=0}^{\infty} P_{n}^{\prime}(x) r^{n}}-r \textcolor{blue}{\displaystyle\sum_{n=0}^{\infty} P_{n}(x) r^{n}}=0 .
			$$
			As before, the coefficient of each power of $r$ is set to zero and we obtain
			\textcolor{blue}{
				$$
				P_{n+1}^{\prime}(x)+P_{n-1}^{\prime}(x)=2 x P_{n}^{\prime}(x)+P_{n}(x),\quad n=1,2,3 \dots  \quad (4).
				$$
			}
			Differentiating  Eq. (3), we get:
			\textcolor{blue}{
				$$
				(n+1) P^{\prime}_{n+1}(x)-(2 n+1)P_{n}(x)-(2 n+1) x P^{\prime}_{n}(x)+n P^{\prime}_{n-1}(x)=0 \quad (5).
				$$
			}
		\end{demo}
		
		\Proposition{From the Rodrigues formula for the Legendre polynomials}{From the Rodrigues formula for the Legendre polynomials, obtain the expression for them in summation form}
		
		\begin{demo}
			
			To remember an expression for the Legendre polynomials is Rodrigues' form:
			$$
			P_{n}(x)=\frac{1}{2^{n} n !}\left[\left(x^{2}-1\right)^{n}\right]^{(n)}
			$$
			
			
			To see why this equals the form you give, see the derivation below.
			$$
			\begin{aligned}
				P_{n}(x)=\frac{1}{2^{n} n !}\left[\left(x^{2}-1\right)^{n}\right]^{(n)} &=\frac{1}{2^{n} n !}\left[\displaystyle\sum_{k=0}^{n}\left(\begin{array}{c}
					n \\
					k
				\end{array}\right)\left(x^{2}\right)^{n-k} \cdot(-1)^{k}\right]^{(n)} \\
				&=\frac{1}{2^{n} n !} \displaystyle\sum_{k=0}^{n}(-1)^{k} \frac{n !}{k !(n-k) !}\left[x^{2 n-2 k}\right]^{(n)} \\
				&=\frac{1}{2^{n}} \displaystyle\sum_{k=0}^{n}(-1)^{k} \frac{1}{k !(n-k) !} \cdot(2 n-2 k)(2 n-2 k-1) \ldots(n-2 k+1) x^{n-2 k} \\
				&=\frac{1}{2^{n}} \displaystyle\sum_{k=0}^{n}(-1)^{k} \frac{1}{k !(n-k) !} \cdot \frac{(2 n-2 k) !}{(n-2 k) !} x^{n-2 k} \\
				&=\frac{1}{2^{n}} \displaystyle\sum_{k=0}^{n} \frac{(-1)^{k}(2 n-2 k) !}{k !(n-k) !(n-2 k) !} x^{n-2 k}
			\end{aligned}
			$$
			Using Rodrigues' from, we can further derive that
			$$
			\begin{aligned}
				P_{n}(x) &=\frac{1}{2^{n} n !}\left[(x+1)^{n} \cdot(x-1)^{n}\right]^{(n)} \\
				&=\frac{1}{2^{n} n !} \displaystyle\sum_{k=0}^{n}\left(\begin{array}{l}
					n \\
					k
				\end{array}\right)\left[(x+1)^{n}\right]^{(k)} \cdot\left[(x-1)^{n}\right]^{(n-k)} \\
				&=\frac{1}{2^{n} n !} \displaystyle\sum_{k=0}^{n}\left(\begin{array}{l}
					n \\
					k
				\end{array}\right) \frac{n !}{(n-k) !}(x+1)^{n-k} \cdot \frac{n !}{k !}(x-1)^{k} \\
				&=\frac{1}{2^{n}} \displaystyle\sum_{k=0}^{n}\left(\begin{array}{l}
					n \\
					k
				\end{array}\right)^{2}(x+1)^{n-k}(x-1)^{k}
			\end{aligned}
			$$
			so
			\textcolor{blue}{
				$$
				P_{2 n}(0)=\frac{1}{2^{2 n}} \displaystyle\sum_{k=0}^{2 n}\left(\begin{array}{c}
					2 n \\
					k
				\end{array}\right)^{2}(-1)^{k}=\frac{1}{2^{2 n}}\left(\begin{array}{c}
					2 n \\
					n
				\end{array}\right)(-1)^{n}=\frac{(-1)^{n}(2 n) !}{2^{2 n} n !^{2}}
				$$
			}
		\end{demo}
		
		\subsection{Funciones de Legendre de segundo tipo}
		\section{Ecuaci\'on de Gegenbauer. Polinomios de Gegenbauer.}
		Ecuaci\'on de Gegenbauer:
		\begin{equation}\label{J} (1-t^2)\,y^{\prime \prime}(t)-(2\lambda+1)t\,y^{\prime}(t)+n(n+2\lambda)\,y(t)=0. \end{equation}
		\textbf{Soluci\'on:}
		\\ Expresando la ecuaci\'on de Gegenbauer en la forma estandar tenemos:
		\begin{equation*} y^{\prime \prime}(t)+\frac{-(2\lambda+1)t}{1-t^2} y^{\prime}(t)+\frac{n(n+2\lambda)}{1-t^2}y(t)=0,
		\end{equation*}
		de donde\, $P(t)=\displaystyle \frac{-(2\lambda+1)t}{1-t^2}\, \, \text{y} \, \,  Q(t)=\displaystyle\frac{n(n+2\lambda)}{1-t^2}$.
		N\'otese que las funciones P(t) y Q(t) son anal\'iticas en el punto $t_{0}=0$, \,por tanto el punto $t_{0}=0$,\,es un punto ordinario de la ecuaci\'on diferencial \eqref{J}. Esto implica, por el teorema \eqref{teosolordinario} que \eqref{J} admite una soluci\'on de la forma $y(t)=\sum_{m=0}^\infty c_{m}(t-t_{0})^{m}.$
		Ahora, sea\, \begin{equation} \label{J4} y(t)=\sum_{m=0}^\infty c_{m}t^{m},\, \text{con}\, c_{0}\neq{0}\,\text{soluci\'on \,de}\, \eqref{J}.\end{equation} Entonces, derivando \eqref{J4} se obtiene:
		\begin{equation*} y^{\prime}(t)=\sum_{m=1}^\infty mc_{m}t^{m-1}\, (*)\, \text{y} \,\, y^{\prime}{\prime}(t)=\sum_{m=2}^\infty m(m-1)c_{m}t^{m-2}\, \, (**).\end{equation*}
		Luego, sustituyendo\eqref{J4},\,$(*)\, \text{y}(**)$ \, en \eqref{J}\, se obtiene:
		\begin{align*}
			(1-t^2)\left(\sum_{m=2}^\infty m(m-1)c_{m}t^{m-2}\right)-(2\lambda+1)t \left(\sum_{m=1}^\infty mc_{m}t^{m-1}\right)
			+n(n+2\lambda)\left(\sum_{m=0}^\infty c_{m}t^{m}\right)=0.\end{align*}
		\begin{align*}
			\Rightarrow \sum_{m=2}^\infty m(m-1)c_{m}t^{m-2}-\sum_{m=2}^\infty m(m-1)c_{m}t^{m}&
			-(2\lambda+1)\left(\sum_{m=1}^\infty mc_{m}t^{m}\right)+\\
			n(n+2\lambda)\left(\sum_{m=0}^\infty c_{m}t^{m}\right)=0
		\end{align*}
		Ahora, con\, $m\rightarrow m+2$\, y \, $m\rightarrow m+1$\, respectivamente obtenemos:
		\begin{align*}
			\Rightarrow \sum_{m=0}^\infty (m+2)(m+1)c_{m+2}t^{m}&-\sum_{m=2}^\infty m(m-1)c_{m}t^{m}
			-(2\lambda+1)\left(\sum_{m=1}^\infty mc_{m}t^{m}\right)+\\
			n(n+2\lambda)\left(\sum_{m=0}^\infty c_{m}t^{m}\right)=0
		\end{align*}
		Luego, tomando\, $m=0$\, y \, $m=1$\, se tiene:
		\begin{align*}
			\Rightarrow 2c_{2}+6c_{3}t+&\sum_{m=2}^\infty (m+2)(m+1)c_{m+2}t^{m}-\sum_{m=2}^\infty m(m-1)c_{m}t^{m}-(2\lambda+1)c_{1}t\\
			-(2\lambda+1)\left(\sum_{m=2}^\infty mc_{m}t^{m}\right)&+n(n+2\lambda)c_{0}+n(n+2\lambda)c_{1}t+n(n+2\lambda)\left(\sum_{m=2}^\infty c_{m}t^{m}\right)=0
		\end{align*}
		Simplificando:
		\begin{align*}
			\Rightarrow &\left(2c_{2}+n(n+2\lambda)c_{0}\right)+\left(6c_{3}-(2\lambda+1)c_{1}+n(n+2\lambda) c_{1})\right)t\\
			&+\left[\sum_{m=2}^\infty (m+2)(m+1)c_{m+2}-\left(m(m-1)-(2\lambda+1)m-n(n+2\lambda)\right)c_{m}\right]t^{m}=0.
		\end{align*}
		\begin{align*}
			\Rightarrow &\left(2c_{2}+n(n+2\lambda)c_{0}\right)+\left(6c_{3}+(n-1)(n+2\lambda+1)c_{1}\right)t\\
			&+\left[\sum_{m=2}^\infty (m+2)(m+1)c_{m+2}+\left((n-m)(n+m+2\lambda)\right)c_{m}\right]t^{m}=0.
		\end{align*}
		Luego, como $t^m\neq0, \, \forall{m}$, se tiene:
		\begin{align*}
			\Rightarrow \begin{cases}
				2c_{2}+n(n+2\lambda)c_{0}=0 & \text{}\\
				6c_{3}+(n-1)(n+2\lambda+1)c_{1}=0 & \text{}\\
				(m+2)(m+1)c_{m+2}-\left((n-m)(n+m+2\lambda)\right)c_{m}=0 & \text{}
			\end{cases}
		\end{align*}
		\begin{align*}
			\Rightarrow \begin{cases}
				c_{2}=-\frac {n(n+2\lambda)c_{0}}{2} & \text{}\\
				c_{3}=-\frac{(n-1)(n+2\lambda+1)c_{1}}{6} & \text{}\\
				c_{m+2}=-\frac{\left((n-m)(n+m+2\lambda)\right)c_{m}}{(m+2)(m+1)} & \text{}\, \forall{m}\in{N}/m\geq{0}
			\end{cases}
		\end{align*}
		As\'i la relaci\'on de recurencia est\'a dada por la expresi\'on:
		\begin{equation}\label{J7}
			c_{m+2}=-\frac{(n-m)(n+m+2\lambda)c_{m}}{(m+2)(m+1)},\, \forall{m}\in{N}/m\geq{0}
		\end{equation}
		
		Ahora d\'andole valores pares a $m$, \, de \eqref{J7} \, se obtiene:
		\begin{align*}
			Si\, m=0 \Rightarrow c_{2}&=-\frac {n(n+2\lambda)c_{0}}{2}.\\
			Si \,m=2 \Rightarrow c_{4}&=-\frac{(n+2+2\lambda))(n-2)}{4\cdot(3)}c_{2}=\frac{(n+2\lambda+2)(n+2\lambda)n(n-2)}{4\cdot{3}\cdot{2}}c_{0}.\\
			Si \,m= 4\Rightarrow c_{6}&=-\frac{(n+4+2\lambda)(n-4)}{6\cdot{5}}c_{4}\\
			c_{6}&=-\frac{(n+2\lambda+4)(n+2\lambda+2)(n+2\lambda)n(n-2)(n-4)}{6\cdot{5}\cdot{4}\cdot{3}\cdot{2}}c_{0}.\\
			\vdots
		\end{align*}
		\begin{equation*}\Rightarrow c_{2m}=(-1)^{m}\frac{(n+2\lambda+2m)(n+2\lambda+2m-2)\cdots(n+2\lambda)n(n-2)\cdots(n-2m+2)}{(2m)!}c_{0}.\end{equation*}
		\begin{equation*}\text{Luego},\, (n+2\lambda+2m)\cdots(n+2\lambda+4)(n+2\lambda+2)(n+2\lambda)=\prod_{k=0}^{m}(n+2k+2\lambda)=2^{m}\prod_{k=0}^{m}(\frac{n}{2}+k+\lambda).\end{equation*}
		\begin{equation*}\Rightarrow \text{Luego},\, (n+2\lambda+2m)\cdots(n+2\lambda+4)(n+2\lambda+2)(n+2\lambda)=2^{m}\frac{\Gamma{(\frac{n}{2}+m+\lambda)}}{\Gamma{(\frac{n}{2}+\frac{1}{2})}},\end{equation*}
		\begin{equation*}y\, n(n-2)(n-4)\cdots(n-2m+2)=\prod_{k=0}^{m}(n-2k+2)=2^{m}\prod_{k=0}^{m}(\frac{n}{2}-k+1)=2^{m}\frac{\Gamma{(\frac{n}{2}+1)}}{\Gamma{(\frac{n}{2}-m+1)}}.\end{equation*}
		\begin{equation*}\text{As'i},\,\label{J8} c_{2m}=\frac{(-1)^{m}}{(2m)!}2^{m}\frac{\Gamma{(\frac{n}{2}+m+\lambda)}}{\Gamma{(\frac{n}{2}+\frac{1}{2})}}2^{m}\frac{\Gamma{(\frac{n}{2}+1)}}{\Gamma{(\frac{n}{2}-m+1)}}c_{0}.\end{equation*}
		\begin{equation}\label{J8} \Rightarrow c_{2m}=(-1)^{m}\frac{2^{2m}\Gamma{(\frac{n}{2}+1)}\Gamma{(\frac{n}{2}+m+\lambda)}}{(2m)!\Gamma{(\frac{n}{2}+\frac{1}{2})}\Gamma{(\frac{n}{2}-m+1)}}c_{0}.
		\end{equation}
		Ahora d\'andole valores impares a $m$, \, de \eqref{J7} \, se obtiene:
		\begin{align*}
			Si\, m=1 \Rightarrow c_{3}&=-\frac{(n+2\lambda+1)(n-1)}{3\cdot{2}}c_{1}.\\
			Si \,m=3 \Rightarrow c_{5}&=-\frac{(n+2\lambda+3)(n-3)}{5\cdot(4)}c_{3}=\frac{(n+2\lambda+3)(n+2\lambda+1)(n-1)(n-3)}{5\cdot{4}\cdot{3}\cdot{2}}c_{1}.\\
			Si \,m= 5\Rightarrow c_{7}&=-\frac{(n+2\lambda+5)(n-5)}{7\cdot{6}}c_{5}.\\
			c_{7}&=-\frac{(n+2\lambda+5)(n+2\lambda+3)(n+2\lambda+1)(n-1)(n-3)(n-5)}{7\cdot{6}\cdot{5}\cdot{4}\cdot{3}\cdot{2}}c_{1}.\\
			\vdots
		\end{align*}
		\begin{equation*}\Rightarrow c_{2m+1}=(-1)^{m}\frac{(n+2\lambda+2m+1)\cdots(n+2\lambda+1)(n-1)(n-3)\cdots(n-2m+1)}{(2m+1)!}c_{1}.\end{equation*}
		\begin{equation*}Luego,\, (n+2\lambda+2m+1)(n+2\lambda+2m-1)\cdots(n+2\lambda+1)=\prod_{k=0}^{m}(n+2\lambda+2k+1)\end{equation*}
		\begin{equation*}=2^{m}\prod_{k=0}^{m}(\frac{n}{2}+\lambda+k+\frac{1}{2})=2^{m}\frac{\Gamma{(\frac{n}{2}+\lambda+m+\frac{1}{2})}}{\Gamma{(\frac{n}{2}+\lambda+\frac{1}{2})}},\end{equation*}
		\begin{equation*}\text{y}\, \, (n-1)(n-3)\cdots(n-2m+1)=\prod_{k=0}^{m}(n-2k+1)=2^{m}\prod_{k=0}^{m}(\frac{n}{2}-k+\frac{1}{2})=2^{m}\frac{\Gamma{(\frac{n}{2}+\frac{1}{2})}}{\Gamma{(\frac{n}{2}-m+\frac{1}{2})}}.\end{equation*}
		As\'i, \, \begin{equation*} c_{2m+1}=\frac{(-1)^{m}}{(2m+1)!}2^{m}\frac{\Gamma{(\frac{n}{2}+\lambda+m+\frac{1}{2})}}{\Gamma{(\frac{n}{2}+\lambda+\frac{1}{2})}}\frac{\Gamma{(\frac{n}{2}+\frac{1}{2})}}{\Gamma{(\frac{n}{2}-m+\frac{1}{2})}}c_{1}.\end{equation*}
		\begin{equation} \label{J9} \Rightarrow c_{2m+1}=(-1)^{m}\frac{2^{2m+1}\Gamma{(\frac{n}{2}+\frac{1}{2})}\Gamma{(\frac{n}{2}+\lambda+m+\frac{1}{2})}}{2\cdot(2m+1)!\Gamma{(\frac{n}{2}+\lambda+\frac{1}{2})}\Gamma{(\frac{n}{2}-m+\frac{1}{2})}}c_{1}.
		\end{equation}
		As\'i, con las sustituciones de \eqref{J8}\,y\,\eqref{J9} en la expresi'on $y(x)=\sum_{m=0}^\infty c_{m}x^{m},\, con\,\\ c_{0}\neq{0},$\, obtenemos la \textit{soluci'on general} de \eqref{J}:
		\begin{align*}
			y(x)&=c_{0}x^{0}\left(1-\frac{n(n+2\lambda)}{2!}x^{2}+\frac{(n+2\lambda+2)(n+2\lambda)n(n-2)}{4!}x^{4}-\cdots \right)\\
			&+c_{1}\left(x-\frac{(n+2\lambda+1)(n-1)}{3!}x^{3}+\frac{(n+2\lambda+3)(n+2\lambda+1)(n-1)(n-3)}{5!}x^{5}- \cdots\right)\end{align*}
		\begin{equation*}\Rightarrow y(x)=c_{0}y_{1}(x)+c_{1}y_{2}(x)=c_{0}\sum_{m=0}^{\infty} (-1)^{m}\frac{2^{2m}\Gamma{(\frac{n}{2}+1)}\Gamma{(\frac{n}{2}+m+\lambda)}}{(2m)!\Gamma{(\frac{n}{2}+\frac{1}{2})}\Gamma{(\frac{n}{2}-m+1)}}x^{2m} +\end{equation*}
		\begin{equation}\label{J11}c_{1}\sum_{m=0}^{\infty}(-1)^{m}\frac{2^{2m+1}\Gamma{(\frac{n}{2}+\frac{1}{2})}\Gamma{(\frac{n}{2}+\lambda+m+\frac{1}{2})}}{2\cdot(2m+1)!\Gamma{(\frac{n}{2}+\lambda+\frac{1}{2})}\Gamma{(\frac{n}{2}-m+\frac{1}{2})}}x^{2m+1}.
		\end{equation}
		Finalmente, para obtener los polinomios de Gegenbauer, tomamos la constante de normalizaci\'on $$c_{n}=\frac{(2\lambda+2n-1)!(\lambda-\frac{1}{2})!}{2^{n}(n!)(2\lambda-1)!(\lambda+n-\frac{1}{2})!},$$\, e iniciamos un proceso regresivo a partir de la f\'ormula de recurrencia \eqref{J7}
		\begin{equation*}
			c_{m+2}=-\frac{(n-m)(n+m+2\lambda)c_{m}}{(m+2)(m+1)},\, \forall{m}\in{N}/m\geq{0}
		\end{equation*}
		Ahora, despejando a $c_{m}$ \, de \eqref{J7} \, se obtiene:
		\begin{equation} \label{J12} c_{m}=-\frac{(m+2)(m+1)}{(n-m)(n+m+2\lambda)}c_{m+2}, \forall{m}\in{N}/m\leq{n+2}. \end{equation}
		Luego, con $m\rightarrow n-2$\, en \eqref{J12} y sustituyendo la constante de normalizaci\'on se obtiene:
		\begin{equation*} c_{n-2}=-\frac{n(m-1)}{2(2n+2\lambda-2)}c_{n}=-\frac{n(m-1)}{2(2n+2\lambda-2)}\frac{(2\lambda+2n-1)!(\lambda-\frac{1}{2})!}{2^{n}(n!)(2\lambda-1)!(\lambda+n-\frac{1}{2})!}.\end{equation*}
		Simplificando:
		\begin{equation} \label{J13} c_{n-2}=-\frac{(2n+2\lambda-3)!}{2^{n}(n-2)!(\lambda+n-\frac{3}{2})!}\frac{(\lambda-\frac{1}{2})!}{(2\lambda-1)!}.\end{equation}
		De igual forma, tomando $m\rightarrow n-4$\,  en \eqref{J12}  y sustituyendo \eqref{J13} respectivamente, se obtiene:
		\begin{equation*}  c_{n-4}=-\frac{(n-2)(n-3)}{4(2n+2\lambda-4)}c_{n-2}=-\frac{(n-2)(n-3)}{4(2n+2\lambda-4)}\left(-\frac{(2n+2\lambda-3)!}{2^{n}(n-2)!(\lambda+n-\frac{3}{2})!}\frac{(\lambda-\frac{1}{2})!}{(2\lambda-1)!}\right). \end{equation*}
		Reduciendo:
		\begin{equation} \label{J14} c_{n-4}=\frac{(2n+2\lambda-5)!}{2\cdot 2^{n}(n-4)!(\lambda+n-\frac{5}{2})!}\frac{(\lambda-\frac{1}{2})!}{(2\lambda-1)!}.\end{equation}
		Siguiendo este proceso de manera iterativa llegamos a la expresi\'on:
		\begin{equation}\label{J18}c_{n-2m}=(-1)^{m}\frac{(2n+2\lambda-2m-1)!}{m!\cdot 2^{n}(n-2m)!(\lambda+n-m-\frac{1}{2})!}\frac{(\lambda-\frac{1}{2})!}{(2\lambda-1)!.}\end{equation}
		\textit{Coeficientes de los polinomios de Gegenbauer.}
		
		Luego, aplicando la f\'ormula de duplicaci\'on de Legendre \textit{(ver ap'endice A, propiedad (6))}, se tienen las siguientes equivalencias que sustituieremos  en la expresi\'on anterior para reescribir los coeficientes de los polinomios de Gengenbauer:
		\begin{equation*}(2n+2\lambda-2m-1)!=\Gamma{(2\cdot(n+\lambda-m))}=\frac{2^{2(n+\lambda-m)-1}\Gamma{(n+\lambda-m)}\cdot \Gamma{(n+\lambda-m+\frac{1}{2})}}{\Gamma{(\frac{1}{2})}},\end{equation*}
		\begin{equation*}\text{y}\, \, (2\lambda-1)!=\Gamma{(2\cdot(\lambda))}=\frac{\Gamma{(\frac{1}{2})}}{2^{2(\lambda)-1}\Gamma{(\lambda)}\cdot \Gamma{(\lambda+\frac{1}{2})}}.\end{equation*}
		\begin{equation}\label{J15}\Rightarrow c_{n-2m}=\frac{(-1)^{m}\Gamma{(n+\lambda-m)}\cdot 2^{n-2m}}{\Gamma{(\lambda)}m!\cdot (n-2m)!}.\end{equation}
		Finalmente, sustituyendo \eqref{J15} en \eqref{J4}
		$ y(t)=\sum_{m=0}^\infty c_{m}t^{m}$\, se concluye:
		$$ y(t)=\sum_{m=0}^\infty c_{m}t^{m}=\sum_{m=0}^{M} c_{n-2m}t^{n-2m}=\sum_{m=0}^{M} \frac{(-1)^{m}\Gamma{(n+\lambda-m)}}{\Gamma{(\lambda)}m!\cdot (n-2m)!}\,(2t)^{n-2m};$$
		donde $M=\frac{n}{2}$\, si n es par y $M=\frac{n-1}{2}$\, si n es impar $\Rightarrow M=[\frac{n}{2}]$ (Funci\'on mayor entero de $\frac{n}{2}$).
		\begin{equation}\label{J16}\Rightarrow C_{n}^{(\lambda)}(t)=\sum_{m=0}^{[\frac{n}{2}]}\frac{(-1)^{m}\Gamma{(n+\lambda-m)}}{\Gamma{(\lambda)}m!\cdot (n-2m)!}\,(2t)^{n-2m}.\,\textbf{Polinomios de Gegenbauer.}\end{equation}
\section{Relaci\'on de los polinomios ortogonales cl\'asicos}
		\subsection{Relaci\'on de los polinomios de Hermite con los de Laguerre}
		
		\begin{eqnarray}\label{hermitelaguerre}
			% \nonumber % Remove numbering (before each equation)
			H_{2 m}(x)=(-1)^m 2^{2 m} m ! L_m^{\left(-\frac{1}{2}\right)}\left(x^2\right), \quad H_{2 m+1}(x)=(-1)^m 2^{2 m+1} m ! x L_m^{\left(\frac{1}{2}\right)}\left(x^2\right)
		\end{eqnarray}
	\section{Ecuaci\'on diferencial de Bessel}
		\textcolor[rgb]{1.00,0.00,0.00}{Resorte envejecido y otras aplicaciones}
		La ecuaci\'on diferencial de Bessel se define por
		\begin{eqnarray}\label{ecuacionbessel}
			% \nonumber % Remove numbering (before each equation)
			x^2 y^{\prime \prime}+x y^{\prime}+\left(x^2-a^2\right) y&=&0
		\end{eqnarray}
		Expresando \ref{ecuacionbessel} a su forma normal, tenemos
		$$\begin{gathered}
			y^{\prime \prime}+\frac{x}{x^2} y^{\prime}+\frac{x^2-a^2}{x^2} y=0 \\
			p_{1}(x)=\frac{1}{x} \quad p_{2}(x)=\frac{x^2-a^2}{x^2} \\
			p(x)=xp_{1}(x)=1 \quad \text { y } \quad x^2 p_{2}(x)=x^2-a^2=q(x)
		\end{gathered}$$
		de donde $x_{0}=0$ es un punto singular regular, por lo tanto existe una soluci\'on dada \ref{solucionfrobenius}.\\
		De \ref{ecuacionindicial} tenemos
		$$
		\begin{aligned}
			&r(r-1)+r+\left(-a^2\right)=0 \\
			&r^2-r+r-a^2=0 \\
			&r^2-a^2=0 \\
			&(r-a)(r+a)=0 \\
			&r=a \quad r_a=-a
		\end{aligned}
		$$
		Calculando las derivadas
		$$\begin{aligned}
			&y(x)=x^a \sum_{m=0}^{\infty} c_m x^m, C_0 \neq 0 \\
			&y^{\prime}(x)=\sum_{m=0}^{\infty}(a+m) C_m x^{m+a-1} \\
			&y^{\prime\prime}(x)=\sum_{m=0}^{\infty}(m+a)(m+a-1) C_m x^{m+a-2}
		\end{aligned}$$
		Reemplazando las derivadas
		$$ \begin{aligned}
			&\quad x^2 y^{\prime \prime}=\sum_{m=0}^{\infty}(m+a)(m+a-1)c_m x^{m+a} \\
			&+\quad\\
			& x y^{\prime}=\sum_{m=0}^{\infty}(m+a)c_{m}x^{m+a} \\
			&x^2 y^{\prime \prime}+x y^{\prime}=\sum_{m=0}^{\infty} c_{m}(m+a)^2 x^{m+a}
		\end{aligned}$$
		Desarrollando \ref{ecuacionbessel}
		\begin{eqnarray*}
			x^2 y^{\prime \prime}+x y^{\prime}+x^2y-a^2 y&=&0\\
			\displaystyle\sum_{m=0}^{\infty}c_{m}(m+a)^{2}x^{m+a}+\displaystyle\sum_{m=0}^{\infty}c_{m}x^{m+a+2}-\displaystyle\sum_{m=0}^{\infty}a^{2}c_{m}x^{m+a}&=&0
		\end{eqnarray*}
		Sumando las series semejantes
		\begin{eqnarray*}
			\sum_{m=0}^{\infty} c_m\left[(m+a)^2-a^2\right] x^{m+a}+\sum_{m=0}^{\infty} c_m x^{m+a+2}&=&0
		\end{eqnarray*}
		Tomando los dos primeros t\'erminos de la primera serie y haciendo un corrimiendo a la segunda
		\begin{eqnarray*}
			c_{0}\left[(a)^{2}-a^{2}\right]x^{a}+c_{1}\left[(1+a)^{2}-a^{2}\right]x^{1+a}+\sum_{m=2}^{\infty}c_{m}\left[(m+a)^{2}-a^{2}\right]x^{m+a}+\sum_{m=2}^{\infty} c_{m-2}x^{m+a}&=&0\\
			c_{1}\left[(1+a)^{2}-a^{2}\right]x^{1+a}+\sum_{m=2}^{\infty}c_{m}\left[(m+a)^{2}-a^{2}+c_{m-2}\right]x^{m+a}&=&0\\
		\end{eqnarray*}
		Igualando los coeficientes a cero
		\begin{eqnarray*}
			c_{1}\left[(1+a)^{2}-a^{2}\right]&=&0\\
			c_{1}\left[(1+a-a)(1+a+a)\right]&=&0\\
			c_{1}\left[(1)(1+2a)\right]&=&0\\
			\text{si $r_{1}-r_{2}=2a \in \mathbb{Z}$}\\
			c_{1}&=&0\\
		\end{eqnarray*}
		De igual manera para serie
		\begin{eqnarray*}
			c_m&=-&\frac{c_{m-2}}{(m+a)^2-a^2}\quad\forall m\geq 2\\
			c_m&=-&\frac{c_{m-2}}{(m+a-a)(m+a+a)}\quad\forall m\geq 2\\
			c_m&=-&\frac{c_{m-2}}{m(m+2a)}\quad\forall m\geq 2
		\end{eqnarray*}
		Obteniendo los coeficientes pares
		\begin{eqnarray*}
			\text{Si m=2} \rightarrow c_{2}&=&-\frac{c_0}{2(2+2a)}=\displaystyle-\frac{c_{0}}{2^{2}(1+a)}\\
			\text{si m=4} \rightarrow c_{4}&=&-\frac{c_{2}}{2^{2}(4+2a)}\\
			c_{4}&=&\frac{c_0}{2^{4}2!(1+a)(2+a)}\\
			\text{Si m=6} \rightarrow c_{6}&=&-\frac{c_{4}}{6(6+2a)}\\
			c_{6}&=&-\displaystyle\frac{1}{2\cdot 6(3+a)}\displaystyle\frac{c_{0}}{2^{4}2!(1+a)(2+a)}=\displaystyle\frac{-c_{0}}{2^{6}3!(1+a)(2+a)(3+a)}\\
			\vdots
		\end{eqnarray*}
		\begin{eqnarray}\label{Besselc2m}
			c_{2m}&=&\frac{(-1)^{m}}{2^{2m}m!\displaystyle\prod_{i=1}^{m}(i+a)}c_{0}
		\end{eqnarray}
		Ahora los coeficientes impares
		\begin{eqnarray*}
			\text{Si m=3} \rightarrow c_3&=&-\frac{c_1}{3(3+2a)}\\
			c_3&=&0\\
			\text{si m=5} \rightarrow c_3&=&-\frac{c_3}{5(5+2a)}\\
			c_5&=&0\\
			\vdots\\
			c_{2 m+1}&=&0 \quad \forall m \geq 1
		\end{eqnarray*}
		de manera que la soluci\'on es de la forma
		$$\begin{array}{r}
			y_1(x)=C_0 x^a\left[1-\displaystyle\frac{1}{2^2 1 !(1+a)} x^2+\displaystyle\frac{1}{2^4 2 !(1+a)(2+a)} x^4\right. \\
			\left.-\displaystyle\frac{1}{2^6 3 !(1+a)(2+a)(3+a)} x^6+\cdots\right]
		\end{array}$$
		Tomando la constante  $c_0=\displaystyle\frac{1}{2^a \Gamma(1+a)}$
		as\'i de \ref{Besselc2m}
		\begin{eqnarray*}
			c_{2m}&=&\displaystyle\frac{(-1)^m}{2^{2m} m ! \displaystyle\prod_{i=1}^{m}(i+a)} \frac{1}{2^{a}\Gamma(1+a)}
		\end{eqnarray*}
		Simplificando el denominador y de \ref{Relaci\'on de recurrencia de la funci\'on gamma} se tiene
		\begin{eqnarray*}
			c_{2m}&=&\displaystyle \frac{(-1)^m}{2^{2m+a} m ! \Gamma(m+1+a)}
		\end{eqnarray*}
		La soluci\'on toma la forma
		\begin{eqnarray}\label{besseluno}
			% \nonumber % Remove numbering (before each equation)
			y_{1}(x) &=& \displaystyle\sum_{m=0}^{\infty}\frac{(-1)^{m}}{m!\Gamma(m+1+a)}\left(\frac{x}{2}\right)^{2m+a}
		\end{eqnarray}
		Si tomamos $a=0$ llegamos a
		\begin{eqnarray}\label{bessela0}
			% \nonumber % Remove numbering (before each equation)
			y_{1}(x) &=& \displaystyle\sum_{m=0}^{\infty}\frac{(-1)^{m}}{m!\Gamma(m+1+a)}\left(\frac{x}{2}\right)^{2m}=\displaystyle\sum_{m=0}^{\infty}\frac{(-1)^{m}}{(m!)^{2}}\left(\frac{x}{2}\right)^{2m}
		\end{eqnarray}
		Para el caso  $r=-a$
		$$\begin{aligned}
			&m(m-2a) c_m=-c_{m-2},\quad m=2, 3, \cdots\\
			&c_m=-\frac{c_{m-2}}{m(m-2 a)}\\
			&m=2 \longrightarrow c_{2}=-\frac{c_0}{2(2-2 a)}=-\frac{c_0}{2^2(1-a)}\\
			&m=4 \rightarrow c_4=-\frac{c_2}{4(4-2 a)}=\frac{c_0}{2^4 2 !(1-a)(2-a)}\\
			&m=6 \rightarrow c_6=-\frac{c_4}{6(6-2 a)}=-\frac{c_0}{2^6 3 !(1-a)(2-a)(3-a)}\\
			\vdots
		\end{aligned}$$
		\begin{eqnarray*}
			c_{2m}&=&\frac{(-1)^{m}}{2^{2m}m!\displaystyle\prod_{i=1}^{m}(i-a)}c_{0}
		\end{eqnarray*}
		Tomando la constante  $c_0=\displaystyle\frac{1}{2^{-a} \Gamma(1-a)}$
		tenemos
		\begin{eqnarray*}
			c_{2m}&=&\frac{(-1)^{m}}{2^{2m}m!\displaystyle\prod_{i=1}^{m}(i-a)}\displaystyle\frac{1}{2^{-a} \Gamma(1-a)}
		\end{eqnarray*}
		de \ref{Relaci\'on de recurrencia de la funci\'on gamma} se tiene
		\begin{eqnarray*}
			c_{2m}&=&\frac{(-1)^{m}}{2^{2m-a}m!\Gamma(m+1-a)}
		\end{eqnarray*}
		La segunda soluci\'on es
		\begin{eqnarray}\label{besseldos}
			y_2(x)&=&\sum_{m=0}^{\infty} \frac{(-1)^m}{m ! \Gamma(m+1-a)}\left(\frac{x}{2}\right)^{2 m-a}
		\end{eqnarray}
		Las soluciones de la ecuaci\'on de Bessel dada por \ref{ecuacionbessel} est\'an dada por
		\begin{eqnarray}\label{ja}
			% \nonumber % Remove numbering (before each equation)
			J_{a}(x)&=& \sum_{m=0}^{\infty} \frac{(-1)^m}{m ! \Gamma(m+1+a)}\left(\frac{x}{2}\right)^{2 m+a}
		\end{eqnarray}
		\begin{eqnarray}\label{-ja}
			% \nonumber % Remove numbering (before each equation)
			J_{-a}(x)&=& \displaystyle\sum_{m=0}^{\infty}\frac{(-1)^{m}}{m!\Gamma(m+1-a)}\left(\frac{x}{2}\right)^{2m-a}
		\end{eqnarray}
		\textbf{Nota}
		Las funciones de Bessel $J_{n}\quad y \quad J_{-n}$ son linealmente independiente, si y solo si, n no es un entero.\\
		Si $n=v \in \mathbb{N}_0$, then
		$$
		J_{-v}(x)=\left(\frac{x}{2}\right)^{-v} \sum_{m=0}^{\infty} \frac{(-1)^m}{m ! \Gamma(m-v+1)}\left(\frac{x}{2}\right)^{2 m} .
		$$
		ya que $1 / \Gamma(m-v+1)=0$ for all $m-v+1 \leq 0$, los t\'erminos para $m=0,1, \ldots, n-1$ son cero
		$$\begin{aligned}
			J_{-v}(x) &=\left(\frac{x}{2}\right)^{-v} \sum_{m=v}^{\infty} \frac{(-1)^m}{m ! \Gamma(m-v+1)}\left(\frac{x}{2}\right)^{2 m} \\
			&=\left(\frac{x}{2}\right)^{-v} \sum_{m=0}^{\infty} \frac{(-1)^{m+v}}{(m+v) ! \Gamma(m+1)}\left(\frac{x}{2}\right)^{2 m+2 v} \\
			&=(-1)^v\left(\frac{x}{2}\right)^v \sum_{m=0}^{\infty} \frac{(-1)^m}{m ! \Gamma(m+v+1)}\left(\frac{x}{2}\right)^{2 m} \\
			&=(-1)^v J_v(x) .
		\end{aligned}$$
		\textcolor{red}{hacer grafica polinomios de bessel}
		$$\begin{aligned}
			&y_0(x)=1, \\
			&y_1(x)=1+x, \\
			&y_2(x)=1+3 x+3 x^2 \\
			&y_3(x)=1+6 x+15 x^2+15 x^3, \\
			&y_4(x)=1+10 x+45 x^2+105 x^3+105 x^4, \\
			&y_6(x)=1+15 x+105 x^2+420 x^8+945 x^4+945 x^5 .
		\end{aligned}$$
		
		\Property{Propiedad}{\begin{eqnarray*}
				% \nonumber % Remove numbering (before each equation)
				I_p(x)&=& i^{-p} J_{p}(ix)
		\end{eqnarray*}}
		
		
		\begin{demo}
			Sabemos que
			\begin{eqnarray*}
				J_{p}(ix)&=&\displaystyle\sum_{n=0}^{\infty}\frac{(-1)^{n}}{n!\Gamma(1+n+p)}(\frac{ix}{2})^{2n+p}\\
				J_{p}(ix)&=&\displaystyle\sum_{n=0}^{\infty}\frac{i^{2n+p}(-1)^{n}}{n!\Gamma(1+n+p)}(\frac{x}{2})^{2n+p}\\
				\text{Desarrollando la potencia de i}\\
				J_{p}(ix)&=&\displaystyle\sum_{n=0}^{\infty}\frac{i^{p}(-1)^{n}(-1)^{n}}{n!\Gamma(1+n+p)}(\frac{x}{2})^{2n+p}\\
				\text{Simplificando y tomando factor com\'un}\\
				J_{p}(ix) &=&i^{p}\displaystyle\sum_{n=0}^{\infty}\frac{1}{n!\Gamma(1+n+p)}(\frac{x}{2})^{2n+p}=i^{p}I_{p}(x)\\
				I_{p}(x)&=&i^{-p}J_{p}(ix)
			\end{eqnarray*}
		\end{demo}
	\subsection{Funci\'on Generadora de la funci\'on de Bessel}
		\Theorem{Funci\'on Generadora de la funci\'on de Bessel}{La funci\'on generadora de la funci\'on de Bessel est\'a dada por la expresi\'on
			\begin{eqnarray}\label{generadorabessel}
				% \nonumber % Remove numbering (before each equation)
				e^{\displaystyle\frac{1}{2} x\left(t-\frac{t}{t}\right)}&=&\displaystyle\sum_{n=0}^{\infty} J_n\left(x\right)t^n
		\end{eqnarray}}
		
		\begin{demo}
			Para demostrar este teorema desarrollaremos el lado izquierdo de la expresion (\ref{generadorabessel}) en potencia de $t$ y mostraremos que el coeficiente de $t^n$ es $J_n(x)$.
			$$
			e^{\displaystyle\frac{1}{2} x\left(t-\frac{1}{t}\right)}=e^{\frac{1}{2} x t} e^{-\frac{x}{2 t}}
			$$
			Por la serie de Maclaurin de la funci\'on exponencial
			$$
			\begin{aligned}
				e^{\displaystyle\frac{1}{2} x\left(t-\frac{1}{t}\right)} & =\sum_{s=0}^{\infty} \frac{(x t)^s}{2^s s !} \sum_{r=0}^{\infty} \frac{(-x)^r}{2^r t^r r !} \\
				& =\sum_{s=0}^{\infty} \sum_{r=0}^{\infty}(-1)^r\left(\frac{x}{2}\right)^{s+r} \frac{t^{s-r}}{s ! r !}
			\end{aligned}
			$$
			Haciendo un cambio de variable en la sumatoria tenemos
			$$
			\sum_{m=-r}^{\infty}\left[\sum_{r=0}^{\infty} \frac{(-1)^r\left(\frac{x}{2}\right)^{2 r+m}}{r !(m+r) !}\right] t^m
			$$
			
			Donde el coeficiente de $t^m$ en la expresi\'on anterior es la funci\'on de Bessel.
		\end{demo}
		\subsection{Relaci\'on de recurrencia de la funci\'on Bessel}
		Ahora demostraremos algunas importantes relaciones de la funci\'on Bessel, incluyendo la relaci\'on de recurrencia.
		
		\Theorem{Bessel}{\begin{eqnarray}\label{besselrelacion}
				% \nonumber % Remove numbering (before each equation)
				x J_a^{\prime}(x)&=&a J_a(x)-x J_{a+1}(x)
		\end{eqnarray}}
		
		\begin{demo}
			Derivando la expresi\'on (\ref{ja}) obtenemos
			$$J_a^{\prime}(x)=\displaystyle\sum_{m=0}^{\infty}\displaystyle \frac{(-1)^m(2 m+a)}{m !(m+a) !} \frac{1}{2}\left(\frac{x}{2}\right)^{2 m+a-1}$$
			As\'i el lado de la izquierda de la expresi\'on (\ref{besselrelacion}) se puede escribir
			$$\begin{aligned}
				x J_a^{\prime}(x)= & a \sum_{m=0}^{\infty} \frac{(-1)^m}{m !(m+a) !}\left(\frac{x}{2}\right)^{2 m+a} \\
				& +x \sum_{m=1}^{\infty} \frac{(-1)^m}{(m-1) !(m+a) !}\left(\frac{x}{2}\right)^{2 m+a-1} \\
				= & a J_a(x)+x \sum_{m=0}^{\infty} \frac{(-1)^{m+1}}{m !(m+a+1) !}\left(\frac{x}{2}\right)^{2 m+a+1} \\
				= & a J_a(x)-x J_{a+1}(x)
			\end{aligned}$$
		\end{demo}
	\subsection{Forma integral de la funci\'on de Bessel}
		\Proposition{Forma integral de la funci\'on de Bessel}{
			Sea $n$ un entero no negativo, tenemos
			\begin{equation}\label{52}
				J_n(x)=\displaystyle\frac{1}{\pi} \int_0^\pi \cos (n \theta-x \sin \theta) d \theta
			\end{equation}
			Esta f\'ormula se conoce como forma integral de  Bessel para $J_n$}
		
		
		Realizando el cambio de variable $t=e^{i \theta}$ en la expresi\'on in (\ref{eq51}), y usando la identidad $e^{i \theta}=\cos \theta+i \sin \theta$ tenemos
		\begin{eqnarray*}
			e^{(x / 2) [ t-(1 / t)]}&=&e^{i x \sin \theta}=\cos (x \sin \theta)+i \sin (x \sin \theta)
		\end{eqnarray*}
		De la expresi\'on (\ref{eq51}) tenemos que
		\begin{eqnarray*}
			\cos (x \sin \theta)+i \sin (x \sin \theta)&=&\displaystyle\sum_{n=-\infty}^{\infty} J_n(x)[\cos n \theta+i \sin n \theta]
		\end{eqnarray*}
		
		Ahora igualando las partes reales e imaginarias y por la identidad (\ref{-jn}) llegamos a
		\begin{equation}\label{eq53}
			\begin{aligned}
				\cos (x \sin \theta)=J_0(x)+2 \displaystyle\sum_{n=1}^{\infty} J_{2 n}(x) \cos 2 n \theta,\quad
				\sin (x \sin \theta)=2 \displaystyle\sum_{n=0}^{\infty} J_{2 n+1}(x) \sin (2 n+1) \theta . \footnote{The validity of these computations depends upon the fact that (\ref{eq51}) is absolutely convergent for all $t \neq 0$, real or complex.}
			\end{aligned}
		\end{equation}
		
		
		Continuing, we now multiply the first of these formulas by $\cos 2 k \theta$, the second by $\sin 2 k \theta$, and integrate the resulting expressions term-by-term over the interval $0 \leq \theta \leq \pi$ (an operation which is legitimate here) to obtain
		
		\begin{equation}\label{eq54}
			\begin{aligned}
				J_{2 k}(x)  =\frac{1}{\pi} \displaystyle\int_0^\pi \cos (x \sin \theta) \cos 2 k \theta d \theta,\quad
				0  =\frac{1}{\pi} \displaystyle\int_0^\pi \sin (x \sin \theta) \sin 2 k \theta d \theta
			\end{aligned}
		\end{equation}
		In exactly the same way we find that
		\begin{equation}\label{eq55}
			\begin{aligned}
				0  =\frac{1}{\pi} \displaystyle\int_0^\pi \cos (x \sin \theta) \cos (2 k+1) \theta d \theta, \quad
				J_{2 k+1}(x)  =\frac{1}{\pi} \displaystyle\int_0^\pi \sin (x \sin \theta) \sin (2 k+1) \theta d \theta .
			\end{aligned}
		\end{equation}
		Finally, by adding these results and using the identity
		$$
		\cos (n \theta-x \sin \theta)=\cos n \theta \cos (x \sin \theta)+\sin n \theta \sin (x \sin \theta)
		$$
		we deduce  If $n$ is a non-negative integer
		$$
		J_n(x)=\frac{1}{\pi} \displaystyle\int_0^\pi \cos (n \theta-x \sin \theta) d \theta .
		$$

		\subsection{Funciones de Bessel de segundo tipo}
		\begin{enumerate}
			\setcounter{enumi}{1}
			\item Caso 2. $a=0$.
		\end{enumerate}
		Tomando $a=0$ la ecuaci\'on de Bessel dada por \ref{ecuacionbessel} se transforma en
		\begin{equation}\label{eq42}
			x y^{\prime \prime}+y^{\prime}+x y=0
		\end{equation}
		De la expresi\'on de la ecuaci\'on indicial \ref{ecuacionindicial} tenemos
		$$F(r)=r(r-1)+r+r^{2}=0$$
		Cuyas soluciones son reales repetidas $r_{1}=r_{2}=0$. De \ref{frobenius3} la segunda soluci\'on linealmente independiente es de la forma
		\begin{equation}\label{eq43}
			K_0(x)=\displaystyle\sum_{k=1}^{\infty} b_k x^k+J_0(x) \ln x
		\end{equation}
		Con $J_0=y_{1}(x)$ dada en la expresi\'on \ref{besseluno}. Nuestro objetivo ahora es determinar el valor de $b_k$. Como \ref{eq43} satisface \ref{ecuacionbessel} tenemos que:
		$$
		x K_0(x) =\displaystyle\sum_{k=1}^{\infty} b_{k} x^{k+1}+x J_0(x) \ln x$$
		Acomodando la sumatoria
		$$x K_0(x) =\displaystyle\sum_{k=3}^{\infty} b_{k-2} x^{k-1}+x J_0(x) \ln x $$
		Derivando \ref{eq43}
		$$K_0^{\prime}(x)=\displaystyle\sum_{k=1}^{\infty} k b_k x^{k-1}+J_0^{\prime}(x) \ln x+\frac{J_0(x)}{x}$$
		Derivando nueva vez expresi\'on anterior y luego multiplicando por $x$
		$$x K_0^{\prime \prime}(x)=\displaystyle\sum_{k=1}^{\infty} k(k-1) b_k x^{k-1}+x J_0^{\prime \prime}(x) \ln x+2 J_0^{\prime}(x)-\frac{J_0(x)}{x}
		$$
		Reemplazando las expresiones anteriores en (\ref{eq42}) obtenemos la expresi\'on
		$$\displaystyle\sum_{k=1}^{\infty} k(k-1) b_k x^{k-1}+x J_0^{\prime \prime}(x) \ln x+2 J_0^{\prime}(x)-\frac{J_0(x)}{x}+ \displaystyle\sum_{k=1}^{\infty} k b_k x^{k-1}+J_0^{\prime}(x) \ln x+\frac{J_0(x)}{x}+\displaystyle\sum_{k=3}^{\infty} b_{k-2} x^{k-1}+x J_0(x) \ln x$$
		Desarrollando los dos primeros t\'erminos de las dos primeras sumatorias tenemos
		$$\textcolor[rgb]{0.00,0.00,0.55}{2b_{2}x}\displaystyle\sum_{k=3}^{\infty} k(k-1) b_k x^{k-1}+x J_0^{\prime \prime}(x) \ln x+2 J_0^{\prime}(x)-\frac{J_0(x)}{x}+\textcolor[rgb]{0.00,0.00,0.55}{b_{1}+2b_{2}x} \displaystyle\sum_{k=3}^{\infty} k b_k x^{k-1}+J_0^{\prime}(x) \ln x+\frac{J_0(x)}{x}+\displaystyle\sum_{k=3}^{\infty} b_{k-2} x^{k-1}+x J_0(x) \ln x$$
		Sumando los t\'erminos semejantes
		$$
		\begin{aligned}
			b_1 & +4 b_2 x+\displaystyle\sum_{k=3}^{\infty}\left[k^2 b_k+b_{k-2}\right] x^{k-1} \\
			& +\left[x J_0^{\prime \prime}(x)+J_0^{\prime}(x)+x J_0(x)\right] \ln x+2 J_0^{\prime}(x) \equiv 0
		\end{aligned}
		$$
		Como $J_{0}$ es soluci\'on de \ref{ecuacion polinomio chebyshev} se concluye que $x J_0^{\prime \prime}(x)+J_0^{\prime}(x)+x J_0(x) \equiv 0$, y as\'i tenemos que
		$$
		b_1+4 b_2 x+\displaystyle\sum_{k=3}^{\infty}\left[k^2 b_k+b_{k-2}\right] x^{k-1} \equiv-2 J_0^{\prime}(x) .
		$$
		Finally, by (\ref{eq39}),
		$$
		J_0^{\prime}(x)=\displaystyle\sum_{k=1}^{\infty}(-1)^k \frac{2 k}{2^{2 k}(k !)^2} x^{2 k-1},
		$$
		whence
		$$
		b_1+4 b_2 x+\sum_{k=3}^{\infty}\left[k^2 b_k+b_{k-2}\right] x^{k-1} \equiv \sum_{k=1}^{\infty}(-1)^{k+1} \frac{4 k}{2^{2 k}(k !)^2} x^{2 k-1} .
		$$
		To facilitate the evaluation of the $b_k$ we now multiply this expression by $x$ and split the series on the left into its even and odd parts to obtain
		$$
		\begin{aligned}
			b_1 x & +\displaystyle\sum_{k=1}^{\infty}\left[(2 k+1)^2 b_{2 k+1}+b_{2 k-1}\right] x^{2 k+1}+4 b_2 x^2 \\
			& +\displaystyle\sum_{k=2}^{\infty}\left[(2 k)^2 b_{2 k}+b_{2 k-2}\right] x^{2 k} \equiv x^2+\displaystyle\sum_{k=2}^{\infty}(-1)^{k+1} \frac{4 k}{2^{2 k}(k !)^2} x^{2 k} .
		\end{aligned}
		$$
		Thus $b_1=b_3=b_5=\cdots=0$, while
		$$
		4 b_2=1 \text {, and }(2 k)^2 b_{2 k}+b_{2 k-2}=(-1)^{k+1} \frac{4 k}{2^{2 k}(k !)^2}, k>1 .
		$$
		Hence
		$$
		\begin{aligned}
			& b_2=\frac{1}{2^2} \\
			& b_4=-\frac{1}{2^2 \cdot 4^2}\left(1+\frac{1}{2}\right)=-\frac{1}{2^4(2 !)^2}\left(1+\frac{1}{2}\right) \\
			& \vdots \\
		\end{aligned}
		$$
		\begin{equation}\label{eq44}
			b_{2k}=(-1)^{k+1} \frac{1}{2^{2 k}(k !)^2}\left(1+\frac{1}{2}+\cdots+\frac{1}{k}\right)
		\end{equation}
		and it follows that
		$$
		K_0(x)=\displaystyle\sum_{k=1}^{\infty} \frac{(-1)^{k+1}}{(k !)^2}\left(1+\frac{1}{2}+\cdots+\frac{1}{k}\right)\left(\frac{x}{2}\right)^{2 k}+J_0(x) \ln x .
		$$
		In theoretical work with Bessel functions it is common practice to replace $K_0$ by a certain linear combination of $J_0$ and $K_0$. The resulting function is known as the Bessel function of order zero of \textbf{the second kind,} and is defined by the formula
		\begin{equation}\label{eq45}
			Y_0(x)=-\frac{2}{\pi} \displaystyle\sum_{k=1}^{\infty} \frac{(-1)^k}{(k !)^2}\left(1+\frac{1}{2}+\cdots+\frac{1}{k}\right)\left(\frac{x}{2}\right)^{2 k}+\frac{2}{\pi} J_0(x)\left[\ln \frac{x}{2}+\gamma\right]
		\end{equation}
		where $\gamma=0.57721566 \ldots$, and is known as Euler's constant. The constant $\gamma$ is defined to be the sum of the series
		$$
		1+\displaystyle\sum_{n=2}^{\infty}\left(\frac{1}{n}+\ln \frac{n-1}{n}\right)
		$$
		The graph of $Y_0$ is shown in Fig.\\
		\begin{enumerate}
			\setcounter{enumi}{2}
			\item  Case 3. $\nu=n$, an integer.
		\end{enumerate}
		This time the roots of (\ref{eq34}) differ by $2 n>0$, the second solution of Bessel's equation is of the form
		$$
		K_n(x)=\sum_{k=0}^{\infty} b_k x^{k+n}+c J_n(x) \ln x,
		$$
		where $c$ is a constant. Here too the $b_k$ and $c$ can be evaluated by the method of undetermined coefficients, but the argument is now exceptionally long and involved. so
		$$
		\begin{aligned}
			K_n(x)= & -\frac{1}{2} \sum_{k=0}^{n-1} \frac{(n-k-1) !}{k !}\left(\frac{x}{2}\right)^{2 k-n}-\frac{H_n}{2 n !}\left(\frac{x}{2}\right)^n \\
			& -\frac{1}{2} \sum_{k=1}^{\infty} \frac{(-1)^k\left[H_k+H_{n+k}\right]}{k !(n+k) !}\left(\frac{x}{2}\right)^{2 k+n}+J_n(x) \ln x,
		\end{aligned}
		$$
		where $H_n=1+\frac{1}{2}+\cdots+1 / n$.
		It is customary to replace $K_n$ by a linear combination of $J_n$ and $K_n$, denoted $Y_n$, and called the \textbf{ Bessel function of order $n$ of the second kind.}
		$$
		\begin{aligned}
			Y_n(x)= & -\frac{1}{\pi} \sum_{k=0}^{n-1} \frac{(n-k-1) !}{k !}\left(\frac{x}{2}\right)^{2 k-n}-\frac{H_n}{\pi(n !)}\left(\frac{x}{2}\right)^n \\
			& -\frac{1}{\pi} \sum_{k=1}^{\infty} \frac{(-1)^k\left[H_k+H_{k+n}\right]}{k !(n+k) !}\left(\frac{x}{2}\right)^{2 k+n}+\frac{2}{\pi} J_n(x)\left[\ln \frac{x}{2}+\gamma\right] .
		\end{aligned}
		$$
		\subsection{Funciones de Bessel de tercer tipo}
trabajar esta parte
		\Proposition{Bessel}{\begin{eqnarray}\label{besselx-n}
				% \nonumber % Remove numbering (before each equation)
				\displaystyle\frac{d}{d x}\left(x^{-a} J_a(x)\right)&=&-x^{-a} J_{a+1}(x)
		\end{eqnarray}}
		
		\begin{demo}
			Aplicando la derivada de un producto en el lado izquerdo de la expresion (\ref{besselx-n})
			\begin{eqnarray*}
				\displaystyle\frac{d}{d x}\left(x^{-a} J_a(x)\right)&=&x^{-a} J_a^{\prime}(x)-a x^{-a-1} J_a(x)
			\end{eqnarray*}
			Tomando factor com\'un en el lado derecho
			\begin{eqnarray*}
				\displaystyle\frac{d}{d x}\left(x^{-a} J_a(x)\right)&=&x^{-a-1}\left[x J_a^{\prime}(x)-a J_a(x)\right]
			\end{eqnarray*}
			Utilizando la expresi\'on (\ref{besselrelacion} ), lo anterior podemos expresarlo como
			\begin{eqnarray*}
				\displaystyle\frac{d}{d x}\left(x^{-a} J_a(x)\right)&=&x^{-a-1}\left(-x J_{a+1}(x)\right)
			\end{eqnarray*}
			Simplificando se llega al resutado esperado
			\begin{eqnarray*}
				\displaystyle\displaystyle\frac{d}{d x}\left(x^{-a} J_a(x)\right)&=&-x^{-a} J_{a+1}(x)
			\end{eqnarray*}
		\end{demo}
		
		\Proposition{Bessel}{\begin{eqnarray}
				% \nonumber % Remove numbering (before each equation)
				J_{1 / 2}(x)&=&\sqrt{\frac{2}{\pi x}}\,\sin (x)
		\end{eqnarray}}\label{bessel/2}
		
		
		\begin{demo}
			Tomando $a=\frac{1}{2}$ en la expresi\'on (\ref{ja}), tenemos
			\begin{eqnarray*}
				J_{\frac{1}{2}}(x)&=&\displaystyle\sum_{m=0}^{\infty} \frac{(-1)^m}{m ! \Gamma\left(m+1+\frac{1}{2}\right)}\left(\frac{x}{2}\right)^{2 m+\frac{1}{2}}
			\end{eqnarray*}
			Simplificando la potencia " $x$ "
			$$
			\begin{aligned}
				J_{\frac{1}{2}}(x) & =\sum_{m=0}^{\infty} \frac{(-1)^m}{m ! \Gamma\left(m+\frac{3}{2}\right)} \frac{x^{2 m+1}}{2^{2 m+1}} \sqrt{\frac{2}{x}} \\
				& =\sqrt{\frac{2}{x}} \sum_{m=0}^{\infty} \frac{(-1)^m x^{2 m+1}}{m ! 2^{2 m+1} \Gamma\left(m+\frac{3}{2}\right)}
			\end{aligned}
			$$
			Utilizando que $m ! 2^{2 m+1} \Gamma\left(m+\frac{3}{2}\right)=(2 m+1) ! \sqrt{\pi}$(\textcolor{red}{demostrar como ejercicio basico})
			$$
			\begin{aligned}
				& J_{\frac{1}{2}}(x)=\sqrt{\frac{2}{\pi x}} \sum_{m=0}^{\infty} \frac{(-1)^m x^{2 m+1}}{(2 m+1) !} \\
				& J_{\frac{1}{2}}(x)=\sqrt{\frac{2}{\pi x}} \sin(x)
			\end{aligned}
			$$
		\end{demo}

		\section{Ecuaci\'on de Airy}
		\begin{eqnarray}\label{Airy equation}
			y^{\prime \prime}-xy&=&0
		\end{eqnarray}
		De \ref{Airy equation} $p(x)=0 \quad \text{y} \quad q(x)=-x$ son funciones anali\'ticas en $x_{0}=0$,por lo tanto tiene una soluci\'on en serie de potencia entorno de dicho punto.\\
		Suponiendo la soluci\'on en series de potencias y calculando respectivamente las derivadas
		\begin{eqnarray*}
			% \nonumber % Remove numbering (before each equation)
			y(x)&=&\sum_{n=0}^{\infty} c_n x^{n}\\
			y^{\prime}(x)&=&\sum_{n=0}^{\infty}n x^{n-1}\\
			y^{\prime \prime}(x)&=&\sum_{n=0}^{\infty}n(n-1) c_n x^{n-2}
		\end{eqnarray*}
		Sustituyendo dichas expresiones en \ref{Airy equation} tenemos la expresi\'on
		\begin{eqnarray*}
			% \nonumber % Remove numbering (before each equation)
			\displaystyle\sum_{n=2}^{\infty} n(n-1) a_{n}x^{n-2}-\displaystyle\sum_{n=0}^{\infty} a_{n} x^{n+1}&=&0
		\end{eqnarray*}
		lo cual es equivalente a
		\begin{eqnarray*}
			% \nonumber % Remove numbering (before each equation)
			\displaystyle\sum_{n=-1}^{\infty}(n+3)(n+2) a_{n+3} x^{n+1}-\displaystyle\sum_{n=0}^{\infty} a_n x^{n+1}&=&0\\
			(2)(1)a_{2}+\displaystyle\sum_{n=0}^{\infty}(n+3)(n+2)a_{n+3} x^{n+1}+\displaystyle\sum_{n=0}^{\infty}-a_{n} x^{n+1}&=&0\\
			(2)(1)a_{2}+\displaystyle\sum_{n=0}^{\infty}\displaystyle\left[(n+3)(n+2) a_{n+3}-a_{n}\right] x^{n+1}&=&0
		\end{eqnarray*}
		Igualando los coeficientes a cero, tenemos
		$$\left \{\begin{array}{c}
			2(1) a_{2}=0 \Rightarrow a_{2}=0\\
			(n+3)(n+2) a_{n+3}-a_{n}=0 \quad \forall n \geq 0 \\
		\end{array} \right.$$
		
		Despejando $a_{n+3}$ de la expresi\'on anterior y realizando un cambio de subindice tenemos $$a_{n}=\displaystyle\frac{1}{n(n-1)} a_{n-3}\quad n \geq 3$$
		Tomando
		$n=2k+2$
		\begin{eqnarray*}
			c_{3 k+2}&=&\frac{1}{(3 k+2)(3 k+1)} c_{3k-1} \quad k=1,2,3,\cdots
		\end{eqnarray*}
		Desarrollando los t\'erminos
		\begin{eqnarray*}
			% \nonumber % Remove numbering (before each equation)
			\text { Si } k=1 \rightarrow c_{5}&=&\frac{1}{(5)(4)}c_{2}\\
			c_{5}&=&\frac{(2)(3)}{(5)(4)(3)(2)}c_{2}\\
			\text { si } k=2 \rightarrow c_{8}&=&\frac{1}{(8)(7)} c_{5}\\
			c_{8}&=&\frac{(2)(3)(6)}{(8)(7)(6)(5)(4)(3)(2)} c_{2}\\
			\text { Si } k=3 \rightarrow c_{11} &=&\frac{1}{(11)(10)}c_{8} \\
			c_{11}&=&\frac{(2)(3)(6)(9)}{(11)(10)(9)(8)(7)(6)(5)(4)(2)} c_{2} \\
			\vdots\\
			c_{3k+2}&=&\frac{1\cdot2 \cdot 3 \cdot 6 \cdot 9 \cdot \ldots(3k)}{(3k+2) !}c_{2}=0
		\end{eqnarray*}
		Ahora si  $n=3k+1$
		\begin{eqnarray*}
			c_{3k+1}&=&\frac{1}{(3k+1)(3k)} c_{3k-2}
		\end{eqnarray*}
		Desarrollando..
		\begin{eqnarray*}
			\text { Si } k=1 \rightarrow c_{4}&=&\frac{1}{(4)(3)}c_{1}\\
			c_{4}&=&\frac{(2)}{(4)(3)(2)}c_{1}\\
			\text { si } k=2 \rightarrow c_{7}&=&\frac{1}{(7)(6)} c_{4}\\
			c_{7}&=&\frac{(2)(5)}{(7)(6)(5)(4)(3)(2)} c_{1}\\
			\text { si } k=3 \rightarrow c_{10}&=&\frac{1}{(10)(9)} c_{7}\\
			c_{10}&=&\frac{(2)(5)(8)}{(10)(9)(8)(7)(6)(5)(4)(3)(2)} c_{1}\\
			\vdots\\
			c_{3k+1}&=&\frac{(2)(5)(8)\cdots (3k-1)}{(3k+1)!} c_{1} \quad k=1,2,3\cdots
		\end{eqnarray*}
		Escogiendo ahora $n=3k$
		\begin{eqnarray*}
			c_{3k}&=&\frac{1}{(3k)(3k-1)} c_{3k-3}
		\end{eqnarray*}
		Desarrollando..
		\begin{eqnarray*}
			\text { Si } k=1 \rightarrow c_{3}&=&\frac{1}{(3)(2)}c_{0}\\
			\text { si } k=2 \rightarrow c_{6}&=&\frac{1}{(6)(5)} c_{3}\\
			c_{6}&=&\frac{(4)}{(6)(5)(4)(3)(2)} c_{0}\\
			\text { si } k=3 \rightarrow c_{9}&=&\frac{1}{(9)(8)} c_{6}\\
			c_{9}&=&\frac{(4)(7)}{(9)(8)(7)(6)(5)(4)(3)(2)} c_{0}\\
			\vdots\\
			c_{3k}&=&\frac{(4)(7)(10)\cdots (3k-2)}{(3k)!} c_{0} \quad k=1,2,3\cdots
		\end{eqnarray*}
		Por lo tanto
		\begin{eqnarray*}
			y(x)=c_{0}+c_{1} x+\sum_{n=1}^{\infty} c_{3n} x^{3n}+\sum_{n=1}^{\infty} c_{3n+1} x^{3n+1}+\sum_{n=1}^{\infty}c_{3n+2} x^{3n+2}
		\end{eqnarray*}
		Y las funciones de Airy est\'an dada por
		\begin{eqnarray*}
			y(x)&=&c_{0}\left[1+\sum_{n=1}^{\infty} \frac{1\cdot4 \cdots(3n-2)}{(3n)!} x^{3n}\right]+c_{1}\left[x+\displaystyle\sum_{n=1}^{\infty} \frac{2\cdot 5 \cdots(3n-1)}{(3n+1)!} x^{3n+1}\right]
		\end{eqnarray*}
		\section{Ecuaci\'on Hipergeom\'etrica}
		\textcolor[rgb]{1.00,0.00,0.00}{Aplicacion del pendulo}
		\begin{eqnarray}\label{ecuacion hipergeometrica}
			% \nonumber % Remove numbering (before each equation)
			x(1-x) y^{\prime \prime}+[c-(a+b+1) x] y^{\prime}-ab y&=&0
		\end{eqnarray}
		Los puntos singulares de \ref{ecuacion hipergeometrica} son $\{0,1\}$, Fij\'emonos en la forma normal:
		\begin{eqnarray*}
			y^{\prime \prime}+\frac{[c-(a+b+1) x]}{x(1-x)} y^{\prime}-\frac{a b}{x(1-x)} y&=&0
		\end{eqnarray*}
		Ahora partiendo de
		\begin{eqnarray*}
			y^{\prime \prime}+P(x) y^{\prime}+Q(x) y&=&0
		\end{eqnarray*}
		\begin{eqnarray}\label{PyQ}
			P(x)=\displaystyle\frac{[c-(a+b+1) x]}{x(1-x)}\quad \text { y }\quad Q(x)=\displaystyle\frac{a b}{x(1-x)}
		\end{eqnarray}
		\begin{eqnarray*}
			P(x)=\frac{\displaystyle\frac{[c-(a+b+1) x]}{(1-x)}}{x} \quad \text { y } \quad Q(x)=\displaystyle\frac{\displaystyle\frac{a b x}{(1-x)}}{x^{2}}
		\end{eqnarray*}
		De donde
		\begin{eqnarray*}
			P_{0}(x)=\frac{[c-(a+b+1) x]}{(1-x)} \quad y \quad Q_{0}(x)=\frac{a b x}{(1-x)}
		\end{eqnarray*}
		$P_{0}(x)$ y $Q_{0}(x)$ son analiticas en $x_{0}=0$ , entonces $x_{0}=0$ es un punto singular regular de \ref{ecuacion hipergeometrica}
		y admite una soluci\'on de la forma dada en \ref{solucionfrobenius}.
		Derivando
		\begin{eqnarray*}
			y^{\prime}&=&\sum_{m=0}^{\infty}c_m(m+r) x^{m+r-1} \\
			y^{\prime \prime}&=&\sum_{m=0}^{\infty}c_m(m+r)(m+r-1) x^{m+r-2}
		\end{eqnarray*}
		Reescribiendo la ecuaci\'on hipergeom\'etrica
		\begin{eqnarray}\label{ecuacion hipergeometrica1}
			x y^{\prime \prime}-x^2 y^{\prime \prime}+c y^{\prime}-(a+b+1) x y^{\prime}-a b y&=&0
		\end{eqnarray}
		As\'i
		\begin{eqnarray*}
			x y^{\prime \prime}&=&x \sum_{m=0}^{\infty} c_{m}(m+r)(m+r-1) x^{m+r-2}=\sum_{m=0}^{\infty}c_{m}(m+r)(m+r-1) x^{m+r-1} \\
			x^2 y^{\prime \prime}&=&x^{2} \sum_{m=0}^{\infty} c_m(m+r)(m+r-1) x^{m+r-2}=\sum_{m=0}^{\infty} c_m(m+r)(m+r-1) x^{m+r} \\
			c y^{\prime}&=&c \sum_{m=0}^{\infty} c_{m}(m+r) x^{m+r-1}=\sum_{m=0}^{\infty} c(m+r) c_{m} x^{m+r-1} \\
			(a+b+1) x y^{\prime}&=&(a+b+1) x \sum_{m=0}^{\infty} c_{m}(m+r) x^{m+r-1}=\sum_{m=0}^{\infty}(a+b+1)(m+r) c_{m} x^{m+r} \\
			a b y&=&a b \sum_{m=0}^{\infty} c_{m} x^{m+r}=\sum_{m=0}^{\infty} a b c_{m} x^{m+r}
		\end{eqnarray*}
		Reemplazando las expresiones anteriores en \ref{ecuacion hipergeometrica1}, tenemos
		\begin{eqnarray*}
			\sum_{m=0}^{\infty}c_{m}(m+r)(m+r-1) x^{m+r-1}-\sum_{m=0}^{\infty} c_{m}(m+r)(m+r-1) x^{m+r}+\sum_{m=0}^{\infty} c(m+r) c_{m} x^{m+r-1} \\
			-\sum_{m=0}^{\infty} c(m+r) c_{m} x^{m+r-1}-\sum_{m=0}^{\infty} a b c_{m} x^{m+r}&=&0
		\end{eqnarray*}
		Agrupando
		\begin{eqnarray*}
			\sum_{m=0}^{\infty}[(m+r)(m+r-1)+(m+r) c] c_{m} x^{m+r-1}
			-\sum_{m=0}^{\infty}c_{m}[(m+r)(m+r-1)+(a+b+1)(m+r)+a b] x^{m+r}&=&0
		\end{eqnarray*}
		Factorizando:
		$$
		\begin{aligned}
			&(m+r)(m+r-1)+(m+r) c=(m+r)(m+r-1+c) \\
			&\begin{aligned}
				(m+r)(m+r&-1)+(a+b+1)(m+r)+a b=(m+r)(m+r+a+b)+a b \\
				&=(m+r)^2+a(m+r)+b(m+r)+a b=(m+r)(m+r+a)+b(m+r+a) \\
				&=(m+r+a)(m+r+b)
			\end{aligned}
		\end{aligned}
		$$
		As\'i la ecuaci\'on toma la forma
		$$\begin{aligned}
			&\sum_{m=0}^{\infty}c_{m}(m+r)(m+r-1+c) x^{m+r-1}-\sum_{m=0}^{\infty}c_{m}(m+r+a)(m+r+b) x^{m+r}=0 \\
			&x^{r-1}\left[\sum_{m=0}^{\infty} c_{m}(m+r)(m+r-1+c) x^{m}-\sum_{m=0}^{\infty}c_{m}(m+r+a)(m+r+b) x^{m+1}\right]=0\\
			&\sum_{m=0}^{\infty}c_{m}(m+r)(m+r-1+c) x^{m}-\sum_{m=0}^{\infty} c_{m-1}(m+r+a)(m+r+b) x^{m+1}=0
		\end{aligned}$$
		Sea $m+1=k, m=k-1$,
		cuando $m=0$ entonces $k=1$. Sutituyendo en la serie y cambiando $k$ por $m$ :
		$$\begin{gathered}
			\sum_{m=0}^{\infty}c_{m}(m+r)(m+r-1+c) x^{m}-\sum_{m=1}^{\infty}c_{m-1}(m-1+r+a)(m-1+r+b) x^{m}=0 \\
			r(r-1+c)c_0+\sum_{m=1}^{\infty}c_{m}(m+r)(m+r-1+c) x^{m}-\sum_{m=1}^{\infty}c_{m-1}(m-1+r+a)(m-1+r+b) x^{m}
			=0
		\end{gathered}$$
		De donde
		$$\begin{gathered}
			r(r-1+c) c_{0}=0 \quad c_{0} \neq 0 \\
			r(r-1+c)=0 \\
			r=0\quad \text{y}\quad(r-1+c)=0 \\
			r=0\quad \text{y}\quad r=1-c
		\end{gathered}$$
		
		\begin{eqnarray*}
			\displaystyle\sum_{m=1}^{\infty}\left[c_m(m+r)(m+r-1+c)-c_{m-1}(m-1+r+a)(m-1+r+b)\right] x^{m}&=&0
		\end{eqnarray*}
		Lo que implica que
		$$\begin{gathered}
			c_m(m+r)(m+r-1+c)-c_{m-1}(m-1+r+a)(m-1+r+b)=0 \\
			c_m(m+r)(m+r-1+c)=c_{m-1}(m-1+r+a)(m-1+r+b)
		\end{gathered}$$
		De la expresi\'on anterior se obtiene la relaci\'on de recurrencia
		\begin{eqnarray*}
			c_{m}=\frac{(m-1+r+a)(m-1+r+b)}{(m+r)(m+r-1+c)} c_{m-1} \quad \forall m \in N / m \geq 1
		\end{eqnarray*}
		Tomando $m\rightarrow m+1$ se tiene que
		\begin{eqnarray}\label{hipergeometricarecurrenciarelacion}
			c_{m+1}&=&\frac{(m+r+a)(m+r+b)}{(m+1+r)(m+r+c)}c_{m} \quad \forall m \in N / m \geq 0
		\end{eqnarray}
		De \ref{hipergeometricarecurrenciarelacion} escogiendo $r=0$ obtenemos la relaci\'on
		\begin{eqnarray}\label{hipergeometricar0}
			c_{m+1}&=&\frac{(m+a)(m+b)}{(m+1)(m+c)}c_{m} \quad \forall m \in N / m \geq 0
		\end{eqnarray}
		Desarrollando los primeros t\'erminos de \ref{hipergeometricar0}
		\begin{eqnarray*}
			\text { Si }\quad  m=0\quad \rightarrow c_{1}&=&\frac{(a)(b)}{(1)(c)}c_{0}\\
			\text { Si }\quad  m=1\quad \rightarrow c_{2}&=&\frac{(1+a)(1+b)}{(2)(1+c)} c_{1}=\frac{a(1+a) b(1+b)}{(1)(2) c(1+c)} c_{0} \\
			\text { Si }\quad  m=2\quad \rightarrow c_{3}&=&\frac{(2+a)(2+b)}{(3)(2+c)} c_{2}=\frac{a(1+a)(2+a) b(1+b)(2+b)}{(1)(2)(3) c(1+c)(2+c)}c_{0} \\
			\text { Si }\quad  m=3\quad \rightarrow c_{4}&=&\frac{(3+a)(3+b)}{(4)(3+c)} c_{2}=\frac{a(1+a)(2+a)(3+a) b(1+b)(2+b)(3+b)}{(1)(2)(3)(4) c(1+c)(2+c)(3+c)}c_{0}\\
			\vdots\\
			\text{De manera que para} \quad m=n \quad \rightarrow c_{n+1}&=&\frac{a(1+a)(2+a)(3+a) \ldots(n+a) b(1+b)(2+b)(3+b) \ldots(n+b)}{(n+1) ! c(1+c)(2+c)(3+c) \ldots(n+c)}c_{0}
		\end{eqnarray*}
		Reemplazando los t\'erminos obtenidos en \ref{solucionfrobenius} llegamos a
		\begin{eqnarray*}
			y&=&c_{0} x^{0}\left(1+\frac{(a)(b)}{(1)(c)} x+\frac{a(1+a) b(1+b)}{(1)(2) c(1+c)} x^{2}+\frac{a(1+a)(2+a) b(1+b)(2+b)}{(1)(2)(3) c(1+c)(2+c)} x^{3}+\cdots\right)
		\end{eqnarray*}
		Tomando $c_{0}=1$ se obtiene una soluci\'on de \ref{ecuacion hipergeometrica}
		\begin{eqnarray*}
			y&=&1+\frac{(a)(b)}{(1)(c)} x+\frac{a(1+a) b(1+b)}{(1)(2) c(1+c)} x^{2}+\frac{a(1+a)(2+a) b(1+b)(2+b)}{(1)(2)(3) c(1+c)(2+c)} x^{3}+\cdots
		\end{eqnarray*}
		Por \ref{P1} la expresi\'on anterior se convierte en la funci\'on hipergeom\'etrica definida en \ref{funcion hipergeometrica}
		\begin{eqnarray}\label{hipersol1}
			y=F(a, b, c, x)&=&\sum_{m=0}^{\infty} \frac{(a)_{m} b_{m}}{m ! c_{m}} x^{m}
		\end{eqnarray}
		Utilizando \ref{Relaci\'on de recurrencia generalizada de la funci\'on gamma} obtenemos la soluci\'on en t\'erminos de la funci\'on Gamma
		\begin{eqnarray*}
			y=F(a, b, c, x)&=&\frac{\Gamma(c)}{\Gamma(a) \Gamma(b)} \sum_{m=0}^{\infty} \frac{\Gamma(m+a) \Gamma(m+b)}{(m) ! \Gamma(m+c)}x^{m}
		\end{eqnarray*}
		Para $r=1-$ c la relacion de recurrencia es la siguiente
		\begin{eqnarray*}
			c_{m+1}&=&\frac{(m+r+a)(m+r+b)}{(m+1+r)(m+r+c)}c_{m} \quad \forall m \in N / m \geq 0 \\
			c_{m+1}&=&\frac{(m+1-c+a)(m+1-c+b)}{(m+1)(m+2-c)} c_{m} \quad \forall m \in N / m \geq 0
		\end{eqnarray*}
		Escogiendo los siguientes par\'metros
		$$
		a^{\prime}=1-c+a, b^{\prime}=1-c+b, c^{\prime}=2-c
		$$
		\begin{eqnarray*}\label{hipergeometricar-c}
			c_{m+1}&=&\frac{\left(m+a^{\prime}\right)\left(m+b^{\prime}\right)}{(m)\left(m+c^{\prime}\right)} c_m \quad \forall m \in N / m \geq 0
		\end{eqnarray*}
		Obteniendo el desarrollo de los coeficientes\\
		
		
		Obtenemos la segunda soluci\'on linealmente independiente
		\begin{eqnarray}\label{hipersol2}
			y&=&x^{1-c} F\left(a^{\prime}, b^{\prime}, c^{\prime}, x\right)\nonumber \\
			y&=&x^{1-c} F(1-c+a, 1-c+b, 2-c, x)
		\end{eqnarray}
		De manera que la soluci\'on general es la combinaci\'on lineal de las dos soluciones independientes.
		\begin{eqnarray}\label{hipersolgeneral}
			y(x)=A_{0}\;F(a, b, c, x)+B_{0}\;x^{1-c} F(1-c+a, 1-c+b, 2-c, x)\quad |x|\leq 1
		\end{eqnarray}
		Ahora obtendremos la soluci\'on de \ref{ecuacion hipergeometrica} entorno al punto singular regular $x_{0}=1$. De \ref{PyQ} obtenemos
		\begin{eqnarray*}
			P(x)=\frac{\frac{[c-(a+b+1) x]}{x}}{(1-x)} \quad y \quad Q(x)=\frac{\frac{a b(1-x)}{x}}{(1-x)^2}
		\end{eqnarray*}
		De donde
		\begin{eqnarray*}
			P_{\circ}(x)=\frac{[c-(a+b+1) x]}{x} \quad y \quad Q_{\circ}(x)=\frac{a b(1-x)}{x}
		\end{eqnarray*}
		$P_{\circ}(x)$ y $Q_{\circ}(x)$ son analiticas en $x_{0}=1$ , entonces $x_{0}=1$ es un punto singular regular de \ref{ecuacion hipergeometrica}
		y admite una soluci\'on de la forma dada \ref{solucionfrobenius}.\\
		La soluci\'on la podemos obtener directamente deduciendo desde las soluciones anteriores por el cambio de variables $t=1-x$. Entonces con esta sustituci\'on la ecuaci\'on dada \ref{ecuacion hipergeometrica} se reduce a:
		$$
		\operatorname{como} y(x)=x(t) \wedge x=1-t, \quad \frac{d y}{d t}=\frac{d y}{d x} \frac{d x}{d t}, \quad \frac{d y}{d t}=(-1) \frac{d y}{d x}, \quad \frac{d^2 y}{d t^2}=\frac{d^2 y}{d x^2}
		$$
		Sustituyendo en \ref{ecuacion hipergeometrica}
		$$
		\begin{gathered}
			(1-t)(1-(1-t)) y^{\prime \prime}+[c-(a+b+1)(1-t)](-1) y^{\prime}-a b y=0 \\
			t(1-t) y^{\prime \prime}+[c-a-b-1+(a+b+1) t](-1) y^{\prime}-a b y=0 \\
			t(1-t) y^{\prime \prime}+[(-c+a+b+1)-(a+b+1) t] y^{\prime}-a b y=0
		\end{gathered}
		$$
		Si $c_1=a+b-c+1$ obtenemos $t(1-t) y^{\prime \prime}+\left[c_1-(a+b+1) t\right] y^{\prime}-a b y=0$\\
		Entonces la soluci\'on general es
		\begin{equation}\label{hiper1}
			y(x)=A_{0}F(a, b, a+b-c+1,1-x)+A_{1}x^{c-a-b} F(c-b, c-a, c-a-b+1,1-x)
		\end{equation}
		
		\section{Funciones El\'ipticas. SEzgo}
		\section{Polinomios de Euler}
		\section{Polinomios de Bernoulli}
		
		\addcontentsline{toc}{section}{\textcolor{uasdblue}{Ejercicio Unidad 2}}
		\begin{Exercise}\label{EX21}
			\vspace{-\baselineskip}% <-- You don't need this line of code if there's some text here
			\Question Eight systems of differential equations and five direction fields are given below.  Determine the system that corresponds to each direction field and sketch the solution curves that correspond to the initial conditions $(x_0, y_0) = (0,1)$ and $(x_0, y_0) = (1,-1)$.
			\begin{tasks}(3)
				\task $\begin{aligned}
					\frac{dx}{dt} & = -x \\     
					\frac{dy}{dt} & = y-1
				\end{aligned}$
				\task $\begin{aligned}
					\frac{dx}{dt} & = x^2 - 1 \\        
					\frac{dy}{dt} & = y
				\end{aligned}$
				\task $\begin{aligned}
					\frac{dx}{dt} & = x+2y \\
					\frac{dy}{dt} & = -y
				\end{aligned}$
				\task $\begin{aligned}
					\frac{dx}{dt} & = 2x \\
					\frac{dy}{dt} & =  y
				\end{aligned}$
				\task $\begin{aligned}
					\frac{dx}{dt} & = x \\
					\frac{dy}{dt}  & = 2y
				\end{aligned}$ 
				\task$\begin{aligned}
					\frac{dx}{dt} & = x-1 \\
					\frac{dy}{dt} & = -y
				\end{aligned}$
				\task$\begin{aligned}
					\frac{dx}{dt} & = x^2-1 \\
					\frac{dy}{dt} & = -y
				\end{aligned}$        
				\task $\begin{aligned}
					\frac{dx}{dt} & = x- 2y \\
					\frac{dy}{dt} & =  -y
				\end{aligned}$
			\end{tasks}
		\end{Exercise}
		\setboolean{firstanswerofthechapter}{true}
		\begin{multicols}{2}
			\begin{Answer}[ref={EX21}]
				\Question 
				\begin{tasks}
					\task This is a solution of Ex 1
					\task This is a solution of Ex 2 
					\task This is a solution of Ex 3 
					\task This is a solution of Ex 4 
					\task This is a solution of Ex 5 
					\task This is a solution of Ex 6 
					\task This is a solution of Ex 7 
					\task This is a solution of Ex 8 
				\end{tasks}
			\end{Answer}
		\end{multicols}
		\setboolean{firstanswerofthechapter}{false}
		
		\begin{Exercise}\label{EX22}
			
			\Question Eight systems of differential equations and five direction fields are given below.  Determine the system that corresponds to each direction field and sketch the solution curves that correspond to the initial conditions $(x_0, y_0) = (0,1)$ and $(x_0, y_0) = (1,-1)$.
			\begin{tasks}(2)
				\task $\left\{\begin{aligned}
					\frac{dx}{dt} & = -x \\     
					\frac{dy}{dt} & = y-1
				\end{aligned}\right.$
				\task $\left\{\begin{aligned}
					\frac{dx}{dt} & = x^2 - 1 \\        
					\frac{dy}{dt} & = y
				\end{aligned}\right.$
				\task $\left\{\begin{aligned}
					\frac{dx}{dt} & = x+2y \\
					\frac{dy}{dt} & = -y
				\end{aligned}\right.$
				\task $\left\{\begin{aligned}
					\frac{dx}{dt} & = 2x \\
					\frac{dy}{dt} & =  y
				\end{aligned}\right.$
				\task $\left\{\begin{aligned}
					\frac{dx}{dt} & = x \\
					\frac{dy}{dt}  & = 2y
				\end{aligned}\right.$ 
				\task$\left\{\begin{aligned}
					\frac{dx}{dt} & = x-1 \\
					\frac{dy}{dt} & = -y
				\end{aligned}\right.$
				\task $\left\{\begin{aligned}
					\frac{dx}{dt} & = x^2-1 \\
					\frac{dy}{dt} & = -y
				\end{aligned}\right.$        
				\task $\left\{\begin{aligned}
					\frac{dx}{dt} & = x- 2y \\
					\frac{dy}{dt} & =  -y
				\end{aligned}\right.$
			\end{tasks}
		\end{Exercise}
		
		\begin{multicols}{2}
			\begin{Answer}[ref={EX22}]
				\Question 
				\begin{tasks}
					\task This is a solution of Ex 1
					\task This is a solution of Ex 2 
					\task This is a solution of Ex 3 
					\task This is a solution of Ex 4 
					\task This is a solution of Ex 5 
					\task This is a solution of Ex 6 
					\task This is a solution of Ex 7 
					\task This is a solution of Ex 8 
				\end{tasks}
			\end{Answer}
		\end{multicols}
		\setboolean{firstanswerofthechapter}{false}
		
	
	
\mychapter{Funciones especiales a partir de la Funci\'on Hipergeom\'etrica }{\begin{wrapfigure}{l}{0.45\textwidth}
		\centering
		\includegraphics[width=0.45\textwidth]{imagen/img9.png}
	\end{wrapfigure} Las funciones especiales constituyen un pilar fundamental en el estudio de las matemáticas aplicadas y la física matemática. Aunque su denominación pueda sugerir un carácter excepcional, en realidad se trata de funciones que aparecen de manera recurrente al resolver ecuaciones diferenciales de gran relevancia en la ciencia y la ingeniería.
	
	\vspace{0.5cm}
	
	En este capítulo abordaremos funciones como la hipergeométrica, la confluente, las funciones de Laguerre y las funciones de Hermite. Todas ellas surgen de problemas concretos: desde la descripción de sistemas cuánticos y modelos de osciladores, hasta la propagación de ondas y fenómenos de difusión.
	
	\vspace{0.5cm}
	
	El estudio de estas funciones no solo amplía el repertorio de herramientas analíticas disponibles, sino que también abre la puerta a una comprensión más profunda de los modelos que gobiernan la naturaleza. Cada subsección desarrollará tanto la teoría matemática como las aplicaciones más relevantes, complementadas con representaciones gráficas que ilustran su comportamiento y propiedades.}
 \addtocontents{toc}{\protect\figuretoc{imagen/img9.png}}

En este capítulo se introducen las funciones hipergeométricas generales \({}_r F_s\), junto con el estudio de sus propiedades fundamentales, tales como la ecuación diferencial que satisfacen, sus relaciones de recurrencia, representaciones integrales, fórmulas de suma, entre otras. Asimismo, se presentan diversos casos particulares, los cuales abarcan la mayoría de las funciones elementales clásicas, así como nuevas funciones de interés, cuya exploración detallada se desarrollará en los capítulos posteriores.

\medskip

\noindent
Además, se mostrará cómo algunas funciones especiales, tales como los polinomios de Legendre, Chebyshev y otras familias ortogonales, pueden obtenerse como casos particulares de la ecuación hipergeométrica, lo que permite un tratamiento unificado y sistemático de dichas funciones dentro del marco general de la teoría hipergeométrica.

\section{Funci\'on Hipergeom\'etrica}
\Definition{Serie Hipergeométrica }{
Sean \( p \) y \( q \) dos enteros positivos, y sean \( a_1, a_2, \ldots, a_p, c_1, c_2, \ldots, c_q \) números reales tales que \( c_j \notin \mathbb{Z}_{\leq 0} \) para todo \( j = 1, 2, \ldots, q \). La serie hipergeométrica con parámetros \( a_1, \ldots, a_p, c_1, \ldots, c_q \) se define para todo \( x \in \mathbb{R} \) mediante

\begin{equation}\label{hiperpq}
{}_p F_q\left(\left. \begin{array}{l}
a_1, \ldots, a_p \\
c_1, \ldots, c_q
\end{array} \right\rvert\, x \right)
= \displaystyle\sum_{k=0}^{+\infty} \dfrac{(a_1)_k \cdots (a_p)_k}{(c_1)_k \cdots (c_q)_k\, k!} \, x^k,
\end{equation}

donde \(\displaystyle (a)_n\) denota el símbolo de Pochhammer, definido en (\ref{P1}).}\label{def:fhper} En esta definición general, se debe suponer que \( \displaystyle c_j \notin \mathbb{Z}_{\leq 0} \) para todo \( j = 1, 2, \ldots, q \). En efecto, si \( \displaystyle c_j \in \mathbb{Z}_{\leq 0} \), entonces para \( \displaystyle n \geq -c_j + 1 \) se tiene

\[
\displaystyle
(c_j)_n = c_j(c_j+1) \cdots \left(c_j + (-c_j + 1) - 1\right) \cdots (c_j + n - 1) = 0
\]

lo que hace que el denominador de los términos de la serie se anule a partir de cierto orden, provocando una indeterminación. Sin embargo, un caso especial importante es el de los polinomios hipergeométricos: la serie hipergeométrica se convierte en en un polinomio si alguno de los parámetros \( a_i \) es un entero negativo o cero. En efecto, si \( a_i \in \mathbb{Z}_{\leq 0} \), entonces la expresión (\ref{hiperpq}) se reduce a

\begin{equation}\label{polihiper}
{}_p F_q\left(\left. \begin{array}{c}
a_1, \ldots, a_p \\
c_1, \ldots, c_q
\end{array} \right\rvert\, x \right)
= \displaystyle\sum_{k=0}^{-a_i} \dfrac{(a_1)_k \cdots (a_p)_k}{(c_1)_k \cdots (c_q)_k\, k!} \, x^k
\end{equation}
ya que \((a_i)_n = 0\) para \( n \geq -a_i + 1 \). En este caso, los parámetros \( c_j \) pueden ser enteros negativos o cero, siempre que se cumpla \( c_j \leq a_i \).\\

Ahora vamos analizar algunos casos particulares permiten extender la definición general de la serie hipergeométrica (\ref{hiperpq}) a situaciones en las que el número de parámetros en el numerador (\(p\)) o en el denominador (\(q\)) es cero. Cuando \(p = 0\), la serie hipergeométrica no contiene factores del tipo \((a_i)_k\) en el numerador, y se reduce a una serie de potencias con coeficientes dependientes únicamente de los parámetros del denominador. Por otro lado, cuando \(q = 0\), no hay restricciones en el denominador, y la serie se convierte en una serie generalizada de tipo binomial, cuya convergencia depende del valor de \(x\).

\begin{align}
{}_0 F_q\left(\begin{array}{c|c}
\cdot & x \\
c_1, \ldots, c_q
\end{array}\right) 
&= \displaystyle\sum_{k=0}^{+\infty} \dfrac{x^k}{(c_1)_k \cdots (c_q)_k\, k!} \label{oq} \\[1em]
%
{}_p F_0\left(\left. \begin{array}{c}
a_1, \ldots, a_p \\
\cdot
\end{array} \right|\, x \right) 
&= \displaystyle\sum_{k=0}^{+\infty} \dfrac{(a_1)_k \cdots (a_p)_k}{k!} \, x^k \label{po} \\[1em]
%
{}_0 F_0(\cdot \mid x) 
&= \displaystyle\sum_{k=0}^{+\infty} \dfrac{x^k}{k!} = e^x \label{ex}
\end{align}
Antes de continuar con el desarrollo formal y el estudio detallado de la serie hipergeométrica general \({}_pF_q\), resulta esencial determinar el intervalo de convergencia de dicha serie, ya que este delimita el dominio sobre el cual las expresiones obtenidas representan funciones bien definidas. El análisis de la convergencia no solo establece la validez de la representación en serie de potencias, sino que también permite comprender las propiedades analíticas de las funciones hipergeométricas, tales como su carácter de funciones enteras, multivaluadas o con singularidades aisladas.
\begin{enumerate}
  \item Si $p > q+1$

\[
{}_p F_q\left(\left.\begin{array}{l}
a_1, \ldots, a_q, a_{q+1}, \ldots, a_{q+i} \\
c_1, \ldots, c_q
\end{array} \right\rvert\, x\right)
= \displaystyle\sum_{k=0}^{\infty} 
\dfrac{(a_1)_k \cdots (a_q)_k (a_{q+1})_k \cdots (a_{q+i})_k}
{(c_1)_k \cdots (c_q)_k \, k!} \, x^k
\]

Sean

\[
\begin{aligned}
B_k &= \displaystyle\dfrac{(a_1)_k \cdots (a_q)_k (a_{q+1})_k \cdots (a_{q+i})_k}
{(c_1)_k \cdots (c_q)_k} \, x^k \\
B_{k+1} &= \displaystyle\dfrac{(a_1)_{k+1} \cdots (a_q)_{k+1} (a_{q+1})_{k+1} \cdots (a_{q+i})_{k+1}}
{(c_1)_{k+1} \cdots (c_q)_{k+1} \, (k+1)!} \, x^{k+1}
\end{aligned}
\]

Por el criterio de la razón y propiedades de Pochhammer

\[
\begin{aligned}
\lim_{k \rightarrow \infty} \left| \dfrac{B_{k+1}}{B_k} \right| 
&= \lim_{k \rightarrow \infty} 
\left| \dfrac{(a_1 + k) \cdots (a_q + k)(a_{q+1} + k) \cdots (a_{q+i} + k)}
{(k+1)(c_1 + k) \cdots (c_q + k)} \right| \\
&= \infty
\end{aligned}
\]

El límite anterior no existe $\forall x \neq 0$.

  \item Para el caso de $p = q + 1$, tenemos

\[
\begin{aligned}
\displaystyle\lim_{k \rightarrow \infty} \left| \dfrac{B_{k+1}}{B_k} \right| 
&=\displaystyle \lim_{k \rightarrow \infty} 
\left| \dfrac{(a_1 + k) \cdots (a_q + k)(a_{q+1} + k)}
{(k+1)(c_1 + k) \cdots (c_q + k)} \right| \\
&= \displaystyle |x|
\end{aligned}
\]

El límite anterior converge si $|x| < 1$.


  \item Si $p \leq q$

\[
\begin{aligned}
\lim_{k \rightarrow \infty} \left| \dfrac{B_{k+1}}{B_k} \right| 
&= \lim_{k \rightarrow \infty} 
\left| \dfrac{(a_1 + k)(a_2 + k) \cdots (a_{q - i} + k)}
{(k+1)(c_1 + k) \cdots (c_q + k)} \, x \right| \\
&= 0 \cdot |x|
\end{aligned}
\]

El límite anterior converge para todo $x \in \mathbb{R}$.

\end{enumerate}
Ya establecidos los intervalos en los que la función hipergeométrica converge, procederemos ahora a definir la función hipergeométrica como un caso particular de la definición (\ref{def:fhper}), lo cual nos permitirá situarla dentro de un marco teórico más amplio y comprender mejor sus propiedades y aplicaciones.
como caso particular de la funci\'on Hipergeom\'etrica.
\Definition{Función Hipergeométrica}{
	
La \textbf{función hipergeométrica generalizada} se denota por ${}_pF_q$ y se define como la serie de potencias

\begin{equation}\label{hipergene}
{}_pF_q\left(\left.\begin{array}{l}
a_1, \ldots, a_p \\
c_1, \ldots, c_q
\end{array} \right\rvert\, x\right)
= \displaystyle\sum_{k=0}^{+\infty} 
\dfrac{(a_1)_k \cdots (a_p)_k}
{(c_1)_k \cdots (c_q)_k \, k!} \, x^k
\end{equation}
}

La función hipergeométrica generalizada posee una notable riqueza estructural que permite representar, de manera unificada, una amplia variedad de funciones conocidas. A través de los siguientes ejemplos, se pondrá de manifiesto cómo diversas expresiones funcionales pueden reescribirse en términos de esta función, revelando así conexiones formales que articulan muchas construcciones del análisis clásico.
\Example{ Serie Geom\'etrica }{ Expresa la serie geom\'etrica
\begin{equation}\label{serieg}
\dfrac{1}{1-x} =\displaystyle\sum_{k=0}^{\infty} x^k  \quad|x|<1
  \end{equation} 
como un caso de la serie hipergeom\'etrica.}
\begin{sol}
Es claro que si en la expresi\'on (\ref{hiperpq}) tomamos $p=1$ y $q=0$ obtenemos
\[
\begin{aligned}
{}_1F_0\left(\left.\begin{array}{c}
1 \\
\cdot
\end{array} \right\rvert\, x\right) 
&= \displaystyle\sum_{k=0}^{+\infty} \dfrac{(1)_k}{k!} x^k \\
&= \displaystyle\sum_{k=0}^{+\infty} x^k \\
&= \dfrac{1}{1 - x}
\end{aligned}
\]
\end{sol}
\Example{ Serie de $\dfrac{\ln(1 - x)}{x}$}{
Expresa la función
\begin{equation*}
  f(x) = \dfrac{\ln(1 - x)}{x}
\end{equation*}
como una serie hipergeométrica en el intervalo \( (-1,0) \cup (0,1) \).
}
\begin{sol}
Integrando la serie geom\'etrica tenemos
\begin{align*}
\displaystyle\sum_{k=0}^{+\infty} \dfrac{x^{k+1}}{k+1} 
&= -\ln(1 - x)\\[1ex]
%
-\ln(1 - x) 
&= \displaystyle x \sum_{k=0}^{+\infty} \dfrac{x^k}{k+1} 
= x \displaystyle\sum_{k=0}^{+\infty} \dfrac{k!}{(k+1)!} x^k \\[1ex]
%
&= \displaystyle x \sum_{k=0}^{+\infty} \dfrac{k! \cdot k!}{(k+1)! \cdot k!} x^k \\[1ex]
&= x\displaystyle \sum_{k=0}^{+\infty} \dfrac{(1)_k \cdot (1)_k}{(2)_k \cdot k!} x^k 
\tag*{para todo \( x \in (-1,1) \)} \\[1ex]
%
\displaystyle{}_2F_1\left(\left.\begin{array}{c}
1,\ 1 \\
2
\end{array} \right\rvert\, x\right) 
&=- \dfrac{\ln(1 - x)}{x}
\tag*{para todo \( x \in (-1,0) \cup (0,1) \)}
\end{align*}
\end{sol}
Presentamos ahora un criterio sencillo que permite determinar si una serie de potencias dada puede expresarse como una serie hipergeométrica.
\Theorem{Caracterización Hipergeométrica de Series de Potencias}{
Sea \( y = \displaystyle\sum_{k=0}^{+\infty} \lambda_k x^k \) una serie de potencias con radio de convergencia \( R > 0 \). Supongamos que existen números reales \( a_1, a_2, \ldots, a_r \), \( c_1, c_2, \ldots, c_s \), y un número real no nulo \( A \) tales que, para todo entero \( k \geq 0 \), se cumple:

\begin{equation}\label{chiperfun}
\lambda_{k+1} = A \cdot \displaystyle\dfrac{(k + a_1)(k + a_2) \cdots (k + a_r)}{(k+1)(k + c_1)(k + c_2) \cdots (k + c_s)} \cdot \lambda_k.
\end{equation}

Entonces,

\begin{equation}\label{hiperfun}
y = \lambda_0 \cdot \displaystyle{}_rF_s\left(\left.\begin{array}{c}
a_1, \ldots, a_r \\
c_1, \ldots, c_s
\end{array} \right\rvert\, A x\right), \quad \left(|x| < \dfrac{R}{|A|}\right).
\end{equation}}\label{thm:fhper}
\begin{demo}
dsd
\end{demo} 
\Example{ Serie hipergeom\'etrica de $\sin(x), \cos(x)$ }{ Obtenga la representacion en la serie hipergeom\'etrica de las funciones trigonom\'etricas seno y coseno}
\begin{sol}
\begin{enumerate}
  \item \textbf{Seno}\\
  Sabemos que \[\sin(x)=\displaystyle\sum_{k=0}^{+\infty} \dfrac{(-1)^k x^{2 k+1}}{(2 k+1)!}=x \displaystyle\sum_{k=0}^{+\infty} \dfrac{(-1)^k\left(x^2\right)^k}{(2 k+1)!} \]
  Por la expresi\'on (\ref{chiperfun}) del teorema (\ref{thm:fhper}) tenemos $\forall k\geq 1$
 \[ \dfrac{\lambda_{k+1}}{\lambda_k}=-\dfrac{(2 k+1)!}{(2 k+3)!}=-\dfrac{1}{2(k+1)(2 k+3)}=-\dfrac{1}{4} \dfrac{1}{(k+1)\left(k+\dfrac{3}{2}\right)} \]
 con 
 \[r=0, s=1, c_1=\dfrac{3}{2}, \lambda_0=1, \text { y } A=-\dfrac{1}{4} \]
 Por lo tanto
 \begin{equation}\label{senohiper}
   \sin(x)=x \cdot{ }_0 F_1\left(\left.\begin{array}{c}
\cdot \\
\dfrac{3}{2}
\end{array} \right\rvert\,-\dfrac{x^2}{4}\right) \quad(x \in \mathbb{R}) .
 \end{equation}
  \item \textbf{Coseno}\\
    Sabemos que \[\cos(x)=\displaystyle\sum_{k=0}^{+\infty} \dfrac{(-1)^k x^{2 k}}{(2 k)!}\]
     Por la expresi\'on (\ref{chiperfun}) del teorema (\ref{thm:fhper}) tenemos $\forall k\geq 1$
 \[ \dfrac{\lambda_{k+1}}{\lambda_k}=-\dfrac{(2 k)!}{(2 k+2)!}=-\dfrac{1}{4(k+1)(k+1/2)}=-\dfrac{1}{4} \dfrac{1}{(k+1)\left(k+\dfrac{1}{2}\right)} \]
  con 
 \[r=0, s=1, c_1=\dfrac{1}{2}, \lambda_0=1, \text { y } A=-\dfrac{1}{4} \]\
  Por lo tanto
 \begin{equation}\label{cosenohiper}
 \cos x={ }_0 F_1\left(\left.\begin{array}{c}
\cdot \\
\dfrac{1}{2}
\end{array} \right\rvert\,-\dfrac{x^2}{4}\right) \quad(x \in \mathbb{R}) .
 \end{equation}
\end{enumerate}
\end{sol}
\subsection{Funciones hipergeométricas clásicas}
\subsubsection{Primer caso: $s=0$}
\begin{itemize}
  \item Si $r=0$ se obteniene la funci\'on exponencial (\ref{ex}).
  \item Cuando $r=1$, la funci\'on hipergeom\'etrica ${ }_1 F_0$ se reduce a la funci\'on binomial. Es decir, $\forall x\in \left(-1,1\right)$
  \begin{equation}\label{hiperbino}
    { }_1 F_0\left(\begin{array}{c|c}
a & x \\
\cdot & x
\end{array}\right)=\displaystyle\sum_{k=0}^{+\infty} \dfrac{(a)_k}{k!} x^k=\displaystyle\sum_{k=0}^{+\infty}\binom{-a}{k}(-x)^k=(1-x)^{-a} 
  \end{equation}
\end{itemize}
\subsubsection{Segundo caso: $s=1$}
Consideramos tres casos. Corresponden a nuevas funciones de gran importancia.
\begin{itemize}

\item Para \( r = 0 \), la función hipergeométrica \( {}_0F_1 \) está relacionada con las funciones de Bessel.
\begin{equation}\label{besselhiper}
{}_0F_1(\cdot \mid x) = \displaystyle\sum_{k=0}^{+\infty} \dfrac{x^k}{(c)_k \, k!} \quad \text{para todo } x \in \mathbb{R}.
\end{equation}

\item Para \( r = 1 \), se obtiene la función hipergeométrica de Kummer \( {}_1F_1 \), también conocida como función hipergeométrica confluente.
\begin{equation}\label{confluentehip}
{}_1F_1\left(\left.\begin{array}{l}
a \\
c
\end{array} \right\rvert\, x\right) = \displaystyle\sum_{k=0}^{+\infty} \dfrac{(a)_k}{(c)_k \, k!} \, x^k \quad \text{para todo } x \in \mathbb{R}.
\end{equation}

\item Para \( r = 2 \), se obtiene la función hipergeométrica de Gauss \( {}_2F_1 \).
\begin{equation}\label{funcionhipergeometrica}
{}_2F_1\left(\left.\begin{array}{c}
a,\, b \\
c
\end{array} \right\rvert\, x\right) = \displaystyle\sum_{k=0}^{+\infty} \dfrac{(a)_k (b)_k}{(c)_k \, k!} \, x^k \quad \text{para todo } x \in (-1,1).
\end{equation}

\end{itemize}
\subsubsection{Funci\'on hipergeom\'etrica modificada}\label{hipermodia}
Hemos visto al comienzo de este capítulo que es necesario, en la definición (\ref{hipergene}) de las funciones hipergeométricas, suponer que los parámetros $c_j$ no son enteros negativos ni cero (excepto posiblemente en el caso de los polinomios hipergeométricos). Esta restricción es a veces problemática. Puede evitarse definiendo las funciones hipergeométricas modificadas ${}_p \mathcal{F}_q$ mediante

\begin{equation}\label{hipermodi}
{}_p \mathcal{F}_q\left(\left.\begin{array}{c}
a_1, \ldots, a_p \\
c_1, \ldots, c_q
\end{array} \right\rvert\, x\right)
= \displaystyle\sum_{k=0}^{+\infty} \dfrac{(a_1)_k \cdots (a_p)_k}{\Gamma(c_1+k) \cdots \Gamma(c_q+k)\, k!} \, x^k
\end{equation}


Como la función $\dfrac{1}{\Gamma}$ está definida en $\mathbb{R}$ (ver \ref{aped.A}), el término general de la serie (\ref{hipermodi}) está siempre definido, incluso si alguno de los $c_j$ es un entero negativo o cero. Además, cuando ninguno de los $c_j$ es un entero negativo ni cero, tenemos por (\ref{gammapoc})
\begin{equation}\label{hipermodigamma}
{}_p \mathcal{F}_q\left(\left.\begin{array}{c}
a_1, \ldots, a_p \\
c_1, \ldots, c_q
\end{array} \right\rvert\, x\right)
= \dfrac{1}{\Gamma(c_1) \cdots \Gamma(c_q)} 
\, {}_p F_q\left(\left.\begin{array}{c}
a_1, \cdots, a_p \\
c_1, \cdots, c_q
\end{array} \right\rvert\, x\right)
\end{equation}
Por lo tanto, utilizaremos ${}_p \mathcal{F}_q$ en lugar de ${}_p F_q$ cada vez que ello simplifique las cosas, y luego usaremos (\ref{hipermodigamma}) para pasar de una función a la otra.

Denotemos $\mathcal{D}_0 = \mathbb{R}$ cuando $p \leq q$ o cuando ${}_p F_q$ y ${}_p \mathcal{F}_q$ son polinomios hipergeométricos, y $\mathcal{D}_0 = \, ]-1,1[ \,$ cuando $p = q+1$ y ${}_p F_q$, ${}_p \mathcal{F}_q$ no son polinomios hipergeométricos. 

\subsubsection{Ecuaciones Diferenciales}\label{EDhipergeometrica}
Mostraremos en esta sección que las funci\'on hipergeométrica ${}_pF_q$, definida en (\ref{hipergene}) , satisface una ecuación diferencial lineal. Para este propósito, introducimos el operador de Euler
\begin{equation}\label{eulerope}
D = x \dfrac{d}{dx}
\end{equation}
\Theorem{ED de las funciones hipergeométricas}{  
Las funciones ${}_pF_q$ y ${}_p \mathcal{F}_q$ satisfacen en $\mathcal{D}_0$ la ecuación diferencial lineal de orden $(q+1)$

\begin{equation}\label{EDhiper}
\left(D + c_1 - 1\right) \cdots \left(D + c_q - 1\right) D y = x \left(D + a_1\right) \cdots \left(D + a_p\right) y
\end{equation}}\label{thm:Edhiper}
\begin{demo}
hsd
\end{demo}
\Example{ ED de la funci\'on hipergeom\'etrica de Gauss}{ La función hipergeométrica de Gauss ${}_2F_1$, definida por (\ref{funcionhipergeometrica})
satisface en $\mathcal{D}_0$ la ecuación diferencial lineal de segundo orden
\begin{equation}\label{edgauhiper}
(D + c - 1)\, D y = x\, (D + a)(D + b)\, y
\end{equation} }
\begin{sol}
La ED (\ref{edgauhiper}) es equivalente a la ecuaci\'on
\[
\begin{aligned}
&\displaystyle (1 - x)\, D^2 y + \left(c - 1 - (a + b)x\right) D y - abx\, y = 0 
\quad \text{Aplicando el operador de Euler (\ref{eulerope})} \\
 &\displaystyle (1 - x)x\left(\dfrac{d y}{d x} + x\, \dfrac{d^2 y}{d x^2}\right) 
+ x\left(c - 1 - (a + b)x\right) \dfrac{d y}{d x} - abx\, y = 0 \\
 &\displaystyle x(1 - x)\, \dfrac{d^2 y}{d x^2} + \left(c - (a + b + 1)x\right) \dfrac{d y}{d x} - ab\, y = 0
\end{aligned}
\]

\end{sol}
Las ecuaciones diferenciales que satisfacen las funciones hipergeométricas ${}_1F_1$ y ${}_0F_1$ pueden obtenerse mediante el mismo método (indicar que se pondran como ejercicios).
\subsection{Funci\'on hipergeom\'etrica de Gauss}\label{fhg}
La función hipergeométrica de Gauss \( {}_2F_1 \) es la más importante dentro de las funciones hipergeométricas, ya que muchas funciones especiales clásicas pueden expresarse como casos particulares de ella. Además, \( {}_2F_1 \) resuelve una ecuación diferencial de segundo orden con tres puntos singulares regulares, lo que la convierte en una herramienta central en el estudio de funciones especiales y ecuaciones diferenciales. Su estructura rica y sus múltiples aplicaciones en física, geometría y análisis la hacen esencial en el desarrollo del análisis matemático.
Ahora discutiremos algunos importantes casos de la serie hipergeom\'etrica definida en (\ref{funcionhipergeometrica})
\subsubsection*{Caso 1}
$a$ o $b$ es un número entero negativo o cero, por ejemplo, $a = -n$, donde $n \geq 0$ es un número entero. En este caso, la serie (\ref{funcionhipergeometrica}) se reduce a un polinomio hipergeométrico

\begin{equation}\label{hiperpoli}
{}_2F_1\left(\left.\begin{array}{c}
-n,\, b \\
c
\end{array} \right\rvert\, x\right) =\displaystyle \sum_{k=0}^{n} \dfrac{(-n)_k (b)_k}{(c)_k\, k!} \, x^k \quad (x \in \mathbb{R}).
\end{equation}
Aquí, $c$ puede ser un número entero negativo o cero, siempre que se cumpla $c \leq -n$.
\subsubsection*{Caso 2}
Ni $a$ ni $b$ son enteros negativos o cero. En este caso, debemos suponer que $c \notin \mathbb{Z}_{\leq 0}$, de modo que $(c)_n$ nunca se anule. Bajo esta condición, la serie (\ref{funcionhipergeometrica}) converge para $|x| < 1$, y se define la función hipergeométrica de Gauss ${}_2F_1$ mediante

\begin{equation}\label{gauupoli}
{}_2F_1\left(\left.\begin{array}{c}
a,\, b \\
c
\end{array} \right\rvert\, x\right) = \displaystyle\sum_{k=0}^{+\infty} \dfrac{(a)_k (b)_k}{(c)_k\, k!} \, x^k \quad (x \in (-1,1).
\end{equation}
Al igual que en el caso general (Subsecci\'on (\ref{hipermodia})), podemos simplificar las expresiones introduciendo la función hipergeométrica modificada ${}_2 \mathcal{F}_1$, definida por
\begin{equation}\label{gauupoli}
{}_2 \mathcal{F}_1\left(\left.\begin{array}{c}
a,\, b \\
c
\end{array} \right\rvert\, x\right) = \displaystyle\sum_{k=0}^{+\infty} \dfrac{(a)_k (b)_k}{\Gamma(c + k)\, k!} \, x^k=\dfrac{1}{\Gamma(c)}{ }_2 F_1\left(\begin{array}{c|c}
a, b & x \\
c & x
\end{array}\right)
\end{equation}
A continuación se analizarán algunas funciones que pueden expresarse como un caso particular de la función hipergeométrica de Gauss. 
\Example{ Representación hipergeométrica del arcoseno }{ 
Exprese la función 
\[
f(x) = \arcsin(x), \quad \forall x \in (-1,1),
\]
en términos de la función hipergeométrica de Gauss \( {}_2F_1 \). }
\begin{sol}
Para todo \( x \in (-1,1) \),

\[
\begin{aligned}
\arcsin x 
&= \displaystyle \int_0^x \left(1 - t^2\right)^{-\dfrac{1}{2}} \, dt\qquad \text{usando (\ref{hiperbino})}\\[2ex]
& =\displaystyle \int_0^x \displaystyle\sum_{k=0}^{+\infty} \dfrac{\displaystyle\left(\dfrac{1}{2}\right)_k}{\displaystyle k!} \, t^{2k} \, dt \\[2ex]
&= \displaystyle x \sum_{k=0}^{+\infty} \dfrac{\displaystyle\left(\dfrac{1}{2}\right)_k}{\displaystyle k!(2k+1)} \, x^{2k}
\end{aligned}
\]

Aplicamos el teorema  (\ref{thm:fhper}), con

\[
\displaystyle \dfrac{\lambda_{k+1}}{\lambda_k} = \dfrac{\left(k + \dfrac{1}{2}\right)^2}{(k+1)\left(k + \dfrac{3}{2}\right)}
\]
\[\arcsin x=x_{\cdot 2} F_1\left(\left.\begin{array}{c}
\dfrac{1}{2}, \dfrac{1}{2} \\
\dfrac{3}{2}
\end{array} \right\rvert\, x^2\right)\]
\end{sol}
\Example{ Representación hipergeométrica de la tangente inversa }{ 
Exprese la función 
\[
f(x) = \arctan(x), \quad \forall x \in (-1,1),
\]
en términos de la función hipergeométrica de Gauss \( {}_2F_1 \). }
\begin{sol}
Sabemos que, para todo \( x \in (-1,1)\)

\[
\begin{aligned}
\arctan x 
&= \displaystyle\int_0^x \dfrac{1}{1 + t^2} \, dt =\displaystyle \sum_{k=0}^{+\infty} (-1)^k \, \dfrac{x^{2k+1}}{2k+1} \\\\[1.5ex]
&= \displaystyle x \sum_{k=0}^{+\infty} \dfrac{1}{2k+1} \left(-x^2\right)^k = \displaystyle \sum_{k=0}^{+\infty} \lambda_k \left(-x^2\right)^k
\end{aligned}
\]

Aquí, se tiene que \( \lambda_0 = 1 \) y

\[
\displaystyle \dfrac{\lambda_{k+1}}{\lambda_k} = \dfrac{2k+1}{2k+3} = \dfrac{\left(k + \dfrac{1}{2}\right)(k+1)}{\left(k + \dfrac{3}{2}\right)(k+1)}
\]

Por lo tanto, el teorema (\ref{thm:fhper}) proporciona 
\[
\arctan x=x_{.2} F_1\left(\left.\begin{array}{c}
1, \dfrac{1}{2} \\
\dfrac{3}{2}
\end{array} \right\rvert\,-x^2\right)
\]

\end{sol}
\subsection{Propiedades de la funci\'on hipergeom\'etrica}\label{prophiper}
Antes de continuar con aplicaciones específicas, es importante establecer algunas identidades fundamentales que satisfacen las funciones hipergeométricas de Gauss. Estas propiedades, derivadas de la estructura de la serie hipergeométrica definida en (\ref{hiperpq}) y de su ecuación diferencial asociada (ver Sección \ref{EDhipergeometrica}), permiten transformar y relacionar funciones con distintos parámetros, simplificando así el análisis de soluciones en contextos teóricos y aplicados.

Las propiedades que se presentan a continuación constituyen herramientas clave en el tratamiento analítico de expresiones hipergeométricas y serán de gran utilidad más adelante, especialmente al establecer relaciones entre las soluciones de ecuaciones diferenciales asociadas a funciones especiales, expresadas en términos de la función hipergeométrica de Gauss.
\Property{Propiedades funcionales de la función hipergeométrica}{ Las funciones hipergeom\'etricas de Gauss satisfacen las siguientes propiedades

\begin{equation} \label{prohiper1}
{}_2F_1\left(\begin{array}{c|c}
a+1,\ b & \\
c & x
\end{array}\right) -
{}_2F_1\left(\begin{array}{c|c}
a,\ b & \\
c & x
\end{array}\right) =
\dfrac{b x}{c}\
{}_2F_1\left(\begin{array}{c|c}
a+1,\ b+1 & \\
c+1 & x
\end{array}\right)
\end{equation}

\begin{equation} \label{prohiper2}
{}_2F_1\left(\begin{array}{c|c}
a,\ b & \\
c & x
\end{array}\right) -
{}_2F_1\left(\begin{array}{c|c}
a,\ b & \\
c-1 & x
\end{array}\right) =
-\dfrac{a b x}{c(c-1)}\
{}_2F_1\left(\begin{array}{c|c}
a+1,\ b+1 & \\
c+1 & x
\end{array}\right)
\end{equation}

\begin{equation} \label{prohiper3}
{}_2F_1\left(\begin{array}{c|c}
a,\ b+1 & \\
c+1 & x
\end{array}\right) -
{}_2F_1\left(\begin{array}{c|c}
a,\ b & \\
c & x
\end{array}\right) =
\dfrac{a(c-b)x}{c(c+1)}\
{}_2F_1\left(\begin{array}{c|c}
a+1,\ b+1 & \\
c+2 & x
\end{array}\right)
\end{equation}

\begin{equation} \label{prohiper4}
{}_2F_1\left(\begin{array}{c|c}
a-1,\ b+1 & \\
c & x
\end{array}\right) -
{}_2F_1\left(\begin{array}{c|c}
a,\ b & \\
c & x
\end{array}\right) =
\dfrac{(a-b-1)x}{c}\
{}_2F_1\left(\begin{array}{c|c}
a,\ b+1 & \\
c+1 & x
\end{array}\right)
\end{equation}}
\begin{demo}
Para demostrar la propiedad (\ref{prohiper1}) tenemos de la expresi\'on (\ref{funcionhipergeometrica})
\[
\begin{aligned}
&{}_2F_1\left( \begin{array}{c|c}
a+1,\ b & \\
c & x
\end{array} \right)
-
{}_2F_1\left( \begin{array}{c|c}
a,\ b & \\
c & x
\end{array} \right)
= \displaystyle \sum_{k=0}^{\infty} \left[ \dfrac{(a+1)_k - (a)_k}{(c)_k\, k!} (b)_k \right] x^k \\[1.5ex]
&= \displaystyle \sum_{k=1}^{\infty} \dfrac{(a)_k (b)_k}{(c)_k\, (k-1)!} \cdot \dfrac{x^k}{a k}
\qquad \text{(usando la identidad de Pochhammer)} \\[1.5ex]
&= x \displaystyle \sum_{k=0}^{\infty} \dfrac{(a+1)_k (b+1)_k}{(c+1)_k\, k!} x^k
\qquad \text{(haciendo el cambio } k = m + 1 \text{)} \\[1.5ex]
&= \dfrac{b x}{c} \cdot
{}_2F_1\left( \begin{array}{c|c}
a+1,\ b+1 & \\
c+1 & x
\end{array} \right)
\end{aligned}
\]
\end{demo}
Las demás propiedades se omiten por brevedad y se dejan al lector como ejercicio (ver secci\'on \ref{hiperprop}), ya que su demostración puede llevarse a cabo siguiendo razonamientos análogos a los expuestos anteriormente. %Esta actividad no solo refuerza la comprensión de las técnicas empleadas, sino que también promueve el desarrollo de habilidades analíticas esenciales para el estudio riguroso de las funciones especiales.
%\begin{demo}
%
%\[
%\begin{gathered}
%\displaystyle F(a+1, b, c, x) = \dfrac{\Gamma(c)}{\Gamma(a+1)\Gamma(b)} \sum_{m=0}^{\infty} \dfrac{\Gamma(m+a+1)\Gamma(m+b)}{m! \Gamma(m+c)} x^m \\
%\displaystyle F(a, b, c, x) = \dfrac{\Gamma(c)}{\Gamma(a)\Gamma(b)} \sum_{m=0}^{\infty} \dfrac{\Gamma(m+a)\Gamma(m+b)}{m! \Gamma(m+c)} x^m
%\end{gathered}
%\]
%
%Sustituyendo ...
%
%\[
%\begin{aligned}
%&= \dfrac{\Gamma(c)}{\Gamma(a+1)\Gamma(b)} \sum_{m=0}^{\infty} \dfrac{\Gamma(m+a+1)\Gamma(m+b)}{m! \Gamma(m+c)} x^m - \dfrac{\Gamma(c)}{\Gamma(a)\Gamma(b)} \sum_{m=0}^{\infty} \dfrac{\Gamma(m+a)\Gamma(m+b)}{m! \Gamma(m+c)} x^m \\
%&= \dfrac{\Gamma(c)}{a\Gamma(a)\Gamma(b)} \sum_{m=0}^{\infty} \dfrac{(m+a)\Gamma(m+a)\Gamma(m+b)}{m! \Gamma(m+c)} x^m - \dfrac{\Gamma(c)}{\Gamma(a)\Gamma(b)} \sum_{m=0}^{\infty} \dfrac{\Gamma(m+a)\Gamma(m+b)}{m! \Gamma(m+c)} x^m \\
%&= \sum_{m=0}^{\infty} \dfrac{\Gamma(c)}{\Gamma(a)\Gamma(b)} \dfrac{\Gamma(m+a)\Gamma(m+b)}{m! \Gamma(m+c)} \left( \dfrac{m+a}{a} - 1 \right) x^m \\
%&= \sum_{m=0}^{\infty} \dfrac{\Gamma(c)}{\Gamma(a)\Gamma(b)} \dfrac{\Gamma(m+a)\Gamma(m+b)}{m! \Gamma(m+c)} \cdot \dfrac{m}{a} x^m \\
%&= \sum_{m=1}^{\infty} \dfrac{\Gamma(c)}{\Gamma(a)\Gamma(b)} \dfrac{\Gamma(m+a)\Gamma(m+b)}{m! \Gamma(m+c)} \cdot \dfrac{m}{a} x^m \\
%&\text{Sea } k = m - 1 \Rightarrow m = k + 1,\ \text{entonces cuando } m \to 1,\ k \to 0,\quad \text{Sustituyendo:} \\
%&= \sum_{k=0}^{\infty} \dfrac{\Gamma(c)}{a\Gamma(a)\Gamma(b)} \cdot \dfrac{\Gamma(k+a+1)\Gamma(k+b+1)(k+1)}{(k+1)! \Gamma(k+c+1)} x^{k+1} \\
%&= \dfrac{\Gamma(c)}{\Gamma(a+1)\Gamma(b)} x \sum_{m=0}^{\infty} \dfrac{\Gamma(m+a+1)\Gamma(m+b+1)}{m! \Gamma(m+c+1)} x^m \\
%&= \dfrac{b}{c} x \cdot \dfrac{c\Gamma(c)}{\Gamma(a+1) b\Gamma(b)} \sum_{m=0}^{\infty} \dfrac{\Gamma(m+a+1)\Gamma(m+b+1)}{m! \Gamma(m+c+1)} x^m \\
%&= \dfrac{b}{c} x \cdot \dfrac{\Gamma(c+1)}{\Gamma(a+1)\Gamma(b+1)} \sum_{m=0}^{\infty} \dfrac{\Gamma(m+a+1)\Gamma(m+b+1)}{m! \Gamma(m+c+1)} x^m \\
%\text{Así:} \\
%\displaystyle F(a+1, b, c, x) - F(a, b, c, x) = \dfrac{b}{c} x F(a+1, b+1, c+1, x)
%\end{aligned}
%\]
%\end{demo}



%\Property{Propiedades}{	
%	\begin{eqnarray}
%		F(a, b, c, x)-F(a, b, c-1, x)&=&-\frac{a b x}{c(c-1)} F(a+1, b+1, c+1, x)
%\end{eqnarray}}
%\begin{demo}
%
%\[
%\begin{aligned}
%\displaystyle F(a, b, c, x) = \dfrac{\Gamma(c)}{\Gamma(a)\Gamma(b)} \sum_{m=0}^{\infty} \dfrac{\Gamma(m+a)\Gamma(m+b)}{m!\Gamma(m+c)} x^m \\
%\displaystyle F(a, b, c-1, x) = \dfrac{\Gamma(c-1)}{\Gamma(a)\Gamma(b)} \sum_{m=0}^{\infty} \dfrac{\Gamma(m+a)\Gamma(m+b)}{m!\Gamma(m+c-1)} x^m
%\end{aligned}
%\]
%
%Sustituyendo...
%
%\begin{eqnarray*}
%&=& \dfrac{\Gamma(c)}{\Gamma(a)\Gamma(b)} \sum_{m=0}^{\infty} \dfrac{\Gamma(m+a)\Gamma(m+b)}{m!\Gamma(m+c)} x^m - \dfrac{\Gamma(c-1)}{\Gamma(a)\Gamma(b)} \sum_{m=0}^{\infty} \dfrac{\Gamma(m+a)\Gamma(m+b)}{m!\Gamma(m+c-1)} x^m \\
%&=& \dfrac{\Gamma(c)}{\Gamma(a)\Gamma(b)} \sum_{m=0}^{\infty} \dfrac{\Gamma(m+a)\Gamma(m+b)}{m!\Gamma(m+c)} x^m - \dfrac{(c-1)\Gamma(c-1)}{(c-1)\Gamma(a)\Gamma(b)} \sum_{m=0}^{\infty} \dfrac{\Gamma(m+a)\Gamma(m+b)(m+c-1)}{m!\Gamma(m+c)} x^m \\
%&=& \sum_{m=0}^{\infty} \dfrac{\Gamma(c)}{\Gamma(a)\Gamma(b)} \dfrac{\Gamma(m+a)\Gamma(m+b)}{m!\Gamma(m+c)} x^m - \sum_{m=0}^{\infty} \dfrac{\Gamma(c)}{(c-1)\Gamma(a)\Gamma(b)} \dfrac{\Gamma(m+a)\Gamma(m+b)(m+c-1)}{m!\Gamma(m+c)} x^m \\
%&=& \sum_{m=0}^{\infty} \dfrac{\Gamma(c)}{\Gamma(a)\Gamma(b)} \dfrac{\Gamma(m+a)\Gamma(m+b)}{m!\Gamma(m+c)} \left[ 1 - \dfrac{m+c-1}{c-1} \right] x^m \\
%&=& \sum_{m=0}^{\infty} \dfrac{\Gamma(c)}{\Gamma(a)\Gamma(b)} \dfrac{\Gamma(m+a)\Gamma(m+b)}{m!\Gamma(m+c)} \left[ \dfrac{-m}{c-1} \right] x^m \\
%&=& \sum_{m=1}^{\infty} \dfrac{\Gamma(c)}{\Gamma(a)\Gamma(b)} \dfrac{\Gamma(m+a)\Gamma(m+b)}{m!\Gamma(m+c)} \left[ \dfrac{-m}{c} \right] x^m \quad \text{Cambiando } k = m - 1 \\
%&=& \sum_{k=0}^{\infty} \dfrac{\Gamma(c)}{\Gamma(a)\Gamma(b)} \dfrac{\Gamma(k+a+1)\Gamma(k+b+1)}{(k+1)!\Gamma(k+c+1)} \left[ \dfrac{-(k+1)}{c-1} \right] x^{k+1} \\
%&=& \sum_{m=0}^{\infty} \dfrac{\Gamma(c)}{\Gamma(a)\Gamma(b)} \dfrac{\Gamma(m+a+1)\Gamma(m+b+1)}{m!\Gamma(m+c+1)} \left[ \dfrac{-1}{c-1} \right] x^{m+1} \\
%&=& -\dfrac{abx}{c(c-1)} \cdot \dfrac{\Gamma(c+1)}{\Gamma(a+1)\Gamma(b+1)} \sum_{m=0}^{\infty} \dfrac{\Gamma(m+a+1)\Gamma(m+b+1)}{m!\Gamma(m+c+1)} x^m
%\end{eqnarray*}
%
%Por lo tanto:
%
%\begin{eqnarray*}
%F(a, b, c, x) - F(a, b, c-1, x) &=& -\dfrac{abx}{c(c-1)} F(a+1, b+1, c+1, x)
%\end{eqnarray*}
%\end{demo}


%\Property{Propiedades}{\begin{eqnarray}
%	F(a, b+1, c+1, x)-F(a, b, c, x)&=&\frac{a(c-b) x}{c(c+1)} F(a+1, b+1, c+2, x)
%	\end{eqnarray}}


%\begin{demo}
%
%\[
%\begin{aligned}
%&\displaystyle F(a, b+1, c+1, x) = \dfrac{\Gamma(c+1)}{\Gamma(a) \Gamma(b+1)} \sum_{m=0}^{\infty} \dfrac{\Gamma(m+a) \Gamma(m+b+1)}{m! \Gamma(m+c+1)} x^m \\
%&\displaystyle F(a, b, c, x) = \dfrac{\Gamma(c)}{\Gamma(a) \Gamma(b)} \sum_{m=0}^{\infty} \dfrac{\Gamma(m+a) \Gamma(m+b)}{m! \Gamma(m+c)} x^m \\
%&\text{Sustituyendo ...} \\
%&= \dfrac{\Gamma(c+1)}{\Gamma(a) \Gamma(b+1)} \sum_{m=0}^{\infty} \dfrac{\Gamma(m+a) \Gamma(m+b+1)}{m! \Gamma(m+c+1)} x^m - \dfrac{\Gamma(c)}{\Gamma(a) \Gamma(b)} \sum_{m=0}^{\infty} \dfrac{\Gamma(m+a) \Gamma(m+b)}{m! \Gamma(m+c)} x^m \\
%&= \sum_{m=0}^{\infty} \dfrac{\Gamma(c+1)}{\Gamma(a) \Gamma(b+1)} \dfrac{\Gamma(m+a) \Gamma(m+b+1)}{m! \Gamma(m+c+1)} x^m - \sum_{m=0}^{\infty} \dfrac{\Gamma(c)}{\Gamma(a) \Gamma(b)} \dfrac{\Gamma(m+a) \Gamma(m+b)}{m! \Gamma(m+c)} x^m \\
%&= \sum_{m=0}^{\infty} \dfrac{c \Gamma(c)}{\Gamma(a) b \Gamma(b)} \dfrac{\Gamma(m+a)(m+b) \Gamma(m+b)}{m! (m+c) \Gamma(m+c)} x^m - \sum_{m=0}^{\infty} \dfrac{\Gamma(c)}{\Gamma(a) \Gamma(b)} \dfrac{\Gamma(m+a) \Gamma(m+b)}{m! \Gamma(m+c)} x^m \\
%&= \sum_{m=0}^{\infty} \dfrac{\Gamma(c)}{\Gamma(a) \Gamma(b)} \dfrac{\Gamma(m+a) \Gamma(m+b)}{m! \Gamma(m+c)} \left[ \dfrac{c(m+b)}{b(m+c)} - 1 \right] x^m \\
%&= \sum_{m=0}^{\infty} \dfrac{\Gamma(c)}{\Gamma(a) \Gamma(b)} \dfrac{\Gamma(m+a) \Gamma(m+b)}{m! \Gamma(m+c)} \left[ \dfrac{m(c-b)}{b(m+c)} \right] x^m \\
%&= \sum_{m=1}^{\infty} \dfrac{\Gamma(c)}{\Gamma(a) \Gamma(b)} \dfrac{\Gamma(m+a) \Gamma(m+b)}{m! \Gamma(m+c)} \left[ \dfrac{m(c-b)}{b(m+c)} \right] x^m \\
%&\text{Sea } k = m - 1 \Rightarrow m = k + 1, \text{ sustituyendo:} \\
%&= \sum_{k=0}^{\infty} \dfrac{\Gamma(c)}{\Gamma(a) \Gamma(b)} \dfrac{\Gamma(k+a+1) \Gamma(k+b+1)}{(k+1)! \Gamma(k+c+1)} \left[ \dfrac{(k+1)(c-b)}{b(k+c+1)} \right] x^{k+1} \\
%&= \dfrac{(c-b) \Gamma(c)}{\Gamma(a) b \Gamma(b)} x \sum_{m=0}^{\infty} \dfrac{\Gamma(m+a+1) \Gamma(m+b+1)(m+1)}{(m+1)! (m+c+1) \Gamma(m+c+1)} x^m \\
%&= \dfrac{a b(c-b)}{c(c+1)} \cdot \dfrac{\Gamma(c+2)}{\Gamma(a+1) \Gamma(b+1)} x \sum_{m=0}^{\infty} \dfrac{\Gamma(m+a+1) \Gamma(m+b+1)}{m! \Gamma(m+c+2)} x^m \\
%&\displaystyle F(a, b+1, c+1, x) - F(a, b, c, x) = \dfrac{a(c-b)x}{c(c+1)} F(a+1, b+1, c+2, x)
%\end{aligned}
%\]
%\end{demo}



%\Property{Propiedades}{\begin{eqnarray}
%		F(a-1, b+1, c, x)-F(a, b, c, x)&=&\frac{(a-b-1) x}{c} F(a, b+1, c+1, x)
%\end{eqnarray}}

%\begin{demo}
%
%\[
%\begin{aligned}
%&\displaystyle F(a-1, b+1, c, x) = \dfrac{\Gamma(c)}{\Gamma(a-1) \Gamma(b+1)} \sum_{m=0}^{\infty} \dfrac{\Gamma(m+a-1) \Gamma(m+b+1)}{m! \Gamma(m+c)} x^m \\
%&\displaystyle F(a, b, c, x) = \dfrac{\Gamma(c)}{\Gamma(a) \Gamma(b)} \sum_{m=0}^{\infty} \dfrac{\Gamma(m+a) \Gamma(m+b)}{m! \Gamma(m+c)} x^m \\
%&\text{Sustituyendo ...} \\
%&= \dfrac{\Gamma(c)}{\Gamma(a-1) \Gamma(b+1)} \sum_{m=0}^{\infty} \dfrac{\Gamma(m+a-1) \Gamma(m+b+1)}{m! \Gamma(m+c)} x^m - \dfrac{\Gamma(c)}{\Gamma(a) \Gamma(b)} \sum_{m=0}^{\infty} \dfrac{\Gamma(m+a) \Gamma(m+b)}{m! \Gamma(m+c)} x^m \\
%&= \dfrac{(a-1)\Gamma(c)}{\Gamma(a) b \Gamma(b)} \sum_{m=0}^{\infty} \dfrac{\Gamma(m+a)(m+b) \Gamma(m+b)}{(m+a-1)m! \Gamma(m+c)} x^m - \dfrac{\Gamma(c)}{\Gamma(a) \Gamma(b)} \sum_{m=0}^{\infty} \dfrac{\Gamma(m+a) \Gamma(m+b)}{m! \Gamma(m+c)} x^m \\
%&= \dfrac{\Gamma(c)}{\Gamma(a) \Gamma(b)} \sum_{m=0}^{\infty} \dfrac{\Gamma(m+a) \Gamma(m+b)}{m! \Gamma(m+c)} \left[ \dfrac{(a-1)(m+b)}{b(m+a-1)} - 1 \right] x^m \\
%&= \dfrac{\Gamma(c)}{\Gamma(a) \Gamma(b)} \sum_{m=0}^{\infty} \dfrac{\Gamma(m+a) \Gamma(m+b)}{m! \Gamma(m+c)} \left[ \dfrac{m(a-1-b)}{b(m+a-1)} \right] x^m \\
%&= \dfrac{\Gamma(c)}{\Gamma(a) \Gamma(b)} \sum_{m=1}^{\infty} \dfrac{\Gamma(m+a) \Gamma(m+b)}{m! \Gamma(m+c)} \left[ \dfrac{m(a-1-b)}{b(m+a-1)} \right] x^m \\
%&\text{Sustituyendo } k = m - 1 \text{ y cambiando } k = m: \\
%&= \dfrac{\Gamma(c)}{\Gamma(a) \Gamma(b)} \sum_{m=0}^{\infty} \dfrac{\Gamma(m+a+1) \Gamma(m+b+1)}{(m+1)! \Gamma(m+c+1)} \left[ \dfrac{(m+1)(a-1-b)}{b(m+a)} \right] x^{m+1} \\
%&= \dfrac{a-1-b}{b} x \cdot \dfrac{\Gamma(c)}{\Gamma(a) \Gamma(b)} \sum_{m=0}^{\infty} \dfrac{\Gamma(m+a+1) \Gamma(m+b+1)(m+1)}{(m+a)(m+1)! \Gamma(m+c+1)} x^m \\
%&= \dfrac{a-1-b}{c} x \cdot \dfrac{\Gamma(c+1)}{\Gamma(a) \Gamma(b+1)} \sum_{m=0}^{\infty} \dfrac{\Gamma(m+a) \Gamma(m+b+1)}{m! \Gamma(m+c+1)} x^m
%\end{aligned}
%\]
%
%As\'i:
%
%\begin{eqnarray*}
%F(a-1, b+1, c, x) - F(a, b, c, x) &=& \dfrac{a-1-b}{c} x F(a, b+1, c+1, x)
%\end{eqnarray*}
%\end{demo}


%\Example{Ejemplo}{
%	\begin{eqnarray*}
%		F(a, b, a, x) &=& (1 - x)^{-b}
%	\end{eqnarray*}
%}

%\begin{demo}
	%Expresamos \( (1 - x)^{-b} \) en series de Maclaurin:
%	\begin{eqnarray*}
%		f(x) &=& (1 - x)^{-b} \\
%		f^{(0)}(x) &=& (1 - x)^{-b} \\
%		f^{(1)}(x) &=& b(1 - x)^{-(1 + b)} \\
%		f^{(2)}(x) &=& b(1 + b)(1 - x)^{-(2 + b)} \\
%		\vdots \\
%		f^{(m)}(x) &=& b(1 + b)(2 + b) \ldots (m - 1 + b)(1 - x)^{-(m + b)} \\
%		f^{(m)}(x) &=& (b)_m (1 - x)^{-(m + b)} \\
%		f^{(m)}(0) &=& (b)_m
%	\end{eqnarray*}

%	Así:
%	\begin{eqnarray}
%		f(x) = (1 - x)^{-b} &=& \displaystyle\sum_{m=0}^{\infty} \dfrac{(b)_m}{m!} x^m \quad |x| < 1
%	\end{eqnarray}

%	Por la definición dada en \ref{funcion_hipergeometrica}, tenemos:
%	\begin{eqnarray*}
%		F(a, b, a, x) &=& \displaystyle\sum_{m=0}^{\infty} \dfrac{(a)_m (b)_m}{m! (a)_m} x^m \\
%		F(a, b, a, x) &=& \displaystyle\sum_{m=0}^{\infty} \dfrac{(b)_m}{m!} x^m
%	\end{eqnarray*}

%	Lo cual verifica la igualdad.
%\end{demo}


%\Example{Ejemplo}{	\begin{eqnarray*}
%		% \nonumber % Remove numbering (before each equation)
%		F\left(\frac{1}{2}, 1, \displaystyle\frac{3}{2},-x^{2}\right)&=&\displaystyle\frac{\tan^{-1}(x)}{x}
%\end{eqnarray*}}

%\begin{demo}
%	Sabemos que
%	\begin{eqnarray*}
%		% \nonumber % Remove numbering (before each equation)
%		\displaystyle\int_0^{x} \frac{1}{1+t^{2}}dt&=&\displaystyle\tan^{-1}(x)  \quad|x|<1\\
%		\text{y}\\
%		\displaystyle\frac{1}{1-t}&=&\displaystyle\sum_{m=0}^{\infty} t^{m} \quad|t|<1
%	\end{eqnarray*}
%%	\begin{eqnarray*}
%		\frac{1}{1-(-t)^{2}}&=&\displaystyle\sum_{m=0}^{\infty}(-1)^{m} t^{2m}
%	\end{eqnarray*}
%	\begin{eqnarray*}
%		\displaystyle\int_{0}^{x} \frac{1}{1+t^{2}} d t=\displaystyle\sum_{m=0}^{\infty} \frac{(-1)^{m} x^{2 m+1}}{2 m+1}&=&\displaystyle\tan ^{-1}(x)
%	\end{eqnarray*}
%	De donde
%	\begin{eqnarray*}
%		\displaystyle\frac{\tan ^{-1}(x)}{x}&=&\displaystyle\sum_{m=0}^{\infty} \frac{(-1)^{m} x^{2m}}{2m+1}\\
%		&=&\displaystyle\frac{\Gamma(1+1/2)}{\Gamma(1/2)\Gamma(1)}\sum_{m=0}^{\infty}\frac{\Gamma(1/2+m)\Gamma(1+m)(-1)^{m}}{m!\Gamma(1+1/2+m)}x^{2m}
%	\end{eqnarray*}
%	Por la definici\'on de \ref{funcionhipergeometrica} en t\'ermino de Gamma se tiene
%	\begin{eqnarray*}
%		\displaystyle\frac{\tan ^{-1}(x)}{x}&=&\displaystyle\sum_{m=0}^{\infty} \frac{(-1)^{m} x^{2m}}{2 m+1}=F\left(\frac{1}{2}, 1, \frac{3}{2},-x^{2}\right) \quad |x|<1
%	\end{eqnarray*}
%\end{demo}
\medskip

\noindent
Uno de los ejemplos más importantes de funciones especiales que se expresan en términos de la función hipergeométrica de Gauss lo constituyen los polinomios de Jacobi. Estos polinomios surgen de manera natural al considerar parámetros específicos en la función ${}_2F_1$ y juegan un papel central en diversos contextos del análisis, como la teoría de ortogonalidad, soluciones de ecuaciones diferenciales y aplicaciones físicas. A continuación, se presenta su definición a partir de la función hipergeométrica.

\Definition{Polinomios de Jacobi}{Los polinomios de jacobi de grado n se define por \begin{align}
		P_{n}^{(\alpha,\beta)}(x) &=\dfrac{(\alpha +1)_{n}}{n!}\,{}_2F_1\left( \begin{array}{c|c}
-n,\ n + \alpha + \beta + 1 & \\
\alpha + 1 & \dfrac{1-x}{2}
\end{array} \right)
	\end{align}
	donde $_{2}F_{1}$ es la funci\'on hipergeom\'etrica de Gauss.}\label{defijacobi}

\medskip

\noindent
A partir de la Definición~\ref{defijacobi}, se pueden derivar expresiones explícitas para los polinomios de Jacobi. A continuación, se presentan algunos ejemplos que ilustran cómo estos polinomios pueden obtenerse utilizando la función hipergeométrica de Gauss ${}_2F_1$. Esto permite no solo calcular los polinomios $P_n^{(\alpha,\beta)}(x)$ de manera concreta, sino también apreciar su conexión estructural con las funciones hipergeométricas.
\Example{Ejemplo}{Utiliza la definici\'on (\ref{defijacobi}) para calcular los primeros $3$ polinomios de Jacobi de orden $(\alpha,\beta)$}

	\begin{sol}
		Si  $n=0$
		\begin{align*}
			P_{0}^{(x, \beta)} &(x)=\frac{(\alpha+1)_{0}}{0 !}\, _{2}F_{1}\left(0, \alpha+\beta+1, \alpha+1 ; \frac{1-x}{2}\right) \\
			&=\sum_{k=0}^{0} \frac{(0)_{k}(\alpha+\beta+1)_{k}}{k !(\alpha+1)_{k}}\left(\frac{1-x}{2}\right)^{k}
		\end{align*}
		\begin{eqnarray}\label{p0(x)}
			% \nonumber % Remove numbering (before each equation)
			P_{0}^{(\alpha, \beta)}(x) &=&1
		\end{eqnarray}
		si $n=1$
		\begin{align*}
			P_{1}^{(\alpha, \beta)}(x) &=\frac{(\alpha+1)!}{1 !}\,_{2}F_{1}\left(-1, \alpha+\beta+2, \alpha+1 ; \frac{1-x}{2}\right) \\
			&=(\alpha+1) \sum_{k=0}^{1} \frac{(-1)_{k}(\alpha+\beta+2)_{k}}{k !(\alpha+1) k}\left(\frac{1-x}{2}\right)^{k}\\
			&=(\alpha+1)\left[\frac{(-1)_{0}(\alpha+\beta+2)_{0}}{0 !(\alpha+1)_{0}}\left(\frac{1-x}{2}\right)^{0}+\frac{(-1)_{1}(\alpha+\beta+2)_{1}}{1 !(\alpha+1)_{1}}\frac{1-x}{2}\right]\\
			\intertext{Simplificando}\\
			&=(\alpha+1)\left[1-\frac{(\alpha+\beta+2)}{(\alpha+1)}\left(\frac{1-x}{2}\right)\right]\\
			&=(\alpha+1)-(\alpha+\beta+2)\left(\frac{1-x}{2}\right) \\
			&=\frac{1}{2}[2 \alpha+2-\alpha-\beta-2+(\alpha+\beta+2) x]
		\end{align*}
		\begin{eqnarray}\label{p1(x)}
			% \nonumber % Remove numbering (before each equation)
			P_{1}^{\left(\alpha_{1} \beta\right)}(x) &=&\frac{1}{2}[\alpha-\beta+(\alpha+\beta+2) x]
		\end{eqnarray}
		Si $n=2 $
		\begin{align*}
			P_{2}^{(\alpha, \beta)}(x)&=\frac{(\alpha+1)_{2}}{2 !}\, _{2}F_{1}\left(-2, \alpha+\beta+3, \alpha+1, \frac{1-x}{2}\right) \\
			&=\frac{(\alpha+1)_{2}}{2 !} \sum_{k=0}^{2} \frac{(-2) k(\alpha+\beta+3) k}{(\alpha+1)_{k} k !}\left(\frac{1-x}{2}\right)^{k}\\
			&=\frac{(-2)_{1}(\alpha+\beta+3)_{1}}{(\alpha+1)_{1} 1 !}\left(\frac{1-x}{2}\right)^{1}+\frac{(-2)_{2}(\alpha+\beta+3)_{2}}{(\alpha+1)_{2} 2 !}\left(\frac{1-x}{2}^{2}\right)\\
			&=\frac{(\alpha+1) 2}{2 !}\left[1-\frac{2(\alpha+\beta+3)}{\alpha+1}\left(\frac{1-x}{2}\right)+\frac{(-2)_{2}(\alpha+\beta+3)_{2}}{4(\alpha+1)_{2} 2 !}\left(1-2 x+x^{2}\right)\right]\\
			&=\frac{1}{2 !}\left[(\alpha+1)_{2}-\frac{2(\alpha+2)(\alpha+\beta+3)}{2}(1-x)+\frac{(-2)_{2}(\alpha+\beta+3)_{2}}{4\times2 !}\left(1-2 x+x^{2}\right)\right]\\
			&=\frac{1}{2 !}\left[(\alpha+1)_{2}-(\alpha+2)(\alpha+\beta+3)+(\alpha+2)(\alpha+\beta+3) x+\frac{(\alpha+\beta+3) 2}{4}\left(1-2 x+x^{2}\right)\right]
		\end{align*}
		\begin{align*}
			P_{2}^{(\alpha, \beta)}(x) =\frac{1}{2!}\left[(\alpha+1)_{2}-(\alpha+2)(\alpha+\beta+3)+(\alpha+2)(\alpha+\beta+3) x+\frac{(\alpha+\beta+3)_{2}}{4}-\frac{(\alpha+\beta+3)_{2}}{2}x..\right] \\
			.. +\frac{1}{2!} \frac{(\alpha+\beta+3)_{2}}{4} x^{2}
		\end{align*}
		Simplificando se obtiene que
		\begin{align*}
			P_{2}^{(\alpha, \beta)}&=\frac{1}{2 !}\left[-4-(\alpha+\beta)+(\alpha-\beta)^{2}+\frac{(\alpha-\beta)(\alpha+\beta+3)}{2}x+\frac{(\alpha+\beta+3)(\alpha+\beta+4)}{4} x^{2}\right]
		\end{align*}
		\begin{equation}\label{p2(x)}
P_{2}^{(\alpha, \beta)} = \dfrac{1}{8} \left\{ -4 - (\alpha+\beta) + (\alpha - \beta)^2 + 2(\alpha - \beta)(\alpha + \beta + 3)x + (\alpha + \beta + 3)(\alpha + \beta + 4)x^2 \right\}
\end{equation}
	\end{sol}
\subsection{Ecuaci\'on hipergeom\'etrica confluente}
Tomando la sustituci\'on $x=\frac{t}{b}$  en \ref{ecuacion hipergeometrica} obtendremos la ecuaci\'on hipergeom\'etrica confluente. Calculando las derivadas tenemos:
\begin{eqnarray*}
	\frac{d y}{d x}&=&b \frac{d y}{d t} \\
	\frac{d^2 y}{d x^2}&=&b^2 \frac{d^2 y}{d t^2}
\end{eqnarray*}
Reemplazando las derivadas en \ref{ecuacion hipergeometrica}
$$
\begin{gathered}
	\left(\frac{t}{b}\right)\left(1-\left(\frac{t}{b}\right)\right) b^2 \frac{d^2 y}{d t^2}+\left[c-(a+b+1)\left(\frac{t}{b}\right)\right] b \frac{d y}{d t}-a b y(t)=0 \\
	\left(\frac{t}{b}-\frac{t^2}{b^2}\right) b^2 \frac{d^2 y}{d t^2}+[c b-(a+b+1) t] \frac{d y}{d t}-a b y=0 \\
	\left(b t-t^2\right) \frac{d^2 y}{d t^2}+[c b-(a+b+1) t] \frac{d y}{d t}-a b y=0 \\
	\left(t-\frac{t^2}{b}\right) \frac{d^2 y}{d t^2}+\left[c-\left(1+\frac{a+1}{b}\right) t\right] \frac{d y}{d t}-a y=0
\end{gathered}
$$
Si hacemos que $ b \rightarrow \infty$, obtenemos la ecuaci\'on hipergeom\'etrica confluente
\begin{eqnarray}\label{ecuacionconfluente}
	% \nonumber % Remove numbering (before each equation)
	t \frac{d^2 y}{d t^2}+(c-t) \frac{d y}{d t}-a y&=&0
\end{eqnarray}
Sabemos que la soluci\'on es de la siguiente forma
$$
\begin{gathered}
	F\left(a, b, c, \frac{t}{b}\right)=\frac{\Gamma(c)}{\Gamma(a)}\left(\sum_{m=0}^{\infty} \frac{\Gamma(a+m)}{\Gamma(c+m) m !}\left\{\frac{\Gamma(b+m)}{b^m \Gamma(b)}\right\} t^m\right) \\
	\lim _{b \rightarrow \infty} F\left(a, b, c, \frac{t}{b}\right)=? \\
	\lim _{b \rightarrow \infty} F\left(a, b, c, \frac{t}{b}\right)=\lim _{b \rightarrow \infty} \frac{\Gamma(c)}{\Gamma(a)}\left(\sum_{m=0}^{\infty} \frac{\Gamma(a+m)}{\Gamma(c+m) m !}\left\{\frac{\Gamma(b+m)}{b^m \Gamma(b)}\right\} t^m\right)
\end{gathered}
$$
Tomamos el $\displaystyle\lim _{b \rightarrow \infty} \frac{\Gamma(b+m)}{b^m \Gamma(b)}$ y resolvamos ese limite ...
Por la propiedades de la funci\'on gamma
\begin{eqnarray*}
	% \nonumber % Remove numbering (before each equation)
	\dfrac{\Gamma(b+m)}{\Gamma(b)}&=&\prod_{i=0}^{m-1}(b+i)=b(b+1) \ldots(b+m-1)\\
	\dfrac{\Gamma(b+m)}{\Gamma(b)}&=&b^m+\cdots\\
	\text{Entonces}\\
	\lim _{b \rightarrow \infty} \frac{b^m+\cdots}{b^m}&=&1
\end{eqnarray*}
esto sigue que
\begin{eqnarray*}
	\displaystyle\lim _{b \rightarrow \infty} F\left(a, b, c, \frac{t}{b}\right)&=&\frac{\Gamma(c)}{\Gamma(a)}\left(\sum_{m=0}^{\infty} \frac{\Gamma(a+m)}{\Gamma(c+m) m !} t^m\right)
\end{eqnarray*}
Por lo que
\begin{eqnarray}\label{serieconfluente}
	\displaystyle\lim _{b \rightarrow \infty} F\left(a, b, c, \frac{t}{b}\right)&=&F(a, c, t)
\end{eqnarray}
La expresi\'on anterior se llama \textbf{serie Hipergeom\'etrica Confluente de Gauss}.
\section{Ecuaciones diferenciales como caso particular de funci\'on Hipergeom\'etrica}
La función hipergeométrica \( {}_2F_1(a,b;c;x) \) ocupa un lugar destacado en el estudio de ecuaciones diferenciales. Muchas ecuaciones clásicas, como las de Legendre, Chebyshev, Hermite o Bessel, pueden verse como casos particulares de la ecuación hipergeométrica. Esto nos permite expresar sus soluciones en términos de una misma función, lo que unifica su análisis y nos brinda una herramienta poderosa para abordar diversos problemas con un enfoque común.\\
Antes de tratar las ecuaciones cl\'asicas analizamos una ecuaci\'on cuya forma general se plantea como ejercicio al lector.
\Example{ Una EDO reducida a la Ecuaci\'on Hipergeom\'etrica}{ Expresa la ecuaci\'on diferencial
\begin{equation*}
  \left( x^{2}+4x+3 \right)\dfrac{d^{2}y}{dx^{2}}+\left(2x+1\right)\dfrac{dy}{dx}+5y=0
\end{equation*} en t\'ermino de la ecuaci\'on Hipergeom\'etrica.}
\begin{sol}
\begin{equation*}
\begin{aligned}
& (x + 3)(x + 1) \dfrac{dy}{dx} + (2x + 1) \dfrac{dy}{dx} + 5y = 0 \\[0.3em]
& z = \dfrac{x+1}{-2} \hspace{3cm} \text{(Cambio de variable)} \\[0.3em]
& \dfrac{dy}{dx} =- \dfrac{1}{2}\dfrac{dy}{dz} \\[0.3em]
& \dfrac{d^2y}{dx^2} = \dfrac{1}{4}\dfrac{d^2y}{dz^2} 
\end{aligned}
\end{equation*}
Reemplazando en la ecuación diferencial, obtenemos
\begin{equation*}
z(1 - z)\, y^{\prime\prime} - \left( \dfrac{1}{2} +2z \right) y^{\prime} - 5y = 0
\end{equation*}
De la expresión (\ref{hipersolgeneral}) tenemos que la solución es
\begin{equation*}
\begin{aligned}
y(x) =\; & c_{0}\;F\left(\dfrac{1 + \sqrt{19}i}{2},\, \dfrac{1 - \sqrt{19}i}{2},\,- \dfrac{1}{2},\, x\right) \\
& + c_{1}\;x^{\frac{3}{2}} F\left(\dfrac{4 + \sqrt{19}i}{2},\, \dfrac{4 - \sqrt{19}i}{2},\, \dfrac{5}{2},\, x\right)
\end{aligned}
\end{equation*}

\end{sol}
\subsection{Ecuaci\'on de Laguerre como caso particular de la Ecuaci\'on Hipergeom\'etrica Confluente de Gauss}
En \ref{ecuacionconfluente} tomando la sustituci\'on $c=\alpha+1\quad \text { y }\quad a=-n$
\begin{eqnarray*}
	t \frac{d^2 y}{d t^2}+(\alpha+1-t) \frac{d y}{d t}+n y&=&0
\end{eqnarray*}
Por lo tanto la soluci\'on est\'a dada por
\begin{eqnarray*}
	F(-n, \alpha+1, t)&=&\frac{\Gamma(\alpha+1)}{\Gamma(-n)} \sum_{m=0}^{\infty} \frac{\Gamma(m-n)}{\Gamma(\alpha+1+m) m !} t^m
\end{eqnarray*}
En el caso de que $\alpha=0$, obtenemos los polinomios de Laguerre de orden cero.
\begin{eqnarray*}
	F(-n,+1, t)&=&\displaystyle\frac{\Gamma(1)}{\Gamma(-n)} \sum_{m=0}^{\infty} \frac{\Gamma(m-n)}{\Gamma(m+1) m !} t^m\\
	&=&\Gamma(1+n) \sum_{m=0}^{\infty} \frac{(-1)^m}{(n-m) ! m ! \Gamma(m+1)} t^m\\
	\text{Partiendo de}\\
	\displaystyle\frac{\Gamma(1+n)}{(n-m) ! m ! \Gamma(m+1)}&=&\frac{n !}{m !(n-m) ! m !}\\
	&=&\displaystyle\frac{1}{m !}\left(\frac{n !}{m !(n-m) !}\right)\\
	&=&\displaystyle\frac{1}{m !}\left(\begin{array}{l}n \\ m\end{array}\right)\\
	\text{Entonces nos queda}\\
	L_n^0(t)&=&\sum_{m=0}^n\displaystyle \frac{1}{m !}\left(\begin{array}{l}n \\ m\end{array}\right)(-t)^m
\end{eqnarray*}
\textcolor{red}{revisar la parte siguiente con la anterior}
A partir de la serie hipergeom\'etrica confluente se obtienen los polinomios de Laguerre lo que motiva a presentar la definici\'on siguiente

\Definition{Serie hipergeom\'etrica}{Sea $_{1}F_{1}$ la serie hipergeom\'etrica confluente de Gauss, los polinonios de Laguerre $L_n^{(\alpha)}(x)$ se obtienen a partir de
	\begin{eqnarray}\label{laguerreiper}
		L_{n}^{(\alpha)}(x)=\left(\begin{array}{c}
			n+\alpha \\
			n
		\end{array}\right) \, _{1}F_{1}(-n, \alpha+1 ; x)
\end{eqnarray}}

Ahora a modo de ilustraci\'on obtenemos los polinomios de Laguerre.

\Example{Polinomios de Laguerre}{	Obtenga los primeros cuatro polinomios de Laguerre $L_{n}(x)$ utilizando la expresi\'on (\ref{laguerreiper}).}

 \begin{sol}
	\begin{equation*}
\begin{aligned}
&\text{Para } n = 0 \\
&L_0^{(0)}(x) = \binom{0}{0} \cdot {}_1F_1(0,1;x) = \binom{0}{0} \cdot 1 = 1 \\[0.8em]
%
&\text{Para } n = 1 \\
&L_1^{(0)}(x) = \binom{1}{1} \cdot {}_1F_1(-1,1;x) 
= \binom{1}{1} \left[1 + \dfrac{-1}{1}x\right] = 1 - x \\[0.8em]
%
&\text{Para } n = 2 \\
&L_2^{(0)}(x) = \binom{2}{2} \cdot {}_1F_1(-2,1;x) 
= 1 + \dfrac{-2}{1}x + \dfrac{(-2)(-1)}{1 \cdot 2 \cdot 2}x^2 
= 1 - 2x + \dfrac{1}{2}x^2 \\[0.8em]
%
&\text{Para } n = 3 \\
&L_3^{(0)}(x) = \binom{3}{3} \cdot {}_1F_1(-3,1;x) 
= 1 + \dfrac{-3}{1}x + \dfrac{(-3)(-2)}{1 \cdot 2 \cdot 2}x^2 + \dfrac{(-3)(-2)(-1)}{1 \cdot 2 \cdot 3 \cdot 6}x^3 \\
&\hspace{3.2cm}= 1 - 3x + \dfrac{3}{2}x^2 - \dfrac{1}{6}x^3
\end{aligned}
\end{equation*}

\end{sol}


\subsection{Ecuaci\'on de Hermite como caso particular de la ecuaci\'on Hipergeom\'etrica confluente}
En la ecuaci\'on definida en \ref{H} se toma la sustituci\'on $z=x^2$. Obteniendo las derivadas
\[
\begin{aligned}
& dz = 2x \, dx \\[0.5em]
& \dfrac{d}{dz}(\,) = \dfrac{1}{2x} \dfrac{d}{dx}(\,) \\[0.5em]
& \dfrac{d}{dz} \left( \dfrac{d}{dz}(\,) \right) = \dfrac{1}{2x} \dfrac{d}{dx} \left( \dfrac{1}{2x} \dfrac{d}{dx}(\,) \right) \\[0.5em]
& \dfrac{d^2}{dz^2}(\,) = \dfrac{1}{4} \cdot \dfrac{1}{x} \dfrac{d}{dx} \left( x^{-1} \dfrac{d}{dx}(\,) \right) \\[0.5em]
& = \dfrac{1}{4x} \left[ x^{-1} \dfrac{d^2}{dx^2}(\,) - x^{-2} \dfrac{d}{dx}(\,) \right] \\[0.5em]
& = \dfrac{1}{4} \dfrac{d^2}{dx^2}(\,) - \dfrac{1}{4x^3} \dfrac{d}{dx}(\,) 
   - \dfrac{1}{2x^2} \cdot \dfrac{1}{2x} \dfrac{d}{dx}(\,) \\[0.5em]
& = \dfrac{d^2}{dz^2}(\,) = \dfrac{1}{4z} \dfrac{d^2}{dx^2}(\,) - \dfrac{1}{2z} \dfrac{d}{dz}(\,) \\[0.5em]
& \dfrac{d^2}{dx^2}(\,) = 4z \dfrac{d^2}{dz^2}(\,) + 2 \dfrac{d}{dz}(\,) \\[0.5em]
& 4z \dfrac{d}{dz}(\,) = 2x \dfrac{d}{dx}(\,)
\end{aligned}
\]

Sustituyendo las derivadas en \ref{H}, tenemos
$$
\begin{gathered}
	4 z y^{\prime \prime}(z)+2 y^{\prime}(z)-4 z y^{\prime}(z)+2 n y(z)=0 \\
	4 z y^{\prime \prime}+2[1-2 z] y^{\prime}+2 n y(z)=0 \\
	z y^{\prime \prime}+\left[\frac{1}{2}-z\right] y^{\prime}+\frac{n}{2} z=0
\end{gathered}
$$
La ecuaci\'on obtenida es un caso de \ref{ecuacionconfluente} con $c=\frac{1}{2}\quad \text{y} \quad a=-\frac{n}{2}$.
As\'i las soluciones est\'an dada por
$$
\begin{aligned}
	&y_{0}(x)= _1F_1\left(-\frac{n}{2} , \frac{1}{2} , x^2\right) \\
	&y_{1}(x)=x _1F_1\left(-\frac{n+1}{2}+1 , \frac{3}{2} , x^2\right)
\end{aligned}
$$

\Example{Representaci\'on integral de la funci\'on Hipergeometrica}{Agregar
	\textcolor[rgb]{1.00,0.00,0.00}{Agregar mas ejemplos y expresar las funciones en el capitulo en termino de la hipergeometrica}}

	

\subsection{Ecuaci\'on de Jacoby como caso particular de la ecuaci\'on Hipergeom\'etrica }

\subsection{Ecuaci\'on de Chevyshev como caso particular de la ecuaci\'on Hipergeom\'etrica }
Tomando el cambio de variable $t=\dfrac{1-x}{2}$ en la ecuaci\'on (\ref{chebyshev equation}) obtenemos la ecuaci\'on hipergeom\'etrica
\begin{equation}\label{chevyhiper}
t\left(1-t\right)y^{\prime\prime}+\left(\dfrac{1}{2}-t\right)y^{\prime}+a^{2}y=0
\end{equation}
Cuya soluci\'on entorno a los puntos $x_{0}=0$ y $x_{0}=1$ se presenta a continuaci\'on. 
\begin{itemize}
  \item Como $x_{0}=0$ es un punto ordinario de (\ref{chevyhiper}) su soluci\'on se obtiene de la expresi\'on (\ref{hipersolgeneral})
  \begin{eqnarray}\label{chevyhiper10}
		y(x)=c_{0}\;F\left(a, -a, \dfrac{1}{2}, \dfrac{1-x}{2}\right)+c_{1}\left(\dfrac{1-x}{2}\right)^{1/2} F\left(\dfrac{1}{2}+a,\dfrac{1}{2}-a,\dfrac{1}{2},\dfrac{1-x}{2}\right)
	\end{eqnarray}
  \item De la expresi\'on (\ref{hiper1}) obtenemos la soluci\'on alrededor de $x_{0}=1$
  \begin{equation}\label{chevyhiper1}
    y(x)=c_0 F\left(a,-a, \dfrac{1}{2}, \dfrac{1-x}{2}\right)+c_1\left(\dfrac{1-x}{2}\right)^{1 / 2} F\left(a+\dfrac{1}{2},-a+\dfrac{1}{2}, \dfrac{3}{2}, \dfrac{1-x}{2}\right)
  \end{equation}
\end{itemize}
\subsection{Ecuaci\'on de Legendre como caso particular de la ecuaci\'on Hipergeom\'etrica }
Sean $\lambda=n\left(n+1\right)$ y $x=1-2t$ en la ecuaci\'on (\ref{ecuacionlegendre}), realizando el cambio de variable llegamos a la ecuaci\'on
\begin{equation}\label{legendrehiper}
  t\left(1-t\right)y^{\prime\prime}+\left(1-2t\right)y^{\prime}+n\left(n+1\right)y=0
\end{equation}
Como la soluci\'on de (\ref{ecuacionlegendre}) son los polinomios de Legendre, la soluc\'ion de (\ref{legendrehiper}) podemos expresarla como
\begin{equation}\label{pnhiper}
P_n(x)=F(-n, n+1,1,(1-x) / 2)
\end{equation}
\medskip

\noindent
En el siguiente ejemplo se ilustrará cómo se pueden obtener los primeros tres polinomios de Legendre a partir de la función hipergeométrica de Gauss. Esta estrategia permite evidenciar la relación que existe entre dichos polinomios clásicos y las soluciones particulares de la ecuación hipergeométrica, reforzando así su importancia dentro del estudio de funciones especiales.

\Example{Polinomios de Legendre a partir de la funci\'on Hipergeom\'etrica}{ Obtenga los polinomios $P_{0}(x), P_{1}(x), P_{2}(x)$ de Legendre utilizando la funci\'on Hipergeom\'etrica}
\begin{sol}
\[
\begin{aligned}
\displaystyle n &= 0 \\
\displaystyle P_0(x) &= \displaystyle\sum_{m=0}^0 \dfrac{(-0)_m (1)_m}{m!(1)_m} \left( \dfrac{1 - x}{2} \right)^m \\
\displaystyle P_0(x) &= 1 \\
\\
\displaystyle n &= 1 \\
\displaystyle P_1(x) &=\displaystyle \sum_{m=0}^1 \dfrac{(-1)_m (2)_m}{m!(1)_m} \left( \dfrac{1 - x}{2} \right)^m \\
\displaystyle P_1(x) &= 1 - 2 \left( \dfrac{1 - x}{2} \right) \\
\displaystyle P_1(x) &= x \\
\\
\displaystyle n &= 2 \\
\displaystyle P_2(x) &=\displaystyle \sum_{m=0}^2 \dfrac{(-2)_m (3)_m}{m!(1)_m} \left( \dfrac{1 - x}{2} \right)^m \\
\displaystyle P_2(x) &= 1 - 3(1 - x) + \dfrac{3}{2}(1 - x)^2 \\
\displaystyle P_2(x) &= \dfrac{1}{2}(3x^2 - 1)
\end{aligned}
\]
\end{sol} 
En el siguiente ejemplo, obtendremos las soluciones de la ecuación de Airy (\ref{Airy equation}), la cual fue analizada previamente en la Sección~\ref{ecuacionairy}, expresándolas en términos de la función hipergeométrica.
\Example{ ED Airy  }{ Obtenga las soluciones de la ED (\ref{Airy equation}) en t\'erminos de la funci\'on hipergeom\'etrica }
\begin{sol}
Tomando $n=k+3$ en la expresi\'on de recurrencia (\ref{Airyrecurr} )
\begin{equation}\label{{Airyrecurrencia}}
a_{k+3}=\dfrac{a_k}{(k+2)(k+3)} \quad \forall k \geq 0
\end{equation}
Ahora reemplazando $k$ sucesivamente por $3k,3k+1,3k+2$
\begin{itemize}
  \item \begin{equation*}
          a_{3 k+3}=\dfrac{a_{3 k}}{3(3 k+2)(k+1)}
        \end{equation*}
        \[
\begin{aligned}
&\begin{array}{ll}
k = 0 ; & \displaystyle a_3 = \dfrac{a_0}{3 \cdot 2 \cdot 1} \\[1.2ex]
k = 1 ; & \displaystyle a_6 = \dfrac{a_3}{3 \cdot 5 \cdot 2} = \dfrac{a_0}{3^2 \cdot 2 \cdot 1 \cdot 5 \cdot 2} \\[1.2ex]
k = 2 ; & \displaystyle a_9 = \dfrac{a_6}{3 \cdot 8 \cdot 3} = \dfrac{a_0}{3^3 \cdot 3 \cdot 2 \cdot 1 \cdot 8 \cdot 5 \cdot 2} \\[1.2ex]
k = 3 ; & \displaystyle a_{12} = \dfrac{a_9}{3 \cdot 11 \cdot 4} = \dfrac{a_0}{3^4 \cdot 4 \cdot 3 \cdot 2 \cdot 1 \cdot 11 \cdot 8 \cdot 5 \cdot 2} \\[1.2ex]
&\vdots  \\
& \displaystyle a_{3k} = \dfrac{a_0}{3^k \cdot k! \cdot (3k-1)(3k-4) \cdots 5 \cdot 2}\qquad \forall k \geq 1  \\[1.2ex]
& = \dfrac{a_0}{3^{2k} \cdot k! \cdot \left(-\dfrac{1}{3}+k\right)\left(-\dfrac{1}{3}+k-1\right) \cdots\left(-\dfrac{1}{3}+2\right) \left(-\dfrac{1}{3}+1\right)} \\[1.2ex]
& = \dfrac{a_0}{3^{2k} \cdot k! \cdot \left[\left(-\dfrac{1}{3}+1\right)+(k-1)\right]\left[\left(-\dfrac{1}{3}+1\right)+(k-2)\right] \cdots \left[\left(-\dfrac{1}{3}+1\right)+1\right] \left(-\dfrac{1}{3}+1\right)} \\[1.2ex]
& \displaystyle a_{3k} = \dfrac{a_0}{3^{2k} \cdot k! \cdot \left(\dfrac{2}{3}\right)_{k}}
\end{array}
\end{aligned}
\]

  \item \[
\begin{aligned}
& a_{3k+4} = \dfrac{a_{3k+1}}{3(k+1)(3k+4)} \\
&\begin{array}{ll}

k = 0 ; & \displaystyle a_4 = \dfrac{a_1}{3 \cdot 1 \cdot 4} \\[1.2ex]
k = 1 ; & \displaystyle a_7 = \dfrac{a_4}{3 \cdot 2 \cdot 7} = \dfrac{a_1}{3^2  \cdot 2 \cdot 1\cdot 4 \cdot 7} = \dfrac{a_1}{3^3 \cdot 3 \cdot 2 \cdot 1 \cdot 10 \cdot 7 \cdot 4} \\[1.2ex]
&\vdots  \\[1.2ex]
 & \displaystyle a_{3k+1} = \dfrac{a_1}{3^k \cdot k! \cdot (3k+1)(3k-2) \cdots 7 \cdot 4}\qquad \forall k \geq 1\\[1.2ex]
& = \dfrac{a_1}{3^{2k} \cdot k! \cdot \left( \dfrac{4}{3} \right)_k}
\end{array}
\end{aligned}
\]
  \item \[a_{3 k+5}=\dfrac{a_{3 k+2}}{(3 k+4)(3 k+5)}\]
  Como en la expresi\'on (\ref{Airyrecurr} ) se obtuvo que $a_{2}=0$ , entonces $a_{3 k+5}=0$
\end{itemize}
Por tanto, una expresi\'on de la soluci\'on general de la ecuaci\'on de Airy es
\begin{equation*}
\displaystyle y =  \displaystyle \sum_{k=0}^{+\infty} a_{3k} x^{3k} + \sum_{k=0}^{+\infty} a_{3k+1} x^{3k+1} 
\end{equation*}
Por la expresi\'on (\ref{hiperfun}) dada en el teorema (\ref{thm:fhper})
\begin{equation}\label{airyhiper}
\displaystyle a_0  \;_{0}F_{1}\left(\left. c_{\dfrac{2}{3}} \right| \dfrac{x^3}{9} \right) + a_1 x\; _{0}F_{1}\left(\left. c_{\dfrac{4}{3}} \right| \dfrac{x^3}{9} \right)
\end{equation}
\end{sol} 
\section{Ejercicios}
\subsection*{Propiedades de la funci\'on hipergeom\'etrica}\label{hiperprop}
\begin{enumerate}
  \item Demostrar la propiedad (\ref{prohiper2}) de la funci\'on hipergeom\'etrica.
  \item Demostrar la propiedad (\ref{prohiper3}) de la funci\'on hipergeom\'etrica.
  \item Demostrar la propiedad (\ref{prohiper4}) de la funci\'on hipergeom\'etrica.
\end{enumerate} 	
\mychapter{Funciones especiales como soluci\'on de un problema de Sturm-Louville }{\begin{wrapfigure}{l}{0.45\textwidth}
		\centering
		\includegraphics[width=0.45\textwidth]{imagen/img8.png}
	\end{wrapfigure} El estudio de las ecuaciones de Sturm–Liouville constituye un puente fundamental entre la teoría de ecuaciones diferenciales y la aparición de funciones especiales en contextos físicos y matemáticos. Estos problemas, caracterizados por operadores diferenciales autoadjuntos y condiciones de frontera, permiten la construcción de sistemas de funciones ortogonales que forman la base para el análisis de soluciones de gran diversidad de fenómenos.
	
	\vspace{0.5cm}
	
	En este capítulo se mostrará cómo diversos conjuntos de funciones especiales clásicas —como los polinomios de Legendre, Hermite, Laguerre, Chebyshev, así como las funciones de Bessel— emergen de manera natural como soluciones de problemas de Sturm–Liouville. Cada una de estas funciones se obtiene al imponer condiciones específicas que reflejan la naturaleza del sistema modelado, desde vibraciones de membranas y ondas esféricas, hasta la descripción cuántica de osciladores armónicos y potenciales centrales.
	
	\vspace{0.5cm}
	
	Asimismo, se introducirá el concepto de función de Green como autofunción asociada a un operador de Sturm–Liouville, junto con la importancia de los problemas con valores en la frontera. De esta manera, el lector comprenderá que las funciones especiales no son entidades aisladas, sino parte de una teoría unificada que surge del análisis de operadores diferenciales y de las condiciones físicas impuestas en los sistemas.}
	
	 \addtocontents{toc}{\protect\figuretoc{imagen/img8.png}}
	
	\textcolor[rgb]{0.00,0.50,0.25}{Teoria de sturm-liouville,definir que es el espacio $L^{2}$, un espacio de hilbert y pre-hilbert, ejemplos, definir el operador, el operador adjunto, propiedades, ejemplos, problemas de sturm liouville y clasificaciones(dirichlet,newman,cauchy)}
	
	\textcolor{red}{clasificacion de los problemas de sturm-liouville de acuerdo a las condiciones, propiedades de los problemas de sturm-liouville(las funciones propias de un problema de sturm-liouville forman un sistema ortogonal)}
	\textcolor{red}{calcular las normas de los polinomios especiales}
	\section{Funci\'on de Green como autofunciones}
	We continue to consider the operator
	$$
	L[y(x)]=\left[p(x) y^{\prime}(x)\right]^{\prime}+q(x) y(x)
	$$
	We want to solve
	$$
	\begin{gathered}
		L[y(x)]=-f(x) \\
		y(0)=0, \quad y(1)=0
	\end{gathered}
	$$
	Suppose that $\left\{\phi_n\right\}$ is a complete orthonormal basis for the vector space consisting of eigenvectors of $L$ that satisfy the boundary conditions, and that $\lambda_n$ is the eigenvalue of $\phi_n$. We assume the inner product
	$$
	\langle f(x), g(x)\rangle=\displaystyle\int_0^1 f(x) g(x) d x .
	$$
	Then we have
	$$
	f(x)=\displaystyle\sum f_n \phi_n(x), \quad y(x)=\displaystyle\sum y_n \phi_n(x),
	$$
	where
	$$
	f_n=\left\langle f(x), \phi_n(x)\right\rangle=\displaystyle\int_0^1 f(x) \phi_n(x) d x .
	$$
	We must find the $y_n \mathrm{~s}$.
	Since $L$ is a linear operator and $\phi_n(x)$ is an eigenvector of $L$ with eigenvalue $\lambda_n$ for each $n$, we have
	$$
	L[y(x)]=L\left(\displaystyle\sum y_n \phi_n(x)\right)=\displaystyle\sum y_n \lambda_n \phi_n(x) .
	$$
	So, from $L[y(x)]=-f(x)$, we get
	$$
	\displaystyle\sum y_n \lambda_n \phi_n(x)=-\displaystyle\sum f_n \phi_n(x),
	$$
	and, since $\left\{\phi_n(x)\right\}$ is a basis,
	$$
	y_n \lambda_n=-f_n .
	$$
	Now, 0 is not an eigenvalue, so $y_n=-\frac{f_n}{\lambda_n}$, and thus
	$$
	y(x)=\displaystyle\sum y_n \phi_n(x)=-\displaystyle\sum \frac{f_n}{\lambda_n} \phi_n(x) .
	$$
	We now want to find $G(x, t)$ so that
	$$
	y(x)=\displaystyle\int_0^1 G(x, t) f(t) d t ;
	$$
	that is, we want to find the Green's function.
	Recall
	$$
	f_n=\displaystyle\int_0^1 f(t) \phi_n(t) d t
	$$
	so
	$$
	\begin{aligned}
		y(x) & =-\displaystyle\sum \frac{\phi_n(x)}{\lambda_n} f_n=-\displaystyle\sum \frac{\phi_n(x)}{\lambda_n} \displaystyle\int_0^1 f(t) \phi_n(t) d t=-\displaystyle\sum \displaystyle\int_0^1 \frac{\phi_n(x) \phi_n(t)}{\lambda_n} f(t) d t \\
		& =-\displaystyle\int_0^1\left(\displaystyle\sum \frac{\phi_n(x) \phi_n(t)}{\lambda_n}\right) f(t) d t
	\end{aligned}
	$$
	where we have assumed moving the summation inside the integral is legitimate. Thus,
	\begin{equation}\label{eq11}
		G(x, t)=-\displaystyle\sum \frac{\phi_n(x) \phi_n(t)}{\lambda_n} .
	\end{equation}
	Clearly, to find the Green's function using this method, the crucial step is to find the eigenvalues and eigenfunctions for $L$ that satisfy the initial conditions.
	We note that some authors consider the problem
	$$
	\bar{L}[y(x)]+\mu y(x)=\left[p(x) y^{\prime}(x)\right]^{\prime}+\bar{q}(x) y(x)+\mu y(x)=-f(x),
	$$
	where $\mu$ is not an eigenvalue of $\bar{L}$ (but now 0 may be an eigenvalue of $\bar{L}$ ) and obtain the Green's function
	\begin{equation}\label{eq12}
		\bar{G}(x, t)=-\displaystyle\sum \frac{\phi_n(x) \phi_n(t)}{\lambda_n-\mu} .
	\end{equation}
	By adjusting either $q(x)$ or $\bar{q}(x)$ either problem can be changed into the other. The preference would be for the version for which the eigenvalues/ eigenvectors are easier to find.
	
	\Example{Use the eigenfunction expansion to find the Green's}{Use the eigenfunction expansion to find the Green's function for
		$$
		y^{\prime \prime}(x)+y(x)=-f(x), \quad y(0)=0, \quad y(1)=0
		$$}
	
	\begin{demo}
		We consider the second form of the problem with $L[y(x)]=y^{\prime \prime}(x)$ and $\mu=1$.
		The eigenvalues and eigenfunctions for are
		$$
		\begin{gathered}
			L[y(x)]=y^{\prime \prime}(\mathrm{x}) \\
			y(x)=\sin (a x), \quad y(x)=\cos (a x),
		\end{gathered}
		$$
		each with eigenvalue $-a^2$.
		We now find the eigenvalues for which the eigenfunctions satisfy the initial conditions. Suppose
		$$
		y(x)=A \sin (a x)+B \cos (a x)
		$$
		and $y(0)=0$. Then $B=0$. If $y(1)=0$, then
		$$
		y(1)=0=A \sin \alpha
		$$
		so $\alpha=n \pi$ where $n$ is an integer. (Otherwise, we have only the trivial solution.) The eigenvalue of $L$ for the eigenfunction $\psi_n(x)=\sin (n \pi x)$ is $-(n \pi)^2$. Since
		$$
		\int_0^1 \sin ^2(n \pi x) d x=\frac{1}{2}, \quad \text { then }\left\|\psi_n(x)\right\|=\frac{1}{\sqrt{2}} .
		$$
		Thus, $\{\sqrt{2} \sin (n \pi x)\}=\left\{\phi_n(x)\right\}$ is an orthonormal set of eigenfunctions for $L$ that satisfy the initial conditions. So, according to equation (\ref{eq12}), the Green's function for this example is
		$$
		G(x, t)=-\displaystyle\sum \frac{\phi_n(x) \phi_n(t)}{\lambda_n-\mu}=-\displaystyle\sum \frac{[\sqrt{2} \sin (n \pi x)][\sqrt{2} \sin (n \pi t)]}{-(n \pi)^2-1} .
		$$
	\end{demo}
	\section{Problemas con valores en la frontera}
	Hasta ahora, nos hemos concentrado s\'olo en problemas de valor inicial, en los que para una ED dada las condiciones suplementarias sobre la funci\'on desconocida y sus derivadas se prescriben en un valor fijo $x_0$ de la variable independiente $x$. Sin embargo, hay una variedad de otras condiciones posibles que son importantes en las aplicaciones. En muchos problemas pr\'acticos, los requisitos adicionales se dan en forma de condiciones de contorno: la funci\'on desconocida y algunas de sus derivadas se fijan en m\'as de un valor de la variable independiente $x$. La ED junto con las condiciones de contorno se conoce como un problema de valor de contorno. En esta secci\'on proporcionaremos una condici\'on necesaria y suficiente para que un problema de valor de contorno dado tenga una soluci\'on \'unica. Antes de tratar la condici\'on de suficiencia y necesidad vamos a tratar la definici\'on de un problema de valor de contorno.
	\Definition{Problema con valores en la frontera}{ Si en la ecuaci\'on diferencial \ref{EDON} consideramos el caso de la ecuaci\'on lineal de segundo orden
		\begin{equation}\label{ED2O}
			a_{2}(x)y^{\prime \prime}(x)+a_{1}(x)y^{\prime}(x)+a_{0}(x)y(x)=g(x)
		\end{equation} donde las funciones $a_{2}(x), a_{1}(x), a_{0}(x)$ y $g(x)$ son continuas en un intervalo $I=[\alpha, \beta]$. La ED (\ref{ED2O}) conjuntamente con las condiciones de frontera de la forma
		\begin{equation}\label{CondFrontera}
			\begin{aligned}
				& \ell_1[y]=a_0 y(\alpha)+a_1 y^{\prime}(\alpha)+b_0 y(\beta)+b_1 y^{\prime}(\beta)=A \\
				& \ell_2[y]=c_0 y(\alpha)+c_1 y^{\prime}(\alpha)+d_0 y(\beta)+d_1 y^{\prime}(\beta)=B
			\end{aligned}
		\end{equation} con $a_i,\, b_i,\, c_i,\, d_i,\, i=0,1$ y $A, B$ son constantes conocidas se denomina \textbf{problema de valor l\'imite lineal de dos puntos no homog\'eneo.}}
	Para el caso de que $g(x)=0$ y $A=B=0$ tenemos la expresi\'on
	\begin{equation}\label{ED2O}
		a_{2}(x)y^{\prime \prime}(x)+a_{1}(x)y^{\prime}(x)+a_{0}(x)y(x)=0
	\end{equation}
	\begin{equation}\label{CondFrontera0}
		\begin{aligned}
			& \ell_1[y]=0 \\
			& \ell_2[y]=0
		\end{aligned}
	\end{equation}
	que se llama \textbf{problema de valor l\'imite lineal de dos puntos homog\'eneo}.\\
	Las condiciones de frontera (\ref{CondFrontera}) son bastante generales y en particular incluyen los casos
	\begin{itemize}
		\item Si en (\ref{CondFrontera}) tenemos que $a_{0}=d_{0}=1,\;a_{1}=b_{0}=b_{1}=c_{0}=c_{1}=d_{1}$ tenemos la condici\'on de frontera que se llama \textbf{condici\'on de Dirichlet}
		\begin{equation}\label{Condirichlet}
			y(\alpha)=A, \quad y(\beta)=B
		\end{equation}
		\item Si en (\ref{CondFrontera}) tenemos que $a_{1}=d_{0}=1,\;a_{0}=b_{0}=b_{1}=c_{0}=c_{1}=d_{1}$ tenemos la condici\'on de frontera que se llama \textbf{condici\'on de Neumann}
		\begin{equation}\label{Condirichlet}
			y^{\prime}(\alpha)=A, \quad y^{\prime}(\beta)=B
		\end{equation}
		\item Si en (\ref{CondFrontera}) tenemos que $a_{0}=b_{1}=c_{1}=d_{1}=1$ con $a_{1}=b_{0}=c_{0}=d_{0}=0$  tenemos la  condici\'on de frontera que se llama \textbf{condici\'on de Cauchy}  \begin{equation}\label{CondCauchy}
			\begin{aligned}
				&y(\alpha)=A, \quad y^{\prime}(\beta)=B\\
				&y^{\prime}(\alpha)=A, \quad y(\beta)=B
			\end{aligned}
		\end{equation}
		\item Si en (\ref{CondFrontera}) tenemos que $b_{0}=b_{1}=c_{0}=c_{1}=0$  tenemos la condici\'on de frontera que se llama \textbf{condici\'on de Sturm-Liouville } 
		\begin{equation}\label{Condliouville}
			\begin{aligned}
				& a_0 y(\alpha)+a_1 y^{\prime}(\alpha)=A \\
				& d_0 y(\beta)+d_1 y^{\prime}(\beta)=B
			\end{aligned}
		\end{equation} con $a_0^2+a_1^2 \neq 0\; \text {y}\; d_0^2+d_1^2 \neq 0$.
		\item Si en (\ref{CondFrontera}) tenemos que $a_{0}=c_{1}=1$, $b_{0}=d_{1}=-1,$ con $a_{1}=b_{1}=c_{0}=c_{1}=0$ y $A=B=0$ tenemos la condici\'on de frontera que se llama \textbf{condici\'on de frontera peri\'odica} 
		\begin{equation}\label{Condliouville}
			y(\alpha)=y(\beta), \quad y^{\prime}(\alpha)=y^{\prime}(\beta)
		\end{equation}
	\end{itemize}
	El problema  de valores en la frontera (\ref{ED2O}) , (\ref{CondFrontera}) se llama \textbf{regular} si $\alpha$ como $\beta$ son valores finito, y la funci\'on $a_{2}(x)\neq 0$ para toda $x \in I$. Si $\alpha=-\infty$ y/o $\beta=\infty$ y/o $a_{2}(x)=0$ para al menos un valor de $x \in I$, entonces el problema  (\ref{ED2O}) , (\ref{CondFrontera}) se llama \textbf{singular}.\\
	Una soluci\'on del problema de valores en la frontera (\ref{ED2O}) , (\ref{CondFrontera}), es una soluci\'on de ED (\ref{ED2O}) que satisface las condiciones de fronteras dada en (\ref{CondFrontera}).\\
	La teoria de la existencia y unicidad de los problemas de valores en la frontera es m\'as exigente que la teor\'ia de los problemas de valores iniciales, en los siguientes ejemplos analizamos la existencia y unicidad de una soluci\'on de cierto problemas de valores iniciales y como cambia el comportamiento de la soluci\'on al resolver la misma ED con condiciones de frontera y viceversa.
	\Example{ Problema de valores iniciales vs problema de valores en la frontera }{ Analiza la soluci\'on de la ED dada por 
		\begin{equation}\label{esim}
			y^{\prime \prime}+y=0
		\end{equation} sujeta a las condiciones
		\begin{enumerate}
			\item \begin{equation}\label{con1}
				y\left(0\right)=w_{1}\qquad y^{\prime}\left(0\right)=w_{2}\qquad w_{1},w_{2}\in \mathrm{R}
			\end{equation}
			\item \begin{equation}\label{con2}
				y\left(0\right)=0\qquad y\left(\pi\right)=B,\quad B\neq 0
			\end{equation}
	\end{enumerate} }
	\begin{sol}
		La ecuaci\'on diferencial (\ref{esim}) es un caso particular de la ecuaci\'on diferencial (\ref{ek}) y cuya soluci\'on general est\'a dada por (\ref{solk}), de manera que la soluci\'on de la ecuaci\'on (\ref{esim}) se obtiene de (\ref{solk}) para $k=1$, as\'i
		\begin{equation}\label{sole}
			y=c_1 \cos \left(x\right)+c_2\sin\left(x\right)
		\end{equation}
		Es la soluci\'on de (\ref{esim}).
		\begin{enumerate}
			\item Ahora aplicando las condiciones iniciales tenemos
			\begin{equation*}
				y\left(0\right)=c_1 \cos \left(0\right)+c_2\sin\left(0\right)=w_{1}
			\end{equation*}
			lo que implica que $c_1=w_{1}$, de manera que
			\begin{equation*}
				y=w_{1}\cos \left(x\right)+c_2\sin\left(x\right)
			\end{equation*}
			Para aplicar la segunda condici\'on derivamos la expresi\'on anterior
			\begin{equation*}
				y=-w_{1}\sin \left(x\right)+c_2\cos\left(x\right)
			\end{equation*} Evaluando la segunda condici\'on
			\begin{equation*}
				y^{\prime}\left(0\right)=-w_{1}\sin \left(0\right)c_2\cos\left(0\right)=w_{2}
			\end{equation*} De donde $c_{2}=w_{2}$, por lo tanto la soluci\'on al problema esta dado por
			\begin{equation*}
				y=w_{1}\cos \left(x\right)+w_2\sin\left(x\right)
			\end{equation*}
			\item Evaluamos las condiciones de frontera en la soluci\'on de (\ref{esim}), teniendo 
			\begin{equation*}
				y\left(0\right)= c_1 \cos \left(0\right)+c_2\sin\left(0\right)=0
			\end{equation*} Implicando que $c_{1}=0$ quedandonos asi la soluci\'on de la forma
			\begin{equation*}
				y=c_2\sin\left(x\right)
			\end{equation*} Ahora aplicamos la segunda condici\'on, evaluando tenemos
			\begin{equation*}
				y\left(0\right)=c_2\sin\left(0\right)=B
			\end{equation*} Lo cual nos llega a una contradicci\'on ya que $B\neq 0$, por lo que no existe una soluci\'on para el problema (\ref{esim}),(\ref{con2}).
		\end{enumerate}
		En el ejercicio anterior hemos visto que las condiciones iniciales y/o de frontera juegan un papel fundamental para la existencia de una soluci\'on de un problema de valores iniciales o de frontera. A continuaci\'on vamos analizar otro ejercicio con igual naturaleza.
	\end{sol}
	\Example{ Soluci\'on \'unica vs Infinitas soluciones }{ Analiza la soluci\'on de la ED dada por 
		\begin{equation*}
			y^{\prime \prime}+y=0
		\end{equation*} sujeta a las condiciones
		\begin{enumerate}
			\item $y\left(0\right)=0\qquad y\left(\beta\right)=B\quad 0<\beta<\pi$
			\item $y\left(0\right)=0\qquad y\left(\pi\right)=0$
	\end{enumerate} }
	\begin{sol}
		\begin{enumerate}
			Sabemos que la soluci\'on de la ED est\'a dada por la expresi\'on (\ref{sole})
			\item Aplicando las condiciones tenemos
			\begin{equation*}
				y\left(0\right)=c_1 \cos \left(0\right)+c_2\sin\left(0\right)=0
			\end{equation*} Lo que implica que $c_{1}=0$, qued\'andonos as\'i la soluci\'on de la forma
			\begin{equation*}
				y=c_2\sin\left(x\right)
			\end{equation*} Por la segunda condici\'on tenemos
			\begin{equation}\label{isol}
				y\left(\beta\right)=c_2\sin\left(\beta\right)=B
			\end{equation} Despejando a $c_{2}$
			\begin{equation*}
				c_{2}=\displaystyle\frac{B}{\sin\left(\beta\right)}
			\end{equation*} De manera que reemplazando este valor en (\ref{isol}) obtenemos la soluci\'on \'unica para este problema.
			\begin{equation*}
				y=\displaystyle\frac{B}{\sin\left(\beta\right)}\sin\left(x\right)
			\end{equation*}
			\item De la primera condici\'on tenemos 
			\begin{equation*}
				y=c_2\sin\left(x\right)
			\end{equation*} Aplicando la segunda condici\'on
			\begin{equation*}
				y\left(\pi\right)=c_2\sin\left(\pi\right)=0
			\end{equation*} Implica que $c_{2}$ puede ser cualquier n\'umero real, obteniendo as\'i infinitas soluciones de la forma
			\begin{equation*}
				y=c_2\sin\left(x\right) \qquad c_{2}\in\mathcal{R}
			\end{equation*}
		\end{enumerate}
	\end{sol}
	Es claro que para el problema (\ref{ED2O}), (\ref{CondFrontera0}) la funci\'on constante $y=0$ (soluci\'on trivial) es siempre una soluci\'on. Sin embargo, con los ejemplos anteriores hemos visto que adem\'as de la soluci\'on trivial el problema con valores en la frontera puede tener soluciones no triviales. Esto nos lleva a cuestionarnos bajo qu\'e condiciones un problema con valores en la frontera tiene soluci\'on \'unica. El siguiente teorema muestra las condiciones suficientes y necesarias para que el problema (\ref{ED2O}), (\ref{CondFrontera0}) tenga \'unicamente la soluci\'on trivial.
	\Theorem{}{ Sean $y_{1}(x)$ y $y_{2}(x)$ dos soluciones linealmente independientes de la ED homog\'enea (\ref{ED2O}). Entonces el problema con valores en la frontera (\ref{ED2O}), (\ref{CondFrontera0}) tiene la soluci\'on trivial si y solo si
		\begin{equation}\label{sol0cond}
			\Delta=\left|\begin{array}{ll}
				\ell_1\left[y_1\right] & \ell_1\left[y_2\right] \\
				\ell_2\left[y_1\right] & \ell_2\left[y_2\right]
			\end{array}\right| \neq 0
	\end{equation}}\label{thm:teoHM}
	\begin{demo}
		Por hip\'otesis $y_{1}(x)$ y $y_{2}(x)$ son soluciones linealmente independiente de (\ref{ED2O}), entonces la soluci\'on general est\'a dada por 
		\begin{equation*}
			y(x)=c_1 y_1(x)+c_2 y_2(x)
		\end{equation*} Esta funci\'on es soluci\'on del problema (\ref{ED2O}), (\ref{CondFrontera0}) si y solo si satisface las condiciones de fronteras, es decir,
		\begin{equation*}
			\begin{aligned}
				\ell_1\left[c_1 y_1+c_2 y_2\right] & =c_1 \ell_1\left[y_1\right]+c_2 \ell_1\left[y_2\right]=0 \\
				\ell_2\left[c_1 y_1+c_2 y_2\right] & =c_1 \ell_2\left[y_1\right]+c_2 \ell_2\left[y_2\right]=0
			\end{aligned}
		\end{equation*}
		Este es un sistema de dos ecuaciones con dos variables y de algebra lineal sabemos que tiene soluci\'on \'unica si $\Delta\neq 0$.
	\end{demo}
	Como consecuencia inmediata del teorema (\ref{teoHM}) se presenta el siguiente corolario.
	\Corollary{Colorario}{ El problema de valores en la frontera (\ref{ED2O}), (\ref{CondFrontera0}) tienes un n\'umero infinito de soluciones no triviales si y solo si $\Delta=0$}
	\Definition{Problema de Sturm-Liouville}{Un problema con valores en la frontera que consiste de una ecuaci\'on diferencial
		\begin{eqnarray}\label{problemaLiouv}
			\left(p(x) y^{\prime}\right)^{\prime}+q(x) y+\lambda r(x) y=p(y)+\lambda r(x) y&=&0
		\end{eqnarray}
		tal que $x \in J=[a, b]$
		con las condiciones de frontera
		
		\begin{eqnarray}\label{condiproblemaLiouv}
			a_0 y(a)+a_1 y^{\prime}(a)&=&0 \quad \quad a_0^2+a_1^2>0 \\
			d_0 y(b)+d_1 y^{\prime}(b)&=&0 \quad d_0^2+d_1^2>0
		\end{eqnarray}Se llama problema de \textbf{Sturm-Liouville}}
	
	En la ED \ref{problemaLiouv}, $\lambda$ es un par\'ametro, y las funciones $q, r \in C(J), p \in C^{\prime}(J)$ y $p(x)>0, r(x)>0 \quad$ en $J$.\\
	El Problema \ref{problemaLiouv} conjunto a las condiciones \ref{condiproblemaLiouv} satisfaciendo las condiciones especificadas anteriormente se dice que es un problema \textbf{Regular de Sturm- Liouville}.\\
	Claramente, si tomamos $y(x) \equiv 0$ siempre cumplir\'a el problema dado \ref{problemaLiouv} conjuntamente con las condiciones \ref{condiproblemaLiouv}. Resolver un problema como el dado en \ref{problemaLiouv} consiste en encontrar el valor de $\lambda$ llamado \textbf{valor propio} y la soluci\'on correspondiente $\phi_\lambda(x)$ denominada \textbf{funci\'on propia}.\\
	
	El conjunto de todo los valores propios de un Problema regular se llama espectro.
	
	\Example{Determina los valores y vectores propios}{	Determina los valores y vectores propios del problema con valores en la frontera
		\begin{eqnarray}\label{prob1}
			y^{\prime \prime}(x)+\lambda y(x)&=&0
		\end{eqnarray}
		sujeto a las condiciones
		\begin{eqnarray}\label{cond1}
			y(0)=y(\pi)=0
	\end{eqnarray}}
	
	
	\begin{sol}
		Para resolver este ejercicio vamos a suponer dos casos:
		\begin{enumerate}
			\item $\lambda =0$\\
			Si $\lambda=0$, la ecuaci\'on se reduce a $y^{\prime \prime}(x)=0 $ y la soluci\'on general es
			\begin{eqnarray}\label{solgnal0}
				% \nonumber % Remove numbering (before each equation)
				y(x)&=&c_1+c_2 x
			\end{eqnarray}
			Ahora aplicando las condiciones
			$$
			y(0)=c_1=0
			$$
			y
			$$y(\pi)=\pi c_2=0$$ lo que implica que $c_2=0$.\\
			Lo que podemos notar que obtenemos la funcion $y(x)\equiv0$ es la solucion trivial y $\lambda=0$ no es un valor propio del problema \ref{prob1} y \ref{cond1}.
			\item $\lambda \neq0$\\
			Ahora supondremos $\lambda \neq 0$, y por conveniencia tomaremos  $\lambda=n^2$, donde $n$ puede no ser un numero real. Asi, la ecuaci\'on \ref{prob1} se transforma en
			\begin{eqnarray*}
				% \nonumber % Remove numbering (before each equation)
				y^{\prime \prime}(x)+n^2 y(x)&=&0
			\end{eqnarray*}
			cuya soluci\'on general es
			\begin{eqnarray}\label{solgnal01}
				y(x)&=&c_1 e^{i n x}+c_2 e^{-i n x}
			\end{eqnarray}
			y en t\'ermino de seno y coseno es
			\begin{eqnarray}\label{solgnal11}
				y(x)&=&\left(c_1+c_2\right) \cos (n x)+\left(c_1-c_2\right) i \sin (n x)
			\end{eqnarray}
			Sustituyendo las condiciones de frontera obtenemos el sistema de ecuaci\'on
			\begin{eqnarray}\label{sis1}
				% \nonumber % Remove numbering (before each equation)
				\left\{\begin{array}{l}
					c_1+c_2=0 \\
					c_1 e^{i n \pi}+c_2 e^{-i n \pi}=0
				\end{array}\right.
			\end{eqnarray}
		\end{enumerate}
		El sistema anterior tiene soluci\'on no trivial, si y solo si, el determinante del sistema es diferente de cero, es decir,
		$$
		\left|\begin{array}{cc}
			1 & 1 \\
			e^{i n \pi} & e^{-i n \pi}
		\end{array}\right|=e^{-i n \pi}-e^{i n \pi}=0
		$$
		Ahora sea $n=k+i t$, donde $k, t \in \mathbb{R}$
		$$
		\begin{aligned}
			& e^{-i \pi(k+i t)}-e^{i \pi(k+i t)}=0 \\
			& e^{(t \pi-k \pi i)}-e^{(-t \pi+k \pi i)}=0
		\end{aligned}
		$$
		Por la identidad de Euler
		\begin{eqnarray*}
			% \nonumber % Remove numbering (before each equation)
			e^{t \pi}[\cos (k \pi)-i\sen(k \pi)]-e^{-t \pi}[\cos (k \pi)+i\sen(k \pi)]&=&0
		\end{eqnarray*}
		Simplificando
		$$
		\left(e^{t \pi}-e^{-t \pi}\right) \cos (k \pi)-i\left(e^{t \pi}-e^{-t \pi}\right)\sen(k \pi)
		$$
		utilizando las identidades del seno y coseno hiperb\'olico tenemos
		$$
		2 \sinh (t \pi) \cos (k \pi)-2 i \cosh (t \pi)\sen(k \pi)=0
		$$
		lo que implica que
		\begin{eqnarray}\label{Aa}
			\sinh (t \pi) \cos (k \pi)=0
		\end{eqnarray}
		y
		\begin{eqnarray}\label{Ba}
			\cosh (t \pi) \operatorname{sen}(k \pi)=0
		\end{eqnarray}
		Como $\cosh (t\pi)>0\; \forall t$, la ecuaci\'on \ref{Ba} requiere que
		$$
		\operatorname{sen}(k \pi)=0
		$$
		lo que implica que $k=m \in Z$. Para esta elecci\'on de $k$, la ecuaci\'on \ref{Aa} se reduce a $$\sinh (t \pi)=0$$ por lo que $t=0$.\\
		Con este valor para $t$, $k$ debe ser distinto de cero, ya que queremos la soluci\'on no trivial,por lo tanto, $n=k$. De manera que el valor propio es
		\begin{eqnarray*}
			% \nonumber % Remove numbering (before each equation)
			\lambda_k&=&n^2=k^2\quad k=1,2,3, \ldots
		\end{eqnarray*}
		Las funciones propias correspondiente a $\lambda_k$
		son  \begin{eqnarray*}
			\phi_k(x) & =&c_1 e^{i k x}-c_1 e^{-i k x} \\
			\phi_k(x) & =&c_1\left(e^{i k x}-e^{-i k x}\right) \\
			\phi_k(x) & =&2 i c_1 \sen(k x)
		\end{eqnarray*}
		o simplemente
		\begin{eqnarray*}
			\phi_k(x)&=&\sen(k x)
		\end{eqnarray*}
	\end{sol}
	
	\Example{Determina los valores y vectores propios del problema \ref{prob1} con las condiciones}{Determina los valores y vectores propios del problema \ref{prob1} con las condiciones
		\begin{eqnarray}\label{cond,2}
			y(0)+y^{\prime}(0)=0, \quad y(1)=0
		\end{eqnarray}
	}
	
	
	
	\begin{sol}
		\begin{enumerate}
			\item Si $\lambda=0$\\
			Para $\lambda=0$, la soluci\'on general est\'a dada por
			\begin{eqnarray*}
				% \nonumber % Remove numbering (before each equation)
				y(x)&=&c_1+c_2 x
			\end{eqnarray*}
			Las condiciones de frontera implican
			\begin{eqnarray*}
				c_1+c_2=0
			\end{eqnarray*}
			es decir $c_1=-c_2$.\\
			Por lo tanto $\lambda=0$ es un valor propio y la correspondiente funci\'on propia es
			\begin{eqnarray*}
				% \nonumber % Remove numbering (before each equation)
				\phi_0(x)&=&1-x
			\end{eqnarray*}
			\item Si $\lambda\neq0$\\
			si $x \neq 0$, y reemplazamos de nuevo $\lambda=n^2$, tenemos que
			\begin{eqnarray*}
				y(x)&=&c_1 e^{i n x}+c_2 e^{-i n x}
			\end{eqnarray*}
			por las condiciones de frontera tenemos
			$$
			\left\{\begin{array}{c}
				\left(c_1+c_2\right)+i n\left(c_1-c_2\right)=0 \\
				c_1 e^{i n}+c_2 e^{-i n}=0
			\end{array}\right.
			$$
			El sistema anterior tiene soluci\'on no trivial, si y solo si,
			$$\left|\begin{array}{cc}
				1+i n & 1-i n \\
				e^{i n} & e^{-i n}
			\end{array}\right|=(1+i n) e^{-i n}-(1-i n) e^{i n}=0$$
			Desarrollando tenemos
			\begin{eqnarray*}
				e^{-i n}+i n e^{-i n}-e^{i n}+i n e^{i n}&=&0
			\end{eqnarray*}
			Por factor com\'un
			\begin{eqnarray*}
				-\left(e^{i n}-e^{-i n}\right)+n i\left(e^{i n}+e^{-i n}\right)&=&0
			\end{eqnarray*}
			por la identidad de seno y coseno hiperb\'olico
			\begin{eqnarray*}
				-i \sen(n)+i \cos (n)&=&0
			\end{eqnarray*}
			lo que es equivalente a
			\begin{eqnarray}\label{n}
				% \nonumber % Remove numbering (before each equation)
				\tan(n) &=& n
			\end{eqnarray}
			para encontrar las soluciones de la ecuaci\'on anterior, graficaremos ambas funciones de manera separada, es decir, $y=n$ y $y=\tan (n)$.\\
			\textcolor{red}{graficar}\\
			A partir de la gr\'afica, es claro que la ecuaci\'on \ref{n} tiene un infinito n\'umeros de ra\'ices positiva de la forma
			\begin{eqnarray*}
				n_k\simeq(2 k+1) \frac{\pi}{2}
			\end{eqnarray*} como en la ecuaci\'on \ref{n} podemos reemplazar $n$ por $-n$, tenemos que
			\begin{eqnarray*}
				n_k \simeq\pm(2 k+1) \frac{\pi}{2}
			\end{eqnarray*}
			por lo tanto, el problema tiene infinitos valores propios, $\lambda_0=0, \lambda_k=(2 k+1)^2 \displaystyle\frac{\pi^2}{4}, k=1,2, \ldots$ y sus funciones propias son
			$$
			\begin{aligned}
				& \phi_0(x)=1-x \\
				& \phi_k(x)=\operatorname{sen}\left(\sqrt{\lambda_k}(1-x)\right), k=1,2, \ldots
			\end{aligned}
			$$
		\end{enumerate}
	\end{sol}
	
	\Theorem{Los valores propios del problema de Sturm-Liouville}{Los valores propios del problema de Sturm-Liouville dado en (\ref{problemaLiouv}) , (\ref{condiproblemaLiouv}) son simples, es decir, si $\lambda$ es un valor propio de (\ref{problemaLiouv}),(\ref{condiproblemaLiouv}) y $\phi_1(x)$,$\phi_2(x)$ son las correspondientes funciones propias, entonces $\phi_1(x)$, $\phi_2(x)$ son linealmente independientes.}
	
	
	\begin{demo}
		Como $\phi_1(x)$ y $\phi_2(x)$ son soluciones de (\ref{problemaLiouv}), tenemos
		\begin{equation}\label{eq21}
			\left(p(x) \phi_1^{\prime}\right)^{\prime}+q(x) \phi_1+\lambda r(x) \phi_1=0
		\end{equation}
		y
		\begin{equation}\label{eq22}
			\left(p(x) \phi_2^{\prime}\right)^{\prime}+q(x) \phi_2+\lambda r(x) \phi_2=0 .
		\end{equation}
		Multiplicando (\ref{eq21}) por $\phi_2(x)$, y (\ref{eq22}) por $\phi_1(x)$ y restando, tenemos
		\begin{equation}\label{eq23}
			\phi_2\left(p(x) \phi_1^{\prime}\right)^{\prime}-\left(p(x) \phi_2^{\prime}\right)^{\prime} \phi_1=0
		\end{equation}
		Sin embargo, desde
		$$
		\begin{aligned}
			{\left[\phi_2\left(p(x) \phi_1^{\prime}\right)-\left(p(x) \phi_2^{\prime}\right) \phi_1\right]^{\prime}} & =\phi_2\left(p(x) \phi_1^{\prime}\right)^{\prime}+\phi_2^{\prime}\left(p(x) \phi_1^{\prime}\right)-\left(p(x) \phi_2^{\prime}\right)^{\prime} \phi_1-\left(p(x) \phi_2^{\prime}\right) \phi_1^{\prime} \\
			& \quad=\phi_2\left(p(x) \phi_1^{\prime}\right)^{\prime}-\left(p(x) \phi_2^{\prime}\right)^{\prime} \phi_1
		\end{aligned}
		$$
		de la expresi\'on (\ref{eq23}) se sigue que
		$$
		\left[\phi_2\left(p(x) \phi_1^{\prime}\right)-\left(p(x) \phi_2^{\prime}\right) \phi_1\right]^{\prime}=0
		$$
		y por lo tanto
		\begin{equation}\label{eq24}
			p(x)\left[\phi_2 \phi_1^{\prime}-\phi_2^{\prime} \phi_1\right]=\mathrm{constant}=C
		\end{equation}
		para encontrar el valor de $C$, notemos que $\phi_1$ y $\phi_2$ satisfacen las condiciones de frontera, y por lo tanto
		$$
		\begin{aligned}
			& a_0 \phi_1(\alpha)+a_1 \phi_1^{\prime}(\alpha)=0 \\
			& a_0 \phi_2(\alpha)+a_1 \phi_2^{\prime}(\alpha)=0
		\end{aligned}
		$$
		lo cual implica que $\phi_1(\alpha) \phi_2^{\prime}(\alpha)-\phi_2(\alpha) \phi_1^{\prime}(\alpha)=0$. Por lo tanto, de la expresi\'on (\ref{eq24}) se sigue que
		$$
		p(x)\left[\phi_2 \phi_1^{\prime}-\phi_2^{\prime} \phi_1\right]=0 \quad \text { for all } \quad x \in[a, b]
		$$
		Como $p(x)>0$, tenemos $\phi_2 \phi_1^{\prime}-\phi_2^{\prime} \phi_1=0$ para todo $x \in[a, b]$. Pero, esto significa que  $\phi_1$ and $\phi_2$ son linealmente dependientes. $\Box$
	\end{demo}
	
	\Theorem{Sturm-Liouville}{	Let $\lambda_n, n=1,2, \cdots$ be the eigenvalues of the regular Sturm-Liouville problem (\ref{eq13}), (\ref{eq14}) and $\phi_n(x), n=1,2, \cdots$ be the corresponding eigenfunctions. Then, the set $\left\{\phi_n(x): n=1,2, \cdots\right\}$ is orthogonal in $[\alpha, \beta]$ with respect to the weight function $r(x)$.}
	
	\begin{demo}
		Let $\lambda_k$ and $\lambda_{\ell},(k \neq \ell)$ be eigenvalues, and $\phi_k(x)$ and $\phi_{\ell}(x)$ be the corresponding eigenfunctions of (\ref{eq13}), (\ref{eq14}). Since $\phi_k(x)$ and $\phi_{\ell}(x)$ are solutions of (\ref{eq13}), we have
		\begin{equation}\label{eq25}
			\left(p(x) \phi_k^{\prime}\right)^{\prime}+q(x) \phi_k+\lambda_k r(x) \phi_k=0
		\end{equation}
		and
		\begin{equation}\label{eq26}
			\left(p(x) \phi_{\ell}^{\prime}\right)^{\prime}+q(x) \phi_{\ell}+\lambda_{\ell} r(x) \phi_{\ell}=0 .
		\end{equation}
		Now following the argument in Theorem-1, we get
		$$
		\left[\phi_{\ell}\left(p(x) \phi_k^{\prime}\right)-\left(p(x) \phi_{\ell}^{\prime}\right) \phi_k\right]^{\prime}+\left(\lambda_k-\lambda_{\ell}\right) r(x) \phi_k \phi_{\ell}=0,
		$$
		which on integration gives
		\begin{equation}\label{eq27}
			\left(\lambda_{\ell}-\lambda_k\right) \displaystyle\int_\alpha^\beta r(x) \phi_k(x) \phi_{\ell}(x) d x=\left.p(x)\left[\phi_{\ell}(x) \phi_k^{\prime}(x)-\phi_{\ell}^{\prime}(x) \phi_k(x)\right]\right|_\alpha ^\beta .
		\end{equation}
		Next since $\phi_k(x)$ and $\phi_{\ell}(x)$ satisfy the boundary conditions (\ref{eq14}), i.e.,
		$$
		\begin{array}{ll}
			a_0 \phi_k(\alpha)+a_1 \phi_k^{\prime}(\alpha)=0, & d_0 \phi_k(\beta)+d_1 \phi_k^{\prime}(\beta)=0 \\
			a_0 \phi_{\ell}(\alpha)+a_1 \phi_{\ell}^{\prime}(\alpha)=0, & d_0 \phi_{\ell}(\beta)+d_1 \phi_{\ell}^{\prime}(\beta)=0
		\end{array}
		$$
		it is necessary that
		$$
		\phi_k(\alpha) \phi_{\ell}^{\prime}(\alpha)-\phi_k^{\prime}(\alpha) \phi_{\ell}(\alpha)=\phi_k(\beta) \phi_{\ell}^{\prime}(\beta)-\phi_k^{\prime}(\beta) \phi_{\ell}(\beta)=0 .
		$$
		Hence, the identity (\ref{eq27}) reduces to
		\begin{equation}\label{eq28}
			\left(\lambda_{\ell}-\lambda_k\right) \displaystyle\int_\alpha^\beta r(x) \phi_k(x) \phi_{\ell}(x) d x=0 .
		\end{equation}
		
		However, since $\lambda_{\ell} \neq \lambda_k$, it follows that $\displaystyle\int_\alpha^\beta r(x) \phi_k(x) \phi_{\ell}(x) d x=0$.\quad $\Box$\\
	\end{demo}
	
	\Corollary{Colorario}{ Let $\lambda_1$ and $\lambda_2$ be two eigenvalues of the regular Sturm-Liouville problem (\ref{eq13}), (\ref{eq14}) and $\phi_1(x)$ and $\phi_2(x)$ be the corresponding eigenfunctions. Then, $\phi_1(x)$ and $\phi_2(x)$ are linearly dependent if and only if $\lambda_1=\lambda_2$.\\}
	
	
	\begin{demo}
		The proof is a direct consequence of equality (\ref{eq28}).
		
	\end{demo}
	
	\Theorem{For the regular Sturm-Liouville}{	For the regular Sturm-Liouville problem (\ref{eq13}), (\ref{eq14}) the eigenvalues are real.}
	
	
	\begin{demo}
		Let $\lambda=a+i b$ be a complex eigenvalue and $\phi(x)=\mu(x)+i \nu(x)$ be the corresponding eigenfunction. Then, we have
		$$
		\left(p(x)(\mu+i \nu)^{\prime}\right)^{\prime}+q(x)(\mu+i \nu)+(a+i b) r(x)(\mu+i \nu)=0
		$$
		and hence
		$$
		\left(p(x) \mu^{\prime}\right)^{\prime}+q(x) \mu+(a \mu-b \nu) r(x)=0
		$$
		and
		$$
		\left(p(x) \nu^{\prime}\right)^{\prime}+q(x) \nu+(b \mu+a \nu) r(x)=0 .
		$$
		Now following exactly the same argument as in  Theorem-1, we get
		$$
		\begin{aligned}
			0=\left.p(x)\left(\nu \mu^{\prime}-\nu^{\prime} \mu\right)\right|_\alpha ^\beta & =\displaystyle\int_\alpha^\beta[-(a \mu-b \nu) \nu r(x)+(b \mu+a \nu) \mu r(x)] d x \\
			& =b \displaystyle\int_\alpha^\beta r(x)\left(\nu^2(x)+\mu^2(x)\right) d x .
		\end{aligned}
		$$
		Hence, it is necessary that $b=0$, i.e., $\lambda$ is real.\quad $\Box$
	\end{demo}
	Since (\ref{eq15}), is a regular Sturm-Liouville problem.\\
	
	In the above results we have established several properties of the eigenvalues and eigenfunctions of the regular Sturm-Liouville problem (\ref{eq13}), (\ref{eq14}). In all these results the existence of eigenvalues is tacitly assumed. We now state the following very important result whose proof involves some advanced arguments.
	
	\Theorem{For the regular Sturm-Liouville}{For the regular Sturm-Liouville problem (\ref{eq13}), (\ref{eq14}) there exists an infinite number of eigenvalues $\lambda_n, n=1,2, \cdots$. These eigenvalues can be arranged as a monotonically increasing sequence $\lambda_1<\lambda_2<\cdots$ such that $\lambda_n \rightarrow \infty$ as $n \rightarrow \infty$. Further, eigenfunction $\phi_n(x)$ corresponding to the eigenvalue $\lambda_n$ has exactly $(n-1)$ zeros in the open interval $(\alpha, \beta)$.}
	
	\begin{demo}
		Hacer
	\end{demo}
	The following examples show that the above properties for singular Sturm-Liouville problems do not always hold.\\
	
	\Example{Sturm-Liouville}{Considere el problema singular de Sturm-Liouville dado  (\ref{prob1}), con las condiciones
		\begin{eqnarray}\label{cond3}
			y(0)&=&0, \quad|y(x)| \leq M<\infty \quad \text { for all } x \in(0, \infty)
	\end{eqnarray}}
	
	
	\begin{sol}
		\begin{enumerate}
			\item $\lambda=0$\\
			de la expresi\'on (\ref{solgnal0}) tenemos que la soluci\'on es
			$$
			y(x)=c_1+c_2(x)
			$$
			aplicando la condici\'on
			$$
			y(0)=c_1+c_2(0)\Rightarrow c_1=0
			$$
			por lo tanto
			$$
			y(x)=c_2 x
			$$
			Por la segunda condici\'on
			$$
			\begin{aligned}
				|y(x)| & =&\left|c_2 x\right|<M  \\
				& =&\left|c_2\right||x|<M
			\end{aligned}
			$$
			como $y=x$ no es acotada en $x\in(0,\infty)$, tenemos
			$c_2=0$.\\
			De manera $\lambda=0$ no es un valor propio
			\item $\lambda\neq0$\\
			Para el caso de  $\lambda\neq0$ lo analizaremos para el caso positivo y negativo.\\
			\begin{itemize}
				\item $\lambda>0$\\
				La soluci\'on est\'a dada en la expresi\'on (\ref{solgnal11}) que es
				\begin{eqnarray*}
					% \nonumber % Remove numbering (before each equation)
					y(x)&=&\left(c_1+c_2\right) \cos (n x)+\left(c_1-c_2\right) i \sin (n x)
				\end{eqnarray*}
				Evaluando la primera condici\'on, tenemos
				\begin{eqnarray*}
					y(0)=c_1+c_2=0
				\end{eqnarray*}
				por la segunda condici\'on
				$$
				\begin{aligned}
					|y(x)| & =\left|\left(c_1-c_2\right) i \sin (n x)\right| \leqslant M  \\
					& =\left|c_1-c_2\right||\sin (n x)| \leqslant M \\
					& =c_1-c_2 \leqslant M
				\end{aligned}
				$$
				Por lo tanto $\lambda_n \in(0, \infty)$ son los valores propios y las funciones propias son $\phi_n(x)=\sin \left(\sqrt{\lambda_n} x\right)$.
				\item $\lambda<0$\\
				Si $\lambda<0$, la soluci\'on es
				\begin{eqnarray*}
					y(x)&=&c_1 e^{n x}+c_2 e^{-n x}
				\end{eqnarray*}
				Aplicando las condiciones
				\begin{eqnarray*}
					y(0)&=&c_1+c_2=0 \\
					|y(x)|&=&\left|c_1 e^{n x}+c_2 e^{-n x}\right| \leqslant M
				\end{eqnarray*}
				utilizando la primera ecuaci\'on, tenemos
				\begin{eqnarray*}
					|y(x)| & =&\left|c_1 e^{n x}-c_1 e^{-n x}\right| \leq M
				\end{eqnarray*}
				por factor com\'un
				\begin{eqnarray*}
					|y(x)|&=&\left|c_1\left(e^{n x}-e^{-n x}\right)\right| \leq M
				\end{eqnarray*}
				tomando $c_{1}=\displaystyle\frac{k}{2}$ e utilizando la definici\'on del seno hiperb\'olico, llegamos a
				\begin{eqnarray*}
					|y(x)|&=&k \sinh (n x) \leq M
				\end{eqnarray*}
				Como $\sinh (x)$ es una funci\'on no acotada en $x \in(0, \infty)$, entonces $\lambda<0$ no es un valor propio.
			\end{itemize}
		\end{enumerate}
	\end{sol}
	
	\Example{Sturm-Liouville}{Considere el problema singular de Sturm-Liouville dado (\ref{prob1}),
		\begin{eqnarray}\label{cond4}
			y(-\pi)=y(\pi), \quad y^{\prime}(-\pi)=y^{\prime}(\pi)
	\end{eqnarray}}
	
	
	
	
	
	\begin{sol}
		\begin{enumerate}
			\item $\lambda=0$\\
			Para $\lambda=0$, tenemos
			$$
			y(x)=c_1+c_2 x
			$$
			por la primera condici\'on
			$$
			y(-\pi)=c_1-\pi c_2 \text { y } y(\pi)=c_1+\pi c_2
			$$
			Asi
			$$
			c_1-\pi c_2=c_1+\pi c_2
			$$
			por 10 tanto $c_2=0$
			de tal forma que la soluci\'on toma la siguiente forma $$y(x)=c_1$$
			La segunda condicion implica que
			$$
			0=0
			$$
			lo que significa $c_1$ es una constante libre.
			Por lo tanto $\lambda_0=0$ es un valor propio $y$ $\phi_0(x)=1$ es la funci\'on propia.
			\item $\lambda\neq0$\\
			\begin{itemize}
				\item $\lambda<0$\\
				Para este caso, la soluci\'on es
				$$
				y(x)=c_1 e^{n x}+c_2 e^{-n x}
				$$
				Aplicando la primera condici\'on t
				$$
				\begin{aligned}
					& y(\pi)=c_1 e^{n \pi}+c_2 e^{-n \pi} \\
					& y(-\pi)=c_1 e^{-m \pi}+c_2 e^{n \pi}
				\end{aligned}$$
				Igualando las expresiones
				$$c_1 e^{n \pi}+c_2 e^{-n \pi}=c_1 e^{-n \pi}+c_2 e^{n \pi}$$
				Tomando factor com\'un y aplicando la definici\'on del seno hiperb\'olico
				$$
				2 \sinh (n \pi) c_1-2 \sinh (n \pi) c_2=0
				$$
				Simplificando
				$$
				c_1-c_2=0
				$$
				Por la segunda condici\'on
				$$
				\begin{aligned}
					& y^{\prime}(\pi)=n c_1 e^{n \pi}-n c_2 e^{-n \pi} \\
					& y^{\prime}(-\pi)=n c_1 e^{-n \pi}-n c_2 e^{n \pi}
				\end{aligned}
				$$
				Igualando
				$$
				n c_1 e^{n \pi}-n c_2 e^{-n \pi}=n c_1 e^{-n \pi}-n c_2 e^{n \pi}
				$$
				Por factor com\'un
				$$
				n c_1\left(e^{n \pi}-e^{-n \pi}\right)+n c_2\left(e^{n \pi}-e^{-n \pi}\right)=0
				$$
				lo que implica que
				$$
				c_1+c_2=0
				$$
				Resolviendo el sistema
				$$
				\left\{\begin{array}{l}
					c_1-c_2=0 \\
					c_1+c_2=0
				\end{array}\right.
				$$
				tenemos que $c_1=c_2=0$. Por lo tanto $\lambda>0$ no es un valor propio.
				\item $\lambda>0$\\
				Evaluando las condiciones en la expresi\'on (\ref{solgnal11}), tenemos
				$$
				\begin{aligned}
					& y(-\pi)=\left(c_1+c_2\right) \cos (n \pi)-\left(c_1-c_2\right) i \sin (n \pi) \\
					& y(-\pi)=(-1)^n\left(c_1+c_2\right) \\
					& y(\pi)=\left(c_1+c_2\right) \cos (n \pi)+\left(c_1-c_2\right) i \sin (n \pi) \\
					& y(\pi)=(-1)^n\left(c_1+c_2\right)
				\end{aligned}
				$$
				Igualando
				$$
				(-1)^n\left(c_1+c_2\right)=(-1)^n\left(c_1+c_2\right)
				$$
				lo que significa $y(-\pi)=y(\pi)$ para todo valor de $x$.\\
				Por la segunda condici\'on
				$$
				\begin{aligned}
					& y^{\prime}(\pi)=-n\left(c_1+c_2\right) \sin (n \pi)+n\left(c_1-c_2\right) i \cos (n \pi) \\
					& y^{\prime}(\pi)=n\left(c_1-c_2\right) i(-1 )^n \\
					& y^{\prime}(-\pi)=n\left(c_1+c_2\right) \sin (n \pi)+n\left(c_1-c_2\right) i \cos (n \pi) \\
					& y^{\prime}(-\pi)=n\left(c_1-c_2\right) i(-1)^n
				\end{aligned}
				$$
				por lo que $y^{\prime}(\pi)=y^{\prime}(-\pi)$ para todo valor de $x$.\\
				As\'i $\lambda_n=n^2$ son valores propios y las funciones propias son $\phi_n(x)=\cos (n x)+\sin (n x)$.
			\end{itemize}
		\end{enumerate}
	\end{sol}
	
	\section{Obtenci\'on de los polinomios de Legendre como soluci\'on a S-L}
	
	\Example{	Considere el problema de Sturm-Liouville}{	Considere el problema de Sturm-Liouville
		\begin{eqnarray}\label{LegendreSL}
			\left(1-x^2\right) y^{\prime \prime}-2 x y^{\prime}+\lambda y=\left(\left(1-x^2\right) y^{\prime}\right)^{\prime}+\lambda y&=&0
		\end{eqnarray}
		sujeto a las condiciones
		\begin{eqnarray}\label{LegendreSLcond}
			\lim _{x \rightarrow-1} y(x)<\infty, \quad \lim _{x \rightarrow 1} y(x)<\infty .
		\end{eqnarray}
		Demuestre que los valores propios de este problema son  $\lambda_n=n(n+1), n=0,1,2, \cdots$ y las correspondientes funciones propias son los polinomios de Legendre $P_n(x)$.}
	
	
	\begin{sol}
		Utilizando la expresi\'on obtenida en (\ref{solegendrepimpar}) dada por
		\begin{eqnarray*}
			y(x)&=&a_0+a_1 x+\displaystyle\sum_{n=2}^{\infty} \frac{n(n+1)-\lambda}{(n+2)(n+1)} a_n x^n
		\end{eqnarray*}
		aplicando las condiciones
		\begin{eqnarray*}
			\displaystyle\lim _{x \rightarrow-1} y(x)&=&\displaystyle\lim _{x \rightarrow-1}\left\{a_0+a_1 x+\displaystyle\sum_{n=2}^{\infty}\displaystyle \frac{n(n+1)-\lambda}{(n+2)(n+1)} a_n x^n\right\}<\infty
		\end{eqnarray*}
		Por propiedad de los l\'imites
		\begin{eqnarray*}
			\displaystyle\lim _{x \rightarrow-1} y(x)&=&a_0-a_1+\displaystyle\sum_{n=2}^{\infty} \displaystyle\frac{n(n+1)-\lambda}{(n+2)(n+1)}(-1)^n a_n<\infty
		\end{eqnarray*}
		La expresi\'on anterior es una serie infinita de t\'erminos constantes, de la forma que puede ser acotada es que los t\'erminos de la serie se anulen a partir de cierto t\'ermino, esto es que $\lambda=n(n+1)$.\\
		Lo que significa que los valores propios del problema (\ref{LegendreSL} ) son $\lambda_n=n(n+1)$ y las correspondientes funciones propias son los poliniomios de legendre $P_{n} (x)$.
	\end{sol}
	
	\Example{	Considere el problema de Sturm-Liouville}{Considere el problema de Sturm-Liouville dado en la expresi\'on \ref{LegendreSL}
		sujeto a las condiciones
		\begin{eqnarray}\label{LegendreSLcond1}
			y^{\prime}(0)&=&0, \quad \lim _{x \rightarrow 1} y(x)<\infty
		\end{eqnarray}
		Demuestre que los valores propios de este problema son  $\lambda_n=2n(2n+1), n=0,1,2, \cdots$ y las correspondientes funciones propias son los polinomios pares de Legendre $P_{2n}(x)$}
	
	\begin{sol}
		De la expresi\'on (\ref{solegendre}) y la primera condici\'on tenemos
		\begin{eqnarray*}
			y(x)&=&a_0\left[1+\displaystyle\sum_{n=1}^{\infty} \frac{(0-\lambda)(6-\lambda) \cdots(2 n(2 n+1)-\lambda)}{(2 n) !} x^{2 n}\right.
		\end{eqnarray*}
		Por la segunda condici\'on se tiene que
		\begin{eqnarray*}
			\displaystyle\lim _{x \rightarrow 1} y(x)&=&a_0\left[1+\displaystyle\sum_{n=1}^{\infty} \frac{(0-\lambda)(6-\lambda) \cdots(2 n(2 n+1)-\lambda)}{(2 n) !}\right]<\infty
		\end{eqnarray*}
		Para que la serie sea acotada, se debe cumplir que $\lambda_n=2 n(2 n+1)$, de manera que son los valores propios del problema (\ref{LegendreSL} )con las condiciones (\ref{LegendreSLcond1}) y las funciones propias correspondientes son polinomios pares de Legendre $P_{2 n}(x)$.
	\end{sol}
	
	\Example{Considere el problema de Sturm-Liouville}{	Considere el problema de Sturm-Liouville dado en la expresi\'on \ref{LegendreSL}
		sujeto a las condiciones
		\begin{eqnarray}\label{LegendreSLcond2}
			y(0)=0, \quad \lim _{x \rightarrow 1} y(x)<\infty
		\end{eqnarray}
		Demuestre que los valores propios de este problema son  $\lambda_n=\lambda_n=(2 n+1)(2 n+2), n=0,1,2, \cdots$ y las correspondientes funciones propias son los polinomios impares de Legendre $P_{2n+1}(x)$}
	
	
	\begin{sol}
		Tomando la expresi\'on obtenida en (\ref{solegendre}) dada por
		
		$$
		\begin{aligned}
			y(x)= & a_0\left[1+\sum_{n=1}^{\infty} \frac{(0-\lambda)(6-\lambda) \ldots(2 n(2 n+1)-\lambda)}{(2 n) !} x^{2 n}\right] \\
			& +a_1\left[x+\sum_{n=1}^{\infty} \frac{(2-\lambda)(12-\lambda) \ldots((2 n+2)(2n+1)-\lambda)}{(2 n+3) !} x^{2 n+1}\right]
		\end{aligned}
		$$
		Por las condiciones del problema (\ref{LegendreSLcond2}), tenemos
		\begin{multline*}
			y(0)=a_0\left[1+\sum_{n=1}^{\infty} \frac{(0-\lambda)(6-\lambda) \cdots(2 n(2 n+1)-\lambda)}{(2 n+1) !}(0)^{2 n}\right] \\
			+a_1\left[0+\sum_{n=1}^{\infty} \frac{(2-\lambda)(12-\lambda) \cdots((2 n+2)(2 n+1)-\lambda)}{(2 n+3) !}(0)^{2 n+1}\right] \\
		\end{multline*}
		$$y(0)= a_0$$
		Por lo tanto $a_0=0$.
		De manera que
		\begin{eqnarray*}
			y(x)&=&a_1\left[x+\displaystyle\sum_{n=1}^{\infty} \displaystyle\frac{(2-\lambda)(12-\lambda) \ldots((2 n+2)(2 n+1)-\lambda)}{(2 n+3) !} x^{2 n+1}\right]
		\end{eqnarray*}
		Ahora queremos que
		\begin{eqnarray*}
			\displaystyle\lim _{x \rightarrow 1} y(x)<\infty
		\end{eqnarray*}
		\begin{eqnarray*}
			\displaystyle\lim _{x \rightarrow 1} y(x)&=&\displaystyle\lim _{x \rightarrow 1}\left\{a_1\left[x+\sum_{n=1}^{\infty} \frac{(2-\lambda)(12-\lambda) \cdots((2 n+2)(2 n+1)-\lambda)}{(2 n+3) !} x^{2 n+1}\right]\right\}<\infty
		\end{eqnarray*}
		Por propiedad de los l\'imites
		$$
		\begin{aligned}
			& \lim _{x \rightarrow 1} y(x)=a_1\left[1+\sum_{n=1}^{\infty} \frac{(2-\lambda)(12-\lambda) \cdots((2 n+2)(2 n+1)-\lambda)}{(2 n+3) !}(1)\right]<\infty \\
			& \lim _{x \rightarrow 1} y(x)=a_1\left[1+\sum_{n=1}^{\infty} \frac{(2-\lambda)(12-\lambda) \cdots((2 n+2)(2 n+1)-\lambda)}{(2 n+3) !}\right]<\infty
		\end{aligned}
		$$
		La serie anterior es finita si los t\'erminos de anulan para eso
		$$
		(2 n+2)(2 n+1)-\lambda=0
		$$
		lo que implica que $\lambda=(2 n+2)(2 n+1)$. De tal manera gue los valores propios son $\lambda_n=(2n+2)(2n+1)$ y las funciones propias son los polinomios $P_{2 n+1}(x)$ de Legendre.
	\end{sol}
	
	\section{Obtenci\'on de los polinomios de Hermite  como soluci\'on a S-L}
	
	\Example{Considere el problema de Sturm-Liouville}{Considere el problema de Sturm-Liouville
		\begin{eqnarray}\label{hermiteSL}
			y^{\prime \prime}-2 x y^{\prime}+\lambda y&=0=&\left(e^{-x^2} y^{\prime}\right)^{\prime}+\lambda e^{-x^2} y
		\end{eqnarray}
		sujeto a las condiciones
		\begin{eqnarray}\label{HermiteSLcond}
			\lim _{x \rightarrow-\infty} \frac{y(x)}{|x|^k}<\infty, \quad \lim _{x \rightarrow \infty} \frac{y(x)}{x^k}<\infty
		\end{eqnarray}
		para alg\'un entero positivo $k$.
		Demuestre que los valores propios de este problema son $\lambda_n=2 n, n=0,1,2, \cdots$ y las correspondientes funciones propias son los polinomios de Hermite $H_{n}(x)$.
	}
	
	
	
	
	\begin{sol}
		kl
	\end{sol}
	\section{Obtenci\'on de los polinomios de Laguerre  como soluci\'on a S-L}
	
	\Example{Ejemplo}{Considere el problema de Sturm-Liouville
		\begin{eqnarray}\label{LaguerreeSL}
			x y^{\prime \prime}+(1-x) y^{\prime}+\lambda y&=0=&\left(x e^{-x} y^{\prime}\right)^{\prime}+\lambda e^{-x} y
		\end{eqnarray}
		sujeto a las condiciones
		\begin{eqnarray}\label{LaguerreSLcond}
			\lim _{x \rightarrow 0}|y(x)|<\infty, \quad \lim _{x \rightarrow \infty} \frac{y(x)}{x^k}<\infty
		\end{eqnarray}
		Para alg\'un entero positivo $k$.
		Demuestre que los valores propios de este problema son $\lambda_n=n, n=0,1,2, \cdots$ y las correspondientes funciones propias son los polinomios de Laguerre $L_{n}(x)$.}
	
	
	\begin{sol}
		kl
	\end{sol}
	
	\section{Obtenci\'on de la funci\'on de Bessel como soluci\'on a S-L}
	
	\Example{Ejemplo}{Sea $a \geq 0$ un n\'umero fijo, y $b_n, n=0,1,2, \cdots$ los ceros de la funci\'on de Bessel  $J_a(x)$. Considere el problema de Sturm-Liouville
		\begin{eqnarray}\label{besselSL}
			x^2 y^{\prime \prime}+x y^{\prime}+\left(\lambda x^2-a^2\right) y&=0=&\left(x y^{\prime}\right)^{\prime}+\left(\lambda x-\frac{a^2}{x}\right)y
		\end{eqnarray}
		sujeto a las condiciones
		\begin{eqnarray}\label{besselSLcond}
			\lim _{x \rightarrow 0} y(x)<\infty, \quad y(1)=0
		\end{eqnarray}
		Demuestre que los valores propios de este problema son $\lambda_n=b_n^2, n=0,1,2, \cdots$  y las correspondientes funciones propias son las funciones de Bessel $J_a\left(b_n x\right)$.}
	
	
	
	\begin{sol}
		kl
	\end{sol}
	
	\section{Obtenci\'on de los polinomios de Chebyshev de 3er y 4to tipo como soluci\'on a S-L}
	\subsection{Obtenci\'on de los polinomios de Chebyshev de 3er tipo como soluci\'on a un S-L}
	La ecuaci\'on diferencial que define los polinomios de Chebyshev de 3er tipo, como vimos en el cap\'itulo anterior es:
	$$
	\left(1-x^2\right) y^{\prime \prime}+(1-2 x) y^{\prime}+\lambda y=0 ;
	$$
	consideremos las condicciones que vimos para un problema de S-L singular, para este caso:
	$$
	\begin{gathered}
		p(1)=0 \\
		p(-1)=0 \\
		y(x) \quad \text { regular en } x=1, x=-1
	\end{gathered}
	$$
	Como vimos en (3.1.3.1) esta ecuaci\'on se puede expres ar en forma de una ecuaci\'on de S-L :
	Con $\lambda=m(m+1)$ Tenemos:
	$$
	\frac{d}{d x}\left[(1+x)^{3 / 2}(1-x)^{1 / 2} \frac{d y_m(x)}{d x}\right]+m(m+1) \sqrt{\frac{1+x}{1-x}} y_m(x)=0
	$$
	donde $p(x)=(1+x)^{3 / 2}(1-x)^{1 / 2}, q(x)=0$ y $w(x)=\sqrt{\frac{1+x}{1-x}}$.
	Es evidente devido a las condiciones que nuestro problemas es singular, de hecho $w(x)$ no es continua en uno de los extremos. Y se verifica que $p(-1)=p(1)=0$
	
	Sabemos que la ecuaci\'on (4.4.1), al resolverla por medio de la sustituci\'on trigonom\'etrica que hicimos en el cap\'itulo anterior, tiene dos soluciones:
	$$
	y_1(x)=\frac{\cos \left[\left(n+\frac{1}{2}\right) \theta\right]}{\cos \left(\frac{\theta}{2}\right)} ; \quad y_2(x)=\frac{\operatorname{sen}\left[\left(n+\frac{1}{2}\right) \theta\right]}{\cos \left(\frac{\theta}{2}\right)}
	$$
	Como $x=\cos (\theta)$, tenemos que $x=-1 \Rightarrow \theta=\pi$ y $x=1 \Rightarrow \theta=0$. Vemos que la singularidad esta en $\theta=\pi$, puesto que $\cos (\pi / 2)=0$.
	$$
	y_2(\pi)=\frac{\operatorname{sen}\left[\left(n+\frac{1}{2}\right) \pi\right]}{\cos \left(\frac{\pi}{2}\right)}=\frac{(-1)^n}{0}=\infty,
	$$
	por otro lado
	$$
	y_1(\pi)=\frac{\cos \left[\left(n+\frac{1}{2}\right) \pi\right]}{\cos \left(\frac{\pi}{2}\right)}=\frac{0}{0},
	$$
	aplicando L'Hopital, tenemos
	$$
	\displaystyle\lim_{\theta \rightarrow \pi} y_1(\cos \theta)=\frac{\left[\left(n+\frac{1}{2}\right)\right] \operatorname{sen}\left[\left(n+\frac{1}{2}\right) \pi\right]}{1 / 2 \operatorname{sen}(\pi / 2)}=\frac{(2 n+1)(-1)^n}{1}=(2 n+1)(-1)^n,
	$$
	lo cual es un valor finito para n finito, asi se concluye que los polinomios de Chebyshev de 3er tipo son soluci\'on de la ecuaci\'on de S-L vista, es decir, $\lambda=m(m+1)$ son los autovalores al problema singular de S-L y $V_m(x)$ sus correspondiente autofunciones. La constante albitraria se toma de modo que el coeficiente principal sea $2^n$.
	
	\subsection{Obtenci\'on de los polinomios de Chebyshev de 4to tipo como soluci\'on a un S-L}
	La ecuaci\'on diferencial que define los polinomios de Chebyshev de 4to tipo, como vimos en el capitulo anterior es:
	$$
	\left(1-x^2\right) y^{\prime \prime}-(1+2 x) y^{\prime}+\lambda y=0
	$$
	consideremos las condicciones que vimos para un problema de S-L singular, para este caso:
	$$
	\begin{gathered}
		p(1)=0 \\
		p(-1)=0 \\
		y(x) \quad \text { regular en } x=1, x=-1
	\end{gathered}
	$$
	Como vimos en (3.2.3.1) esta ecuaci\'on se puede expresar en forma de una ecuaci\'on de S-L : Con $\lambda=m(m+1)$ Tenemos:
	$$
	\begin{gathered}
		\frac{d}{d x}\left[(1-x)^{3 / 2}(1+x)^{1 / 2} \frac{d y_m(x)}{d x}\right]+m(m+1) \sqrt{\frac{1-x}{1+x}} y_m(x)=0 \\
		\text { donde } p(x)=(1-x)^{3 / 2}(1+x)^{1 / 2}, q(x)=0 \mathrm{y} w(x)=\sqrt{\frac{1-x}{1+x}} .
	\end{gathered}
	$$
	Es evidente devido a las condiciones que nuestro problemas es singular.\\
	Como vimos en el cap\'itulo anterior al resolver esta EDO por medio de una sustituci\'on trigonom\'etrica, tenemos dos soluciones. Puesto que es evidente que $p(-1)=p(1)=0$, veremos que una de las dos soluciones obtenidas verifica la condicci\'on restante.
	$$
	y_1(x)=\frac{\cos \left[\left(n+\frac{1}{2}\right) \theta\right]}{\operatorname{sen}\left(\frac{\theta}{2}\right)} ; \quad y_2(x)=\frac{\operatorname{sen}\left[\left(n+\frac{1}{2}\right) \theta\right]}{\operatorname{sen}\left(\frac{\theta}{2}\right)}
	$$
	Como $x=\cos (\theta)$, tenemos que $x=-1 \Rightarrow \theta=\pi \mathrm{y} x=1 \Rightarrow \theta=0$. Vemos que la singularidad esta en $\theta=0$, puesto que $\operatorname{sen}(\pi / 2)=0$.
	$$
	y_1(0)=\frac{\cos (0)}{\operatorname{sen}(0)}=\frac{1}{0}=\infty,
	$$
	por otro lado
	$$
	y_2(0)=\frac{\operatorname{sen}(0)}{\operatorname{sen}(0)}=\frac{0}{0},
	$$
	aplicando L'Hopital, tenemos
	$$
	\displaystyle\lim_{\theta \rightarrow 0} y_2(\cos 0)=2 \frac{\left[\left(n+\frac{1}{2}\right)\right] \cos (0)}{\cos (0)}=2 n+1,
	$$
	lo cual es un valor finito para n finito, asi se concluye que los polinomios de Chebyshev de 4to tipo son soluci\'on de la ecuaci\'on de S-L vista, es decir, $\lambda=m(m+1)$ son los autovalores al problema irregular de S-L y $W_m(x)$ sus correspondiente autofunciones. La constante albitraria se tomo de modo que el coeficiente principal sea $2^n$.
	\textcolor[rgb]{1.00,0.00,0.00}{anterior de tesis de pablo}
	
	\Example{Ejemplo}{Considere el problema de Sturm-Liouville
		\begin{equation}\label{eq36}
			\left(1-x^2\right) y^{\prime \prime}-2 x y^{\prime}+\lambda y=\left(\left(1-x^2\right) y^{\prime}\right)^{\prime}+\lambda y=0
		\end{equation}
		con las condiciones
		\begin{equation}\label{eq38}
			y^{\prime}(0)=0, \quad \lim _{x \rightarrow 1} y(x)<\infty
		\end{equation}
		Demuestre que los valores propios del problema son $\lambda_n=2 n(2 n+1), n=$ $0,1,2, \cdots$ y las correspondientes funciones propias son los polinomios pares de Legendre $P_{2 n}(x)$.}
	
	
	
	
	\begin{sol}
		Para resolver este problema de Sturm-Liouville necesitamos conocer las soluciones de la ecuaci\'on de Legendre, la cual se analiz\'o en la secci\'on  \ref{Legendre}, donde la soluci\'on general dada en \ref{solegendre} es
		\begin{multline*}
			y(x) =a_{0}\bigg[1+\displaystyle\sum_{n=1}^{\infty}\left[\displaystyle\frac{(0-\lambda)(6-\lambda)\cdots (2n(2n+1)-\lambda)}{(2n)!} \right]x^{2n}\bigg] \\
			+a_{1}\bigg[x+\displaystyle\sum_{n=1}^{\infty}\left[ \displaystyle\frac{(2-\lambda)(12-\lambda) \cdots((2n+1)(2 n+2)-\lambda)}{(2 n+3)!}\right]x^{2n+1}\bigg]
		\end{multline*}
		Aplicando las condiciones, tenemos
		$$
		\begin{aligned}
			y^{\prime}(x)= & a_0 \sum_{n=1}^{\infty}\left[\frac{(0-1)(6-\lambda) \ldots(2 n(2 n+1)-\lambda)}{(2 n) !}\right](2 n) x^{2n-1} \\
			& +a_1 \sum_{\pi=1}^{\infty}\left[\frac{(2-\lambda)(12-\lambda) \cdots((2n+1)(2 n+2)-\lambda)}{(2 n+3) !}\right](2n+1) x^{2n}
		\end{aligned}
		$$
		As\'i
		$$
		y^{\prime}(0)=a_1=0
		$$
		De manera que la soluci\'on es
		\begin{eqnarray}\label{12.}
			y(x)=a_0\left[1+\sum_{n=1}^{\infty}\left[\frac{(0-x)(6-x) \cdots(2 n(2 n+1)-x)}{(2 n) !}\right] x^{2 n}\right.
		\end{eqnarray}
		Por la segunda condici\'on
		$$
		\lim _{x \rightarrow 1} a_0\left[1+\sum_{n=1}\left[\frac{(2-\lambda)(6-\lambda) \cdots(2 n(2 n+1)-\lambda)}{(2 n) !}\right] x^{2 n}<\infty\right.
		$$
		como los coeficientes est\'an dado en t\'ermino de productoria, de manera que algunos de ellos es cero si $\exists n \geq 1\quad \text{tal que} \quad \lambda=2n(2n-1)$.
		Por lo tanto la expresi\'on \ref{12.} representa los polinomios pares de Legendre
		\begin{eqnarray*}
			P_{2n}(x)&=&a_0\left[1+\sum_{n=1}^{\infty}\left[\frac{(0-x)(6-x) \cdots(2 n(2 n+1)-x)}{(2 n) !}\right] x^{2 n}\right.
		\end{eqnarray*}
	\end{sol}
	
	\Example{Ejemplo}{Considere la ED \ref{eq36} con las condiciones
		\begin{equation}\label{eq39}
			y(0)=0, \quad \displaystyle\lim _{x \rightarrow 1} y(x)<\infty
		\end{equation}
		Demuestre que los valores propios son  $\lambda_n=(2 n+1)(2 n+2), n=$ $0,1,2, \cdots$ y las correspondientes funciones propias son los polinomios impares de de Legendre $P_{2 n+1}(x)$.}
	
	
	\begin{sol}
		Sabemos que la soluci\'on general de ED de Legendre es
		\begin{multline*}
			y(x) =a_{0}\bigg[1+\displaystyle\sum_{n=1}^{\infty}\left[\displaystyle\frac{(0-\lambda)(6-\lambda)\cdots (2n(2n+1)-\lambda)}{(2n)!} \right]x^{2n}\bigg] \\
			+a_{1}\bigg[x+\displaystyle\sum_{n=1}^{\infty}\left[ \displaystyle\frac{(2-\lambda)(12-\lambda) \cdots((2n+1)(2 n+2)-\lambda)}{(2 n+3)!}\right]x^{2n+1}\bigg]
		\end{multline*}
		Ahora aplicaremos las condiciones indicadas en la expresi\'on \ref{eq39}. Evaluando la soluci\'on en $x=0$
		\begin{multline*}
			y(0)=a_{0}\bigg[1+\displaystyle\sum_{n=1}^{\infty}\left[\displaystyle\frac{(0-\lambda)(6-\lambda)\cdots (2n(2n+1)-\lambda)}{(2n)!} \right](0)^{2n}\bigg] \\
			+a_{1}\bigg[0+\displaystyle\sum_{n=1}^{\infty}\left[ \displaystyle\frac{(2-\lambda)(12-\lambda) \cdots((2n+1)(2 n+2)-\lambda)}{(2 n+3)!}\right](0)^{2n+1}\bigg]=0
		\end{multline*}
		La expresi\'on anterior implica que $a_{1}=0$. As\'i
		\begin{eqnarray*}
			% \nonumber % Remove numbering (before each equation)
			y(x) &=& a_{1}\bigg[0+\displaystyle\sum_{n=1}^{\infty}\left[ \displaystyle\frac{(2-\lambda)(12-\lambda) \cdots((2n+1)(2 n+2)-\lambda)}{(2 n+3)!}\right](0)^{2n+1}\bigg]
		\end{eqnarray*}
		La expresi\'on anterior debe ser finita por la segunda condici\'on, as\'i
		\begin{eqnarray*}
			% \nonumber % Remove numbering (before each equation)
			\displaystyle\lim_{x\rightarrow 1} a_{1}\bigg[x+\displaystyle\sum_{n=1}^{\infty}\left[ \displaystyle\frac{(2-\lambda)(12-\lambda) \cdots((2n+1)(2 n+2)-\lambda)}{(2 n+3)!}\right]x^{2n+1}\bigg]<\infty
		\end{eqnarray*}
		podemos notar que la funci\'on es finita si algunos de sus coeficientes para alg\'un valor $\lambda$, por lo tanto esto ocurre si $\lambda=(2n+1)(2n+2)$, as\'i obteniendo los polinomios impares de Legendre
		\begin{eqnarray*}
			% \nonumber % Remove numbering (before each equation)
			P_{2n+1}(x) &=& a_{1}\bigg[x+\displaystyle\sum_{n=1}^{\infty}\left[ \displaystyle\frac{(2-\lambda)(12-\lambda) \cdots((2n+1)(2 n+2)-\lambda)}{(2 n+3)!}\right]x^{2n+1}\bigg]
		\end{eqnarray*}
	\end{sol}
	
	\Example{ Sturm-Liouville}{
		
		Considere el problema singular de  Sturm-Liouville
		\begin{equation}\label{eq40}
			y^{\prime \prime}-2 x y^{\prime}+\lambda y=0=\left(e^{-x^2} y^{\prime}\right)^{\prime}+\lambda e^{-x^2} y
		\end{equation}
		Sujeto a las condiciones
		\begin{equation}\label{eq41}
			\displaystyle\lim _{x \rightarrow-\infty} \frac{y(x)}{|x|^k}<\infty, \quad \displaystyle\lim_{x \rightarrow \infty} \frac{y(x)}{x^k}<\infty
		\end{equation}
		para alg\'in entero positvo  $k$. Demuestre que los valores propios son  $\lambda_n=2 n, n=0,1,2, \cdots$ y las correspondientes funciones propias son los polinomios de Hermite $H_n(x)$.}
	
	
	
	\begin{sol}
		De la secci\'on \ref{Hermite} Sabemos que la soluci\'on  de la ED de Hermite est\'a dada por la expresi\'on
		\begin{eqnarray*}
			% \nonumber % Remove numbering (before each equation)
			y(x) &=&\displaystyle\sum_{n=0}^{\infty} c_{n}x^{n}
		\end{eqnarray*}
		donde $c_{n}$ est\'a dada por la expresi\'on \ref{H4}, de manera que la expresi\'on anterior es
		$$
		y(x)=a_{0}+a_{1}x+\displaystyle\sum_{n=2}^{\infty} \frac{(\lambda-2 n)c_{n}}{(n+2)(n+1)} x^n
		$$
		Aplicando las candiciones tenemos
		\begin{eqnarray*}
			% \nonumber % Remove numbering (before each equation)
			\lim _{x \rightarrow \infty}[a_{0}+a_{1}x+ \sum_{n=2}^{\infty}  \frac{(\lambda-2 n)c_{n}}{(n+2)(n-1))} \frac{x^n}{x^k}]<\infty
		\end{eqnarray*}
		La expresi\'on anterior es finta, si existe un $n \in Z^{+}$ tal que $\lambda-2n=0$ lo que implica que $ \lambda=2 n$.\\
		Por la otra condici\'on
		$$
		\lim _{x \rightarrow-\infty} [a_{0}+a_{1}x+ \sum_{n=2}^{\infty} \frac{(\lambda-2 n)c_{n}}{(n+2)(n+1)} \frac{x^n}{|x|^k}]<\infty
		$$
		de iqual manera, para que la expresi\'on anterior sea finita, se debe cumplir que $\lambda-2 n=0$ para ang\'un $n \in z^{+}$, lo que implica que $\lambda=2 n$.\\
		Por lo tanto los valores propios son $\lambda n=2 n$ y las correspondientes fanciones propias son los polinomios de Hermite $H_{n}(x).$
	\end{sol}
	
	\Example{Considere el problema regular de Sturm-Liouville}{
		
		Considere el problema regular de Sturm-Liouville
		\begin{equation}\label{eq42}
			x y^{\prime \prime}+(1-x) y^{\prime}+\lambda y=0=\left(x e^{-x} y^{\prime}\right)^{\prime}+\lambda e^{-x} y
		\end{equation}
		con
		\begin{equation}\label{eq43}
			\begin{aligned}
				\displaystyle\lim _{x \rightarrow 0}|y(x)|<\infty, \quad \displaystyle\lim_{x \rightarrow \infty} \frac{y(x)}{x^k}<\infty \text { para alg\'un entero positivo } k .
			\end{aligned}
		\end{equation}
		Demuestre que los valores propios del problema son  $\lambda_n=n, n=0,1,2, \cdots$ y las correspondientes funciones propias est\'an dadas por los polinomios de Laguerre  $L_n(x)$.}
	
	
	\begin{sol}
		khkhkf
	\end{sol}
	
	
	\setboolean{firstanswerofthechapter}{true}
	\begin{multicols}{2}
		\begin{Answer}[ref={EX41}]
			\Question 
			\begin{tasks}
				\task This is a solution of Ex 1
				\task This is a solution of Ex 2 
				\task This is a solution of Ex 3 
				\task This is a solution of Ex 4 
				\task This is a solution of Ex 5 
				\task This is a solution of Ex 6 
				\task This is a solution of Ex 7 
				\task This is a solution of Ex 8 
				\task[9] This is a solution of Ex 9
				\task[10] This is a solution of Ex 10 
				\task[11] This is a solution of Ex 11
				\task[12] This is a solution of Ex 12
				\task[13] This is a solution of Ex 13
				\task[14] This is a solution of Ex 14 
				\task[15] This is a solution of Ex 15
				\task[16] This is a solution of Ex 16
			\end{tasks}
		\end{Answer}
	\end{multicols}

\mychapter{Series de Fourier en bases a funciones especiales }{\begin{wrapfigure}{l}{0.45\textwidth}
		\centering
		\includegraphics[width=0.45\textwidth]{imagen/img11.png}
	\end{wrapfigure} En el estudio del análisis matemático y sus aplicaciones, las series especiales constituyen una herramienta fundamental para la representación y aproximación de funciones. Entre estas destacan, por un lado, las series de Fourier-Bessel, que surgen de la resolución de problemas con condiciones de frontera en coordenadas cilíndricas, y que tienen una importancia crucial en la física matemática, especialmente en fenómenos de propagación de ondas y conducción de calor. Por otro lado, los polinomios de Chebyshev representan una de las familias de polinomios ortogonales más relevantes en la aproximación de funciones, ya que permiten minimizar el error de interpolación y facilitan la construcción de expansiones eficientes.
	
	\vspace{0.5cm}
	
	En este capítulo se abordarán, en primer lugar, las series de Fourier-Bessel, enfatizando sus propiedades, condiciones de ortogonalidad y aplicaciones típicas en problemas con simetría radial. Posteriormente, se presentará la aproximación de funciones mediante los polinomios de Chebyshev de tercera y cuarta especie, resaltando su papel en el desarrollo de series de Fourier generalizadas. La combinación de estas herramientas ofrece un marco poderoso no solo para la teoría matemática, sino también para la resolución práctica de ecuaciones diferenciales y problemas aplicados en ingeniería, física y computación científica.}
	
	\addtocontents{toc}{\protect\figuretoc{imagen/img11.png}}
	
	
	% Capitulo 5
	
	\Example{Find the Fourier series}{	$$
		|\sin x|=\displaystyle\sum_{n=-\infty}^{\infty} c_n e^{i 2 n x},\quad \text { for } x \in[-\pi, \pi].
		$$}
	
	
	\begin{demo}
		The coefficients $\left\{c_n\right\}$ are given by
		$$
		c_n=\frac{1}{\pi} \int_0^\pi|\sin x| e^{-i 2 n x} d x=\frac{1}{\pi} \int_0^\pi \sin x e^{-i 2 n x} d x .
		$$
		We can expand $\sin x$ into exponentials to obtain
		$$
		\begin{aligned}
			c_n & =\frac{1}{2 \pi i} \int_0^\pi\left(e^{i x}-e^{-i x}\right) e^{-i 2 n x} d x \\
			& =\frac{1}{2 \pi i}\left[\int_0^\pi e^{-i(2 n-1) x} d x-\int_0^\pi e^{-i(2 n+1) x} d x\right] \\
			& =\frac{1}{2 \pi i}\left[\frac{i}{2 n-1}\left(e^{-i(2 n-1) \pi}-1\right)-\frac{i}{2 n+1}\left(e^{-i(2 n+1) \pi}-1\right)\right] \\
			& =\frac{1}{2 \pi i}(-2 i)\left[\frac{1}{2 n-1}-\frac{1}{2 n+1}\right] \\
			& =-\frac{2}{\pi} \frac{1}{4 n^2-1}
		\end{aligned}
		$$
		Therefore,
		$$
		|\sin x|=\sum_{n=-\infty}^{\infty}-\frac{2}{\pi} \frac{1}{4 n^2-1} e^{i 2 n x}=\frac{2}{\pi}-\sum_{n=1}^{\infty} \frac{4}{\pi} \frac{1}{4 n^2-1} \cos 2 n x .
		$$
		Notice that this can be used in order to compute infinite sums. Evaluating this at $x=0$, we have that
		$$
		\sum_{n=1}^{\infty} \frac{1}{4 n^2-1}=\frac{1}{2},
		$$
		whereas evaluating this at $x=\pi / 2$, we have that
		$$
		\sum_{n=1}^{\infty} \frac{(-1)^n}{4 n^2-1}=\frac{2-\pi}{4}
		$$
	\end{demo}
	
	\section{Series Fourier-Bessel}
	\textcolor[rgb]{1.00,0.00,0.00}{sistemas ortogonales milane}
	
	\Example{Find the Bessel series expansion}{Find the Bessel series expansion with respect to $J_\nu\left(\lambda_k x\right)$ for the function $x^\nu, \nu \geq 0$}
	
	
	\begin{demo}
		In this case the coefficients in the series are given by the formula
		$$
		c_k=\frac{2}{\left[J_{\nu+1}\left(\lambda_k\right)\right]^2} \displaystyle\int_0^1 x^{\nu+1} J_\nu\left(\lambda_k x\right) d x
		$$
		and can be evaluated by setting $t=\lambda_k x$ and using Formula (\ref{eq46}), as follows:
		$$
		\begin{aligned}
			\displaystyle\int_0^1 x^{\nu+1} J_\nu\left(\lambda_k x\right) d x  =\frac{1}{\lambda_k^{\nu+2}} \displaystyle\int_0^{\lambda_k} t^{\nu+1} J_\nu(t) d t  =\frac{1}{\lambda_k^{\nu+2}} \displaystyle\int_0^{\lambda_k} \frac{d}{d t}\left[t^{\nu+1} J_{\nu+1}(t)\right] d t  &=\frac{1}{\lambda_k^{\nu+2}}\left[t^{\nu+1} J_{\nu+1}(t)\right]_0^{\lambda_k}\\
			&=\frac{1}{\lambda_k} J_{\nu+1}\left(\lambda_k\right)
		\end{aligned}
		$$
		Thus
		$$
		c_k=\frac{2}{\lambda_k J_{\nu+1}\left(\lambda_k\right)}, \quad k=1,2, \ldots,
		$$
		\begin{equation}\label{eq75}
			x^\nu=2\left[\frac{J_\nu\left(\lambda_1 x\right)}{\lambda_1 J_{\nu+1}\left(\lambda_1\right)}+\frac{J_\nu\left(\lambda_2 x\right)}{\lambda_2 J_{\nu+1}\left(\lambda_2\right)}+\frac{J_\nu\left(\lambda_3 x\right)}{\lambda_3 J_{\nu+1}\left(\lambda_3\right)}+\cdots\right],
		\end{equation}
		where the series converges in the mean in $\mathcal{P C}[0,1]$, pointwise in $(0,1)$, and uniformly on any closed subinterval of $(0,1)$.
		In particular, when $\nu=0,(\ref{eq75})$ yields the formula
		\begin{equation}\label{eq76}
			\displaystyle\sum_{k=1}^{\infty} \frac{J_0\left(\lambda_k x\right)}{\lambda_k J_1\left(\lambda_k\right)}=\frac{1}{2}, \quad 0<x<1 .
		\end{equation}
	\end{demo}
	
	\Example{Ejemplo}{Expand  $x^2, 0<x<1$, as a series in  $J_0\left(\lambda_k x\right)$}
	
	
	\begin{demo}
		\vspace{.1cm}
		
		Here
		$$
		c_k=\frac{2}{\left[J_1\left(\lambda_k\right)\right]^2} \displaystyle\int_0^1 x^3 J_0\left(\lambda_k x\right) d x
		$$
		and, reasoning as in the preceding example, we find that
		$$
		\begin{aligned}
			\displaystyle\int_0^1 x^3 J_0\left(\lambda_k x\right) d x & =\frac{1}{\lambda_k^4} \displaystyle\int_0^{\lambda_k} t^3 J_0(t) d t \\
			& =\frac{1}{\lambda_k^4} \displaystyle\int_0^{\lambda_k} t^2 \frac{d}{d t}\left[t J_1(t)\right] d t \\
			& =\frac{1}{\lambda_k^4}\left[\left.t^3 J_1(t)\right|_0 ^{\lambda_k}-2 \displaystyle\int_0^{\lambda_k} t^2 J_1(t) d t\right] \\
			& =\frac{J_1\left(\lambda_k\right)}{\lambda_k}-\frac{2}{\lambda_k^4} \displaystyle\int_0^{\lambda_k} t^2 J_1(t) d t .
		\end{aligned}
		$$
		But
		$$
		\begin{aligned}
			\displaystyle\int_0^{\lambda_k} t^2 J_1(t) d t & =-\displaystyle\int_0^{\lambda_k} t^2 \frac{d}{d t} J_0(t) d t \\
			& =-\left.t^2 J_0(t)\right|_0 ^{\lambda_k}+2\displaystyle \int_0^{\lambda_k} t J_0(t) d t \\
			& =2 \displaystyle\int_0^{\lambda_k} \frac{d}{d t}\left[t J_1(t)\right] d t \\
			& =2 \lambda_k J_1\left(\lambda_k\right)
		\end{aligned}
		$$
		and it follows that
		$$
		\displaystyle\int_0^1 x^3 J_0\left(\lambda_k x\right) d x=\frac{J_1\left(\lambda_k\right)}{\lambda_k}-\frac{4 J_1\left(\lambda_k\right)}{\lambda_k^3}
		$$
		Thus
		$$
		c_k=\frac{2}{J_1\left(\lambda_k\right)}\left[\frac{1}{\lambda_k}-\frac{4}{\lambda_k^3}\right],
		$$
		and
		$$
		x^2=2 \displaystyle\sum_{k=1}^{\infty} \frac{1}{J_1\left(\lambda_k\right)}\left(\frac{1}{\lambda_k}-\frac{4}{\lambda_k^3}\right) J_0\left(\lambda_k x\right), \quad 0<x<1 $$
		$$
		x^2=2 \displaystyle\sum_{k=1}^{\infty} \frac{1}{J_1\left(j_{0,k}\right)}\left(\frac{1}{j_{0,k}}-\frac{4}{j_{0,k}^3}\right) J_0\left(j_{0,k} x\right), \quad 0<x<1
		$$
		
		$$
		x^2\approx\sum_{k=1}^1 \frac{2\left(\frac{1}{j_{0, k}}-\frac{4}{\left(j_{0, k}\right)^3}\right) J_0\left(x j_{0, k}\right)}{J_1\left(j_{0, k}\right)}=\frac{2\left(\left(j_{0,1}\right)^2-4\right) J_0\left(x j_{0,1}\right)}{\left(j_{0,1}\right)^3 J_1\left(j_{0,1}\right)}
		$$
		\textcolor[rgb]{1.00,0.00,0.00}{agregar grafica}
	\end{demo}
	
	\Example{Ejemplo}{Prove that
		$$
		1-x^2=8 \displaystyle\sum_{k=1}^{\infty} \frac{J_0\left(\lambda_k x\right)}{\lambda_k^3 J_1\left(\lambda_k\right)}, \quad \text{for $0<x<1$.}
		$$
	}
	
	\begin{demo}
		$$
		f(x)=\displaystyle\sum_{k=1}^{\infty} c_k J_\nu\left(\lambda_k x\right),\qquad  c_k=\frac{2}{\left[J_{\nu+1}\left(\lambda_k\right)\right]^2} \displaystyle\int_0^1 f(x) J_\nu\left(\lambda_k x\right) x d x .
		$$
		$$
		\begin{aligned}
			& C_k=\frac{2}{\left[J_1\left(\lambda_k\right)\right]^2} \displaystyle\int_0^1 (1- x^2)x J_0\left(\lambda_k x\right) d x \\
			& C_k=\frac{2}{\left[J_1\left(\lambda_k\right)\right]^2}\left[\displaystyle\int_0^1 x J_0\left(\lambda_kx\right) d x-\displaystyle\int_0^1 x^3 J_0\left(\lambda_k x\right) d x\right]
		\end{aligned}
		$$
		$t=\lambda_k x \quad 0<t<\lambda_k$
		$$
		\begin{aligned}
			& C_k=\frac{2}{\left[J_1\left(\lambda_k\right)\right]^2}\left[\displaystyle\int_0^{\lambda_k} \frac{t}{\lambda_k} J_0(t) \frac{d t}{\lambda_k}-\displaystyle\int_0^\lambda x^3 J_0\left(\lambda_k x\right) d x\right] \\
			& C_k=\frac{2}{\left[J_1\left(\lambda_k\right)\right]^2}\left[\frac{1}{\lambda_k^2} \displaystyle\int_0^{\lambda_k} t J_0(t) d t-\left(\frac{J_1\left(\lambda_k\right)}{\lambda_k}-\frac{4 J_1\left(\lambda_k\right)}{\lambda^3}\right)\right]\\
			& C_k=\frac{2}{\left[J_1\left(\lambda_k\right)\right]^2}\left[\frac{1}{\lambda_k^2} \displaystyle\int_0^{\lambda_k} \frac{d}{d t}\left[t J_1(t)\right] d t-J_1\left(\lambda_k\right)\left(\frac{1}{\lambda_k}-\frac{4}{\lambda_k^3}\right)\right] \\
			& C_k=\frac{2}{\left[J_1\left(\lambda_k\right)\right]^2}\left[\frac{1}{\lambda_k^2}\left[t J_1(t)\right]_0^{\lambda_k}-J_1\left(\lambda_k\right)\left(\frac{1}{\lambda_k}-\frac{4}{\lambda_k^3}\right)\right]\\
			& C_k=\frac{2}{J_1\left(\lambda_k\right)}\left[\frac{1}{\lambda_k}-\frac{1}{\lambda_k}+\frac{4}{\lambda_k^3}\right] \\
			& C_k=\frac{8}{\lambda_k^3 J_1\left(\lambda_k\right)}
		\end{aligned}
		$$
		$$
		1-x^2=8 \displaystyle\sum_{k=1}^{\infty} \frac{J_0\left(\lambda_k x\right)}{\lambda_k^3 J_1\left(\lambda_k\right)} ;\quad  0<x<1
		$$
	\end{demo}
	
	
	\section{Aproximaci\'on de funciones utilizando los polinomios de Chebyshev III, IV}
	Como hemos estado viendo los polinomios de chebyshev de 3er y 4to tipo forman un sistema ortogonal cada uno, uno de los resultados importante de la teor\'ia de Sturm-Liouville es que las autofunciones de los problemas de autovalores de S-L son importantes para aproximar funciones continuas o suaves a trozo. A continuaci\'on veremos como usar los polinomios que est amos tratando para tales fines.
	\subsection{ Series de Fourier Chebyshev III }
	Sea $\mathrm{f}$ una funci\'on continua en $(-1,1)$, supongamos que esta tiene un desarrollo convergente en series de Fourier de la siguiente forma:
	$$
	f(x) \approx \sum_{n=0}^{\infty} c_n V_n(x),
	$$
	Donde $V_n(x)$ son los polinomios de Chebyshev de 3er tipo.
	"Nuestro problema seria encontrar los $c_n "$, con
	$$
	w(x)=\sqrt{\frac{1+x}{1-x}},
	$$
	Si multiplicamos esta expresi\'on en ambos lados por $w(x) V_m(x)$ ), (con $m<n$ fijo) e integrando entre $-1$ y 1 , (suponiendo que la integraci\'on de la serie es t\'ermino a t\'ermino), tenemos que:
	$$
	\begin{gathered}
		\int_{-1}^1 w(x) V_m(x) f(x) d x=\int_{-1}^1 \sum_{n=0}^{\infty} c_n w(x) V_m(x) V_n(x) d x=\sum_{n=0}^{\infty} c_n \int_{-1}^1 w(x) V_m(x) V_n(x) d x \\
		=c_0 \int_{-1}^1 w(x) V_m(x) V_0(x) d x+c_1 \int_{-1}^1 w(x) V_m(x) V_1(x) d x+\ldots+ \\
		c_m \int_{-1}^1 w(x) V_m(x) V_m(x) d x+\ldots+c_n \int_{-1}^1 w(x) V_m(x) V_n(x) d x
	\end{gathered}
	$$
	Por ortogonalidad,
	$$
	\int_{-1}^1 w(x) V_m(x) f(x) d x=c_m \int_{-1}^1 w(x) V_m^2(x) d x
	$$
	Haciendo un cambio $x=\cos \theta \Rightarrow d x=-\operatorname{sen} \theta d \theta$ se tiene
	$$
	\int_{-1}^1 w(x) V_m^2(x) d x=-\int_0^\pi \sqrt{\frac{1+x}{1-x}} \frac{\cos ^2\left[\left(n+\frac{1}{2}\right) \theta\right]}{\cos ^2\left(\frac{\theta}{2}\right)}(-\operatorname{sen} \theta d \theta)=2 \int_0^\pi \cos ^2\left[n+\frac{1}{2}\right] \theta d \theta=\pi
	$$
	por lo que
	$$
	\int_{-1}^1 w(x) V_m(x) f(x) d x=c_m \int_{-1}^1 w(x) V_m^2(x) d x=c_m \pi,
	$$
	$$
	c_m=\frac{1}{\pi} \int_{-1}^1 w(x) V_m(x) f(x) d x
	$$
	y de esta forma obtenemos los coeficientes.
	
	\Example{Ejemplo}{Obtenga los primeros 5 t\'erminos del desarrollo en series de Fourier-Chebyshev III de la funci\'on $f(x)=x$.}
	
	\begin{demo}
		Tenemos
		$$
		\begin{gathered}
			f(x)=x \approx \sum_{n=0}^4 c_n V_n(x) \\
			c_n=\frac{1}{\pi} \int_{-1}^1 x w(x) V_n(x) \\
			x \approx \sum_{n=0}^4 c_n V_n(x)=c_0 V_0(x)+c_1 V_1(x)+c_2 V_2(x)+c_3 V_3(x)+c_4 V_4(x)
		\end{gathered}
		$$
		donde
		Sabemos que:\\
		$ V_0(x)=1, V_1(x)=2 x-1, V_2(x)=4 x^2-2 x-1 V_3(x)=8 x^3-4 x^2-4 x+1, V_4(x)=16 x^4-8 x^3-12 x^2+4 x+1 $
		$\begin{aligned}
			& c_0=\frac{1}{\pi} \int_{-1}^1 x \sqrt{\frac{1+x}{1-x}} V_0(x) d x=\frac{1}{\pi} \int_{-1}^1 x \sqrt{\frac{1+x}{1-x}} d x=\frac{1}{\pi} \frac{\pi}{2}=\pi \\
			& c_1=\frac{1}{\pi} \int_{-1}^1 x \sqrt{\frac{1+x}{1-x}}(2 x-1) d x=\frac{1}{\pi} \int_{-1}^1\left(2 x^2-x\right) \sqrt{\frac{1+x}{1-x}} d x=\frac{2}{\pi} \int_{-1}^1 x^2 \sqrt{\frac{1+x}{1-x}} d x-\frac{1}{\pi} \int_{-1}^1 x \sqrt{\frac{1+x}{1-x}} d x \\
			& =\frac{2}{\pi}\left(\frac{\pi}{2}\right)-\frac{\pi}{2}=\frac{1}{2} \\
			& c_2=\frac{1}{\pi} \int_{-1}^1 x\left(4 x^2-2 x-1\right) \sqrt{\frac{1+x}{1-x}} d x=\frac{1}{\pi}\left(4 \int_{-1}^1 x^3 \sqrt{\frac{1+x}{1-x}} d x-2 \int_{-1}^1 x^2 \sqrt{\frac{1+x}{1-x}} d x-\int_{-1}^1 x \sqrt{\frac{1+x}{1-x}} d x\right) \\
			& =\frac{1}{\pi}\left[\frac{3 \pi}{2}-\pi-\frac{\pi}{2}\right]=0 \\
			&
		\end{aligned}
		$
		$\begin{aligned}
			& c_3=\frac{1}{\pi} \int_{-1}^1 x\left(8 x^3-4 x^2-4 x+1\right) \sqrt{\frac{1+x}{1-x}} d x=\frac{1}{\pi} \int_{-1}^1\left(8 x^4-4 x^3-4 x^2+x\right) \sqrt{\frac{1+x}{1-x}} d x \\
			& =\frac{1}{\pi}\left[8 \int_{-1}^1 x^4 \sqrt{\frac{1+x}{1-x}} d x-4 \int_{-1}^1 x^3 \sqrt{\frac{1+x}{1-x}} d x-4 \int_{-1}^1 x^2 \sqrt{\frac{1+x}{1-x}} d x+\int_{-1}^1 x \sqrt{\frac{1+x}{1-x}} d x\right] \\
			& =\frac{1}{\pi}\left[3 \pi-\frac{3 \pi}{2}-2 \pi+\frac{\pi}{2}\right]=0 \\
			& c_4=\frac{1}{\pi} \int_{-1}^1 x\left(16 x^4-8 x^3-12 x^2+4 x+1\right) \sqrt{\frac{1+x}{1-x}} d x \\
			& =\frac{1}{\pi}\left[16 \int_{-1}^1 x^5 \sqrt{\frac{1+x}{1-x}} d x-8 \int_{-1}^1 x^4 \sqrt{\frac{1+x}{1-x}} d x-12 \int_{-1}^1 x^3 \sqrt{\frac{1+x}{1-x}} d x\right] \\
			& +\frac{1}{\pi}\left[4 \int_{-1}^1 x^2 \sqrt{\frac{1+x}{1-x}} d x+\int_{-1}^1 x \sqrt{\frac{1+x}{1-x}} d x\right] \\
			& =\frac{1}{\pi}\left[5 \pi-3 \pi-\frac{9 \pi}{2}+2 \pi+\frac{\pi}{2}\right]=0 \\
			&
		\end{aligned}$
		Por lo que tenemos que:
		$$
		\begin{gathered}
			c_0=\frac{1}{2}, c_1=\frac{1}{2}, c_2=c_3=c_4 \\
			f(x)=x \approx \frac{1}{2}(1+2 x-1)=x
		\end{gathered}
		$$
		Por lo que la funci\'on aproximada es igual a la funci\'on original.
		\textcolor[rgb]{1.00,0.00,0.00}{hacer grafica}
	\end{demo}
	
	\Example{Ejemplo}{	Obtenga los primeros 5 t\'erminos del desarrollo en serie Fourier-Chebyshev III de la funci\'on $f(x)=(1-x)^{1 / 2}(1+x)^{-1 / 2}$ en $(-1,1)$.}
	
	\begin{demo}
		Tenemos
		$$
		f(x) \approx \sum_{n=0}^4 c_n V_n(x),
		$$
		donde
		$$
		\begin{gathered}
			c_n=\frac{1}{\pi} \int_{-1}^1 \sqrt{\frac{1-x}{1+x}} \sqrt{\frac{1+x}{1-x}} V_n(x) d x=\frac{1}{\pi} \int_{-1}^1 V_n(x) d x \\
			f(x) \approx \sum_{n=0}^4 c_n V_n(x)=c_0 V_0(x)+c_1 V_1(x)+c_2 V_2(x)+c_3 V_3(x)+c_4 V_4(x) \\
			c_0=\frac{1}{\pi} \int_{-1}^1 V_0(x) d x=\frac{1}{\pi}(1+1)=\frac{2}{\pi}
		\end{gathered}
		$$
		$$\begin{gathered}
			c_1=\frac{1}{\pi} \int_{-1}^1 V_1(x) d x=\frac{1}{\pi} \int_{-1}^1(2 x-1) d x=\frac{2}{\pi} \int_{-1}^1 x d x-\frac{1}{\pi} \int_{-1}^1 d x=-\frac{2}{\pi} \\
			c_2=\frac{1}{\pi} \int_{-1}^1 V_2(x) d x=\frac{1}{\pi} \int_{-1}^1\left(4 x^2-2 x-1\right) d x=\frac{2}{3 \pi} \\
			c_3=\frac{1}{\pi} \int_{-1}^1 V_3(x) d x=\frac{1}{\pi} \int_{-1}^1\left(8 x^3-4 x^2-4 x+1\right) d x=-\frac{2}{3 \pi} \\
			c_4=\frac{1}{\pi} \int_{-1}^1 V_4(x) d x=\frac{1}{\pi} \int_{-1}^1\left(16 x^4-8 x^3-12 x^2+4 x+1\right) d x=\frac{2}{5 \pi},
		\end{gathered}$$
		por lo que,
		$$
		c_0=\frac{2}{\pi}, c_1=-\frac{2}{\pi}, c_2=\frac{2}{3 \pi}, c_3=-\frac{2}{3 \pi}, c_4=\frac{2}{5 \pi}
		$$
		$$
		\sqrt{\frac{1-x}{1+x}} \approx \frac{2}{\pi}-\frac{2}{\pi}(2 x-1)+\frac{2}{3 \pi}\left(4 x^2-2 x-1\right)-\frac{2}{3 \pi}\left(8 x^3-4 x^2-4 x+1\right)+\frac{2}{5 \pi}\left(16 x^4-8 x^3-12 x^2+4 x+1\right)
		$$
		\textcolor[rgb]{1.00,0.00,0.00}{hacer la grafica}
	\end{demo}
	
	\Example{Ejemplo}{Obtenga los primeros 5 t\'erminos del desarrollo en serie Fourier-Chebyshev III de la funci\'on $f(x)=e^x$ en $(-1,1)$.}
	
	\begin{demo}
		$$\begin{gathered}
			e^x \approx \sum_{n=0}^4 c_n V_n(x) \\
			c_0=\frac{1}{\pi} \int_{-1}^1 e^x \sqrt{\frac{1+x}{1-x}} d x=\frac{5.75}{\pi} \\
			c_1=\frac{1}{\pi} \int_{-1}^1 e^x \sqrt{\frac{1+x}{1-x}}(2 x-1) d x=\frac{0.20}{\pi} \\
			c_2=\frac{1}{\pi} \int_{-1}^1 e^x \sqrt{\frac{1+x}{1-x}}\left(4 x^2-2 x-1\right) d x=\frac{0.496}{\pi} \\
			c_3=\frac{1}{\pi} \int_{-1}^1 e^x \sqrt{\frac{1+x}{1-x}}\left(8 x^3-4 x^2-4 x+1\right) d x=\frac{0.08}{\pi} \\
			c_4=\frac{1}{\pi} \int_{-1}^1 e^x \sqrt{\frac{1+x}{1-x}}\left(16 x^4-8 x^3-12 x^2+4 x+1\right) d x=\frac{0.009}{\pi}
		\end{gathered}$$
		Por lo que
		$$
		e^x \approx \frac{5.75}{\pi}+\frac{0.20}{\pi}(2 x-1)+\frac{0.496}{\pi}\left(4 x^2-2 x-1\right)+\frac{0.08}{\pi}\left(8 x^3-4 x^2-4 x+1\right)+\frac{0.009}{\pi}\left(16 x^4-8 x^3-12 x^2+4 x+1\right)
		$$
		\textcolor[rgb]{1.00,0.00,0.00}{hacer la grafica}
	\end{demo}
	
	\Example{Ejemplo}{	Obtenga los primeros 5 t\'erminos del desarrollo en serie Fourier-Chebyshev III de la funci\'on $f(x)=\operatorname{sen}(x)$ en $(-1,1)$.}
	
	\begin{demo}
		$$\begin{gathered}
			\operatorname{sen}(x) \approx \sum_{n=0}^4 c_n V_n(x) \\
			c_0=\frac{1}{\pi} \int_{-1}^1 \operatorname{sen}(x) \sqrt{\frac{1+x}{1-x}} d x=\frac{1.38}{\pi} \\
			c_1=\frac{1}{\pi} \int_{-1}^1 \operatorname{sen}(x) \sqrt{\frac{1+x}{1-x}}(2 x-1) d x=\frac{1.38}{\pi} \\
			c_2=\frac{1}{\pi} \int_{-1}^1 \operatorname{sen}(x) \sqrt{\frac{1+x}{1-x}}\left(4 x^2-2 x-1\right) d x=-\frac{0.06}{\pi} \\
			c_3=\frac{1}{\pi} \int_{-1}^1 \operatorname{sen}(x) \sqrt{\frac{1+x}{1-x}}\left(8 x^3-4 x^2-4 x+1\right) d x=-\frac{0.06}{\pi} \\
			c_4=\frac{1}{\pi} \int_{-1}^1 \operatorname{sen}(x) \sqrt{\frac{1+x}{1-x}}\left(16 x^4-8 x^3-12 x^2+4 x+1\right) d x=\frac{0.0008}{\pi}
		\end{gathered}$$
		Por lo que
		$$
		\operatorname{sen}(x) \approx \frac{1.38}{\pi}+\frac{1.38}{\pi}(2 x-1)-\frac{0.06}{\pi}\left(4 x^2-2 x-1\right)-\frac{0.06}{\pi}\left(8 x^3-4 x^2-4 x+1\right)+\frac{0.0008}{\pi}\left(16 x^4-8 x^3-12 x^2+4 x+1\right)
		$$
		\textcolor[rgb]{1.00,0.00,0.00}{hacer la grafica}
	\end{demo}
	
	\Example{Ejemplo}{Obtenga los primeros 5 t\'erminos del desarrollo en serie Fourier-Chebyshev III de la funci\'on $f(x)=\cos (x)$ en $(-1,1)$.}
	
	\begin{demo}
		$$
		\begin{gathered}
			\cos (x) \approx \sum_{n=0}^4 c_n V_n(x) \\
			c_0=\frac{1}{\pi} \int_{-1}^1 \cos (x) \sqrt{\frac{1+x}{1-x}} d x=\frac{2.40}{\pi} \\
			c_1=\frac{1}{\pi} \int_{-1}^1 \cos (x) \sqrt{\frac{1+x}{1-x}}(2 x-1) d x=-\frac{0.36}{\pi} \\
			c_2=\frac{1}{\pi} \int_{-1}^1 \cos (x) \sqrt{\frac{1+x}{1-x}}\left(4 x^2-2 x-1\right) d x=-\frac{0.36}{\pi} \\
			c_3=\frac{1}{\pi} \int_{-1}^1 \cos (x) \sqrt{\frac{1+x}{1-x}}\left(8 x^3-4 x^2-4 x+1\right) d x=\frac{0.008}{\pi} \\
			c_4=\frac{1}{\pi} \int_{-1}^1 \cos (x) \sqrt{\frac{1+x}{1-x}}\left(16 x^4-8 x^3-12 x^2+4 x+1\right) d x=\frac{0.008}{\pi}
		\end{gathered}
		$$
		Por lo que
		$$
		\cos (x) \approx \frac{2.40}{\pi}-\frac{0.36}{\pi}(2 x-1)-\frac{0.36}{\pi}\left(4 x^2-2 x-1\right)+\frac{0.008}{\pi}\left(8 x^3-4 x^2-4 x+1\right)+\frac{0.008}{\pi}\left(16 x^4-8 x^3-12 x^2+4 x+1\right)
		$$
	\end{demo}
	
	\subsection{ Series de Fourier Chebyshev IV }
	En el caso de los polinomios de Chebyshev IV podemos aproximar funciones al igual que como lo hicimos para los de 3er tipo. Si $f(x)$ es una funci\'on continua o suave a trozos podemos aproximarla de la siguiente manera:
	$$
	f(x) \approx \sum_{n=0}^{\infty} c_n W_n(x),
	$$
	Donde $W_n(x)$ son los polinomios de Chebyshev de 4to tipo $\mathrm{y}$
	$$
	\begin{gathered}
		c_n=\frac{1}{\pi} \int_{-1}^1 w(x) W_n(x) f(x) d x \\
		w(x)=\sqrt{\frac{1-x}{1+x}}
	\end{gathered}
	$$
	Veamos algunos ejemplos de funciones polinomiales, ya que cuando la funci\'on $\mathrm{f}$ es polinomial la aproximaci\'on es muy buena.
	
	\Example{Ejemplo}{	Sea $f(x)=x^2$, encuentre la aproximaci\'on Fourier-Chebyshev IV con 5 t\'erminos en $(-1,1)$}
	
	\begin{demo}
		$$
		x^2 \approx \sum_{n=0}^4 c_n W_n(x)=c_0 W_0(x)+c_1 W_1(x)+c_2 W_2(x)+c_3 W_3(x)+c_4 W_4(x)
		$$
		donde
		$$
		c_0=\frac{1}{\pi} \int_{-1}^1 \sqrt{\frac{1-x}{1+x}} W_0(x) x^2 d x=\frac{1}{\pi} \int_{-1}^1 x^2 \sqrt{\frac{1-x}{1+x}} d x=\frac{1}{\pi}\left(\frac{\pi}{2}\right)=\frac{1}{2}
		$$
		$$\begin{gathered}
			c_1=\frac{1}{\pi} \int_{-1}^1 \sqrt{\frac{1-x}{1+x}} W_1(x) x^2 d x=\frac{1}{\pi} \int_{-1}^1 x^2 \sqrt{\frac{1-x}{1+x}}(2 x+1) d x=\frac{1}{\pi}\left(2 \int_{-1}^1 x^3 \sqrt{\frac{1-x}{1+x}} d x+\int_{-1}^1 x^2 \sqrt{\frac{1-x}{1+x}} d x\right) \\
			=\frac{1}{\pi}\left(-2 \frac{3 \pi}{8}+\frac{\pi}{2}\right)=\frac{1}{2}-\frac{3}{4}=-\frac{1}{4} \\
			c_2=\frac{1}{\pi} \int_{-1}^1 \sqrt{\frac{1-x}{1+x}} W_2(x) x^2 d x=\frac{1}{\pi} \int_{-1}^1 x^2 \sqrt{\frac{1-x}{1+x}}\left(4 x^2+2 x-1\right) d x \\
			=\frac{1}{\pi}\left(4 \int_{-1}^1 x^4 \sqrt{\frac{1-x}{1+x}} d x+2 \int_{-1}^1 x^3 \sqrt{\frac{1-x}{1+x}} d x-\int_{-1}^1 x^2 \sqrt{\frac{1-x}{1+x}} d x\right) \\
			=\frac{1}{\pi}\left(4\left(\frac{3 \pi}{8}\right)+2\left(-\frac{3 \pi}{8}\right)-\frac{\pi}{2}\right)=\frac{1}{4}
		\end{gathered}$$
		
		$$\begin{gathered}
			c_3=\frac{1}{\pi} \int_{-1}^1 \sqrt{\frac{1-x}{1+x}} W_3(x) x^2 d x=\frac{1}{\pi} \int_{-1}^1 x^2 \sqrt{\frac{1-x}{1+x}}\left(8 x^3+4 x^2-4 x-1\right) d x\\
			c_3=\frac{1}{\pi}\left(8 \int_{-1}^1 x^5 \sqrt{\frac{1-x}{1+x}} d x+4 \int_{-1}^1 x^4 \sqrt{\frac{1-x}{1+x}} d x-4 \int_{-1}^1 x^3 \sqrt{\frac{1-x}{1+x}} d x-\int_{-1}^1 x^2 \sqrt{\frac{1-x}{1+x}} d x\right) \\
			=\frac{1}{\pi}\left(8\left(-\frac{5 \pi}{16}\right)+4\left(\frac{3 \pi}{8}\right)-4\left(-\frac{3 \pi}{8}\right)-\frac{\pi}{2}\right)=0
		\end{gathered}$$
		$$\begin{gathered}
			c_4=\frac{1}{\pi} \int_{-1}^1 \sqrt{\frac{1-x}{1+x}} W_4(x) x^2 d x=\frac{1}{\pi} \int_{-1}^1 \sqrt{\frac{1-x}{1+x}}\left(16 x^4+8 x^3-12 x^2-4 x+1\right) x^2 d x \\
			=\frac{1}{\pi}\left(16 \int_{-1}^1 x^6 \sqrt{\frac{1-x}{1+x}} d x+8 \int_{-1}^1 x^5 \sqrt{\frac{1-x}{1+x}} d x-12 \int_{-1}^1 x^4 \sqrt{\frac{1-x}{1+x}} d x-4 \int_{-1}^1 x^3 \sqrt{\frac{1-x}{1+x}} d x\right) \\
			+\frac{1}{\pi}\left(\int_{-1}^1 x^2 \sqrt{\frac{1-x}{1+x}} d x\right) \\
			=\frac{1}{\pi}\left(16\left(\frac{5 \pi}{16}\right)+8\left(-\frac{5 \pi}{16}\right)-12\left(\frac{3 \pi}{8}\right)+4\left(\frac{3 \pi}{8}\right)+\left(\frac{\pi}{2}\right)\right)=0 \\
		\end{gathered}$$
		Por lo que
		$$
		c_0=\frac{1}{2}, \quad c_1=-\frac{1}{4}, \quad c_2=\frac{1}{4}, c_3=c_4=0
		$$
		As\'i tenemos:
		$$
		\begin{gathered}
			x^2 \approx c_0 W_0(x)+c_1 W_1(x)+c_2 W_2(x)+c_3 W_3(x)+c_4 W_4(x) \\
			=\frac{1}{2}-\frac{1}{4}(2 x+1)+\frac{1}{4}\left(4 x^2+2 x-1\right)=x^2
		\end{gathered}
		$$
		\textcolor[rgb]{1.00,0.00,0.00}{hacer la grafica}\\
		Por lo que se tiene que la aproximaci\'on es exacta. A demas se puede observar que las constantes $c_n$ se hacen cero cuando $n+1$ es mayor que el grado del polinomio que se esta aproximando.
	\end{demo}
	
	\Example{Ejemplo}{Sea $f(x)=5 x^3-3 x+8$, encuentre la aproximaci\'on Fourier-Chebyshev IV con 4 t\'erminos en $(-1,1)$}
	
	
	\begin{demo}
		$$
		5 x^3-3 x+8 \approx \sum_{n=0}^3 c_n W_n(x)=c_0 W_0(x)+c_1 W_1(x)+c_2 W_2(x)+c_3 W_3(x),
		$$
		donde
		$$
		\begin{gathered}
			c_0=\frac{1}{\pi} \int_{-1}^1 \sqrt{\frac{1-x}{1+x}} W_0(x)\left(5 x^3-3 x+8\right) d x=\frac{1}{\pi} \int_{-1}^1\left(5 x^3-3 x+8\right) \sqrt{\frac{1-x}{1+x}} d x \\
			=\frac{1}{\pi}\left(5 \int_{-1}^1 x^3 \sqrt{\frac{1-x}{1+x}} d x-3 \int_{-1}^1 x \sqrt{\frac{1-x}{1+x}} d x+8 \int_{-1}^1 \sqrt{\frac{1-x}{1+x}} d x\right) \\
			=\frac{1}{\pi}\left(-5\left(\frac{3 \pi}{8}\right)+3\left(\frac{\pi}{2}\right)+8(\pi)\right)=\frac{37}{8}
		\end{gathered}
		$$
		$$\begin{gathered}
			c_1=\frac{1}{\pi} \int_{-1}^1 \sqrt{\frac{1-x}{1+x}} W_1(x)\left(5 x^3-3 x+8\right) d x=\frac{1}{\pi} \int_{-1}^1\left(5 x^3-3 x+8\right)(2 x+1) \sqrt{\frac{1-x}{1+x}} d x \\
			=\frac{1}{\pi}\left(\int_{-1}^1\left(10 x^4+5 x^3-6 x^2+13 x+8\right) \sqrt{\frac{1-x}{1+x}} d x\right) \\
			=\frac{1}{\pi}\left(10 \int_{-1}^1 x^4 \sqrt{\frac{1-x}{1+x}} d x+5 \int_{-1}^1 x^3 \sqrt{\frac{1-x}{1+x}} d x-6 \int_{-1}^1 x^2 \sqrt{\frac{1-x}{1+x}} d x\right) \\
			+\left(13 \int_{-1}^1 x \sqrt{\frac{1-x}{1+x}} d x+8 \int_{-1}^1 \sqrt{\frac{1-x}{1+x}} d x\right) \\
			=\frac{1}{\pi}\left(10\left(\frac{3 \pi}{8}\right)+5\left(-\frac{3 \pi}{8}\right)-6\left(\frac{\pi}{2}\right)+13\left(-\frac{\pi}{2}\right)+8\left(-\frac{\pi}{2}\right)\right)=-\frac{93}{8}
		\end{gathered}$$
		$$
		\begin{aligned}
			c_2=\frac{1}{\pi} \int_{-1}^1 \sqrt{\frac{1-x}{1+x}}\left(4 x^2+2 x-1\right)\left(5 x^3-3 x+8\right) d x & =\frac{1}{\pi} \int_{-1}^1\left(20 x^5+10 x^4-17 x^3+26 x^2+19 x-8\right) \sqrt{\frac{1-x}{1+x}} d x \\
			=\frac{20}{\pi} \int_{-1}^1 x^5 \sqrt{\frac{1-x}{1+x}} d x+ & \frac{10}{\pi} \int_{-1}^1 x^4 \sqrt{\frac{1-x}{1+x}} d x-\frac{17}{\pi} \int_{-1}^1 x^3 \sqrt{\frac{1-x}{1+x}} d x+\frac{26}{\pi} \int_{-1}^1 x^2 \sqrt{\frac{1-x}{1+x}} d x \\
			& +\frac{19}{\pi} \int_{-1}^1 x \sqrt{\frac{1-x}{1+x}} d x+\frac{8}{\pi} \int_{-1}^1 \sqrt{\frac{1-x}{1+x}} d x \\
			& =-\frac{100}{16}+\frac{30}{8}+\frac{51}{8}+\frac{26}{2}-\frac{19}{2}-8=-\frac{57}{8}
		\end{aligned}$$
		$$
		\begin{gathered}
			c_3=\frac{1}{\pi} \int_{-1}^1 \sqrt{\frac{1-x}{1+x}}\left(8 x^3+4 x^2-2 x-1\right)\left(5 x^3-3 x+8\right) d x\\
			=\frac{1}{\pi} \int_{-1}^1\left(40 x^6+20 x^5-34 x^4+47 x^3+38 x^2-13 x-8\right) \sqrt{\frac{1-x}{1+x}} d x \\
			=\frac{40}{\pi} \int_{-1}^1 x^6 \sqrt{\frac{1-x}{1+x}} d x+\frac{20}{\pi} \int_{-1}^1 x^5 \sqrt{\frac{1-x}{1+x}} d x-\frac{34}{\pi} \int_{-1}^1 x^4 \sqrt{\frac{1-x}{1+x}} d x \\
			+\frac{47}{\pi} \int_{-1}^1 x^3 \sqrt{\frac{1-x}{1+x}} d x+\frac{38}{\pi} \int_{-1}^1 x^2 \sqrt{\frac{1-x}{1+x}} d x-\frac{13}{\pi} \int_{-1}^1 x \sqrt{\frac{1-x}{1+x}} d x-\frac{8}{\pi} \int_{-1}^1 \sqrt{\frac{1-x}{1+x}} d x \\
			=40\left(\frac{5}{16}\right)-\frac{100}{16}-\frac{51}{4}-47\left(\frac{3}{8}\right)+19+\frac{13}{2}-8=\frac{47}{8}
		\end{gathered}
		$$
		$$
		\begin{gathered}
			c_4=\frac{1}{\pi} \int_{-1}^1 \sqrt{\frac{1-x}{1+x}}\left(16 x^4+8 x^3-12 x^2-4 x+1\right)\left(5 x^3-3 x+8\right) d x \\
			=\frac{1}{\pi} \int_{-1}^1\left(80 x^7+40 x^6-108 x^5+84 x^4+105 x^3-84 x^2-35 x+8\right) \sqrt{\frac{1-x}{1+x}} d x \\
			=\frac{80}{\pi} \int_{-1}^1 x^7 \sqrt{\frac{1-x}{1+x}} d x+\frac{40}{\pi} \int_{-1}^1 x^6 \sqrt{\frac{1-x}{1+x}} d x-\frac{108}{\pi} \int_{-1}^1 x^5 \sqrt{\frac{1-x}{1+x}} d x \\
			+\frac{84}{\pi} \int_{-1}^1 x^4 \sqrt{\frac{1-x}{1+x}} d x+\frac{105}{\pi} \int_{-1}^1 x^3 \sqrt{\frac{1-x}{1+x}} d x-\frac{84}{\pi} \int_{-1}^1 x^2 \sqrt{\frac{1-x}{1+x}} d x \\
			\quad-\frac{35}{\pi} \int_{-1}^1 x \sqrt{\frac{1-x}{1+x}} d x+\frac{8}{\pi} \int_{-1}^1 \sqrt{\frac{1-x}{1+x}} d x=0
		\end{gathered}
		$$
		Por lo que
		$$
		c_0=\frac{37}{8}, \quad c_1=-\frac{93}{8}, \quad c_2=-\frac{57}{8}, c_3=\frac{47}{8}, c_4=0
		$$
		As\'i tenemos:
		$$
		\begin{gathered}
			5 x^3-3 x+8 \approx c_0 W_0(x)+c_1 W_1(x)+c_2 W_2(x)+c_3 W_3(x)+c_4 W_4(x) \\
			=\frac{37}{8}-\frac{93}{8}(2 x+1)-\frac{57}{8}\left(4 x^2+2 x-1\right)+\frac{47}{8}\left(8 x^3+4 x^2-4 x-1\right)=5 x^3-3 x+8
		\end{gathered}
		$$
		Por lo que se tiene que la aproximaci\'on es exacta.
	\end{demo}
	
	\Example{Ejemplo}{Sea $f(x)=\tan (x)$, encuentre la aproximaci\'on Fourier-Chebyshev IV con 5 t\'erminos en $(-1,1)$}
	
	\begin{demo}
		$$ \begin{aligned}
			& \qquad \tan (x) \approx \sum_{n=0}^4 c_n W_n(x)=c_0 W_0(x)+c_1 W_1(x)+c_2 W_2(x)+c_3 W_3(x)+c_4 W_4(x) \\
			& c_0=\frac{1}{\pi} \int_{-1}^1 \tan (x) \sqrt{\frac{1-x}{1+x}} d x=-\frac{2.17}{\pi} \\
			& \text { donde } \\
			& c_1=\frac{1}{\pi} \int_{-1}^1 \tan (x) \sqrt{\frac{1-x}{1+x}}(2 x+1) d x=\frac{2.17}{\pi} \\
			& c_2=\frac{1}{\pi} \int_{-1}^1 \tan (x) \sqrt{\frac{1-x}{1+x}}\left(4 x^2+2 x-1\right) d x=-\frac{0.243}{\pi} \\
			& c_3=\frac{1}{\pi} \int_{-1}^1 \tan (x) \sqrt{\frac{1-x}{1+x}}\left(8 x^3+4 x^2-4 x-1\right) d x=\frac{0.243}{\pi} \\
			& c_4=\frac{1}{\pi} \int_{-1}^1 \tan (x) \sqrt{\frac{1-x}{1+x}}\left(16 x^4+8 x^3-12 x^2-4 x+1\right) d x=-\frac{0.03}{\pi} \\
			& \text { As\'i tenemos: } \tan (x) \approx c_0 W_0(x)+c_1 W_1(x)+c_2 W_2(x)+c_3 W_3(x)+c_4 W_4(x) \\
			& =-\frac{2.17}{\pi}+\frac{2.17}{\pi}(2 x+1)-\frac{0.243}{\pi}\left(4 x^2+2 x-1\right)+\frac{0.243}{\pi}\left(8 x^3+4 x^2-4 x-1\right)-\frac{0.03}{\pi}\left(16 x^4+8 x^3-12 x^2-4 x+1\right)
		\end{aligned}$$
	\end{demo}
	
	\Example{Ejemplo}{Sea $f(x)=5^x-4 \cos (x)$, encuentre la aproximaci\'on Fourier-Chebyshev IV con 5 t\'erminos en $(-1,1)$}
	
	\begin{demo}
		$$
		5^x-4 \cos (x) \approx \sum_{n=0}^4 c_n W_n(x)=c_0 W_0(x)+c_1 W_1(x)+c_2 W_2(x)+c_3 W_3(x)+c_4 W_4(x),
		$$
		donde
		$$
		\begin{gathered}
			c_0=\frac{1}{\pi} \int_{-1}^1\left(5^x-4 \cos (x)\right) \sqrt{\frac{1-x}{1+x}} d x=-\frac{7.53}{\pi} \\
			c_1=\frac{1}{\pi} \int_{-1}^1\left(5^x-4 \cos (x)\right) \sqrt{\frac{1-x}{1+x}}(2 x+1) d x=\frac{0.74}{\pi} \\
			c_2=\frac{1}{\pi} \int_{-1}^1\left(5^x-4 \cos (x)\right) \sqrt{\frac{1-x}{1+x}}\left(4 x^2+2 x-1\right) d x=\frac{2.379}{\pi} \\
			c_3=\frac{1}{\pi} \int_{-1}^1\left(5^x-4 \cos (x)\right) \sqrt{\frac{1-x}{1+x}}\left(8 x^3+4 x^2-4 x-1\right) d x=\frac{0.289}{\pi} \\
			c_4=\frac{1}{\pi} \int_{-1}^1\left(5^x-4 \cos (x)\right) \sqrt{\frac{1-x}{1+x}}\left(16 x^4+8 x^3-12 x^2-4 x+1\right) d x=\frac{0.02}{\pi}
		\end{gathered}
		$$
		As\'i tenemos:
		$$
		\begin{gathered}
			5^x-4 \cos (x) \approx c_0 W_0(x)+c_1 W_1(x)+c_2 W_2(x)+c_3 W_3(x)+c_4 W_4(x) \\
			=-\frac{7.53}{\pi}+\frac{0.74}{\pi}(2 x+1)+\frac{2.379}{\pi}\left(4 x^2+2 x-1\right)+\frac{0.289}{\pi}\left(8 x^3+4 x^2-4 x-1\right)+\frac{0.02}{\pi}\left(16 x^4+8 x^3-12 x^2-4 x+1\right)
		\end{gathered}
		$$
		\textcolor[rgb]{1.00,0.00,0.00}{hacer grafica}
	\end{demo}
	
	\Example{Ejemplo}{Sea $f(x)=\sqrt{1-x^2}$, encuentre la aproximaci\'on Fourier-Chebyshev IV con 5 t\'erminos en $(-1,1)$}
	
	\begin{demo}
		$$
		\begin{gathered}
			\sqrt{1-x^2} \approx \sum_{n=0}^4 c_n W_n(x)=c_0 W_0(x)+c_1 W_1(x)+c_2 W_2(x)+c_3 W_3(x)+c_4 W_4(x), \\
			\left.c_0=\frac{1}{\pi} \int_{-1}^1 \sqrt{1-x^2}\right) \sqrt{\frac{1-x}{1+x}} d x=\frac{2}{\pi} \\
			\text { donde } \\
			c_1=\frac{1}{\pi} \int_{-1}^1 \sqrt{1-x^2} \sqrt{\frac{1-x}{1+x}}(2 x+1) d x=\frac{0.67}{\pi} \\
			c_2=\frac{1}{\pi} \int_{-1}^1 \sqrt{1-x^2} \sqrt{\frac{1-x}{1+x}}\left(4 x^2+2 x-1\right) d x=-\frac{0.67}{\pi} \\
			c_3=\frac{1}{\pi} \int_{-1}^1 \sqrt{1-x^2} \sqrt{\frac{1-x}{1+x}}\left(8 x^3+4 x^2-4 x-1\right) d x=\frac{0.13}{\pi} \\
			c_4=\frac{1}{\pi} \int_{-1}^1 \sqrt{1-x^2} \sqrt{\frac{1-x}{1+x}}\left(16 x^4+8 x^3-12 x^2-4 x+1\right) d x=-\frac{0.13}{\pi}
		\end{gathered}
		$$
		As\'i tenemos:
		$$
		\begin{gathered}
			\sqrt{1-x^2} \approx c_0 W_0(x)+c_1 W_1(x)+c_2 W_2(x)+c_3 W_3(x)+c_4 W_4(x) \\
			=\frac{2}{\pi}+\frac{0.67}{\pi}(2 x+1)-\frac{0.67}{\pi}\left(4 x^2+2 x-1\right)+\frac{0.13}{\pi}\left(8 x^3+4 x^2-4 x-1\right)-\frac{0.13}{\pi}\left(16 x^4+8 x^3-12 x^2-4 x+1\right)
		\end{gathered}
		$$
	\end{demo}
	
	\Example{Ejemplo}{Existe una funci\'on, denominada la funci\'on del amor, puest o que al graficarla, la gr\'afica toma la forma de un coraz\'on. Sea $f_1(x)=\left(x^2\right)^{1 / 3}+\sqrt{1-x^2}$ y $f_2(x)=\left(x^2\right)^{1 / 3}-\sqrt{1-x^2}$, si graficamos estas dos funciones en el mismo plano, tenemos:\\
		\textcolor[rgb]{1.00,0.00,0.00}{hacer la grafica}\\
		encuentre la aproximaci\'on Fourier-Chebyshev IV con 5 t\'erminos en $(-1,1)$ de ambas funciones y su gr\'afica en el mismo plano.}
	
	\begin{demo}
		Para $f_1(x)$
		$$
		\left(x^2\right)^{1 / 3}+\sqrt{1-x^2} \approx \sum_{n=0}^4 c_n W_n(x)=c_0 W_0(x)+c_1 W_1(x)+c_2 W_2(x)+c_3 W_3(x)+c_4 W_4(x),
		$$
		donde
		$$\begin{gathered}
			c_0=\frac{1}{\pi} \int_{-1}^1\left(\left(x^2\right)^{1 / 3}+\sqrt{1-x^2}\right) \sqrt{\frac{1-x}{1+x}} d x=\frac{4.24}{\pi} \\
			c_1=\frac{1}{\pi} \int_{-1}^1\left(\left(x^2\right)^{1 / 3}+\sqrt{1-x^2}\right) \sqrt{\frac{1-x}{1+x}}(2 x+1) d x=\frac{0.107}{\pi} \\
			c_2=\frac{1}{\pi} \int_{-1}^1\left(\left(x^2\right)^{1 / 3}+\sqrt{1-x^2}\right) \sqrt{\frac{1-x}{1+x}}\left(4 x^2+2 x-1\right) d x=-\frac{0.107}{\pi} \\
			c_3=\frac{1}{\pi} \int_{-1}^1\left(\left(x^2\right)^{1 / 3}+\sqrt{1-x^2}\right) \sqrt{\frac{1-x}{1+x}}\left(8 x^3+4 x^2-4 x-1\right) d x=\frac{0.29}{\pi} \\
			c_4=\frac{1}{\pi} \int_{-1}^1\left(\left(x^2\right)^{1 / 3}+\sqrt{1-x^2}\right) \sqrt{\frac{1-x}{1+x}}\left(16 x^4+8 x^3-12 x^2-4 x+1\right) d x=-\frac{0.29}{\pi}
		\end{gathered}$$
		As\'i tenemos:
		$$
		\begin{gathered}
			\left(x^2\right)^{1 / 3}+\sqrt{1-x^2} \approx c_0 W_0(x)+c_1 W_1(x)+c_2 W_2(x)+c_3 W_3(x)+c_4 W_4(x) \\
			=\frac{4.24}{\pi}+\frac{0.107}{\pi}(2 x+1)-\frac{0.107}{\pi}\left(4 x^2+2 x-1\right)+\frac{0.29}{\pi}\left(8 x^3+4 x^2-4 x-1\right)-\frac{0.29}{\pi}\left(16 x^4+8 x^3-12 x^2-4 x+1\right)
		\end{gathered}
		$$
		Para $f_2(x)$
		$$
		\left(x^2\right)^{1 / 3}-\sqrt{1-x^2} \approx \sum_{n=0}^4 c_n W_n(x)=c_0 W_0(x)+c_1 W_1(x)+c_2 W_2(x)+c_3 W_3(x)+c_4 W_4(x) \text {, }
		$$
		donde
		$$\begin{gathered}
			c_0=\frac{1}{\pi} \int_{-1}^1\left(\left(x^2\right)^{1 / 3}-\sqrt{1-x^2}\right) \sqrt{\frac{1-x}{1+x}} d x=\frac{0.24}{\pi} \\
			c_1=\frac{1}{\pi} \int_{-1}^1\left(\left(x^2\right)^{1 / 3}-\sqrt{1-x^2}\right) \sqrt{\frac{1-x}{1+x}}(2 x+1) d x=-\frac{1.27}{\pi} \\
			c_2=\frac{1}{\pi} \int_{-1}^1\left(\left(x^2\right)^{1 / 3}-\sqrt{1-x^2}\right) \sqrt{\frac{1-x}{1+x}}\left(4 x^2+2 x-1\right) d x=\frac{1.27}{\pi} \\
			c_3=\frac{1}{\pi} \int_{-1}^1\left(\left(x^2\right)^{1 / 3}-\sqrt{1-x^2}\right) \sqrt{\frac{1-x}{1+x}}\left(8 x^3+4 x^2-4 x-1\right) d x=\frac{0.027}{\pi} \\
			c_4=\frac{1}{\pi} \int_{-1}^1\left(\left(x^2\right)^{1 / 3}-\sqrt{1-x^2}\right) \sqrt{\frac{1-x}{1+x}}\left(16 x^4+8 x^3-12 x^2-4 x+1\right) d x=-\frac{0.027}{\pi}
		\end{gathered}$$
		As\'i tenemos:
		$$
		\begin{gathered}
			\left(x^2\right)^{1 / 3}-\sqrt{1-x^2} \approx c_0 W_0(x)+c_1 W_1(x)+c_2 W_2(x)+c_3 W_3(x)+c_4 W_4(x) \\
			=\frac{0.24}{\pi}-\frac{1.27}{\pi}(2 x+1)+\frac{1.27}{\pi}\left(4 x^2+2 x-1\right)+\frac{0.027}{\pi}\left(8 x^3+4 x^2-4 x-1\right)-\frac{0.027}{\pi}\left(16 x^4+8 x^3-12 x^2-4 x+1\right)
		\end{gathered}
		$$
		Si graficamos estas dos expresiones en el mismo plano mostrado anteriormente tenemos
		\textcolor[rgb]{1.00,0.00,0.00}{hacer grafica con la aproximacion}
	\end{demo}
	
	\Example{Ejemplo}{Sea $f(x)=|x|$, encuentre la aproximaci\'on Fourier-Chebyshev IV con 5 t\'erminos en $(-1,1)$}
	
	\begin{demo}
		$$
		|x| \approx \sum_{n=0}^4 c_n W_n(x)=c_0 W_0(x)+c_1 W_1(x)+c_2 W_2(x)+c_3 W_3(x)+c_4 W_4(x),
		$$
		donde
		$$\begin{gathered}
			c_0=\frac{1}{\pi} \int_{-1}^1|x| \sqrt{\frac{1-x}{1+x}} d x=\frac{2}{\pi} \\
			c_1=\frac{1}{\pi} \int_{-1}^1|x| \sqrt{\frac{1-x}{1+x}}(2 x+1) d x=-\frac{0.07}{\pi} \\
			c_2=\frac{1}{\pi} \int_{-1}^1|x| \sqrt{\frac{1-x}{1+x}}\left(4 x^2+2 x-1\right) d x=\frac{0.07}{\pi} \\
			c_3=\frac{1}{\pi} \int_{-1}^1|x| \sqrt{\frac{1-x}{1+x}}\left(8 x^3+4 x^2-4 x-1\right) d x=\frac{0.13}{\pi} \\
			c_4=\frac{1}{\pi} \int_{-1}^1|x| \sqrt{\frac{1-x}{1+x}}\left(16 x^4+8 x^3-12 x^2-4 x+1\right) d x=-\frac{0.13}{\pi}
		\end{gathered}$$
		As\'i tenemos:
		$$
		\begin{gathered}
			|x| \approx c_0 W_0(x)+c_1 W_1(x)+c_2 W_2(x)+c_3 W_3(x)+c_4 W_4(x) \\
			=\frac{2}{\pi}-\frac{0.07}{\pi}(2 x+1)+\frac{0.07}{\pi}\left(4 x^2+2 x-1\right)+\frac{0.13}{\pi}\left(8 x^3+4 x^2-4 x-1\right)-\frac{0.13}{\pi}\left(16 x^4+8 x^3-12 x^2-4 x+1\right)
		\end{gathered}
		$$
		\textcolor[rgb]{1.00,0.00,0.00}{hacer grafica}
	\end{demo}
	
	\Example{Ejemplo}{	Sea $f(x)=\operatorname{sen}\left(\frac{x}{2}\right)+3^x$, encuentre la aproximaci\'on Fourier-Chebyshev IV con 5 t\'erminos en $(-1,1)$}
	
	\begin{demo}
		$$
		\operatorname{sen}\left(\frac{x}{2}\right)+3^x \approx \sum_{n=0}^4 c_n W_n(x)=c_0 W_0(x)+c_1 W_1(x)+c_2 W_2(x)+c_3 W_3(x)+c_4 W_4(x),
		$$
		donde
		$$
		\begin{gathered}
			c_0=\frac{1}{\pi} \int_{-1}^1\left(\operatorname{sen}\left(\frac{x}{2}\right)+3^x\right) \sqrt{\frac{1-x}{1+x}} d x=\frac{1.4}{\pi} \\
			c_1=\frac{1}{\pi} \int_{-1}^1\left(\operatorname{sen}\left(\frac{x}{2}\right)+3^x\right) \sqrt{\frac{1-x}{1+x}}(2 x+1) d x=\frac{2.24}{\pi} \\
			c_2=\frac{1}{\pi} \int_{-1}^1\left(\operatorname{sen}\left(\frac{x}{2}\right)+3^x\right) \sqrt{\frac{1-x}{1+x}}\left(4 x^2+2 x-1\right) d x=\frac{0.44}{\pi} \\
			c_3=\frac{1}{\pi} \int_{-1}^1\left(\operatorname{sen}\left(\frac{x}{2}\right)+3^x\right) \sqrt{\frac{1-x}{1+x}}\left(8 x^3+4 x^2-4 x-1\right) d x=\frac{0.07}{\pi} \\
			c_4=\frac{1}{\pi} \int_{-1}^1\left(\operatorname{sen}\left(\frac{x}{2}\right)+3^x\right) \sqrt{\frac{1-x}{1+x}}\left(16 x^4+8 x^3-12 x^2-4 x+1\right) d x=\frac{0.01}{\pi}
		\end{gathered}
		$$
		As\'i tenemos:
		$$
		\begin{gathered}
			\operatorname{sen}\left(\frac{x}{2}\right)+3^x \approx c_0 W_0(x)+c_1 W_1(x)+c_2 W_2(x)+c_3 W_3(x)+c_4 W_4(x) \\
			=\frac{1.4}{\pi}+\frac{2.24}{\pi}(2 x+1)+\frac{0.44}{\pi}\left(4 x^2+2 x-1\right)+\frac{0.07}{\pi}\left(8 x^3+4 x^2-4 x-1\right)+\frac{0.01}{\pi}\left(16 x^4+8 x^3-12 x^2-4 x+1\right)
		\end{gathered}
		$$
		\textcolor[rgb]{1.00,0.00,0.00}{hacer grafica}
	\end{demo}
	
	
	\Example{Ejemplo}{Sea $f(x)=\cosh (x)$, encuentre la aproximaci\'on Fourier-Chebyshev IV con 5 t\'erminos en $(-1,1)$}
	
	\begin{demo}
		$$\begin{aligned}
			& \qquad \cosh (x) \approx \sum_{n=0}^4 c_n W_n(x)=c_0 W_0(x)+c_1 W_1(x)+c_2 W_2(x)+c_3 W_3(x)+c_4 W_4(x) \\
			& \text { donde } \\
			& c_0=\frac{1}{\pi} \int_{-1}^1 \cosh (x) \sqrt{\frac{1-x}{1+x}} d x=\frac{3.98}{\pi} \\
			& c_1=\frac{1}{\pi} \int_{-1}^1 \cosh (x) \sqrt{\frac{1-x}{1+x}}(2 x+1) d x=-\frac{0.426}{\pi} \\
			& c_2=\frac{1}{\pi} \int_{-1}^1 \cosh (x) \sqrt{\frac{1-x}{1+x}}\left(4 x^2+2 x-1\right) d x=\frac{0.426}{\pi} \\
			& c_3=\frac{1}{\pi} \int_{-1}^1 \cosh (x) \sqrt{\frac{1-x}{1+x}}\left(8 x^3+4 x^2-4 x-1\right) d x=-\frac{0.009}{\pi} \\
			& c_4=\frac{1}{\pi} \int_{-1}^1 \cosh (x) \sqrt{\frac{1-x}{1+x}}\left(16 x^4+8 x^3-12 x^2-4 x+1\right) d x=\frac{0.009}{\pi} \\
			& \cosh (x) \approx c_0 W_0(x)+c_1 W_1(x)+c_2 W_2(x)+c_3 W_3(x)+c_4 W_4(x) \\
			& \text { As\'i tenemos: } \\
			& =\frac{3.98}{\pi}-\frac{0.426}{\pi}(2 x+1)+\frac{0.426}{\pi}\left(4 x^2+2 x-1\right)-\frac{0.009}{\pi}\left(8 x^3+4 x^2-4 x-1\right)+\frac{0.009}{\pi}\left(16 x^4+8 x^3-12 x^2-4 x+1\right)
		\end{aligned}$$
	\end{demo}
	
	\Example{Ejemplo}{Sea $f(x)=\operatorname{senh}(x)$, encuentre la aproximaci\'on Fourier-Chebyshev IV con 5 t\'erminos en $(-1,1)$}
	
	\begin{demo}
		$$\begin{aligned}
			& \qquad \operatorname{senh}(x) \approx \sum_{n=0}^4 c_n W_n(x)=c_0 W_0(x)+c_1 W_1(x)+c_2 W_2(x)+c_3 W_3(x)+c_4 W_4(x) \\
			& c_0=\frac{1}{\pi} \int_{-1}^1 \operatorname{senh}(x) \sqrt{\frac{1-x}{1+x}} d x=-\frac{1.78}{\pi} \\
			& \text { donde } \\
			& c_1=\frac{1}{\pi} \int_{-1}^1 \operatorname{senh}(x) \sqrt{\frac{1-x}{1+x}}(2 x+1) d x=\frac{1.78}{\pi} \\
			& c_2=\frac{1}{\pi} \int_{-1}^1 \operatorname{senh}(x) \sqrt{\frac{1-x}{1+x}}\left(4 x^2+2 x-1\right) d x=-\frac{0.07}{\pi} \\
			& c_3=\frac{1}{\pi} \int_{-1}^1 \operatorname{senh}(x) \sqrt{\frac{1-x}{1+x}}\left(8 x^3+4 x^2-4 x-1\right) d x=\frac{0.07}{\pi} \\
			& c_4=\frac{1}{\pi} \int_{-1}^1 \operatorname{senh}(x) \sqrt{\frac{1-x}{1+x}}\left(16 x^4+8 x^3-12 x^2-4 x+1\right) d x=-\frac{8.5(10)^{-4}}{\pi} \\
			& \operatorname{senh}(x) \approx c_0 W_0(x)+c_1 W_1(x)+c_2 W_2(x)+c_3 W_3(x)+c_4 W_4(x) \\
			& \text { As\'i tenemos: } \\
			& =-\frac{1.78}{\pi}+\frac{1.78}{\pi}(2 x+1)-\frac{0.07}{\pi}\left(4 x^2+2 x-1\right)+\frac{0.07}{\pi}\left(8 x^3+4 x^2-4 x-1\right)-\frac{8.5(10)^{-4}}{\pi}\left(16 x^4+8 x^3-12 x^2-4 x+1\right)
		\end{aligned}$$
	\end{demo}
	
	\Example{Ejemplo}{	Sea $f(x)=|x|$
		$$
		f(x)=\sum_{n=0}^{\infty} c_n p_n(x)\quad-1<x<1
		$$}
	
	\begin{demo}
		Para obtener la representaci\'on haremos uso de (Demostrar la identidad)
		\textcolor{blue}{
			$$
			P_{n+1}^{\prime}(x)-P_{n-1}^{\prime}(x)=(2 n+1) P_{n}(x), \quad n=1,2, \ldots  \quad (6)
			$$
		}
		$$
		\begin{gathered}
			f(x)=|x| \text { es una funci\'on par } \\
			f(x) \sim \displaystyle\sum_{n=0}^{\infty} a_{n} P_{n}(x),\quad-1<x<1
		\end{gathered}
		$$
		
		\[
		a_{n}=\left(n+\frac{1}{2}\right) \displaystyle\int_{-1}^{1} f(x) P_{n}(x) d x, \quad n=0,1,2, \ldots
		\]
		Se puede ver que si n es un n es un n\'umero par, entonces los polinomios de Legendre seran pares ya que el producto de funciones pares es par, los coeficientes se cancelaran para el caso que sea par, por esta raz\'on tendremos los coeficientes para el caso impar.\\ we can see that if n is an odd number, then the Legendre polynomials will be odd and since the product of an even function by an odd function is odd, the coefficients will cancel for the odd case, for this reason, we will only have coefficients left for the even case .
		$$
		\begin{aligned}
			&\text { That is, }\quad -P_{2 n+1}(x)=P_{2 n+1}(-x)\quad  \text { are odd functions } \\
			&f(-x)=f(x)\quad  \text { is an even function  } \\
			&\Rightarrow \quad f(-x) P_{2 n+1}(-x)=-f(x) P_{2 n+1}(x)\quad  \text {are odd functions }
		\end{aligned}
		$$
		
		
		therefore $$
		c_{2 n+1}=0 \quad \forall n \geq 0
		$$
		Thus,
		
		$$
		\begin{aligned}
			&c_{2 n}=\left(2 n+\frac{1}{2}\right) \displaystyle\int_1^1|x| P_{2 n}(x) d x ; \quad P_{2 n}(-x)=P_{2 n}(x) \\
			&c_{2 n}=2\left(2 n+\frac{1}{2}\right) \displaystyle\int_0^1 x P_{2 n}(x) d x
		\end{aligned}
		$$
		
		$$
		\begin{aligned}
			&c_{2 n}=(4 n+1) \displaystyle\int_0^1 x P_{2 n}(x) d x \\
			&\frac{c_{2 n}}{(4 n+1)}=\displaystyle\int_0^1 x P_{2 n}(x) d x
		\end{aligned}
		$$
		$c_0 =\displaystyle\int_0^1 x P_{0}(x) d x=\frac{1}{2},\qquad c_2= 5 \int_0^1 \frac{1}{2} x\left(-1+3 x^2\right) d x=\frac{5}{8}$
		
		integrating by parts
		$$
		\begin{array}{rl}
			u=x & d v=P_{2 n}(x) d x \\
			d u=d x & \displaystyle\int d v=\displaystyle\int P_{2 n}(x) d x, \quad  v=\displaystyle\int P_{2 n}(x) d x
		\end{array}
		$$
		we know that:
		$$
		P_n(x)=\frac{1}{2 n+1}\left[P_{n+1}^{\prime}(x)-P_{n-1}^{\prime}(x)\right] ; n=1,2,3,\dots
		$$
		so,
		$$
		P_{2 n}(x)=\frac{1}{4 n+1}\left[P_{2 n+1}^{\prime}(x)-P_{2 n-1}^{\prime}(x)\right] ; n=2,3,\dots
		$$
		then,
		$$
		\begin{aligned}
			&v=\displaystyle\int P_{2 n}(x) d x \\
			&v=\frac{1}{(4 n+1)} \displaystyle\int\left[P_{2 n+1}^{\prime}(x)-P_{2 n-1}^{\prime}(x)\right] d x\\
			&v=\frac{1}{(4 n+1)}\left[P_{2 n +1}(x)-P_{2 n-1}(x)\right]
		\end{aligned}
		$$
		$$
		\frac{c_{2 n}}{(4 n+1)}=\left.\frac{x}{(4 n+1)}\left[P_{2 n+1}(x)-P_{2 n-1}(x)\right]\right|_0 ^1-\displaystyle\int_0^1 \frac{1}{(4 n+1)}\left[P_{2 n+1}(x)-P_{2n-1}(x)\right] d x
		$$
		we know that:\quad $P_n(1)=1 \quad \forall n \in N$
		so,
		$$
		\frac{c_{2 n}}{(4 n+1)}=\frac{1}{(4 n+1)} \displaystyle\int_0^1\left[P_{2 n-1}(x)-P_{2 n+1}(x)\right] d x
		$$
		$$
		c_{2 n}=\displaystyle\int_0^1 P_{2 n-1}(x) d x-\displaystyle\int_0^1 P_{2 n+1}(x) d x
		$$
		Following the same process as before, we get
		$$
		\begin{aligned}
			&I_1=\displaystyle\int_0^1 P_{2 n-1}(x) d x \\
			&I_1=\frac{2(2 n-1)+1}{2(2n-1)+1} \displaystyle\int_0^1 P_{2 n-1}(x) d x \\
			&I_1=\frac{1}{(4 n-1)} \displaystyle\int_0^1(4 n-1) P_{2 n-1}(x) d x=\frac{1}{(4 n-1)} \displaystyle\int_0^1\left[P_{2 n-1+1}^{\prime}(x)-P_{2 n-1-1}^{\prime}(x)\right] d x\\
			&I_1=\left.\frac{1}{(4 n-1)}\left[P_{2 n}(x)-P_{2 n-2}(x)\right]\right|_0 ^1=\frac{1}{(4 n-1)}\left[P_{2 n-2}(0)-P_{2 n}(0)\right]
		\end{aligned}
		$$
		We can use the Generator function to get
		$$
		P_{2 n}(0)=(-1)^n \displaystyle\frac{(2 n-1) ! !}{(2 n) ! !}
		$$
		$$
		H\left(x,r\right)=\frac{1}{\sqrt{1-2 x r+r^{2}}}=\displaystyle\sum_{n=0}^{\infty} P_{n}(x) r^{n} \quad (1).
		$$
		
		$$
		\begin{aligned}
			H(0,r)=\displaystyle\sum_{n=0}^{\infty} P_n(0) r^n =P_0(0)+P_1(0) t+P_2(0) r^2+P_3(0) r^3+\ldots
		\end{aligned}
		$$
		
		$$
		\begin{aligned}
			\text{Remembering:}\quad (1-u)^{\alpha} =\displaystyle\sum_{k=0}^{\infty}(-1)^k \left(\begin{array}{l}
				\alpha \\
				k
			\end{array}\right) u^{k}
			=1-\alpha u+\frac{\alpha(\alpha-1)}{2 !} u^{2}-\cdots
		\end{aligned}
		$$
		
		$$
		H\left(0,r\right)=\frac{1}{\sqrt{1+r^2}}=1-\frac{1}{2} r^2+\frac{3}{8} r^4+\ldots
		$$
		Comparing these expansions, we have the $P_n(0)=0$ for $n$ odd and for even integers one can show, that
		$$
		P_{2 n}(0)=(-1)^n \frac{(2 n-1) ! !}{(2 n) ! !},
		$$
		where $n ! !$ is the double factorial,
		$$
		n ! != \begin{cases}n(n-2) \ldots(3) 1, & n>0, \text { odd } \\ n(n-2) \ldots(4) 2, & n>0, \text { even } \\ 1, & n=0,-1\end{cases},\quad (2 n) ! !=2^n n !,\quad (2 n-1) ! !=\frac{(2 n) !}{(2 n) ! !}=\frac{(2 n) !}{2^n n !}
		$$
		$$
		\begin{aligned}
			I_1 &=\frac{1}{(4 n-1)}\left[P_{2 n-2}(0)-P_{2 n}(0)\right] \\
			&=\frac{1}{(4 n-1)}\left[(-1)^{n-1} \frac{(2 n-3) ! !}{(2 n-2) ! !}-(-1)^n \frac{(2 n-1) ! !}{(2 n) !}\right] \\
			&=\frac{-1}{(4 n-1)}(-1)^n \frac{(2 n-3) ! !}{(2 n-2) ! !}\left[1+\frac{2 n-1}{2 n}\right] \\
			&=\frac{-1}{(4 n-1)}(-1)^n \frac{(2 n-3) ! !}{(2 n-2) ! !}\left[\frac{4 n-1}{2 n}\right]\\
			&=-\frac{1}{2 n}(-1)^n \frac{(2 n-3) ! !}{(2 n-2) ! !}\\
			&=-(-1)^n \frac{(2 n-3) ! !}{(2 n) ! !}\\
			&=(-1)^{n+1} \frac{(2 n) !}{2^n n !((n-1)} \div 2^n n !\\
			&=(-1)^{n+1} \frac{(2 n) !}{\left[2^n n !\right]^2(2 n-1)}
		\end{aligned}
		$$
		$(2 n-1) ! !=(2 n-1)(2 n-3) ! !\Rightarrow(2 n-3) !!=\frac{(2 n-1) ! !}{(2 n-1)}=\frac{(2 n) !}{2^n n !(2 n-1)}$
		Let take
		$$I_1=B_n $$
		Then
		$$
		\begin{aligned}
			B_n=\frac{1}{(4 n-1)}\left[P_{2 n-2}(0)-P_{2 n}(0)\right]=(-1)^{n+1} \frac{(2 n) !}{\left[2^n n !\right]^2(2 n-1)}
		\end{aligned}
		$$
		Then, we have:
		$$
		\begin{aligned}
			I_2=B_{n+1}=\frac{1}{[4(n+1)-1]}\left[P_{2(n+1)-2}(0)-P_{2(n+1)}(0)\right]=(-1)^{n+2} \frac{[2 n+2] !}{\left[2^{n+1}(n+1) !\right]^2(2 n+1)}\\
		\end{aligned}
		$$
		$$
		\begin{aligned}
			-I_2=-B_{n+1}=(-1)^{n+1} \frac{(2 n+2)(2 n+1)(2 n) !}{\left[2*2^n(n+1)(n) !\right]^2(2 n+1)}
		\end{aligned}
		$$
		$$
		\begin{aligned}
			&c_{2 n}=I_1-I_2\\
			&=(-1)^{n+1} \frac{(2 n) !}{\left[2^n n !\right]^2(2 n-1)}+\frac{(-1)^{n+1}(2 n) ! 2(n+1)(2 n+1)}{\left[2^n n !\right]^2 2^2(n+1)^2(2 n+1)}\\
			&=(-1)^{n+1} \frac{(2 n) !}{[2 n !]^n}\left[\frac{1}{2 n-1}+\frac{1}{2 n+2}\right]\\
			&c_{2 n}=(-1)^{n+1} \frac{(2 n) !}{2(2 n-1)(n+1)\left[2^n n !\right]^2}
		\end{aligned}
		$$
		
		$$
		\Rightarrow|x|= \frac{1}{2} P_0(x)+\frac{5}{8} P_2(x)+\frac{1}{2} \displaystyle\sum_{n=2}^{\infty} \frac{(-1)^n \quad P_{2 n}(x)}{(1-2 n)(n+1)\left[2^n n !\right]^2};\qquad -1<x<1.
		$$
	\end{demo}
	
	\Example{Ejemplo}{Prove the following.
		
		i) If $f$ is periodic and equal to $\operatorname{sign} x$ in $(-\pi, \pi)$, then
		$$
		f(x) \sim \frac{4}{\pi}\left\{\sin x+\frac{\sin 3 x}{3}+\frac{\sin 5 x}{5}+\cdots\right\}
		$$
		ii) Let $0<h<\frac{1}{2} \pi$, and let $f$ be the "triangular" function defined as follows: $f$ is periodic, even, continuous, $f(0)=1, f(x)=0$ for $2 h \leq x \leq \pi, f$ is linear in $(0,2 h)$. Then
		$$
		f \sim \frac{2 h}{\pi}\left[\frac{1}{2}+\displaystyle\sum_{k=1}^{\infty}\left(\frac{\sin k h}{k h}\right)^{2} \cos k x\right]=\frac{h}{\pi}\left[1+\displaystyle\sum_{-\infty}^{+\infty}\left(\frac{\sin k h}{k h}\right)^{2} e^{i k x}\right]
		$$
		iii) Let $g$ be periodic and equal to $\frac{1}{2} \log \left[1 /\left|2 \sin \frac{1}{2} x\right|\right]$ in $(-\pi, \pi)$. Then $\quad
		g \sim \displaystyle\sum_{k=1}^{\infty} \frac{\cos k x}{k}
		$\\
		
		[HINT: For iii), one may either integrate by parts in the formula for the cosine coefficients of $g$, or consider the real part of the series integrate by parts in the formula for the cosine coefficients of $g$, or consider the real part of the series
		$$
		\displaystyle\sum_{k=1}^{\infty} \frac{z^{k}}{k}=\log \frac{1}{1-z}, \quad z=r e^{i x}
		$$
		for $r<1$, and then let $r \rightarrow 1$.]}
	
	\begin{demo}
		i) We know that the Fourier series of a function $f(t)$ over $[-\pi, \pi]$, is given by
		$$
		f(t)=\frac{1}{2} a_{0}+\displaystyle\sum_{n=1}^{\infty} a_{n} \cos (n t)+\displaystyle\sum_{n=1}^{\infty} b_{n} \sin (n t)
		$$
		where
		$$
		\begin{aligned}
			&a_{0}=\frac{1}{\pi} \displaystyle\int_{-\pi}^{\pi} f(t) d t \\
			&a_{n}=\frac{1}{\pi} \displaystyle\int_{-\pi}^{\pi} f(t) \cos (n t) d t \\
			&b_{n}=\frac{1}{\pi} \displaystyle\int_{-\pi}^{\pi} f(t) \sin (n t) d t
		\end{aligned}
		$$
		$
		\operatorname{sign}(t)=\left\{\begin{aligned}
			1 & \text { if } t>0 \\
			0 & \text { if } t=0 \\
			-1 & \text { if } t<0
		\end{aligned}\right.
		$
		The function $f(t)$  is odd, therefore;
		$$
		\begin{aligned}
			&a_{0}=\frac{1}{\pi} \displaystyle\int_{-\pi}^{\pi} sign(t) d t=0 \\
			&a_{n}=\frac{1}{\pi} \displaystyle\int_{-\pi}^{\pi} sign(t) \cos (n t) d t=0 \\
		\end{aligned}
		$$
		
		And,
		$$
		b_{n}=\frac{2}{\pi} \displaystyle\int_{0}^{\pi} sign(t) \sin (n t) dt =\frac{2}{\pi} \displaystyle\int_{0}^{\pi} \sin (n t) dt=\displaystyle\frac{2(1-\cos (\pi n))}{\pi n}=\frac{4}{\pi (2n-1)}
		$$
		
		Then,
		$$
		\textcolor{blue}{\operatorname{sign}(t)\sim \frac{4}{\pi}\displaystyle\sum_{n=1}^{\infty}\frac{\sin(2n-1)t}{(2n-1)}}
		$$
		ii) Let $0<h<\frac{1}{2} \pi$, and let $f$ be the "triangular" function defined as follows: $f$ is periodic, even, continuous, $f(0)=1, f(x)=0$ for $2 h \leq x \leq \pi, f$ is linear in $(0,2 h)$. Then
		$$
		f \sim \frac{2 h}{\pi}\left[\frac{1}{2}+\displaystyle\sum_{k=1}^{\infty}\left(\frac{\sin k h}{k h}\right)^{2} \cos k x\right]=\frac{h}{\pi}\left[1+\displaystyle\sum_{-\infty}^{+\infty}\left(\frac{\sin k h}{k h}\right)^{2} e^{i k x}\right]
		$$
		\textbf{\textcolor{red}{Soluci\'on}}
		$$
		\begin{aligned}
			&f(x)=\displaystyle\sum_{-\infty}^{\infty} C_{n} e^{i \frac{\pi n x}{p}} \\
			&C_{n}=\frac{1}{2 p} \displaystyle\int_{-p}^{p} f(x) e^{\frac{-i n \pi x}{p}} d x
		\end{aligned}
		$$
		$$
		\begin{aligned}
			&C_{0}=\frac{1}{2 \pi} \displaystyle\int_{-\pi}^{\pi} f(x) d x \\
			&C_{0}=\frac{1}{\pi} \displaystyle\int_{0}^{\pi} f(x) d x \\
			&C_{0}=\frac{1}{\pi} \displaystyle\int_{0}^{\pi}-\frac{1}{2 h}(x-2 h) d x\\
			&C_{0}=\frac{1}{\pi} \displaystyle\int_{0}^{2h}-\frac{1}{2 h}(x-2 h) d x\\
			&C_{0}=\frac{h}{\pi}
		\end{aligned}
		$$
		\textcolor{blue}{$C_{0}=\frac{h}{\pi}$}
		$
		C_{n}=\frac{1}{2 \pi} \displaystyle\int_{-\pi}^{\pi} f(x) e^{-i n x} d x,\qquad  C_{n}=\frac{1}{2 \pi} \displaystyle\int_{-\pi}^{\pi} f(x)[\cos (n x)-i \sin(n x)] d x
		$
		$$
		\begin{aligned}
			&\text { because } f(-x)=f(x) \text { is even, then}\\
			&\displaystyle\int_{-\pi}^{\pi} f(x) \sin(n x) d x=0\\
			&C_{n}=\frac{1}{\pi} \displaystyle\int_{0}^{\pi} f(x) \cos (n x) d x ; \quad n \geq 1
		\end{aligned}
		$$
		$$
		\begin{aligned}
			&C_{n}=\frac{1}{\pi} \displaystyle\int_{0}^{\pi}-\frac{1}{2 h}(x-2 h) \cos (n x) d x \\
			&C_{n}=\frac{-1}{2 \pi h} \displaystyle\int_{0}^{2 h}(x-2 h) \cos (n x) d x
		\end{aligned}
		$$
		Integrating by parts we obtain
		$$
		\begin{aligned}
			&\textcolor{blue}{C_{n}=\frac{\operatorname{san}^{2}(h n)}{\pi h n^{2}} ; n \geq 1} \\
			&f(x)=C_{0}+\displaystyle\sum_{n=-\infty}^{\infty} C_{n} e^{i n x} \\
			&f(x)=\frac{h}{\pi}+\displaystyle\sum_{-\infty}^{\infty} \frac{\sin^{2}(h n)}{\pi h n^{2}} e^{i n x} \\
			&\textcolor{blue}{f(x)=\frac{h}{\pi}\left[1+\displaystyle\sum_{n=-\infty}^{\infty}\left(\frac{\sin(h n)}{h n}\right)^{2} e^{i n x}\right]}
		\end{aligned}
		$$
		Now I am going to demonstrate the equivalence of the representation of the Fourier series, and I will do it by converting the left side of the equality to the right side.
		$$
		f\sim \frac{h}{\pi}\left[1+\displaystyle\sum_{-\infty}^{+\infty}\left(\frac{\sin k h}{k h}\right)^{2} e^{i k x}\right]= \frac{2 h}{\pi}\left[\frac{1}{2}+\displaystyle\sum_{k=1}^{\infty}\left(\frac{\sin k h}{k h}\right)^{2} \cos k x\right]
		$$
		$
		f(x)=\displaystyle\frac{h}{\pi}\left[1+\displaystyle\sum_{n=-\infty}^{\infty}\left(\frac{\sin(h n)}{h n}\right)^{2} e^{i n x}\right],\quad \text{Let}\left(\frac{\sin(h n)}{h n}\right)^{2}=B_{n} \Rightarrow B_{n}=B_{-n},\quad  f(x)=\frac{h}{\pi}\left[1+\displaystyle\sum_{-\infty}^{\infty} B_{n} e^{i n x}\right]$
		$$
		\begin{aligned}
			f(x)=\frac{2 h}{\pi}\left[\frac{1}{2}+\frac{1}{2} \displaystyle\sum_{-\infty}^{\infty} B_{n} e^{i n x}\right] &=\frac{2 h}{\pi}\left[\frac{1}{2}+\frac{1}{2} \displaystyle\sum_{n=1}^{\infty} B_{n} e^{i n x}+\frac{1}{2} \displaystyle\sum_{-\infty}^{-1} B_{n} e^{i n x}\right].\\
			&f(x)=\frac{2 h}{\pi}\left[\frac{1}{2}+\frac{1}{2} \displaystyle\sum_{n=1}^{\infty} B_{n}{e}^{i n x}+\frac{1}{2} \displaystyle\sum_{n=1}^{\infty} B_{-n} e^{-i n x}\right]\\
			&f(x)=\frac{2 h}{\pi}\left[\frac{1}{2}+\displaystyle\sum_{n=1}^{\infty} B_{n}\left(\frac{e^{i n x}+e^{-i nx}}{2}\right)\right]\\
			&f(x)=\frac{2 h}{\pi}\left[\frac{1}{2}+\displaystyle\sum_{n=1}^{\infty} B_{n} \cos (n x)\right],\quad \cos (n x)=\frac{e^{i n x}+e^{-i n x}}{2}
		\end{aligned}
		$$
		\textcolor{blue}{
			$$
			\Rightarrow f(x)=\frac{2 h}{\pi}\left[\frac{1}{2}+\displaystyle\sum_{n=1}^{\infty}\left(\frac{\sin(h n)}{n h}\right)^{2} \cos (n x)\right]
			$$
		}
		iii) Let $g$ be periodic and equal to $\frac{1}{2} \log \left[1 /\left|2 \sin \frac{1}{2} x\right|\right]$ in $(-\pi, \pi)$. Then
		$$
		g \sim \displaystyle\sum_{k=1}^{\infty} \frac{\cos k x}{k}
		$$
		
		[HINT: For iii), one may either integrate by parts in the formula for the cosine coefficients of $g$, or consider the real part of the series integrate by parts in the formula for the cosine coefficients of $g$, or consider the real part of the series
		$$
		\displaystyle\sum_{k=1}^{\infty} \frac{z^{k}}{k}=\log \frac{1}{1-z}, \quad z=r e^{i x}
		$$
		for $r<1$, and then let $r \rightarrow 1$.]
		\vspace{0.2cm}
		
		\textbf{\textcolor{red}{Soluci\'on}}
		\vspace{0.2cm}
		
		
		The goal is to compute the Fourier series of $g(x)=-\frac{1}{2}\log |2\sin( x/2)|$ over $[-\pi, \pi],$ or the Fourier series of $g(x)=-\frac{1}{2}\log |2\sin(x)|$ over $[-2\pi, 2\pi].$
		I'll do this, getting the fourier series of $f(x)=\log \cos \frac{x}{2}$ over $[-\pi, \pi]$, and then I will arrive at the desired result, making a translation of the series of f (x).\\
		Since $f(x)$ is an even function, we have to compute:
		$$
		a_{k}=\frac{1}{\pi} \displaystyle\int_{-\pi}^{+\pi} \cos (k x) \log \cos \frac{x}{2} d x=\frac{2}{\pi} \displaystyle\int_{0}^{\pi} \cos (k x) \log \cos \frac{x}{2} d x
		$$
		for any $k \geq 1$ to be able to state:
		$$
		f(x)=\frac{1}{2 \pi} \displaystyle\int_{-\pi}^{\pi} f(x) d x+\displaystyle\sum_{k \geq 1} a_{k} \cos (k x)=\frac{a_{0}}{2}+\displaystyle\sum_{k \geq 1} a_{k} \cos (k x)
		$$
		for any $x \in(-\pi, \pi)$. Integration by parts gives:
		$$
		a_{k}=\frac{2}{\pi}\left(\left.\frac{1}{k} \sin (n x) \log \cos \frac{x}{2}\right|_{0} ^{\pi}+\frac{1}{2 k} \displaystyle\int_{0}^{\pi} \sin (k x) \tan \frac{x}{2} d x\right)
		$$
		or just:
		$$
		a_{k}=\frac{1}{\pi k} \displaystyle\int_{0}^{\pi} \frac{\sin (k x) \sin (x / 2)}{\cos (x / 2)} d x=\frac{2}{\pi k} \displaystyle\int_{0}^{\pi / 2} \frac{\sin (2 k x) \sin x}{\cos x} d x
		$$
		Since $\cos ((2 k+1) x)=2 \cos x \cos (2 k x)-\cos ((2 k-1) x)$, we have:
		$$
		a_{k}=\frac{1}{\pi k} \displaystyle\int_{0}^{\pi / 2} \displaystyle\sum_{n=1}^{k} \cos ((2 n-1) x) d x=\frac{(-1)^{k+1}}{k}
		$$
		$$
		\text{This gives:}\quad \log \cos \frac{x}{2}=\frac{a_{0}}{2}+\displaystyle\sum_{k \geq 1} \frac{(-1)^{k+1}}{k} \cos (k x)
		$$
		for any $x \in(-\pi, \pi)$. In order to find $a_{0}$, we can simply match $f(0)=0$ with the series on the right hand side. Since:
		$$
		\displaystyle\sum_{k \geq 1} \frac{(-1)^{k+1}}{k}=\displaystyle\int_{0}^{1} \frac{d x}{1+x}=\log 2
		$$
		we have:
		$$
		\log \cos \frac{x}{2}=-\log 2+\displaystyle\sum_{k \geq 1} \frac{(-1)^{k+1}}{k} \cos (k x) \quad \forall x \in(-\pi, \pi)
		$$
		and by translating the variable:
		$$
		\log \cos \left(\frac{x-\pi}{2}\right)=-\log 2+\displaystyle\sum_{k \geq 1} \frac{(-1)^{k+1}}{k} \cos (k( x-\pi)) \quad \forall x \in(-2\pi, 2\pi)
		$$
		then
		\textcolor{blue}{
			$$
			\begin{array}{ll}
				\log \sin \frac{x}{2}=-\log 2-\displaystyle\sum_{k \geq 1} \frac{1}{k} \cos (k x) & \forall x \in(0,2 \pi),\quad \heartsuit
			\end{array}
			$$
		}
		or
		$$
		\begin{array}{ll}
			\log \sin x=-\log 2-\displaystyle\sum_{k \geq 1} \frac{1}{k} \cos (2 k x) & \forall x \in(0, \pi)
		\end{array}
		$$
		
		now we are ready to write the series of g (x).\\
		
		$g(x)=-\frac{1}{2}\log(2) -\frac{1}{2}\log \sin( x/2).$\\
		
		Then from \textcolor{blue}{$\heartsuit,$}  we have,
		\textcolor{blue}{
			$$
			\begin{array}{ll}
				-\frac{1}{2}\log \sin \frac{x}{2}=\frac{1}{2}\log 2+\frac{1}{2}\displaystyle\sum_{k \geq 1} \frac{1}{k} \cos (k x) & \forall x \in(0,2 \pi),\quad \heartsuit
			\end{array}
			$$
		}
		\textcolor{blue}{
			$$
			\begin{array}{ll}
				g(x)=-\frac{1}{2}\log 2-\frac{1}{2}\log \sin \frac{x}{2}=\frac{1}{2}\displaystyle\sum_{k \geq 1} \frac{1}{k} \cos (k x) & \forall x \in(0,2 \pi),\quad \heartsuit
			\end{array}
			$$
		}
		as wanted.
	\end{demo}
	
	



\begin{comment}
Pedro cambio
\mychapter{Ecuaciones Diferenciales Parciaales}{ \begin{wrapfigure}{l}{0.45\textwidth}
		\centering
		\includegraphics[width=0.45\textwidth]{Sir.jpg}
\end{wrapfigure} 
Las ecuaciones diferenciales son expresiones matemáticas que relacionan una función desconocida 
con sus derivadas. Estas ecuaciones surgen de manera natural al modelar
 fenómenos que involucran cambios continuos, como el crecimiento 
 poblacional, la velocidad de un objeto en movimiento, la propagación del 
 calor o la evolución de sistemas eléctricos. Dependiendo del tipo
  de derivadas involucradas, las ecuaciones diferenciales pueden ser 
  ordinarias (EDO), si solo dependen de una variable independiente, o 
  parciales (EDP), si dependen de varias. Su estudio permite predecir el 
  comportamiento de sistemas dinámicos y comprender profundamente las leyes
   que rigen muchos procesos físicos, biológicos, económicos y sociales.

El análisis de ecuaciones diferenciales no solo consiste en encontrar 
soluciones exactas, sino también en comprender su existencia, unicidad
 y estabilidad. A lo largo de la historia, se han desarrollado diversos 
 métodos analíticos y numéricos para resolver diferentes tipos de ecuaciones diferenciales, 
 desde las más simples hasta sistemas altamente no lineales. 
 Las soluciones, cuando no pueden obtenerse de forma explícita,
  pueden analizarse cualitativamente o aproximarse mediante algoritmos 
  computacionales. Esta versatilidad convierte a las ecuaciones
   diferenciales en una herramienta fundamental  en la ciencia aplicada, la ingeniería y la matemática pura.
 
   Pedro Jose 
   }


\section{Bianca}
 
\lipsum[1-12]
\section{Bianca}
 
\lipsum[1-12]
\section{Bianca}
 
\lipsum[1-12]

\part{Ecuaciones Diferenciales Ordinarias}

\begin{fullwidth}[%\begin{fullwidth}[%
	width=\dimexpr\textwidth+\marginparsep+\marginparwidth,
	outermargin=\dimexpr-\marginparsep-\marginparwidth,
	]
	
	\chapter{Ecuaciones Diferenciales Ordinarias}
\end{fullwidth}
\end{comment}
\appendix 

%\mychapter{Funciones gamma, beta y s\'imbolo de Pochhammer.}{Funciones gamma, beta y s\'imbolo de Pochhammer.}
\chapter{Funciones gamma, beta y s\'imbolo de Pochhammer.}\label{aped.A}

Las funciones gamma y beta son de las funciones m\'as importantes en matem\'atica, m\'as all\'a de las funciones exponenciales y logar\'itmicas. \\
La funci\'on gamma se denota por $\Gamma{(a)}$,\, (gamma de a) y tiene su representaci\'on integral:
\begin{eqnarray}\label{representacion de Gamma}
	% \nonumber % Remove numbering (before each equation)
	\Gamma{(a)}&=&\int_{0}^{\infty} e^{-t}t^{a-1}dt\quad \, Re\, a>0
\end{eqnarray}

A partir de dicha representaci\'on integral, vamos a deducir ciertas propiedades o relaciones de la funci\'on gamma, que utilizaremos frecuentemente en el desarrollo de este trabajo.

\section{Propiedades de la funci\'on Gamma}
\begin{eqnarray}\label{Relaci\'on de recurrencia de la funci\'on gamma}
	% \nonumber % Remove numbering (before each equation)
	\Gamma{(a+1)}&=&a\Gamma{(a)}
\end{eqnarray}

\begin{demo}
	De \ref{representacion de Gamma}
	\begin{eqnarray*}
		\Gamma{(a+1)}&=&\int_{0}^{\infty} e^{-t}t^{a}dt
	\end{eqnarray*}
	integrando por partes $u=t^{a}\Rightarrow du=a t^{a-1}dt\quad y \quad dv=e^{-t}\Rightarrow v=-e^{-t}$
	\begin{eqnarray*}
		\Gamma{(a+1)}&=&-t^{a}e^{-t}\bigg|_{0}^{\infty}+a \int_{0}^{\infty} e^{-t}t^{a-1}dt=a\Gamma{(a)} \quad Re\, a>0
	\end{eqnarray*}
\end{demo}
 
\begin{eqnarray}\label{Relaci\'on de recurrencia generalizada de la funci\'on gamma}
	% \nonumber % Remove numbering (before each equation)
	\Gamma{(a+n)}&=&(a+n-1)(a+n-2)\cdots(a+2)(a+1)a\Gamma{(a)}\,:\, n=1,2,3,...
\end{eqnarray}
\begin{demo}
	\begin{eqnarray*}
		\Gamma{(a+n)}&=&(a+n-1)\Gamma{(a+n-1)}\\
		&=&(a+n-1)(a+n-2)\Gamma{(a+n-2)}\\
		&=&(a+n-1)(a+n-2)(a+n-3)\Gamma{(a+n-3)}\\
		& \vdots \\
		\Gamma{(a+n)}&=&(a+n-1)(a+n-2)\cdots(a+2)(a+1)a\,\Gamma{(a)}\,:\,n=1,2,3,...
	\end{eqnarray*}
\end{demo}


\begin{eqnarray}\label{Funci\'on gamma en t\'erminos de la productoria}
	\frac{\Gamma{(a+n)}}{\Gamma{(a)}}&=&\prod_{k=0}^{n-1}(a+k)\quad n=1,2,3,...
\end{eqnarray}
\begin{demo}
	\begin{eqnarray*}
		\Gamma{(a+n)}&=&(a+n-1)(a+n-2)\cdots(a+2)(a+1)a\,\Gamma{(a)}\\
		\frac{\Gamma{(a+n)}}{\Gamma{(a)}}&=&(a+n-1)(a+n-2)\cdots(a+2)(a+1)a\\
		\frac{\Gamma{(a+n)}}{\Gamma{(a)}}&=&\prod_{k=0}^{n-1}(a+k)=a(a+1)(a+2)(a+3)\cdots(a+n-1)\quad\,n=1,2,3,...
	\end{eqnarray*}
\end{demo}



En la siguiente propiedade demostraremos la relaci\'on entre la funci\'on gamma y el s\'imbolo de Pochhammer
\begin{eqnarray}\label{Relaci\'on entre la funci\'on gamma y el s\'imbolo de Pochhammer}
	\frac{\Gamma{(a+n)}}{\Gamma{(a)}}&=&(a)_{n}
\end{eqnarray}
$(a)_{n}$ representa el \textit{shifted factorial o s\'imbolo de Pochhammer} definido por:
\begin{equation}\label{P1}
	(a)_{n}=\begin{cases}
		1 & \text{si $n= 0$}\\
		a(a+1)(a+2)\cdots{(a+n-1)} & \text{si $n=1,\,2,\,3,\,...$}
	\end{cases}\\ \end{equation}
\begin{demo}
	Esta prueba es inmediata, pues por definici\'on:
	\begin{equation}\label{P1}(a)_{n}=\begin{cases}
			1 & \text{si $n= 0$}\\
			a(a+1)(a+2)\cdots{(a+n-1)}=\displaystyle\prod_{k=0}^{n-1}(a+k)=\frac{\Gamma{(a+n)}}{\Gamma{(n)}} & \text{si $n=1,\,2,\,3,\,...$}
		\end{cases}\\ \end{equation}
	Lo que se concluye
	$$ \frac{\Gamma{(a+n)}}{\Gamma{(n)}}=(a)_{n}\,:\,n=1,2,3,...$$
\end{demo}


Ahora obtendremos la relaci\'on entre la funci\'on gamma y la funci\'on beta
\begin{eqnarray}\label{relaci\'on entre la funci\'on gamma y la funci\'on beta}
	\frac{\Gamma{(a)}\Gamma{(b)}}{\Gamma{(a+b)}}&=& B(a,b)
\end{eqnarray}
$B(a,b)$ denota la funci\'on beta y tiene su representaci\'on integral:
\begin{eqnarray}
	B{(a,b)}&=&\int_{0}^{1}t^{a-1} (1-t)^{b-1}dt \quad Re\, a>0, \, Re\, b>0
\end{eqnarray}
\begin{demo}
	Sea $x=rcos^{2}\theta,\, y=rsin^{2}(\theta),\, 0\leq r<\infty,\, 0\leq \theta \leq{\frac{\pi}{2}}.$\, Entonces el Jacobiano es $2rcos(\theta) sin(\theta).$\quad As\'i:
	
	\begin{eqnarray*}
		\Gamma{(a)}\Gamma{(b)}&=&\int_{0}^{\infty} e^{-x}x^{a-1}dx\int_{0}^{\infty} e^{-y}y^{b-1}dy=\int_{0}^{\infty}\int_{0}^{\infty} x^{a-1}y^{b-1}e^{-x}e^{-y}dxdy\\
		\Gamma{(a)}\Gamma{(b)}&=&\int_{0}^{\infty}\int_{0}^{\frac{\pi}{2}}
		(rcos^{2}(\theta))^{a-1}(rsin^{2}(\theta))^{b-1}e^{-rcos^{2}(\theta)}e^{-rsin^{2}(\theta)}(2rcos(\theta) sin(\theta) drd\theta)\\
		\Gamma{(a)}\Gamma{(b)}&=&\int_{0}^{\infty}r^{a-1}r^{b-1}e^{-r}rdr \cdot{2}\int_{0}^{\frac{\pi}{2}} (cos^{2}(\theta))^{a-1}(sin^{2}(\theta))^{b-1}(cos(\theta) sin(\theta) d\theta)\\
		\Gamma{(a)}\Gamma{(b)}&=&\int_{0}^{\infty}r^{a+b-1}e^{-r}dr \cdot{2}\int_{0}^{\frac{\pi}{2}} (cos^{2}(\theta))^{a-1}(1-cos^{2}(\theta))^{b-1}(cos(\theta) sin(\theta) d\theta)\\
		\Gamma{(a)}\Gamma{(b)}&=&\Gamma{(a+b)} \cdot{2}\int_{0}^{\frac{\pi}{2}} (cos^{2}(\theta))^{a-1}(1-cos^{2}(\theta))^{b-1}(cos(\theta) sin(\theta) d\theta)
	\end{eqnarray*}
	Sea $$t=cos^{2}(\theta) \Rightarrow dt=-2cos(\theta) sin(\theta) d\theta$$
	
	\begin{equation}\begin{cases}
			\text{si}\quad \theta \rightarrow 0 &  t\rightarrow 1\\
			\text{si}\quad \theta \rightarrow \frac{\pi}{2} &  t\rightarrow 0
		\end{cases}\\ \end{equation}
	\begin{eqnarray*}
		\Gamma{(a)}\Gamma{(b)}&=&\Gamma{(a+b)} \left(-\int_{1}^{0} t^{a-1}(1-t)^{b-1}dt\right)=\Gamma{(a+b)} \int_{0}^{1} t^{a-1}(1-t)^{b-1}dt\\
		\Gamma{(a)}\Gamma{(b)}&=&\Gamma{(a+b)} B(a,b)\Rightarrow \frac{\Gamma{(a)}\Gamma{(b)}}{\Gamma{(a+b)}}= B(a,b)
	\end{eqnarray*}
	Acontinuaci\'on analizaremos la F\'ormula de duplicaci\'on de Legendre
	\begin{eqnarray}\label{F\'ormula de duplicaci\'on de Legendre}
		\Gamma{(2a)}&=&\frac{2^{2a-1}}{\Gamma{(\frac{1}{2})}}\Gamma{(a)}\Gamma{(a+\frac{1}{2})}
	\end{eqnarray}
	\begin{demo}
		Como $$\frac{\Gamma{(a)}\Gamma{(b)}}{\Gamma{(a+b)}}= B(a,b) \Rightarrow \frac{\Gamma{(a)}\Gamma{(a)}}{\Gamma{(2a)}}= B(a,a)=\int_{0}^{1}t^{a-1} (1-t)^{a-1}dt=\int_{0}^{1}[t(1-t)]^{a} \frac{dt}{t(1-t)}.$$
		Tomando la sustituci\'on $u=4t(1-t)$,\, en el intervalo $0 \leq t \leq \frac{1}{2}\Rightarrow du=4(1-2t)dt$\\
		$\Rightarrow du=4\sqrt{1-u}dt\Rightarrow dt=\frac{du}{4\sqrt{1-u}},$\, y reemplaz\'adola en la expresi\'on anterior tenemos:
		$$\Rightarrow \frac{\Gamma{(a)}\Gamma{(a)}}{\Gamma{(2a)}}=2\int_{0}^{\frac{1}{2}}[t(1-t)]^{a} \frac{dt}{t(1-t)}=2\int_{0}^{1}\left(\frac{u}{4}\right)^{a} \frac{1}{t(1-t)}\frac{du}{4\sqrt{1-u}}.$$
		$$\Rightarrow \frac{\Gamma{(a)}\Gamma{(a)}}{\Gamma{(2a)}}=2\int_{0}^{1}\left(\frac{u}{4}\right)^{a} \frac{du}{u\sqrt{1-u}}$$
		$$\Rightarrow \frac{\Gamma{(a)}\Gamma{(a)}}{\Gamma{(2a)}}=\frac{2}{4^a}\int_{0}^{1} u^{a-1} (1-u)^{-\frac{1}{2}}du=2^{1-2a}B(a,\frac{1}{2})=2^{1-2a}\frac{\Gamma{(a)}\Gamma{(\frac{1}{2})}}{\Gamma{(a+\frac{1}{2})}}.$$
		Finalmente, al despejar se obtiene:
		$$\Gamma{(2a)}=\frac{2^{2a-1}}{\Gamma{(\frac{1}{2})}}\Gamma{(a)}\Gamma{(a+\frac{1}{2})}.$$
	\end{demo}
\end{demo}

\begin{eqnarray}
	\Gamma{(\frac{1}{2})}&=&\sqrt{\pi}
\end{eqnarray}
\begin{demo}
	\begin{eqnarray*}
		% \nonumber % Remove numbering (before each equation)
		\left[\Gamma{\left(\frac{1}{2}\right)}\right]^{2} &=& \displaystyle\Gamma{\left(\frac{1}{2}\right)}\cdot \Gamma{\left(\frac{1}{2}\right)}\\
		&=&\displaystyle\int_{0}^{\infty} e^{-x}x^{-\frac{1}{2}}dx\int_{0}^{\infty} e^{-y}y^{-\frac{1}{2}}dy
	\end{eqnarray*}
	Tomando el siguiente cambio de variable
	\begin{equation}\begin{cases}
			\text{sea}\quad x=rcos^{2}(\theta)& y=rsin^{2}(\theta)\\
			0\leq r<\infty &0\leq \theta \leq{\frac{\pi}{2}}
		\end{cases}\\ \end{equation}
	Entonces el Jacobiano es $2rcos(\theta) sin(\theta).$\, As\'i:
	$$\left[\Gamma{\left(\frac{1}{2}\right)}\right]^{2}=\int_{0}^{\infty} \int_{0}^{\frac{\pi}{2}}  e^{-r(sin^{2}\theta+cos^{2}\theta)}(rcos^{2}\theta)^{-\frac{1}{2}}(rsin^{2}(\theta))^{-\frac{1}{2}}(2rcos(\theta) sin(\theta) dr\cdot d{\theta}).$$
	$$\Rightarrow \left[\Gamma{\left(\frac{1}{2}\right)}\right]^{2}=\left(2 \int_{0}^{\infty}e^{-r} dr\right) \left(\int_{0}^{\frac{\pi}{2}}d{\theta} \right)=2(-e^{-r}|_{0}^{\infty})(\theta|_{0}^{\frac{\pi}{2}})=2(1)(\frac{\pi}{2})=\pi \Rightarrow \Gamma{(\frac{1}{2})}=\sqrt{\pi} .$$
\end{demo}

\section{Variable Compleja}

\begin{comment}
	\begin{teo}{Polo Simple}{}\label{PoloSimple}
		Si $f$ tiene un polo simple en $z=z_0$, entonces,
		\begin{eqnarray}\label{polosimple}
			Res\left(f(z), z_0\right)&=&\displaystyle\lim _{z \rightarrow z_0}\left(z-z_0\right) f(z)
		\end{eqnarray}
	\end{teo}
	
\end{comment}

\begin{demo}
	Puesto que $f$ tiene un polo simple en $z=z_0$, su desarrollo de Laurent convergente en un disco perforado $0<\left|z-z_0\right|<R$ tiene la forma
	\begin{eqnarray*}
		f(z)&=&\displaystyle\frac{a_{-1}}{z-z_0}+a_0+a_1\left(z-z_0\right)+a_2\left(z-z_0\right)+\cdots,
	\end{eqnarray*}
	donde $a_{-1} \neq 0$. Al multiplicar ambos lados de esta serie por $z-z_0$ y luego tomando el límite cuando $z \rightarrow z_0$ obtenemos
	$$
	\begin{aligned}
		\displaystyle\lim _{z \rightarrow z_0}\left(z-z_0\right) f(z) & =\lim _{z \rightarrow z_0}\left[a_{-1}+a_0\left(z-z_0\right)+a_1\left(z-z_0\right)^2+\cdots\right] \\
		& =a_{-1}=\operatorname{Res}\left(f(z), z_0\right)
	\end{aligned}
	$$
\end{demo}

\begin{comment}
	\begin{teo}{Polo de Orden}{}\label{PoloN}
		Si $f$ tiene un polo de orden $n$ en $z=z_0$, entonces,
		\begin{eqnarray}\label{poloN}
			Res\left(f(z), z_0\right)&=&\frac{1}{(n-1) !} \lim _{z \rightarrow z_0} \frac{d^{n-1}}{d z^{n-1}}\left[\left(z-z_0\right)^n f(z)\right]
		\end{eqnarray}
	\end{teo}
\end{comment}


\begin{demo}
	Debido a que se supone que $f$ tiene polo de orden $n$ en $z=z_0$, su desarrollo de Laurent convergente en un disco perforado $0<\left|z-z_0\right|<R$ debe tener la forma
	\begin{eqnarray*}
		f(z)&=&\frac{a_{-n}}{\left(z-z_0\right)^n}+\cdots+\frac{a_{-2}}{\left(z-z_0\right)^2}+\frac{a_{-1}}{z-z_0}+a_0+a_1\left(z-z_0\right)+\cdots,
	\end{eqnarray*}
	donde $a_{-n} \neq 0$. Multiplicamos la \'ultima expresi\'on por $\left(z-z_0\right)^n$,
	\begin{eqnarray*}
		\left(z-z_0\right)^n f(z)&=&a_{-n}+\cdots+a_{-2}\left(z-z_0\right)^{n-2}+a_{-1}\left(z-z_0\right)^{n-1}+a_0\left(z-z_0\right)^n+a_1\left(z-z_0\right)^{n+1}+\cdots
	\end{eqnarray*}
	y despu\'es derivando $n-1$ veces ambos lados de la igualdad:
	\begin{eqnarray}\label{ins}
		\displaystyle\frac{d^{n-1}}{d z^{n-1}}\left[\left(z-z_0\right)^n f(z)\right]&=&(n-1) ! a_{-1}+n ! a_0\left(z-z_0\right)+\cdots
	\end{eqnarray}
	Ya que todos los t\'erminos del lado derecho despu\'es del primero involucran potencias enteras positivas de $z-z_0$, el l\'imite de (\ref{ins}), cuando $z \rightarrow z_0$ es
	\begin{eqnarray*}
		\lim _{z \rightarrow z_0} \frac{d^{n-1}}{d z^{n-1}}\left[\left(z-z_0\right)^n f(z)\right]&=&(n-1) ! a_{-1}
	\end{eqnarray*}
	Resolviendo la \'ultima ecuaci\'on para $a_{-1}$ se obtiene (\ref{poloN}).
\end{demo}

\begin{comment}
	\begin{teo}{Cauchy}{}\label{cauchy}
		Sea $D$ un dominio simplemente conexo y $C$ un contorno simple y cerrado que se encuentra en el interior de $D$. Si una funci\'on $f$ es anal\'itica sobre y dentro de $C$, excepto en un n\'umero finito de singularidades aisladas $z_1, z_2, \ldots, z_n$ dentro de $C$, entonces
		\begin{eqnarray}\label{cauchyteo}
			\oint_C f(z) d z&=&2 \pi i \sum_{k=1}^n \operatorname{Res}\left(f(z), z_k\right)
		\end{eqnarray}
	\end{teo}
	
\end{comment}

\begin{demo}
	Supongamos que $C_1, C_2, \ldots, C_n$ son circunferencias con centro en $z_1, z_2, \ldots, z_n$, respectivamente. Adem\'as que cada circunferencia $C_k$ tiene un radio $r_k$ suficientemente peque\~no tal que $C_1, C_2, \ldots, C_n$
	
	son mutuamente disjuntas y est\'an en el interior de la curva cerrada simple $C$. Vea la figura 6.5.1. Ahora en (20) de la secci\'on 6.3 vimos que $\oint_{C_k} f(z) d z=2 \pi i \operatorname{Res}\left(f(z), z_k\right)$, y así por el teorema 5.3.2 tenemos
	$$
	\oint_C f(z) d z=\sum_{k=1}^n \oint_{C_k} f(z) d z=2 \pi i \sum_{k=1}^n \operatorname{Res}\left(f(z), z_k\right)
	$$
\end{demo}

% -- La caja ---
\newtcolorbox{mybox}[1]{
	enhanced,
	sharp corners=all,
	colback=white,
	colframe=gray,
	toprule=0pt,
	bottomrule=0pt,
	leftrule=1pt,
	rightrule=1pt,
	overlay={
		\draw[gray,line width=1pt] (frame.north west) -- ++(2cm,0pt);
		\draw[gray,line width=1pt] (frame.south east) -- ++(-2cm,0pt);
	},
	attach boxed title to top left={yshift=-20pt},
	boxed title style={frame hidden,interior hidden},
	top=.75cm,
	title={\bfseries\color{black}#1},
	breakable % Permite que la caja se divida entre páginas
}

%---------------------------------------

 \chapter{Referencias}
 

\section{Reglas de la Derivación}

\begin{fullwidth}[%\begin{fullwidth}[%
	width=\dimexpr\textwidth+\marginparsep+\marginparwidth,
	outermargin=\dimexpr-\marginparsep-\marginparwidth,
	]


\begin{mybox}{\textbf{\large \color{blue}Fórmulas generales}}
	\begin{minipage}[t]{0.48\textwidth}
		\begin{itemize}
			\item[1.] \( \dfrac{d}{dx} (c) = 0 \)
			\item[2.] \( \dfrac{d}{dx} [cf(x)] = cf'(x) \)
			\item[3.] \( \dfrac{d}{dx} [f(x) + g(x)] = f'(x) + g'(x) \)
			\item[4.] \( \dfrac{d}{dx} [f(x) - g(x)] = f'(x) - g'(x) \)
		\end{itemize}
	\end{minipage}
	\hfill
	\begin{minipage}[t]{0.48\textwidth}
		\begin{itemize}
			\item[5.] \( \dfrac{d}{dx} [f(x)g(x)] = f(x) g'(x) + g(x) f'(x) \)
			\item[6.] \( \dfrac{d}{dx} \left[ \dfrac{f(x)}{g(x)} \right] = \dfrac{g(x)f'(x) - f(x)g'(x)}{[g(x)]^2} \)
			\item[7.] \( \dfrac{d}{dx} f(g(x)) = f'(g(x)) g'(x) \)
			\item[8.] \( \dfrac{d}{dx} (x^n) = n x^{n-1} \)
		\end{itemize}
	\end{minipage}
\end{mybox}

\vspace{0.5cm}

\begin{mybox}{\textbf{\large \color{blue}Funciones exponenciales y logarítmicas}}
	\begin{minipage}[t]{0.48\textwidth}
		\begin{enumerate}
			\setcounter{enumi}{8}
			\item \( \dfrac{d}{dx} (e^x) = e^x \)
			\item \( \dfrac{d}{dx} (b^x) = b^x \ln b \)
		\end{enumerate}
	\end{minipage}
	\hfill
	\begin{minipage}[t]{0.48\textwidth}
		\begin{enumerate}
			\setcounter{enumi}{10}
			\item \( \dfrac{d}{dx} \ln |x| = \dfrac{1}{x} \)
			\item \( \dfrac{d}{dx} (\log_b x) = \dfrac{1}{x \ln b} \)
		\end{enumerate}
	\end{minipage}
\end{mybox}

\vspace{0.5cm}

\begin{mybox}{\textbf{\large \color{blue}Funciones trigonométricas}}
	\begin{minipage}[t]{0.48\textwidth}
		\begin{enumerate}
			\setcounter{enumi}{12}
			\item \( \dfrac{d}{dx} (\sen x) = \cos x \)
			\item \( \dfrac{d}{dx} (\cos x) = -\sen x \)
			\item \( \dfrac{d}{dx} (\tan x) = \sec^2 x \)
		\end{enumerate}
	\end{minipage}
	\hfill
	\begin{minipage}[t]{0.48\textwidth}
		\begin{enumerate}
			\setcounter{enumi}{15}
			\item \( \dfrac{d}{dx} (\csc x) = -\csc x \cot x \)
			\item \( \dfrac{d}{dx} (\sec x) = \sec x \tan x \)
			\item \( \dfrac{d}{dx} (\cot x) = -\csc^2 x \)
		\end{enumerate}
	\end{minipage}
\end{mybox}



\begin{mybox}{\textbf{\large \color{blue}Funciones trigonométricas inversas}}
	\begin{minipage}[t]{0.48\textwidth}
		\begin{enumerate}
			\setcounter{enumi}{18}
			\item \( \dfrac{d}{dx} (\sen^{-1} x) = \dfrac{1}{\sqrt{1 - x^2}} \)
			\item \( \dfrac{d}{dx} (\cos^{-1} x) = -\dfrac{1}{\sqrt{1 - x^2}} \)
			\item \( \dfrac{d}{dx} (\tan^{-1} x) = \dfrac{1}{1 + x^2} \)
		\end{enumerate}
	\end{minipage}
	\hfill
	\begin{minipage}[t]{0.48\textwidth}
		\begin{enumerate}
			\setcounter{enumi}{21}
			\item \( \dfrac{d}{dx} (\csc^{-1} x) = -\dfrac{1}{x\sqrt{x^2 - 1}} \)
			\item \( \dfrac{d}{dx} (\sec^{-1} x) = \dfrac{1}{x\sqrt{x^2 - 1}} \)
			\item \( \dfrac{d}{dx} (\cot^{-1} x) = -\dfrac{1}{1 + x^2} \)
		\end{enumerate}
	\end{minipage}
\end{mybox}

\vspace{0.5cm}

\begin{mybox}{\textbf{\large \color{blue}Funciones hiperbólicas}}
	\begin{minipage}[t]{0.48\textwidth}
		\begin{enumerate}
			\setcounter{enumi}{24}
			\item \( \dfrac{d}{dx} (\senh x) = \cos{h}^2 x \)
			\item \( \dfrac{d}{dx} (\cosh x) = \sen{h}^2 x \)
			\item \( \dfrac{d}{dx} (\tanh x) = \sec{h}^2 x \)
		\end{enumerate}
	\end{minipage}
	\hfill
	\begin{minipage}[t]{0.48\textwidth}
		\begin{enumerate}
			\setcounter{enumi}{27}
			\item \( \dfrac{d}{dx} (\csch x) = -\csch x \coth x \)
			\item \( \dfrac{d}{dx} (\sech x) = -\sech x \tanh x \)
			\item \( \dfrac{d}{dx} (\coth x) = -csch^2  x \)
		\end{enumerate}
	\end{minipage}
\end{mybox}

\vspace{0.5cm}

\begin{mybox}{\textbf{\large \color{blue}Funciones hiperbólicas inversas}}
	\begin{minipage}[t]{0.48\textwidth}
		\begin{enumerate}
			\setcounter{enumi}{30}
			\item \( \dfrac{d}{dx} (\senh^{-1} x) = \dfrac{1}{\sqrt{1 + x^2}} \)
			\item \( \dfrac{d}{dx} (\cosh^{-1} x) = \dfrac{1}{\sqrt{x^2 - 1}} \)
			\item \( \dfrac{d}{dx} (\tanh^{-1} x) = \dfrac{1}{1 - x^2} \)
		\end{enumerate}
	\end{minipage}
	\hfill
	\begin{minipage}[t]{0.48\textwidth}
		\begin{enumerate}
			\setcounter{enumi}{33}
			\item \( \dfrac{d}{dx} (\csch^{-1} x) = -\dfrac{1}{|x|\sqrt{x^2 + 1}} \)
			\item \( \dfrac{d}{dx} (\sech^{-1} x) = -\dfrac{1}{x\sqrt{1 - x^2}} \)
			\item \( \dfrac{d}{dx} (\coth^{-1} x) = \dfrac{1}{1 - x^2} \)
		\end{enumerate}
	\end{minipage}
\end{mybox}



\end{fullwidth}

\newpage

\begin{fullwidth}[%\begin{fullwidth}[%
	width=\dimexpr\textwidth+\marginparsep+\marginparwidth,
	outermargin=\dimexpr-\marginparsep-\marginparwidth,
	]

\section{Tabla de Integrales}

\begin{mybox}{\textbf{\large \color{blue}Formas básicas}}
	\begin{minipage}[t]{0.34\textwidth}
		\begin{enumerate}
			\item \( \displaystyle \int u \, dv = uv - \int v \, du \)
			\item \( \displaystyle \int u^n \, du = \dfrac{u^{n+1}}{n+1} + C, \quad n \neq -1 \)
			\item \( \displaystyle \int \dfrac{du}{u} = \ln |u| + C \)
			\item \( \displaystyle \int e^u \, du = e^u + C \)
			\item \( \displaystyle \int b^u \, du = \dfrac{b^u}{\ln b} + C \)
			\item \( \displaystyle \int \sen u \, du = -\cos u + C \)
			\item \( \displaystyle \int \cos u \, du = \sen u + C \)
			\item \( \displaystyle \int \sec^2 u \, du = \tan u + C \)
		\end{enumerate}
	\end{minipage}
	\hfill
	\begin{minipage}[t]{0.28\textwidth}
		\begin{enumerate}
			\setcounter{enumi}{7}
		
			\item \( \displaystyle \int \csc^2 u \, du = -\cot u + C \)
			\item \( \displaystyle \int \sec u \tan u \, du = \sec u + C \)
			\item \( \displaystyle \int \csc u \cot u \, du = -\csc u + C \)
			\item \( \displaystyle \int \tan u \, du = \ln |\sec u| + C \)
			\item \( \displaystyle \int \cot u \, du = \ln |\sen u| + C \)
			\item \( \displaystyle \int \sec u \, du = \ln |\sec u + \tan u| + C \)
		\end{enumerate}
	\end{minipage}
	\hfill
	\begin{minipage}[t]{0.28\textwidth}
		\begin{enumerate}
			\setcounter{enumi}{14}
			\item \( \displaystyle \int \csc u \, du = \ln |\csc u - \cot u| + C \)
			\item \( \displaystyle \int \dfrac{du}{\sqrt{a^2 - u^2}} = \sen^{-1} \dfrac{u}{a} + C, \quad a > 0 \)
			\item \( \displaystyle \int \dfrac{du}{a^2 + u^2} = \dfrac{1}{a} \tan^{-1} \dfrac{u}{a} + C \)
			\item \( \displaystyle \int \dfrac{du}{u\sqrt{u^2 - a^2}} = \dfrac{1}{a} \sec^{-1} \dfrac{u}{a} + C \)
			\item \( \displaystyle \int \dfrac{du}{a^2 - u^2} = \dfrac{1}{2a} \ln \left| \dfrac{u+a}{u-a} \right| + C \)
			\item \( \displaystyle \int \dfrac{du}{u^2 - a^2} = \dfrac{1}{2a} \ln \left| \dfrac{u-a}{u+a} \right| + C \)
		\end{enumerate}
	\end{minipage}
\end{mybox}


\vspace{-0.5cm}

\begin{mybox}{\textbf{\large \color{blue}Formas que involucran \( \sqrt{a^2 + u^2} \), \quad \( a > 0 \)}}
	\begin{enumerate}
		\setcounter{enumi}{20}
		\item \( \displaystyle \int \sqrt{a^2 + u^2} \, du = \dfrac{u}{2} \sqrt{a^2 + u^2} + \dfrac{a^2}{2} \ln (u + \sqrt{a^2 + u^2}) + C \)
		\item \( \displaystyle \int u^2 \sqrt{a^2 + u^2} \, du = \dfrac{u}{8} (a^2 + 2u^2) \sqrt{a^2 + u^2} - \dfrac{a^4}{8} \ln (u + \sqrt{a^2 + u^2}) + C \)
		\item \( \displaystyle \int \dfrac{\sqrt{a^2 + u^2}}{u} \, du = \sqrt{a^2 + u^2} - a \ln \left| \dfrac{a + \sqrt{a^2 + u^2}}{u} \right| + C \)
		\item \( \displaystyle \int \dfrac{\sqrt{a^2 + u^2}}{u^2} \, du = -\dfrac{\sqrt{a^2 + u^2}}{u} + \ln (u + \sqrt{a^2 + u^2}) + C \)
		\item \( \displaystyle \int \dfrac{du}{\sqrt{a^2 + u^2}} = \ln (u + \sqrt{a^2 + u^2}) + C \)
		\item \( \displaystyle \int \dfrac{u^2 \, du}{\sqrt{a^2 + u^2}} = \dfrac{u}{2} \sqrt{a^2 + u^2} - \dfrac{a^2}{2} \ln (u + \sqrt{a^2 + u^2}) + C \)
		
		\begin{minipage}[t]{0.48\textwidth}
			\item \( \displaystyle \int \dfrac{du}{u\sqrt{a^2 + u^2}} = -\dfrac{1}{a} \ln \left| \dfrac{\sqrt{a^2 + u^2} + a}{u} \right| + C \)
		\end{minipage}%
		\hfill
		\begin{minipage}[t]{0.48\textwidth}
			\item \( \displaystyle \int \dfrac{du}{u^2 \sqrt{a^2 + u^2}} = -\dfrac{\sqrt{a^2 + u^2}}{a^2 u} + C \)
		\end{minipage}
			
			\item \( \displaystyle \int \dfrac{du}{(a^2 + u^2)^{3/2}} = \dfrac{u}{a^2 \sqrt{a^2 + u^2}} + C \)
		\end{enumerate}
	\end{mybox}



\begin{mybox}{\textbf{\large \color{blue}Formas que involucran $\sqrt{a^2 - u^2}$, \quad $a > 0$}}
	\begin{enumerate}
		\setcounter{enumi}{29}
		\item $\displaystyle \int \sqrt{a^2 - u^2} \, du = \frac{u}{2} \sqrt{a^2 - u^2} + \frac{a^2}{2} \sin^{-1} \frac{u}{a} + C$
		\item $\displaystyle \int u^2 \sqrt{a^2 - u^2} \, du = \frac{u}{8} (2u^2 - a^2) \sqrt{a^2 - u^2} + \frac{a^4}{8} \sin^{-1} \frac{u}{a} + C$
		\item $\displaystyle \int \frac{\sqrt{a^2 - u^2}}{u} \, du = \sqrt{a^2 - u^2} - a \ln \left| \frac{a + \sqrt{a^2 - u^2}}{u} \right| + C$
		\item $\displaystyle \int \frac{\sqrt{a^2 - u^2}}{u^2} \, du = -\frac{1}{u} \sqrt{a^2 - u^2} - \sin^{-1} \frac{u}{a} + C$
		\item $\displaystyle \int \frac{u^2 \, du}{\sqrt{a^2 - u^2}} = \frac{u}{2} \sqrt{a^2 - u^2} - \frac{a^2}{2} \sin^{-1} \frac{u}{a} + C$
		\item $\displaystyle \int \frac{du}{u \sqrt{a^2 - u^2}} = \frac{1}{a} \ln \left| \frac{a + \sqrt{a^2 - u^2}}{u} \right| + C$
		\item $\displaystyle \int \frac{du}{u^2 \sqrt{a^2 - u^2}} = -\frac{\sqrt{a^2 - u^2}}{a^2 u} + C$
		\item $\displaystyle \int \frac{(a^2 - u^2)^{3/2} \, du}{u} = \frac{u}{8} (2u^2 - 5a^2) \sqrt{a^2 - u^2} + \frac{3a^4}{8} \sin^{-1} \frac{u}{a} + C$
		\item $\displaystyle \int \frac{du}{(a^2 - u^2)^{3/2}} = \frac{u}{a^2 \sqrt{a^2 - u^2}} + C$
	\end{enumerate}
\end{mybox}




\begin{mybox}{\textbf{\large \color{blue}Formas que involucran $\sqrt{u^2 - a^2}$, \quad $a > 0$}}
	\begin{enumerate}
		\setcounter{enumi}{38}
		\item $\displaystyle \int \sqrt{u^2 - a^2} \, du = \frac{u}{2} \sqrt{u^2 - a^2} - \frac{a^2}{2} \ln |u + \sqrt{u^2 - a^2}| + C$
		\item $\displaystyle \int u^2 \sqrt{u^2 - a^2} \, du = \frac{u}{8} (2u^2 - a^2) \sqrt{u^2 - a^2} - \frac{a^4}{8} \ln |u + \sqrt{u^2 - a^2}| + C$
		\item $\displaystyle \int \frac{\sqrt{u^2 - a^2}}{u} \, du = \sqrt{u^2 - a^2} - a \cosh^{-1} \frac{u}{a} + C$
		\item $\displaystyle \int \frac{\sqrt{u^2 - a^2}}{u^2} \, du = -\frac{\sqrt{u^2 - a^2}}{u} + \ln |u + \sqrt{u^2 - a^2}| + C$
		\item $\displaystyle \int \frac{du}{\sqrt{u^2 - a^2}} = \ln |u + \sqrt{u^2 - a^2}| + C$
		\item $\displaystyle \int \frac{u^2 \, du}{\sqrt{u^2 - a^2}} = \frac{u}{2} \sqrt{u^2 - a^2} - \frac{a^2}{2} \ln |u + \sqrt{u^2 - a^2}| + C$
		
		\begin{minipage}[t]{0.48\textwidth}
			\item $\displaystyle \int \frac{du}{u \sqrt{u^2 - a^2}} = \frac{1}{a} \ln \left| \frac{\sqrt{u^2 - a^2} + a}{u} \right| + C$
		\end{minipage}%
		\hfill
		\begin{minipage}[t]{0.48\textwidth}
			\item $\displaystyle \int \frac{du}{u^2 \sqrt{u^2 - a^2}} = -\frac{\sqrt{u^2 - a^2}}{a^2 u} + C$
		\end{minipage}
		
		\item $\displaystyle \int \frac{du}{(u^2 - a^2)^{3/2}} = \frac{u}{a^2 \sqrt{u^2 - a^2}} + C$
	\end{enumerate}
\end{mybox}



\begin{mybox}{\textbf{\Large \color{blue}Formas que involucran $a + bu$}}
	\begin{enumerate}
		\setcounter{enumi}{46} % Para continuar la numeración
		\item $\displaystyle \int \frac{u \,du}{a + bu} = \frac{1}{b^2} (a + bu - a \ln |a + bu|) + C$
		\item $\displaystyle \int \frac{u^2 \,du}{a + bu} = \frac{1}{2b^3} \left[ (a + bu)^2 - 4a(a + bu) + 2a^2 \ln |a + bu| \right] + C$
		\item $\displaystyle \int \frac{du}{a + bu} = \frac{1}{a} \ln \left| \frac{u}{a + bu} \right| + C$
		\item $\displaystyle \int \frac{du}{u^2(a + bu)} = \frac{1}{a} + \frac{b}{a^2} \ln \left| \frac{a + bu}{u} \right| + C$
		\item $\displaystyle \int \frac{u \,du}{(a + bu)^2} = \frac{1}{b^2(a + bu)} + \frac{1}{b^2} \ln |a + bu| + C$
		\item $\displaystyle \int \frac{du}{u(a + bu)^2} = \frac{1}{a(a + bu)} - \frac{a}{a^2} \ln \left| \frac{a + bu}{u} \right| + C$
		\item $\displaystyle \int \frac{u^2 \,du}{(a + bu)^2} = \frac{1}{b^3} \left( a + bu - \frac{a^2}{a + bu} - 2a \ln |a + bu| \right) + C$
		\item $\displaystyle \int \sqrt{a + bu} \, du = \frac{2}{15b^2} (3bu - 2a)(a + bu)^{3/2} + C$
		\item $\displaystyle \int \frac{du}{\sqrt{a + bu}} = \frac{2}{3b^2} (bu - 2a) \sqrt{a + bu} + C$
		\item $\displaystyle \int \frac{u^2 \,du}{\sqrt{a + bu}} = \frac{2}{15b^3} (8a^2 + 3b^2u^2 - 4abu)\sqrt{a + bu} + C$
		\item $\displaystyle \int \frac{du}{\sqrt{a + bu}} = \frac{1}{\sqrt{a + bu}} - \frac{\sqrt{a + bu}}{\sqrt{a}} + C, \quad \text{si } a > 0$
		\item $\quad \quad = \frac{2}{\sqrt{-a}} \tan^{-1} \sqrt{\frac{a + bu}{-a}} + C, \quad \text{si } a < 0$
		\item $\displaystyle \int \frac{du}{u \sqrt{a + bu}} = 2 \sqrt{a + bu} + a \int \frac{du}{\sqrt{a + bu}}$
		\item $\displaystyle \int \frac{du}{u^2} = -\frac{\sqrt{a + bu}}{u} + \frac{b}{2} \int \frac{du}{\sqrt{a + bu}}$
		\item $\displaystyle \int u^n \sqrt{a + bu} \, du = \frac{2}{b(2n + 3)} u^n(a + bu)^{3/2} - \frac{n}{b(2n + 1)} \int u^{n-1} \sqrt{a + bu} \, du$
		\item $\displaystyle \int \frac{du}{\sqrt{a + bu}} = \ln |\sqrt{a + bu} + a| + C$
		\item $\displaystyle \int \frac{u \,du}{\sqrt{a + bu}} = \frac{2u^2 + a}{b(2n + 1)} - \frac{2na}{b(2n + 1)} \int \frac{du}{\sqrt{a + bu}}$
		\item $\displaystyle \int \frac{du}{\sqrt{a + bu}} = \frac{b(2n - 3)}{2a(2n - 1)} \int \frac{du}{\sqrt{a + bu}}$
	\end{enumerate}
\end{mybox}


\newpage


\section*{\textbf{\color{blue}Formas trigonométricas}}

\noindent
\begin{minipage}{0.32\textwidth}
	\begin{align*}
		63. & \int \sen^2 u \, du = \frac{1}{2} u - \frac{1}{4} \sen 2u + C \\[8pt]
		64. & \int \cos^2 u \, du = \frac{1}{2} u + \frac{1}{4} \sen 2u + C \\[8pt]
		65. & \int \tan^2 u \, du = \tan u - u + C \\[8pt]
		66. & \int \cot^2 u \, du = -\cot u - u + C \\[8pt]
		67. & \int \sen^3 u \, du = -\frac{1}{3} (2 + \sen^2 u) \cos u + C \\[8pt]
		68. & \int \cos^3 u \, du = \frac{1}{3} (2 + \cos^2 u) \sen u + C \\[8pt]
		69. & \int \tan^3 u \, du = \frac{1}{2} \tan^2 u + \ln |\cos u| + C \\[8pt]
		70. & \int \cot^3 u \, du = -\frac{1}{2} \cot^2 u - \ln |\sen u| + C
	\end{align*}
\end{minipage}
%\hfill

\begin{minipage}{0.32\textwidth}
	\begin{align*}
		79. & \int \sen au \sen bu \, du = \frac{\sen (a-b)u}{2(a-b)} - \frac{\sen (a+b)u}{2(a+b)} + C \\[8pt]
		80. & \int \cos au \cos bu \, du = \frac{\sen (a-b)u}{2(a-b)} + \frac{\sen (a+b)u}{2(a+b)} + C \\[8pt]
		81. & \int \sen au \cos bu \, du = -\frac{\cos (a-b)u}{2(a-b)} - \frac{\cos (a+b)u}{2(a+b)} + C \\[8pt]
		82. & \int u \sen u \, du = \sen u - u \cos u + C \\[8pt]
		83. & \int u \cos u \, du = \cos u + u \sen u + C \\[8pt]
		84. & \int u^n \sen u \, du = -u^n \cos u + n \int u^{n-1} \cos u \, du \\[8pt]
		85. & \int u^n \cos u \, du = u^n \sen u - n \int u^{n-1} \sen u \, du \\[8pt]
		86. & \int \sen^n u \cos^m u \, du = -\frac{\sen^{n-1} u \cos^{m+1} u}{n+m} + \frac{n-1}{n+m} \int \sen^{n-2} u \cos^m u \, du \\[8pt]
		& = \frac{\sen^{n+1} u \cos^{m-1} u}{n+m} + \frac{m-1}{n+m} \int \sen^{n-2} u \cos^{m-2} u \, du
	\end{align*}
\end{minipage}

%\hfill

\begin{minipage}{0.32\textwidth}
	\begin{align*}
		71. & \int \sec^3 u \, du = \frac{1}{2} \sec u \tan u + \frac{1}{2} \ln |\sec u + \tan u| + C \\[8pt]
		72. & \int \csc^3 u \, du = -\frac{1}{2} \csc u \cot u + \frac{1}{2} \ln |\csc u - \cot u| + C \\[8pt]
		73. & \int \sen^n u \, du = -\frac{1}{n} \sen^{n-1} u \cos u + \frac{n-1}{n} \int \sen^{n-2} u \, du \\[8pt]
		74. & \int \cos^n u \, du = \frac{1}{n} \cos^{n-1} u \sen u + \frac{n-1}{n} \int \cos^{n-2} u \, du \\[8pt]
		75. & \int \tan^n u \, du = \frac{1}{n-1} \tan^{n-1} u - \int \tan^{n-2} u \, du \\[8pt]
		76. & \int \cot^n u \, du = \frac{-1}{n-1} \cot^{n-1} u - \int \cot^{n-2} u \, du \\[8pt]
		77. & \int \sec^n u \, du = \frac{1}{n-1} \tan u \sec^{n-2} u + \frac{n-2}{n-1} \int \sec^{n-2} u \, du \\[8pt]
		78. & \int \csc^n u \, du = \frac{-1}{n-1} u \csc^{n-2} u + \frac{n-2}{n-1} \int \csc^{n-2} u \, du
	\end{align*}
\end{minipage}


\vspace{0.5cm}

% Trigonométricas inversas
\section*{\textbf{\color{blue}Formas trigonométricas inversas}}

	
		\begin{minipage}{0.48\textwidth}
		\begin{align*}
			87. & \int \sen^{-1} u \, du = u \sen^{-1} u + \sqrt{1 - u^2} + C \\[8pt]
			88. & \int \cos^{-1} u \, du = u \cos^{-1} u - \sqrt{1 - u^2} + C \\[8pt]
			89. & \int \tan^{-1} u \, du = u \tan^{-1} u - \frac{1}{2} \ln(1 + u^2) + C \\[8pt]
			90. & \int \sen^{-1} u \, du = \frac{2u^2 - 1}{4} \sen^{-1} u + \frac{\sqrt{1 - u^2}}{4} + C \\[8pt]
			91. & \int \cos^{-1} u \, du = \frac{2u^2 - 1}{4} \cos^{-1} u - \frac{\sqrt{1 - u^2}}{4} + C
		\end{align*}
	\end{minipage}
	\hfill
	\begin{minipage}{0.48\textwidth}
		\begin{align*}
			92. & \int u \tan^{-1} u \, du = \frac{u^2 + 1}{2} \tan^{-1} u - \frac{u}{2} + C \\[8pt]
			93. & \int u^n \sen^{-1} u \, du = \frac{1}{n+1} \left[ u^{n+1} \sen^{-1} u - \int \frac{u^{n+1} du}{\sqrt{1 - u^2}} \right], \quad n \neq -1 \\[8pt]
			94. & \int u^n \cos^{-1} u \, du = \frac{1}{n+1} \left[ u^{n+1} \cos^{-1} u - \int \frac{u^{n+1} du}{\sqrt{1 - u^2}} \right], \quad n \neq -1 \\[8pt]
			95. & \int u^n \tan^{-1} u \, du = \frac{1}{n+1} \left[ u^{n+1} \tan^{-1} u - \int \frac{u^{n+1} du}{1 + u^2} \right], \quad n \neq -1
		\end{align*}
	\end{minipage}
	
	\newpage


% Exponenciales y logarítmicas
\section*{\textbf{\color{blue}Formas exponenciales y logarítmicas}}

\begin{minipage}{0.48\textwidth}
	\begin{align*}
		96. & \int u e^u \, du = \frac{1}{a^2} (au - 1)e^u + C \\[8pt]
		97. & \int u^n e^{au} \, du = \frac{1}{a} \left[ u^n e^{au} - \frac{n}{a} \int u^{n-1} e^{au} \, du \right] \\[8pt]
		98. & \int e^{au} \sen bu \, du = \frac{e^{au}}{a^2 + b^2} (a \sen bu - b \cos bu) + C \\[8pt]
		99. & \int e^{au} \cos bu \, du = \frac{e^{au}}{a^2 + b^2} (a \cos bu + b \sin bu) + C
	\end{align*}
\end{minipage}
\hfill
\begin{minipage}{0.48\textwidth}
	\begin{align*}
		100. & \int \ln u \, du = u \ln u - u + C \\[8pt]
		101. & \int u^n \ln u \, du = \frac{u^{n+1}}{(n+1)^2} \left[ (n+1) \ln u - 1 \right] + C \\[8pt]
		102. & \int \frac{1}{u} \ln u \, du = \ln |\ln u| + C
	\end{align*}
\end{minipage}







\vspace{10pt}

% Hiperbólicas

\section*{\textbf{\color{blue}Formas hiperbólicas}}

	\begin{minipage}{0.48\textwidth}
	\begin{align*}
		103. & \int \senh u \, du = \cosh u + C \\[8pt]
		104. & \int \cosh u \, du = \senh u + C \\[8pt]
		105. & \int \tanh u \, du = \ln |\cosh u| + C \\[8pt]
		106. & \int \coth u \, du = \ln |\senh u| + C \\[8pt]
		107. & \int \sech u \, du = \tan^{-1} |\senh u| + C
	\end{align*}
\end{minipage}
\hfill
\begin{minipage}{0.48\textwidth}
	\begin{align*}
		108. & \int \csch u \, du = \ln |\tanh \frac{u}{2}| + C \\[8pt]
		109. & \int \sech^2 u \, du = \tanh u + C \\[8pt]
		110. & \int \csch^2 u \, du = -\coth u + C \\[8pt]
		111. & \int \sech u \tanh u \, du = -\sech u + C \\[8pt]
		112. & \int \csch u \coth u \, du = -\csch u + C
	\end{align*}
\end{minipage}



\section*{\textbf{\color{blue}Transformadas de Laplace}}

\begin{center}
\renewcommand{\arraystretch}{2}
\begin{tabular}{ll}
	\hline
	\( f(t) \) & \( \mathcal{L} \{ f(t) \} = F(s) \) \\
	\hline
	1. \( 1 \) & \( \displaystyle \frac{1}{s} \) \\
	
	2. \( t \) & \( \displaystyle \frac{1}{s^2} \) \\
	
	3. \( t^n \) & \( \displaystyle \frac{n!}{s^{n+1}}, \quad n \text{ entero positivo} \) \\
	
	4. \( t^{-1/2} \) & \( \displaystyle \sqrt{\frac{\pi}{s}} \) \\
	
	5. \( t^{1/2} \) & \( \displaystyle \frac{\sqrt{\pi}}{2s^{3/2}} \) \\
	
	6. \( t^\alpha \) & \( \displaystyle \frac{\Gamma(\alpha+1)}{s^{\alpha+1}}, \quad \alpha > -1 \) \\
	
	7. \( \sen kt \) & \( \displaystyle \frac{k}{s^2 + k^2} \) \\
	
	8. \( \cos kt \) & \( \displaystyle \frac{s}{s^2 + k^2} \) \\
	
	9. \( \sen^2 kt \) & \( \displaystyle \frac{2k^2}{s(s^2 + 4k^2)} \) \\
	
	10. \( \cos^2 kt \) & \( \displaystyle \frac{s^2 + 2k^2}{s(s^2 + 4k^2)} \) \\
	
	11. \( e^{at} \) & \( \displaystyle \frac{1}{s - a} \) \\
	
	12. \( \senh kt \) & \( \displaystyle \frac{k}{s^2 - k^2} \) \\

	13. \( \cosh kt \) & \( \displaystyle \frac{s}{s^2 - k^2} \) \\
	
	14. \( \senh^2 kt \) & \( \displaystyle \frac{2k^2}{s(s^2 - 4k^2)} \) \\
	
	15. \( \cosh^2 kt \) & \( \displaystyle \frac{s^2 - 2k^2}{s(s^2 - 4k^2)} \) \\
	
	16. \( t e^{at} \) & \( \displaystyle \frac{1}{(s-a)^2} \) \\
	
	17. \( t^n e^{at} \) & \( \displaystyle \frac{n!}{(s-a)^{n+1}}, \quad n \text{ entero positivo} \) \\
	18. \( e^{at} \operatorname{sen}kt \) &  \(\displaystyle \dfrac{k}{(s-a)^{2}+k^{2}} \)

\end{tabular}
\end{center}

\begin{comment}
	

\begin{center}
	\renewcommand{\arraystretch}{2}
	\begin{tabular}{lcl}
		f(0) && \displaystyle 2tf(0)=F(0) \\
		\\
		18. & \displaystyle e^{at}\operatorname{sen}kt & \displaystyle \dfrac{k}{(s-a)^{2}+k^{2}} \\
		\\
		19. & \displaystyle e^{at}\cos kt & \displaystyle \dfrac{s-a}{(s-a)^{2}+k^{2}} \\
		\\
		20. & \displaystyle e^{at}\operatorname{senh}kt & \displaystyle \dfrac{k}{(s-a)^{2}-k^{2}} \\
		\\
		21. & \displaystyle e^{at}\cosh kt & \displaystyle \dfrac{s-a}{(s-a)^{2}-k^{2}} \\
		\\
		22. & \displaystyle t\operatorname{sen}kt & \displaystyle \dfrac{2ks}{(s^{2}+k^{2})^{2}} \\
		\\
		23. & \displaystyle t\cos kt & \displaystyle \dfrac{s^{3}-k^{2}}{(s^{2}+k^{2})^{2}} \\
		\\
		24. & \displaystyle \operatorname{sen}kt+kt\cos kt & \displaystyle \dfrac{2ks^{2}}{(s^{2}+k^{2})^{2}} \\
		\\
		25. & \displaystyle \operatorname{sen}kt-kt\cos kt & \displaystyle \dfrac{2k^{2}}{(s^{2}+k^{2})^{2}} \\
		\\
		26. & \displaystyle t\operatorname{senh}kt & \displaystyle \dfrac{2ks}{(s^{2}-k^{2})^{2}} \\
		\\
		27. & \displaystyle t\cosh kt & \displaystyle \dfrac{s^{3}+k^{2}}{(s^{2}-k^{2})^{2}} \\
		\\
		28. & \displaystyle \dfrac{e^{at}-e^{bt}}{a-b} & \displaystyle \dfrac{1}{(s-a)(s-b)} \\
		\\
		29. & \displaystyle \dfrac{ae^{at}-be^{bt}}{a-b} & \displaystyle \dfrac{s}{(s-a)(s-b)} \\
		\\
		30. & \displaystyle 1-\cos kt & \displaystyle \dfrac{k^{2}}{s(s^{2}+k^{2})} \\
		\\
		31. & \displaystyle kt-\operatorname{sen}kt & \displaystyle \dfrac{k^{2}}{s^{2}(s^{2}+k^{2})} \\
		\\
		32. & \displaystyle \dfrac{a\operatorname{sen}bt-b\operatorname{sen}at}{ab(a^{2}-b^{2})} & \displaystyle \dfrac{1}{(s^{2}+a^{2})(s^{2}+b^{2})} \\
		\\
		33. & \displaystyle \dfrac{\cos bt-\cos at}{a^{2}-b^{2}} & \displaystyle \dfrac{s}{(s^{2}+a^{2})(s^{2}+b^{2})} \\
		\\
		34. & \displaystyle \operatorname{sen}kt\operatorname{senh}kt & \displaystyle \dfrac{2k^{2}s}{s^{4}+4k^{4}} \\
		\\
		35. & \displaystyle \operatorname{sen}kt\cosh kt & \displaystyle \dfrac{k(k^{2}+2k^{2})}{s^{4}+4k^{4}} \\
		\\
		36. & \displaystyle \cos kt\operatorname{senh}kt & \displaystyle \dfrac{k(k^{2}-2k^{2})}{s^{4}+4k^{4}} \\
		\\
		37. & \displaystyle \cos kt\cosh kt & \displaystyle \dfrac{s^{3}}{s^{4}+4k^{4}} \\
	\end{tabular}
\end{center}


\end{comment}

\end{fullwidth}















\backmatter


%----  Respuesta del libro ---- % 

\chapter{Respuesta a todos los problemas}


\begin{multicols}{2}\raggedcolumns
	\shipoutAnswer
\end{multicols}

\pagestyle{headings}   
 %------ Bibliografia ------%

\addcontentsline{toc}{chapter}{Bibliografía}
\bibliographystyle{unsrt}

\begin{thebibliography}{99}
	\bibitem{Piheira}
	A D\'iaz; H Pijeira ; J Quintero.
	{\it Electrostatic models for zeros of Laguerre-Sobolev polynomials}. Agosto 2023
	\bibitem{Szego1939}
	G Szego.
	{\it \text { ORTHOGONAL POLYNOMIALS }}. American Mathematical Society Providence, Rhode Island 1939
\end{thebibliography}



\makefrontcoverback






\end{document} 

